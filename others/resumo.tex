No âmbito industrial, o custo empregado na manutenção de equipamentos ainda representa uma grande
parcela dos investimentos. Dessa forma, o desenvolvimento de técnicas de manutenção, e o seu correto
planejamento, cada vez mais estão assumindo papeis de grande importância nesse setor, visto que
impactam diretamente no fator econômico das empresas. Neste sentido, o trabalho em questão apresenta
a proposta de uma arquitetura orientada a serviços para um sistema de manutenção inteligente, a fim
de possibilitar a integração de forma facilitada entre equipamentos e as ferramentas de análise de
degradação. A arquitetura é composta por diferentes entidades, cada uma responsável por determinada
tarefa para o funcionamento do sistema. No trabalho, as entidades são implementadas e experimentos
são realizados a fim de validar a solução proposta.
