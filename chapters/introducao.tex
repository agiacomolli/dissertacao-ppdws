\chapter{Introdução}

Atualmente, a importância do emprego de técnicas de manutenção no âmbito industrial está em
constante ascensão devido a necessidade de aumentar a disponibilidade e segurança dos equipamentos,
bem como a qualidade do processo produtivo~\cite{muller2008formalisation}. O custo empregado
anualmente com processos de manutenção está na faixa de 15\% para a indústria de manufatura, entre
20\% a 30\% para a indústria química e na faixa de 40\% para a indústria do aço e
siderúrgica~\cite{chu1998predictive, nguyen2008new}. Dessa forma, o desenvolvimento de novas
técnicas de manutenção para uso nas mais diversas áreas e o correto planejamento dos processos de
manutenção estão cada vez mais importantes, uma vez que impactam diretamente no fator econômico,
alterando a disponibilidade do sistema e também a segurança~\cite{zhao2010soabased}.

Nos últimos anos, tem-se observado um crescimento no uso de um novo paradigma de manutenção
denominado de manutenção inteligente~\cite{zhang2013performance}. Este novo paradigma visa
transformar a forma como as técnicas de manutenção são utilizadas. Diferentemente dos métodos
tradicionais, conhecidos por aplicar o conserto aos equipamento somente após a falha ou por manterem
processos de manutenção agendados baseado no histórico de falhas dos componentes, o paradigma de
manutenção inteligente visa predizer a condição do sistema e prevenir uma possível falha.
Segundo~\cite{bloch2012machinery}, 99\% das falhas em sistemas mecânicos podem ser observadas por
indicadores perceptíveis. Dessa forma, é possível a utilização de técnicas de manutenção inteligente
empregadas no monitoramento contínuo da saúde do sistema, de forma a não interromper a operação dos
equipamentos.

As tecnologias empregadas na manutenção contínua do sistema, bem como diagnóstico de falhas, tiveram
grande desenvolvimento nas últimas décadas e visam predizer o estado do
sistema~\cite{heng2009rotating}. Assinatura de sinais de vibração e de emissão acústica puderam ser
obtidos, processados e analisados através de sensores e softwares computacionais. Novas pesquisas
nestas áreas estão em constante evolução e, cada vez mais, utilizam técnicas modernas para análise e
processamento de sinais. Isso se torna possível com os avanços da eletrônica e computação, que cada
vez mais propiciam ferramentas e técnicas para a resolução de problemas~\cite{zhao2010predictive}.
Como exemplo, pode-se citar o uso de métodos baseados em redes neurais ou de mapas auto-organizáveis
para detecção de padrões em sinais~\cite{goncalves2011fault}.

Outra linha de pesquisa também está em constante crescimento: o uso de \gls{SOA} ou Arquiteturas
Orientadas a Serviços. O uso do padrão \gls{SOA} está evoluindo e cada vez mais presente em
aplicações nos mais diversos segmentos, sejam eles a nível de dispositivos, na implementação de
camadas de negócios ou mesmo no setor industrial~\cite{candido2010soa, choi2010impact,
ragavan2012service, papazoglou2007service}. É um conceito de arquitetura que suporta acoplamento
mínimo entre componentes, possibilitando ganhos em flexibilidade e interoperabilidade. Dessa forma,
qualquer tipo de aplicação pode ser representada como um conjunto complexo de serviços.

Com a utilização de \gls{SOA}, um recurso ou componente é identificado como um serviço. Cada
entidade apresenta comportamento bem definido e é composta por módulos autocontidos, os quais
permitem que um determinado serviço seja independente do estado ou contexto de outros
serviços~\cite{papazoglou2007service}. As funcionalidades agregadas a um serviço são publicadas e
disponibilizadas através de uma interface padrão, o que possibilita a troca de informações ou
requisição da execução de alguma tarefa entre os componentes~\cite{ragavan2012service}.

O uso de \gls{SOA} no âmbito industrial, como forma de integração dos sistemas, se mostra factível
pelos sucesso de vários projetos~\cite{karnouskos2010towards, bohn2006sirena, de2006soda,
colombo2010factory}. Os projetos demonstram a viabilidade na utilização de serviços em sistemas
embarcados a fim de integrá-los com sistemas \gls{MES} e \gls{ERP}, localizados nos níveis mais
altos da corporação. A utilização de \gls{SOA} em ambientes industriais possibilita o aumento da
flexibilidade do sistema, resultado em rápida adaptação para situações onde são impostas demandas do
mercado~\cite{starke2013flexible}. A configuração de equipamentos de forma flexível utilizando
\gls{SOA}, possibilita aumento na agilidade como os processos desta natureza são executados.


%Motivação

Nesse contexto, um sistema de manutenção inteligente também pode se valer da utilização dos
conceitos empregados pelo padrão \gls{SOA}. Do ponto de vista da arquitetura \gls{SOA}, o sistema de
manutenção pode conter serviços para relatórios de saúde e falhas, informações sobre o prognóstico
do tempo de operação sem necessidade de manutenção, além de serviços de configuração de ferramentas
de diagnóstico ou dos modos de operação suportados pelo equipamento. O monitoramento remoto do
sistema também possibilita a integração das informações de saúde dos equipamentos em sistemas
\gls{MES} e \gls{ERP}, a fim de se obter o correto gerenciamento da cadeia de suprimentos de peças
de reposição~\cite{oldham2003delivering}


%Proposta do trabalho

Mesmo que as pesquisas envolvendo as áreas de sistemas de manutenção e arquiteturas orientadas a
serviços estejam em constante evolução, as iniciativas para integração das duas tecnologias ainda
são escassas. Dessa forma, o trabalho em questão apresenta a proposta de uma arquitetura orientada a
serviços para um sistema de manutenção inteligente. A proposta tem por objetivo possibilitar a
integração de equipamentos com um sistema de manutenção inteligente de forma facilitada e flexível.
Portanto, todas as entidades propostas foram construídas obedecendo os padrões definidos pelas
arquiteturas \gls{SOA}.

A arquitetura proposta engloba diversas entidades, cada uma destinada a um propósito específico. As
entidades foram construídas conforme a demanda encontrada para a integração entre os sistemas aqui
descritos. Dentre as principais, estão o Analisador de Dispositivos, que possibilita a análise
automáticas de equipamentos e o Gerenciador de Análises, o qual permite a criação de planos de
análise pelo operador do sistema, com a definição da utilização dos dados e ferramentas de análise.


%Objetivos

Como objetivos do trabalho, estão a definição da arquitetura proposta e implementação das entidades.
A validação do sistema é feita com a implementação de um dispositivo que representa um conjunto
atuador elétrico e válvula, utilizado para controle de fluxo em redes de distribuição de petróleo.
Com a definição do dispositivo que representa o atuador elétrico, o sistema é posto em
funcionamento, onde o operador do sistema tem acesso às configurações dos equipamentos e planos de
análise, podendo criar novos planos e verificar o resultado dos níveis de degradação obtidos com as
análises.

Por fim, são ilustrados experimentos para determinar a viabilidade da solução proposta. Os
experimentos tem por objetivo verificar a interoperabilidade entre as entidades propostas, bem como
determinar o correto funcionamento dos componentes implementados. São analisados pontos positivos em
relação à utilização do sistema proposto, tendo em vista a comparação com a utilização de métodos
tradicionais de obtenção dos níveis de degradação em equipamentos. Também são verificadas as
vantagens na utilização da arquitetura proposta em relação ao aumento do número de dispositivos
monitorados, levando em conta o gerenciamento de vários dispositivos similares para a obtenção dos
estados de saúde de todos.


% Estrutura

Dando continuidade ao capítulo introdutório, esta dissertação apresenta a seguinte estrutura: no
\cref{cha:conceituacao-teorica} são apresentados os conteúdos teóricos utilizados ao longo do
trabalho, divididos nas duas áreas de estudo, sendo elas a de manutenção inteligente e também
arquiteturas orientadas a serviços; a análise do estado da arte é apresentada no
\cref{cha:estado-arte}, onde são descritos os estudos atuais envolvendo os dois assuntos
pertinentes a este trabalho; a proposta de uma arquitetura orientada a serviços faz parte do
\cref{cha:arquitetura-proposta}, onde é apresentado o projeto dos componentes utilizados no sistema
proposto, além da definição do estudo de caso e dos experimentos de validação; no
\cref{cha:implementacao-resultados}, são apresentados os resultados obtidos com a implementação do
sistema proposto e o resultado dos experimentos definidos no capítulo anterior; por fim, o
\cref{cha:conclusao} apresenta as conclusões e pontos relevantes para a continuidade do trabalho.
