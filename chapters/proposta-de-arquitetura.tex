%\chapter{Arquitetura proposta para um sistema de manutenção inteligente}
\chapter{Arquitetura proposta}

\todo[inline]{Introdução do capítulo. Revisar o capítulo para se adequar às novas alterações.}


%%
\section{Arquitetura orientada a serviços proposta}

A arquitetura proposta neste trabalho tem por objetivo a integração de sistemas de manutenção
inteligente, dos diversos equipamentos que precisam ser monitorados, além de outras entidades que
auxiliam no funcionamento do sistema. A troca de informações entre todos os elementos que compõem a
arquitetura é abstraída na forma de serviços, o que facilita a integração, inserção e remoção de
novas entidades no sistema. Dessa forma, a especificação da arquitetura é definida utilizando os
padrões \gls{SOA}. Uma visão geral das entidades que fazem parte da arquitetura proposta é
apresentada na \cref{fig:soa-proposed-architecture}\todo{Alterar figura. A aplicação SOA deve
englobar mais entidades}. Na figura, nota-se que cada entidade tem uma função específica no sistema
e o acesso às suas funcionalidades é realizado através de serviços. A seguir, para melhor
entendimento, todos os elementos serão detalhados.

\includefigure
    {images/soa-proposed-architecture}
    {Arquitetura orientada a serviços proposta para integração de um sistema de manutenção
        inteligente.}
    {fig:soa-proposed-architecture}


%%%
\subsection{Serviço}

No contexto da arquitetura proposta neste trabalho, serviço é um componente de software que
encapsula uma funcionalidade acessível através dos padrões definidos pela tecnologia \gls{SOA}.
Todos os componentes da arquitetura expõem as suas funcionalidades na forma de serviços,
possibilitando que a interação entre eles seja feita de forma transparente. Como parte do padrão
\gls{SOA}, serviços podem ser descobertos e utilizados por clientes que desejam executar uma
determinada tarefa. Neste contexto, também é possível a definição de serviços mais especializados
com base em outros serviços, praticando a técnica da composição de serviços. Estas características
se tornam inerentes à proposta, devido a utilização do padrão \gls{SOA}.


%%%
\subsection{Dispositivo}

Um dispositivo é um componente de software utilizado para encapsular um elemento físico da aplicação
proposta. Por ser executado em um dispositivo físico, o componente é denominado dispositivo lógico e
disponibiliza serviços para acesso à funcionalidades previamente definidas. As funcionalidades podem
ser relativas ao dispositivo físico em que o componente está executando ou outras que auxiliam em
alguma tarefa específica não relacionada diretamente com o hardware hospedeiro.

Do ponto de vista da aplicação, os dispositivos são entidades que hospedam serviços. Dentre os
serviços hospedados, alguns estão presentes em todos os dispositivos da arquitetura, servindo de
base para a comunicação entre todos os elementos desta classe. Dessa forma, conhecendo os serviços
básicos, uma interface mínima para troca de informações entre as entidades do sistema é definida,
facilitando a inserção de novos dispositivos.

Além dos serviços base, outros podem ser executados no dispositivo. A arquitetura permite o envio de
novos serviços para os dispositivos do sistema. O dispositivo recebe o novo serviço, que possui as
mesmas funcionalidades de um serviço padrão, e o carrega para ser executado normalmente.


%%%
\subsection{Aplicação orientada a serviços}

A \todo{Alterar texto para englobar mais entidades da arquitetura}aplicação orientada a serviços
nada mais é do que o resultado da utilização dos diversos serviços propostos na arquitetura. Através
da composição ou utilização direta dos serviços de uma forma coordenada, a aplicação é construída.
De um modo minimalista, a aplicação pode ser vista como a interação progressiva de funcionalidades
simples provenientes de serviços básicos, resultando em um componente complexo de software. Visto
que o produto final da composição dos elementos do sistema não é definido \textit{a priori}, o
resultado da aplicação pode ser qualquer construção ou interação entre os componentes.


%%%
\subsection{Gerenciador de dispositivos}

Como o nome sugere, o Gerenciador de Dispositivos é um componente de software utilizado para
gerenciar os dispositivos da arquitetura proposta. A configuração ou obtenção da lista dos
dispositivos na aplicação orientada a serviços são exemplos de funcionalidades do gerenciador. Como
os dispositivos são entidades que mapeiam funcionalidades dos dispositivos físicos encontrados em um
sistema de manutenção inteligente, o gerenciador pode, por exemplo, obter a lista de sensores do
equipamento e configurar alguns parâmetros, como a taxa de atualização dos dados dos sensores.
Também é possível definir comportamentos adicionais para o caso do equipamento estar operando em
diferentes níveis de degradação. Os diferentes comportamentos podem ser utilizados quando há
necessidade de operar o equipamento em condições onde manter os níveis de degradação estáveis é mais
importante do que a operação a pleno. Portanto, pode-se manter o equipamento funcionando até que uma
manutenção \textit{in loco} possa ser realizada.

Grupos de dispositivos podem ser criados e gerenciados. O Gerenciador de Dispositivos permite a
criação de grupos, a fim de facilitar a alteração simultânea de vários dispositivos. Supondo que a
aplicação é inerentemente escalável, o número de dispositivos tente a aumentar, o que dificulta o
gerenciamento ou configuração de dispositivos similares. Considerando que as configurações aplicadas
a uma mesma classe de dispositivos é muito parecida, o gerenciador permite que um grupo receba os
mesmos parâmetros, automatizando o processo de customização dos dispositivos.

Outra funcionalidade que o gerenciador provê é a obtenção da topologia dos componentes do sistema.
Em uma arquitetura \gls{SOA} todos os serviços estão localizados no mesmo nível hierárquico. Essa
característica pode ser considerada positiva do ponto de vista de integração e reuso de serviços. Em
contrapartida, ao definir níveis hierárquicos entre componentes de um mesmo dispositivo físico, a
complexidade para o estabelecimento de uma hierarquia lógica para representação das diferentes
partes desse dispositivo aumenta. Contudo, ao aplicar um identificador único para cada dispositivo,
ou parte de dispositivo, do sistema, torna-se possível definir e estabelecer uma relação
hierárquica. O identificador pode ser uma \gls{URI}, por exemplo, onde cada parte do endereço se
refere a um dos níveis hierárquicos.

O Gerenciador de Dispositivos também é utilizado na resolução de problemas encontrados durante a
execução dos componentes do sistema. A entidade se enquadra na categoria de \gls{IHM},
possibilitando que a rede seja mapeada em busca de dispositivos ou serviços de forma interativa.
Além disso, é possível obter o estado dos dispositivo, visualizar dados ou testar os serviços
encontrados.


%%%
\subsection{Gerenciador de análises}

O Gerenciador de Análises é utilizado para definir o plano que será aplicado a determinado
equipamento, ou grupo de equipamentos, \todo{Em ou com?}com vista a obter os níveis de degradação.
Nesta entidade, é possível selecionar e definir a ordem em que serão executadas as ferramentas de
análise de dados através de um plano. O gerenciador permite o agendamento dos planos e a execução
periódica em intervalos de tempo. Os planos também podem ser gerenciados, possibilitando alteração
e/ou exclusão.

No plano são definidos os comportamentos a serem adotados com base nos valores de degradação do
equipamento. O gerenciador obtém os comportamentos previamente definidos para o equipamento a ser
analisado, os quais podem ser mapeados pelo usuário para diferentes níveis de degradação. Dessa
forma, o equipamento pode ser adaptado a diferentes condições, por exemplo, evitando o aumento da
degradação até que uma manutenção possa ser realizada.

Esta entidade também coordena as análises realizadas em um grupo de equipamentos similares. O
processo de obtenção dos valores de confiança de um conjunto de equipamentos pode ser simplificado
pelo fato de que, se são similares, o nível de degradação de um equipamento pode ser aproximado para
o nível de degradação do grupo. Dessa maneira, é possível elaborar um plano de análise que seja
aplicável a vários equipamentos.

\todo[inline]{Utilização da base de dados para armazenamento dos planos.}


%%%
\subsection{Analisador de dispositivos}

O Analisador de Dispositivos concentra todas as ferramentas necessárias para a realização da análise
dos dados de determinado dispositivo. A entidade opera com base nos planos criados e agendados pelo
\todo{Operador do sistema?}usuário no Gerenciador de Análises. De posse do plano, o analisador
executa as análises utilizando os algoritmos necessários para obtenção do resultado esperado. A
execução de um mesmo plano é feita de forma automática.

O analisador permite verificar quais planos estão sendo executados ou na fila para execução. Sempre
que um novo plano é encontrado, o analisador o coloca na fila de execução, a fim de que seja
finalizado o mais breve possível. A entidade permite o monitoramento destes planos pelo usuário, o
qual pode verificar se a execução está de acordo com o esperado.


%%%
\subsection{Base de dados}

A base de dados é utilizada para armazenar todos os dados relativos à aplicação. Desde os dados
obtidos nas análises de degradação dos equipamentos como também os planos de análise agendados pelo
usuário. A base centraliza as informações, mas não necessariamente faz com que a aplicação seja
centralizada. Nada impede a utilização de diferentes \todo{Remotas e locais? Alterar figura}bases de
dados e o compartilhamento de informações entre as entidade do sistema. Por exemplo, o armazenamento
dos dados de treinamento de um equipamento, necessário para obtenção do nível de degradação, pode
ser feito em diferentes bases de dados. O Analisador de Dispositivos pode combinar estas
informações, a fim de melhorar o resultado esperado.

Por ser defina como um componente \gls{SOA}, o acesso aos dados é feito através de serviços. A
inclusão, deleção e modificação de dados são realizadas por serviços especializados. Um dispositivo
remoto pode acessar a base de dados a fim de obter o histórico dos últimos valores de confiança
calculados ou valores de testes intermediários para determinado equipamento, utilizando-os para um
novo tipo de análise.


%%%
\subsection{Repositório de serviços}

O repositório de serviços é uma entidade que armazena os dispositivos lógicos e seus serviços, os
quais podem ser obtidos dinamicamente e implantados em dispositivos físicos. Esta entidade pode
estar localizada na rede local ou remota, com acesso disponível para várias aplicações. Dessa forma,
uma nova aplicação, isolada de outra já existente, pode ser construída com o reúso de componentes
obtidos de um repositório de serviços compartilhado pelas duas. Esta abordagem também permite o
compartilhamento de uma base de dados de componentes lógicos entre aplicações.


%%
\section{Casos de uso da arquitetura proposta}

Os elementos que compõem a arquitetura proposta encaixam-se em alguns casos de uso no contexto de um
sistema de manutenção inteligente. Os casos de uso propostos a seguir exemplificam a interação do
operador do sistema, quando há a necessidade de busca ou configuração dos dispositivos, além da
ferramenta de análise de dados, que executa autonomamente sobre os dados obtidos durante o processo
de amostragem dos equipamentos.\todo{Introdução da seção com definição dos atores da aplicação.}


%%%
\subsection{Descoberta e configuração de dispositivos}

A descoberta dos dispositivos e serviços na rede pode ser feita a qualquer momento utilizando o
explorador de dispositivos. Cada um dos dispositivos encontrados é identificado e colocado em uma
lista de dispositivos ativos. Também são identificados os serviços hospedados em cada um dos
dispositivos. Dessa forma, é possível determinar quais dispositivos estão atualmente disponíveis na
rede e quais possuem serviços que poderão ser utilizados em determinada funcionalidade do sistema.

A geração da hierarquia de recursos encontrados na rede também é possível. Aplicando um
identificador único para cada recurso -- utilizando uma \gls{URI}, por exemplo --, a reconstrução
dos componentes do sistema pode ser obtida agrupando-os em classes ou categorias. O identificador é
utilizado para auxiliar a definir uma topologia para a interação entre os diferentes componentes do
sistema.

O diagrama \gls{UML} de casos de uso para descoberta e configuração de dispositivos é apresentado na
\cref{fig:uml-discovery-setup-devices}. O diagrama apresenta os casos de uso do explorador de
dispositivos e a sua interação com os dispositivos que fazem parte da arquitetura orientada a
serviços. Em vias gerais, o explorador será utilizado pelo operador do sistema para descobrir os
dispositivos da rede, obter o estado ou realizar modificações.

\includefigure
    {images/uml-discovery-setup-devices}
    {Diagrama de casos de uso para descoberta e configuração de dispositivos.}
    {fig:uml-discovery-setup-devices}

\todo[inline]{Novo diagrama com o envio de novos comportamentos para o dispositivo ou inclusão de
informações no mesmo.}


%%%
%\subsection{Acesso à base de dados}

\subsection{Gerenciamento de análises}


%%%
\subsection{Análise dos dados}

\includefigure
    {images/uml-analysis-create}
    {Diagrama de casos de uso para a criação e execução de análise de um equipamento pelo operador
        do sistema.}
    {fig:uml-analysis-create}
