%\chapter{Arquitetura proposta para um sistema de manutenção inteligente}
\chapter{Arquitetura proposta}

\todo[inline]{Introdução do capítulo}


%%
\section{Arquitetura orientada a serviços}

A arquitetura proposta neste trabalho tem por objetivo a integração de sistemas de manutenção
inteligente, dos diversos equipamentos que o utilizam além de outros componentes que auxiliam no
funcionamento do sistema. Todos os elementos que compõem a arquitetura são abstraídos como serviços,
o que facilita a integração, inserção e remoção de novos componentes no sistema. Para tanto, a
especificação da arquitetura é definida utilizando os padrões \gls{SOA}. Uma visão geral das
entidades que fazem parte da arquitetura proposta é apresentada na \cref{fig:soa-proposed-architecture}.

\includefigure
    {images/soa-proposed-architecture}
    {Arquitetura orientada a serviços proposta para integração de um sistema de manutenção
        inteligente.}
    {fig:soa-proposed-architecture}

Como observado na \cref{fig:soa-proposed-architecture}, a arquitetura proposta é composta de
diversas entidades. Cada entidade tem uma função específica no sistema e o acesso às suas
funcionalidades é realizado através de serviços. A seguir, para melhor entendimento, todos
componentes da arquitetura são detalhados.


%%%
\subsection{Serviço}

No contexto da arquitetura proposta neste trabalho, um serviço é um componente de software que
encapsula uma funcionalidade acessível através dos padrões definidos pela tecnologia \gls{SOA}.
Todos os componentes da arquitetura são definidos como serviços, possibilitando que a interação
entre eles seja feita de forma transparente. Como parte do padrão \gls{SOA}, serviços podem ser
descobertos e utilizados por clientes que desejam executar uma determinada tarefa. Também é possível
a definição de serviços mais especializados com base em outros serviços, praticando a técnica da
composição de serviços. Estas características se tornam inerentes à proposta, devido a utilização de
\gls{SOA}.


%%%
\subsection{Dispositivo}

Um dispositivo é um componente de software utilizado para encapsular um elemento da aplicação.
Normalmente é denominado dispositivo lógico, por ser executado em um dispositivo físico,
disponibilizando serviços para acesso à funcionalidades previamente definidas. As funcionalidades
podem ser relativas ao dispositivo físico em que o componente está executando ou outras que auxiliam
em alguma tarefa específica não relacionada diretamente com o hardware em questão.

Do ponto de vista da aplicação, os dispositivos hospedam serviços. Dentre os serviços hospedados,
alguns estão presentes em todos os dispositivos da arquitetura. Dessa forma, uma interface mínima
para troca de informações entre as entidades do sistema é definida, facilitando a entrada de novos
dispositivos.


%%%
\subsection{Aplicação orientada a serviços}

A aplicação orientada a serviços nada mais é do que o resultado da utilização dos diversos serviços
propostos na arquitetura. Através da composição ou utilização direta dos serviços de uma forma
coordenada, a aplicação é construída. De um modo minimalista, a aplicação pode ser vista como a
interação progressiva de funcionalidades simples provenientes de serviços básicos, resultando em um
componente complexo de software. Visto que o produto final da composição dos elementos do sistema
não é definido \textit{a priori}, a aplicação final pode resultar em
qualquer construção ou interação entre os componentes.

%%%
\subsection{Implantador de serviços}

Como forma de implantar novos serviços nos dispositivos físicos encontrados na rede, é necessário um
software dedicado para esta tarefa. O implantador disponibiliza formas de alterar os dispositivos na
rede, seja enviando novos serviços para serem executados ou modificando as informações do
dispositivo. Como cada dispositivo do sistema possui alguns serviços padrão, entre eles o que
permite o recebimento de novos serviços para serem executados no dispositivo, a integração das
funcionalides do implantador pode ser feita de forma facilitada.


%%%
\subsection{Explorador de dispositivos}

O explorador de dispositivos é um componente de software utilizado na análise e configuração dos
dispositivos na aplicação orientada a serviços. De certa forma, também é utilizado na resolução de
problemas encontrados durante a execução dos componentes do sistema. O explorador se enquadra na
categoria de \gls{IHM}, possibilitando que a rede seja mapeada em busca de dispositivos ou serviços
de forma interativa. Além disso, é possível obter o estado dos dispositivo, visulizar dados ou
testar os serviços encontrados.

A implantação de novos serviços em um dispositvo físico também pode ser feita utilizando o
explorador. Ao buscar pelos dispositivos do sistema, também são encontrados os serviços que cada um
disponibiliza. Com os serviços básicos oferecidos por todos os dispositivos da rede, como o de
recebimento de novos serviços, o explorador pode também pode enviar e configurar novos serviços no
dispositivo.

A topologia dos componentes do sistema também pode ser obtida com o explorador. Em um sistema
\gls{SOA} todos os serviços estão localizados no mesmo nível hierárquico. Essa característica pode
ser considerada positiva do ponto de vista de integração e reuso de serviços. Em contrapartida, ao
definir níveis hierárquicos entre componentes de um mesmo dispositivo físico, fica difícil
estabelecer uma hierarquia lógica para representação das diferentes partes desse dispositivo.
Contudo, ao aplicar um identificador único para cada dispositivo do sistema, torna-se possível
definir e estabelecer uma relação hierárquica. O identificador pode ser uma \gls{URI}, por exemplo,
onde cada parte do endereço define um nível hierárquico do dispositivo.


%%%
\subsection{Base de dados}

A base de dados é utilizada para armazenar os dados relativos às análises de degradação dos
componentes do sistema de manutenção inteligente. Isso também inclui os dados de treinamentos de
cada equipamento e as análises intermediárias que são utilizadas para gerar o valor de confiança do
equipamento. Dessa forma, toda a informação dos componentes do sistema que estão sob o monitoramento
de degradação são armazenadas em um local centralizado.

Por ser defina como um componente \gls{SOA}, o acesso aos dados é feito através de serviços. A
inclusão, deleção e modificação dos dados são realizadas por sreviços especializados. Um dispositivo
remoto pode acessar a base de dados a fim de obter um histórico dos últimos valores de confiança
calculados ou valores de testes intermediários, utilizando-os para um novo tipo de análise.


%%%
\subsection{Gerenciador de análise de dados}

O gerenciador de análise de dados é utilizado para analisar os dados dos equipamentos sob inspeção
pelo \gls{IMS}. Nele estão contidos os algoritmos para análise, onde a interface de acesso é
oferecida através de serviços. Definida uma ferramenta para análise de determinado equipamento, o
gerenciador obtém os dados da base de dados e executa a ferramenta selecionada. A partir dos
resultados, novos dados são gerados, os quais são inseridos na base de dados do equipamento em
análise\todo{Precisa ser finalizada.}.


%%%
\subsection{Repositório de serviços}

O repositório de serviços é uma entidade que armazena os dispositivos lógicos e seus serviços, os
quais podem ser obtidos dinamicamente e implantados em dispositivos físicos. Essa entidade pode
estar localizada na rede local ou remota, com acesso disponível para várias aplicações. Dessa forma,
uma nova aplicação, isolada de outra já existente, pode ser construída com o reúso de componentes
obtidos de um repositório de serviços compartilhado pelas duas. Esta abordagem também permite o
compartilhamento de uma base de dados de componentes lógicos entre aplicações.


%%
\section{Casos de uso da arquitetura proposta}

\todo[inline]{Introdução da seção com definição dos atores da aplicação.}


%%%%%%%%%%%%%%%%%%%%%%%%%%%%%%%%%%%%%%%%%%%%%%%%%%%%%%%%%%%%%%%%%%%%%%%%%%%%%%%%%%%%%%%%%%%%%%%%%%%%
\subsection{Descoberta e configuração de dispositivos}

A descoberta dos dispositivos e serviços \todo{Também falar sobre os serviços?} na rede pode ser
feita a qualquer momento utilizando o explorador de dispositivos. Cada um dos dispositivos
encontrados é identificado e colocado em uma lista de dispositivos ativos. Também são identificados
os serviços hospedados em cada um dos dispositivos. Dessa forma, é possível determinar quais
dispositivos estão atualmente disponíveis na rede e quais possuem serviços que poderão ser
utilizados em determinada funcionalidade do sistema.

A geração da hierarquia de recursos encontrados na rede também é possível. Aplicando um
identificador único para cada recurso -- uma \gls{URI}, por exemplo --, a reconstrução dos
componentes do sistema pode ser obtida agrupando-os em classes ou categorias. O identificador é
utilizado com intuito de definir uma topologia para a interação entre os diferentes componentes do
sistema.

\includefigure
    {images/uml-discovery-setup-devices}
    {Diagrama UML para descoberta e configuração de dispositivos.}
    {fig:uml-discovery-setup-devices}

\todo[inline]{Adicionar exemplo com figura ilustrativa e explicação do funcionamento. Utilizar o
atuador elétrico e seus componentes como exemplo.}
\todo[inline]{E a configuração de novos dispositivos?}


%%%%%%%%%%%%%%%%%%%%%%%%%%%%%%%%%%%%%%%%%%%%%%%%%%%%%%%%%%%%%%%%%%%%%%%%%%%%%%%%%%%%%%%%%%%%%%%%%%%%
\subsection{Busca de serviços nos dispositivos}

\todo[inline]{Não faz referência direta à imagem que descreve a arquitetura}


%%%%%%%%%%%%%%%%%%%%%%%%%%%%%%%%%%%%%%%%%%%%%%%%%%%%%%%%%%%%%%%%%%%%%%%%%%%%%%%%%%%%%%%%%%%%%%%%%%%%
\subsection{Troca dinâmica de dispositivos}


%%%%%%%%%%%%%%%%%%%%%%%%%%%%%%%%%%%%%%%%%%%%%%%%%%%%%%%%%%%%%%%%%%%%%%%%%%%%%%%%%%%%%%%%%%%%%%%%%%%%
\subsection{Acesso à base de dados}


%%%%%%%%%%%%%%%%%%%%%%%%%%%%%%%%%%%%%%%%%%%%%%%%%%%%%%%%%%%%%%%%%%%%%%%%%%%%%%%%%%%%%%%%%%%%%%%%%%%%
%%%%%%%%%%%%%%%%%%%%%%%%%%%%%%%%%%%%%%%%%%%%%%%%%%%%%%%%%%%%%%%%%%%%%%%%%%%%%%%%%%%%%%%%%%%%%%%%%%%%
\section{Definição dos dispositivos do sistema}

No contexto deste trabalho, um dispositivo é a entidade lógica principal que abstrai um elemento da
aplicação. Pode representar uma entidade física, como um sensor ou atuador, ou lógica, como uma
máquina, composta por diversas entidades de hardware mapeadas como entidades lógicas. Cada um dos
dispositivo hospeda serviços, representando tarefas ou funcionalidades específicas possíveis de
serem executadas. Tanto os dispositivos, bem como os serviços por eles hospedados, podem ser
descobertos e identificados na rede. Os dispositivos podem descobrir outros dispositivos e utilizar
os serviços do segundo, a fim de criar um serviço composto mais complexo para execução de
determinada tarefa. A \cref{fig:device-services-overview} ilustra a topologia utilizada neste
trabalho, onde os clientes utilizam os serviços hospedados pelos dispositivos. Um serviço também
pode ser considerado um cliente caso utilize de uma funcionalidade remota para prover a sua
funcionalidade ou tarefa. Além disso, da mesma forma, um dispositivo também pode ser considerado um
cliente, o que flexibiliza a integração entre os componentes do sistema.

\includefigure
    {images/device-services-overview}
    {Visão geral dos clientes, dispositivos e serviços hospedados.}
    {fig:device-services-overview}

O modelo de dispositivos empregados neste estudo é apresentado na figura\todo{Incluir figura}. Nela,
o dispositivo físico é abstraído por um dispositivo lógico, identificado por\textit{XX*}. O
dispositivo lógico inclui alguns serviços padrão, referenciando funcionalidades que podem ser
encontradas em todos os dispositivos presentes na arquitetura proposta. Dentre os serviços padrão,
tem-se o serviço de implantação de novos serviços. Este serviço permite que outros serviços sejam
adicionados ao dispositivo físico. Além disso, é possível a implantação de novos dispositivos
lógicos, a fim de abstrair componentes da aplicação, juntamente com serviços do
usuário\todo{Melhorar o texto com base na imagem que será incluída.}.


%%%%%%%%%%%%%%%%%%%%%%%%%%%%%%%%%%%%%%%%%%%%%%%%%%%%%%%%%%%%%%%%%%%%%%%%%%%%%%%%%%%%%%%%%%%%%%%%%%%%
\subsection{Serviços genéricos}

\todo[inline]{Falar sobre os serviços built-in.}

%%%%%%%%%%%%%%%%%%%%%%%%%%%%%%%%%%%%%%%%%%%%%%%%%%%%%%%%%%%%%%%%%%%%%%%%%%%%%%%%%%%%%%%%%%%%%%%%%%%%
\subsection{Serviços implantados}


%%%%%%%%%%%%%%%%%%%%%%%%%%%%%%%%%%%%%%%%%%%%%%%%%%%%%%%%%%%%%%%%%%%%%%%%%%%%%%%%%%%%%%%%%%%%%%%%%%%%
%%%%%%%%%%%%%%%%%%%%%%%%%%%%%%%%%%%%%%%%%%%%%%%%%%%%%%%%%%%%%%%%%%%%%%%%%%%%%%%%%%%%%%%%%%%%%%%%%%%%
\section{Estudo de caso}

O objeto de estudo de caso deste trabalho é uma bacada de testes


%%%%%%%%%%%%%%%%%%%%%%%%%%%%%%%%%%%%%%%%%%%%%%%%%%%%%%%%%%%%%%%%%%%%%%%%%%%%%%%%%%%%%%%%%%%%%%%%%%%%
\subsection{Componentes do estudo de caso}


%%%%%%%%%%%%%%%%%%%%%%%%%%%%%%%%%%%%%%%%%%%%%%%%%%%%%%%%%%%%%%%%%%%%%%%%%%%%%%%%%%%%%%%%%%%%%%%%%%%%
\subsection{IMS}


%%%%%%%%%%%%%%%%%%%%%%%%%%%%%%%%%%%%%%%%%%%%%%%%%%%%%%%%%%%%%%%%%%%%%%%%%%%%%%%%%%%%%%%%%%%%%%%%%%%%
\subsection{Base de dados}


%%%%%%%%%%%%%%%%%%%%%%%%%%%%%%%%%%%%%%%%%%%%%%%%%%%%%%%%%%%%%%%%%%%%%%%%%%%%%%%%%%%%%%%%%%%%%%%%%%%%
\subsection{Válvulas}

\todo[inline]{Dispositivos devem ser informados sobre a configuração da base de dados e das
ferramentas que serão utilizadas para análise dos dados.}



%%%%%%%%%%%%%%%%%%%%%%%%%%%%%%%%%%%%%%%%%%%%%%%%%%%%%%%%%%%%%%%%%%%%%%%%%%%%%%%%%%%%%%%%%%%%%%%%%%%%
\subsection{Ferramentas de análise de dados}

\subsubsection{Módulo de manipulação dos dados}

\subsubsection{Módulo de avaliação da saúde do sistema}

\subsubsection{Fusão de sensores}


%%%%%%%%%%%%%%%%%%%%%%%%%%%%%%%%%%%%%%%%%%%%%%%%%%%%%%%%%%%%%%%%%%%%%%%%%%%%%%%%%%%%%%%%%%%%%%%%%%%%
\subsection{Serviços de clientes (?)}


%%%%%%%%%%%%%%%%%%%%%%%%%%%%%%%%%%%%%%%%%%%%%%%%%%%%%%%%%%%%%%%%%%%%%%%%%%%%%%%%%%%%%%%%%%%%%%%%%%%%
\subsection{Interface de acesso aos algoritmos de manutenção inteligente}

As ferramentas de análise dos sinais são distribuídas através do pacote Watchdog~Agent.

\iffalse
Funcionalidade para carregar módulos de diagnóstico nos dispositivos. As ferramentas do Watchdog
poderiam ser estendidas e embarcadas em um arquivo JAR. As ferramentas poderiam ser enviadas para os
dispositivos através de serviços e utilizadas localmente, sem necessidade de acesso remoto ao
Watchdog Agent.
\fi


%%%%%%%%%%%%%%%%%%%%%%%%%%%%%%%%%%%%%%%%%%%%%%%%%%%%%%%%%%%%%%%%%%%%%%%%%%%%%%%%%%%%%%%%%%%%%%%%%%%%
\subsection{Estruturação da base de dados}




%%%%%%%%%%%%%%%%%%%%%%%%%%%%%%%%%%%%%%%%%%%%%%%%%%%%%%%%%%%%%%%%%%%%%%%%%%%%%%%%%%%%%%%%%%%%%%%%%%%%
%%%%%%%%%%%%%%%%%%%%%%%%%%%%%%%%%%%%%%%%%%%%%%%%%%%%%%%%%%%%%%%%%%%%%%%%%%%%%%%%%%%%%%%%%%%%%%%%%%%%
\subsection{Armazenamento das informações}

\todo[inline]{Armazenamento das informações dos dispositivos e análises de degradação.}


%%%%%%%%%%%%%%%%%%%%%%%%%%%%%%%%%%%%%%%%%%%%%%%%%%%%%%%%%%%%%%%%%%%%%%%%%%%%%%%%%%%%%%%%%%%%%%%%%%%%
\subsection{Resultados esperados}


\iffalse
Dados de treinamento e teste.
Definição de alarmes para alteração do valor de confiança.
\fi
