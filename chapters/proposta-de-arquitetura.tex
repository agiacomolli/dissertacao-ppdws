%\chapter{Arquitetura proposta para um sistema de manutenção inteligente}
\chapter{Arquitetura proposta}

\todo[inline]{Introdução do capítulo. Revisar o capítulo para se adequar às novas alterações.}


%%
\section{Arquitetura orientada a serviços}

A arquitetura proposta neste trabalho tem por objetivo a integração de sistemas de manutenção
inteligente, dos diversos equipamentos que os utilizam além de outras entidades que auxiliam no
funcionamento do sistema. Todos os elementos que compõem a arquitetura são abstraídos como serviços,
o que facilita a integração, inserção e remoção de novas entidades no sistema. Para tanto, a
especificação da arquitetura é definida utilizando os padrões \gls{SOA}. Uma visão geral das
entidades que fazem parte da arquitetura proposta é apresentada na
\cref{fig:soa-proposed-architecture}\todo{Alterar figura}.

\includefigure
    {images/soa-proposed-architecture}
    {Arquitetura orientada a serviços proposta para integração de um sistema de manutenção
        inteligente.}
    {fig:soa-proposed-architecture}

Como observado na \cref{fig:soa-proposed-architecture}, a arquitetura proposta é composta de
diversas entidades. Cada entidade tem uma função específica no sistema e o acesso às suas
funcionalidades é realizado através de serviços. A seguir, para melhor entendimento, todos os
elementos serão detalhados.


%%%
\subsection{Serviço}

No contexto da arquitetura proposta neste trabalho, um serviço é um componente de software que
encapsula uma funcionalidade acessível através dos padrões definidos pela tecnologia \gls{SOA}.
Todos os componentes da arquitetura são definidos como serviços, possibilitando que a interação
entre eles seja feita de forma transparente. Como parte do padrão \gls{SOA}, serviços podem ser
descobertos e utilizados por clientes que desejam executar uma determinada tarefa. Também é possível
a definição de serviços mais especializados com base em outros serviços, praticando a técnica da
composição de serviços. Estas características se tornam inerentes à proposta, devido a utilização do
padrão \gls{SOA}.


%%%
\subsection{Dispositivo}

Um dispositivo é um componente de software utilizado para encapsular um elemento da aplicação.
Normalmente é denominado dispositivo lógico, por ser executado em um dispositivo físico, que
disponibiliza serviços para acesso à funcionalidades previamente definidas. As funcionalidades podem
ser relativas ao dispositivo físico em que o componente está executando ou outras que auxiliam em
alguma tarefa específica não relacionada diretamente com o hardware hospedeiro.

Do ponto de vista da aplicação, os dispositivos hospedam serviços. Dentre os serviços hospedados,
alguns estão presentes em todos os dispositivos da arquitetura. Dessa forma, uma interface mínima
para troca de informações entre as entidades do sistema é definida, facilitando a entrada de novos
dispositivos\todo{Explicar melhor os serviços da interface compartilhada}.


%%%
\subsection{Aplicação orientada a serviços}

A aplicação orientada a serviços nada mais é do que o resultado da utilização dos diversos serviços
propostos na arquitetura. Através da composição ou utilização direta dos serviços de uma forma
coordenada, a aplicação é construída. De um modo minimalista, a aplicação pode ser vista como a
interação progressiva de funcionalidades simples provenientes de serviços básicos, resultando em um
componente complexo de software. Visto que o produto final da composição dos elementos do sistema
não é definido \textit{a priori}, a aplicação final pode resultar em qualquer construção ou
interação entre os componentes.

%%%
\iffalse
\subsection{Implantador de serviços}

Como forma de implantar novos serviços nos dispositivos físicos encontrados na rede, é necessário um
software dedicado para esta tarefa. O implantador disponibiliza formas de alterar os dispositivos na
rede, seja enviando novos serviços para serem executados ou modificando as informações do
dispositivo. Como cada dispositivo do sistema possui alguns serviços padrão, entre eles o que
permite o recebimento de novos serviços para serem executados no dispositivo, a integração das
funcionalidades do implantador pode ser feita de forma facilitada.
\fi



%%%
\subsection{Gerenciador de dispositivos}

O Gerenciador de Dispositivos \todo{Enviar novos comportamentos para um grupo de dispositivos.} é um
componente de software utilizado para obter a lista dos dispositivos encontrados na aplicação
orientada a serviços. A entidade também possibilita configurar os dispositivos de forma individual
ou em grupo. Como os dispositivos são entidades que mapeiam funcionalidades dos dispositivos físicos
encontrados em um sistema de manutenção inteligente, o gerenciador pode obter a lista de sensores do
equipamento e configurar alguns parâmetros, como a taxa de atualização dos dados dos sensores.
Também é possível definir comportamentos adicionais para o caso do equipamento estar operando em
diferentes níveis de degradação. Os diferentes comportamentos podem ser utilizados dependendo dos
resultados obtidos com a análise dos dados e da saúde do equipamento.

A topologia dos componentes do sistema também pode ser obtida com esta entidade. Em um sistema
\gls{SOA} todos os serviços estão localizados no mesmo nível hierárquico. Essa característica pode
ser considerada positiva do ponto de vista de integração e reuso de serviços. Em contrapartida, ao
definir níveis hierárquicos entre componentes de um mesmo dispositivo físico, a complexidade para o
estabelecimento de uma hierarquia lógica para representação das diferentes partes desse dispositivo
aumenta. Contudo, ao aplicar um identificador único para cada dispositivo do sistema, torna-se
possível definir e estabelecer uma relação hierárquica. O identificador pode ser uma \gls{URI}, por
exemplo, onde cada parte do endereço é responsável pela definição de um dos níveis hierárquicos.

De certa forma, o Gerenciador de Dispositivos também pode ser utilizado na resolução de problemas
encontrados durante a execução dos componentes do sistema. A entidade se enquadra na categoria de
\gls{IHM}, possibilitando que a rede seja mapeada em busca de dispositivos ou serviços de forma
interativa. Além disso, é possível obter o estado dos dispositivo, visualizar dados ou testar os
serviços encontrados.


%%%
\subsection{Gerenciador de análises}

O Gerenciador de Análises é utilizado para definir o plano que será aplicado a determinado
equipamento com vista a obter os níveis de degradação. Através desta entidade, pode-se selecionar e
definir a ordem de execução das ferramentas de análise de dados em um plano a ser executado em
intervalos de tempo definidos. Os planos podem ser gerenciados, possibilitando alteração e/ou
exclusão.

No plano também são definidos os comportamentos a serem adotados com base nos valores de degradação
obtidos. O gerenciador obtém os comportamentos previamente definidos para o equipamento a ser
analisado podendo mapeá-los para diferentes níveis de degradação. Dessa forma, o equipamento pode
ser adaptado a diferentes condições, evitando o aumento da degradação até que uma manutenção possa
ser realizada.

Esta entidade também coordena as análises realizadas em um grupo de equipamentos similares. O
processo de obtenção dos valores de confiança de um conjunto de equipamentos pode ser simplificado
pelo fato de que, se são similares, os níveis de degradação se aproximam. Dessa maneira, é possível
elaborar um plano de análise que seja aplicável a vários equipamentos.

\todo[inline]{Utilização da base de dados para armazenamento dos planos.}


%%%
\subsection{Ferramenta de análise}

\todo[inline]{Melhorar o título da seção.}

Para a análise dos dados dos dispositivos, a arquitetura conta com a Ferramenta de Análise. A
entidade executa os planos criados e agendados no Gerenciador de Análises. É no analisador que as
ferramentas de análise de dados estão contidas. Cada um dos algoritmos também pode ser acessado
independentemente via serviço.

Os dados utilizados em cada análise são obtidos da base de dados.

\iffalse
O gerenciador de análise de dados é utilizado para analisar os dados dos equipamentos sob inspeção
pelo \gls{IMS}. Nele estão contidos os algoritmos para análise, onde a interface de acesso é
oferecida através de serviços. Definida uma ferramenta para análise de determinado equipamento, o
gerenciador obtém os dados da base de dados e executa a ferramenta selecionada. A partir dos
resultados, novos dados são gerados, os quais são inseridos na base de dados do equipamento em
análise\todo{Precisa ser finalizada.}.
\fi


%%%
\subsection{Base de dados}

A base de dados é utilizada para armazenar os dados relativos às análises de degradação dos
equipamentos. Isso também inclui os dados de treinamentos de cada equipamento e as análises
intermediárias que são utilizadas para gerar os valores de confiança. Dessa forma, toda as
informações dos componentes do sistema que estão sob o monitoramento de degradação são armazenadas
na base de dados.

Por ser defina como um componente \gls{SOA}, o acesso aos dados é feito através de serviços. A
inclusão, deleção e modificação de dados são realizadas por serviços especializados. Um dispositivo
remoto pode acessar a base de dados a fim de obter o histórico dos últimos valores de confiança
calculados ou valores de testes intermediários para determinado equipamento, utilizando-os para um
novo tipo de análise.



%%%
\subsection{Repositório de serviços}

\todo[inline]{Falar sobre os comportamentos que podem ser adotados para cada dispositivo. Os
comportamentos podem ser armazenados como serviços. Também pode armazenar os dados de treinamentos.}

O repositório de serviços é uma entidade que armazena os dispositivos lógicos e seus serviços, os
quais podem ser obtidos dinamicamente e implantados em dispositivos físicos. Esta entidade pode
estar localizada na rede local ou remota, com acesso disponível para várias aplicações. Dessa forma,
uma nova aplicação, isolada de outra já existente, pode ser construída com o reúso de componentes
obtidos de um repositório de serviços compartilhado pelas duas. Esta abordagem também permite o
compartilhamento de uma base de dados de componentes lógicos entre aplicações.


%%
\section{Casos de uso da arquitetura proposta}

Os elementos que compõem a arquitetura proposta encaixam-se em alguns casos de uso no contexto de um
sistema de manutenção inteligente. Os casos de uso propostos a seguir exemplificam a interação do
operador do sistema, quando há a necessidade de busca ou configuração dos dispositivos, além da
ferramenta de análise de dados, que executa autonomamente sobre os dados obtidos durante o processo
de amostragem dos equipamentos.\todo{Introdução da seção com definição dos atores da aplicação.}


%%%
\subsection{Descoberta e configuração de dispositivos}

A descoberta dos dispositivos e serviços na rede pode ser feita a qualquer momento utilizando o
explorador de dispositivos. Cada um dos dispositivos encontrados é identificado e colocado em uma
lista de dispositivos ativos. Também são identificados os serviços hospedados em cada um dos
dispositivos. Dessa forma, é possível determinar quais dispositivos estão atualmente disponíveis na
rede e quais possuem serviços que poderão ser utilizados em determinada funcionalidade do sistema.

A geração da hierarquia de recursos encontrados na rede também é possível. Aplicando um
identificador único para cada recurso -- utilizando uma \gls{URI}, por exemplo --, a reconstrução
dos componentes do sistema pode ser obtida agrupando-os em classes ou categorias. O identificador é
utilizado para auxiliar a definir uma topologia para a interação entre os diferentes componentes do
sistema.

O diagrama \gls{UML} de casos de uso para descoberta e configuração de dispositivos é apresentado na
\cref{fig:uml-discovery-setup-devices}. O diagrama apresenta os casos de uso do explorador de
dispositivos e a sua interação com os dispositivos que fazem parte da arquitetura orientada a
serviços. Em vias gerais, o explorador será utilizado pelo operador do sistema para descobrir os
dispositivos da rede, obter o estado ou realizar modificações.

\includefigure
    {images/uml-discovery-setup-devices}
    {Diagrama de casos de uso para descoberta e configuração de dispositivos.}
    {fig:uml-discovery-setup-devices}

\todo[inline]{Novo diagrama com o envio de novos comportamentos para o dispositivo ou inclusão de
informações no mesmo.}


%%%
%\subsection{Acesso à base de dados}

\subsection{Gerenciamento de análises}


%%%
\subsection{Análise dos dados}

\includefigure
    {images/uml-analysis-create}
    {Diagrama de casos de uso para a criação e execução de análise de um equipamento pelo operador
        do sistema.}
    {fig:uml-analysis-create}



%%
\section{Definição dos dispositivos do sistema}

No contexto deste trabalho, um dispositivo é a entidade lógica principal que abstrai um elemento da
aplicação. Pode representar uma entidade física, como um sensor ou atuador, ou lógica, como uma
máquina, composta por diversas entidades de hardware mapeadas como entidades lógicas. Cada um dos
dispositivo hospeda serviços, representando tarefas ou funcionalidades específicas possíveis de
serem executadas. Tanto os dispositivos, bem como os serviços por eles hospedados, podem ser
descobertos e identificados na rede. Os dispositivos podem descobrir outros dispositivos e utilizar
os serviços do segundo, a fim de criar um serviço composto mais complexo para execução de
determinada tarefa. A \cref{fig:device-services-overview} ilustra a topologia utilizada neste
trabalho, onde os clientes utilizam os serviços hospedados pelos dispositivos. Um serviço também
pode ser considerado um cliente caso utilize de uma funcionalidade remota para prover a sua
funcionalidade ou tarefa. Além disso, da mesma forma, um dispositivo também pode ser considerado um
cliente, o que flexibiliza a integração entre os componentes do sistema.

\includefigure
    {images/device-services-overview}
    {Visão geral dos clientes, dispositivos e serviços hospedados.}
    {fig:device-services-overview}

O modelo de dispositivos empregados neste estudo é apresentado na figura\todo{Incluir figura}. Nela,
o dispositivo físico é abstraído por um dispositivo lógico, identificado por\textit{XX*}. O
dispositivo lógico inclui alguns serviços padrão, referenciando funcionalidades que podem ser
encontradas em todos os dispositivos presentes na arquitetura proposta. Dentre os serviços padrão,
tem-se o serviço de implantação de novos serviços. Este serviço permite que outros serviços sejam
adicionados ao dispositivo físico. Além disso, é possível a implantação de novos dispositivos
lógicos, a fim de abstrair componentes da aplicação, juntamente com serviços do
usuário\todo{Melhorar o texto com base na imagem que será incluída.}.


%%
\section{Estudo de caso}

O objeto de estudo de caso deste trabalho é uma bacada de testes


%%%
\subsection{Componentes do estudo de caso}


%%%
\subsection{IMS}


%%%
\subsection{Base de dados}


%%%
\subsection{Válvulas}

\todo[inline]{Dispositivos devem ser informados sobre a configuração da base de dados e das
ferramentas que serão utilizadas para análise dos dados.}



%%%
\subsection{Ferramentas de análise de dados}

\subsubsection{Módulo de manipulação dos dados}

\subsubsection{Módulo de avaliação da saúde do sistema}

\subsubsection{Fusão de sensores}


%%%
\subsection{Serviços de clientes (?)}


%%%
\subsection{Interface de acesso aos algoritmos de manutenção inteligente}

As ferramentas de análise dos sinais são distribuídas através do pacote Watchdog~Agent.

\iffalse
Funcionalidade para carregar módulos de diagnóstico nos dispositivos. As ferramentas do Watchdog
poderiam ser estendidas e embarcadas em um arquivo JAR. As ferramentas poderiam ser enviadas para os
dispositivos através de serviços e utilizadas localmente, sem necessidade de acesso remoto ao
Watchdog Agent.
\fi


%%%
\subsection{Estruturação da base de dados}


%%%
\subsection{Armazenamento das informações}

\todo[inline]{Armazenamento das informações dos dispositivos e análises de degradação.}


%%%
\subsection{Resultados esperados}


\iffalse
Dados de treinamento e teste.
Definição de alarmes para alteração do valor de confiança.
\fi
