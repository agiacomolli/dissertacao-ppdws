\chapter{Conclusão}
\label{cha:conclusao}

Este trabalho apresentou a proposta de uma arquitetura orientada a serviços para um sistema de
manutenção inteligente. Na dissertação, foi apresentado o projeto das diversas entidades que fazem
parte da arquitetura e propostos casos de uso onde o sistema se encaixa. Através dos casos de uso,
foi possível determinar as funcionalidades que cada entidade de software deveria apresentar. Por
fim, o sistema foi implementado e verificou-se que a solução proposta é viável, visto que integra um
ambiente que auxilia na determinação dos coeficientes de degradação de equipamentos de forma
facilitada.

Com as entidades propostas, mais especificamente o Gerenciador de Dispositivos, foi apresentado que
é possível gerenciar e configurar os dispositivos. É ofertado ao operador do sistema um software
dedicado para a configuração dos dispositivos remotamente. O gerenciador possibilita, entre outras
funcionalidades, a capacidade de obtenção de todos os dispositivos presentes na rede e a
apresentação de informações detalhadas sobre cada um deles ao usuário. Além disso, é possível a
obtenção da topologia encontrada na rede. O gerenciador organiza os dispositivos apresentando-os de
forma hierárquica, facilitando a verificação da estrutura dos equipamentos. Também no Gerenciador de
Dispositivos, é disponibilizada a funcionalidade de obtenção de relatórios de análise dos
equipamentos. Dessa forma, o operador pode verificar o histórico de degradação dos equipamentos e,
futuramente, utilizar os dados para tarefas de prognóstico.

Como forma de gerenciar as análises empregadas nos dispositivos, foi proposto o Gerenciador de
Análises. O gerenciador possibilita a criação de planos de análise para dispositivos individuais ou
em grupo. Esta entidade integra todo o processo de gerenciamento das análises a que os equipamentos
estão submetidos. É disponibilizado ao operador do sistema a definição das ferramentas que serão
utilizadas na análise dos dados, bem como a configuração de comportamentos em função do nível de
degradação obtido com a análise. Novamente, como no caso anterior, o Gerenciador de Análises é uma
ferramenta de configuração remota, permitindo, assim, o gerenciamento dos planos de análise de forma
facilitada.

As análises agendadas pelo operador no Gerenciador de Análises são executadas pela entidade
Analisador de Dispositivos. A arquitetura proposta dispões desta entidade dedicada para a execução
dos planos de análise dos equipamentos. O analisador verifica os dados dos dispositivos e executa as
análises com base no plano definido pelo operador do sistema. Como apresentado, o analisador utiliza
os algoritmos de análise presentes na ferramenta Watchdog Agent. Porém, levando em conta a forma com
o que software foi projetado, a extensão do módulo de análise para outras implementações comerciais
ou específicas é possível de forma facilitada.

Por fim, coma definição do estudo de caso e implementação dos dispositivos, foi possível testar a
interoperabilidade da solução proposta. O estudo de caso apresentado foi implementado na forma de
dispositivos \gls{DPWS}. A fim de facilitar os testes da arquitetura, os dados foram obtidos
anteriormente em situações distintas. Com isso, o dispositivo se comportou como um simulador,
gerando os dados dos sensores na mesma forma como foram obtidos.

Em relação aos experimentos propostos, verificou-se a vantagem no uso da arquitetura proposta em
relação a utilização dos métodos tradicionais para verificação da degradação em equipamentos
utilizando as técnicas de manutenção inteligente. Ficou claro que, nos métodos tradicionais,
utilizando somente o software Watchdog Agent, por exemplo, é necessário o emprego de um operador
qualificado para a configuração das ferramentas e obtenção dos resultados. Isso se deve ao fato de
que, para cada nova análise, o software precisa ser reconfigurado. Com as hipóteses levantadas, foi
possível concluir que a utilização de planos de análise, empregados nesta proposta, facilitam o
processo de análise dos dados, visto que, após a criação de um plano, o usuário não necessita
modificá-lo nem reconfigurar o software de análise. Também foi constatada a facilidade de obtenção
dos níveis de degradação. Ao passo que, nos métodos tradicionais o operador necessita coordenar a
análise, com a entidade proposta essas são feitas de maneira automática. O operador do sistema
necessita somente obter os relatórios com os valores calculados durante o período estipulado.

As vantagens na utilização da proposta são mais evidentes quando comparadas à configuração e análise
de múltiplos equipamentos. As dificuldades encontradas na utilização do software convencional para
obtenção dos níveis de degradação de um equipamento são expandidas quando se faz necessária a mesma
avaliação para um conjunto de dispositivos. Dessa forma, o operador necessita, a cada nova análise,
iniciar todo o processo de configuração das ferramentas e obtenção dos dados. Com a definição dos
planos de análise no Gerenciador de Análises, é possível a utilização de um mesmo plano para um
grupo de dispositivos. Os dispositivo que fazem parte do grupo de análise serão monitorados de forma
igual pelo Analisador de Dispositivos. Dessa forma, o analisador se encarrega de obter os níveis de
degradação de forma automática.

Visando a continuidade do desenvolvimento da proposta, pode-se citar alguns pontos verificados como
passíveis de melhoramento. Um deles diz respeito aos modos de aquisição de dados de treinamento. No
estado atual, o operador do sistema necessita enviar os dados de treinamento para os dispositivos de
forma manual. A cada novo treinamento que deseja-se incluir na base de dados, é necessária a
utilização das funcionalidades empregadas no Gerenciador de Dispositivos para envio dos dados.
Portanto, com o aumento no número de treinamentos, aumenta também a dificuldade de envio dos dados,
por representar tarefas repetitivas. Uma melhoria constatada está na utilização da entidade que
representa o dispositivo configurada em um modo específico de aquisição de dados de treinamento. Ao
contrário do que é utilizado na implementação atual, o dispositivo poderia ser configurado para
enviar os dados das aquisições diretamente para a base de dados como dados de treinamento.

O aumento no número de dispositivos também é considerado um problema na visão do Analisador de
Dispositivos. Se muitos dispositivos estão em processo de análise, o analisador manterá uma fila
demasiadamente grande. O processamento das análises enfileiradas pode representar atraso
considerável na obtenção dos valores de degradação. Considerando isso, a proposta de inclusão de
mais entidades para análise de dispositivos pode solucionar o problema. Uma possibilidade é a
comunicação ente os processos e verificação da disponibilidade. Dessa forma, evitando que a
sobrecarga dos planos de análise agendados recaiam sobre somente um ponto de processamento de dados.

Por último, como forma de melhorias, a reimplementação das ferramentas de análise se faz necessário.
Neste estudo, foram utilizadas as ferramentas que integram o software Watchdog Agent. Essas são
disponibilizadas em forma de arquivos de \textit{script} para execução no software Matlab. Pelo fato
de que arquivos deste tipo são interpretados, verificou-se perda considerável no desempenho em
função do aumento do lote de dados analisado. Dessa forma, propõe-se a reimplementação dos
algoritmos utilizando linguagem compiladas, como C ou C++, e, em pontos onde for possível,
processamento paralelo, como CUDA ou OpenCL~\cite{kirk2012programming}.
