\chapter{Implementação e resultados}

Este capítulo descreve a implementação e resultados obtidos com o sistema proposto. A implementação
faz referências diretas sobre o que foi apresentado no capítulo anterior, onde a proposta geral do
trabalho foi definida. A forma como as entidades que fazem parte da arquitetura orientada a serviços
foram implementadas é descrita e são apresentados resultados.

Alguns experimentos visam comprovar as vantagens na utilização da proposta definida neste estudo.
São analisadas questões como a facilidade no gerenciamento das análises e equipamentos, bem como a
escalabilidade da solução. Também são testadas as entidades implementadas, verificando a
interoperabilidade entre os diferentes softwares desenvolvidos.


%%
\section{Implementação das entidades da arquitetura proposta}

Para a implementação de todas as entidades de software utilizou-se a linguagem de programação
Java~\cite{java2013homepage}. Aplicativos construídos utilizando esta linguagem executam em uma
máquina virtual. Dessa forma, é possível a execução em diferentes sistemas operacionais sem a
necessidade de reescrita ou reimplementação de partes do software. A única premissa é que exista uma
versão compatível da máquina virtual para o sistema operacional alvo.

Neste trabalho, a especificação \gls{SOA} escolhida foi a \gls{DPWS}. Conforme visto na
\cref{sec:arquiteturas-orientadas-servicos}, como \gls{SOA} é um padrão que não define a forma de
implementação, é possível encontrar diversas. Dentre as várias opções, algumas são voltadas para o
cenário corporativo, para a integração entre grandes empresas, e que estão focadas na manipulação e
gerência de um grande volume de dados e operações. Outras são empregadas em cenários mais simples do
ponto de vista de número de tarefas e conjunto diferenciado de serviços. Dentre as várias
especificações, o \gls{DPWS} foi o mais indicado para este trabalho devido às funcionalidades
disponibilizadas, as quais cumprem com os requisitos estipulados no projeto da arquitetura.

A especificação \gls{DPWS} possui algumas diferentes implementações em meio à diferentes projetos.
Dentre os projetos, pode-se citar WS4D, SOA4D e .Net Microframework~\cite{marcelo2013analise}. Os
dois primeiros contemplam implementações em linguagens Java, C e C++, o que reduz a dependência por
um sistema operacional específico. A terceira faz parte de um conjunto de ferramentas desenvolvido
pela Microsoft para sistemas embarcados, possuindo implementações em C\#. Neste trabalho optou-se
pela utilização da implementação \gls{JMEDS}, que faz parte do projeto
WS4D~\cite{jmeds2013homepage}. Esta escolha partiu do fato de que o projeto em questão conta com
grande atividade e contribuidores e, como as entidades de software são implementadas em Java, a
integração entre a biblioteca se dá de forma facilitada.


%%%
\subsection{Implementação do Gerenciador de Dispositivos}

Como visto na \cref{sub:proposta-gerenciador-dispositivos}, o Gerenciador de Dispositivos se trata
de um componente de software responsável pelo gerenciamento dos dispositivos na arquitetura
proposta. De forma geral, o gerenciador é utilizado pelo operador do sistema para obter os
dispositivos presentes na rede e configurá-los. A \cref{fig:device-manager-main-screen} ilustra a
tela principal do Gerenciador de Dispositivos.

\includefiguretmp
  %{images/device-manager-main-screen}
  {Tela principal do Gerenciador de Dispositivos.}
  {fig:device-manager-main-screen}


%%%%
\subsubsection{Obtenção da lista de dispositivos e serviços}

Uma das funcionalidades do gerenciador é a obtenção da lista de dispositivos presentes na rede. O
conjunto de ferramentas \gls{JMEDS}~\cite{jmeds2013homepage} implementa a especificação
{WS-Discovery}, permitindo a descoberta de serviços através de um protocolo padronizado. Dessa forma,
toda a busca e troca de mensagens para a descoberta e identificação dos serviços é disponibilizada
de forma transparente para o desenvolvedor da aplicação.

%\todo[inline]{Falar sobre o processo de busca utilizando a biblioteca JMEDS.}

Com a lista de dispositivos, é possível saber o estado de cada um. Ainda conforme a
\cref{fig:device-manager-main-screen}, é possível observar que o gerenciador apresenta uma área onde
são exibidas as informações pertinentes do dispositivo, como fabricante, nome e número do modelo,
número serial uma descrição sobre o dispositivo e o seu \gls{URI}. Estas informações são
apresentadas quando o operador do sistema seleciona um dos elementos da lista de dispositivos
encontrados, podendo ser um dispositivo ou subdispositivo, e referem-se à \cref{tab:device-metadata}.

A lista de serviços de cada dispositivo é obtida pelo usuário através do botão localizado na área de
informações do dispositivo. Ao ser pressionado, um novo diálogo é aberto, listando cada um dos
serviços. Em complementação às informações do dispositivo, apresentadas na tela principal do
gerenciador, o novo diálogo exibe informações mais detalhadas, como uma \gls{URL} para o número do
modelo e outra para o endereço eletrônico do fabricante. A fim de atualizar estas informações, o
operador do sistema poderá editá-las utilizando outro botão. Ao ser pressionado, um novo diálogo é
aberto, possibilitando a inserção, edição ou atualização de todas as informações do dispositivo
selecionado. A \cref{fig:device-manager-services-dialog} ilustra a janela de diálogo contendo as
informações do dispositivo, a lista de serviços e os botões mencionados anteriormente.

\includefiguretmp
  %{images/device-manager-services-dialog}
  {Diálogo com informações detalhadas sobre o dispositivo e lista de serviços disponíveis.}
  {fig:device-manager-services-dialog}

Dentre os serviços listados no diálogo de informações do dispositivo, estão aqueles comuns a todos
os dispositivos da arquitetura proposta juntamente com os específicos. Os serviços específicos são
aqueles que diferem entre os dispositivos, como, por exemplo, o serviço para obtenção de dados de um
determinado sensor. É possível listar somente os serviços específicos do dispositivo através de uma
caixa de seleção. Dessa forma, lista presente no diálogo da
\cref{fig:device-manager-services-dialog} exibirá somente os serviços não padronizados.

Informações detalhadas sobre cada um dos serviços também podem ser obtidas através do Gerenciador de
Dispositivos. No diálogo de informações do dispositivo, após a seleção de determinado serviço, é
habilitado um botão que possibilita obter mais informações sobre o componente. O diálogo lista as
operações do serviço selecionado. Também lista as informações de cada uma das operações, bastando o
operador do sistema selecioná-las na lista. A \cref{fig:device-manager-service-dialog} apresenta o
diálogo com informações adicionais sobre determinado serviço selecionado pelo operador do sistema.

\includefiguretmp
  %{images/device-manager-service-dialog}
  {Diálogo com informações adicionais sobre o serviço selecionado.}
  {fig:device-manager-service-dialog}


%%%%
\subsubsection{Topologia da rede e configuração da base de dados}

A topologia da rede é obtida pelo Gerenciador de Dispositivos no momento em que o operador do
sistema realiza uma nova busca de dispositivos. Como anteriormente apresentado na
\cref{fig:device-manager-main-screen}, a tela principal do gerenciador lista todos os dispositivos
encontrados, bem como os subdispositivos, localizados em níveis hierárquicos inferiores. Isso é
possível pelo fato de que cada dispositivo da arquitetura proposta deve apresentar um identificador
único. O identificador, no caso implementado, uma \gls{URL}, é utilizada para construir a hierarquia
dos dispositivos da rede.

Na \gls{URL} de identificação de cada dispositivo, estão codificadas informações para que o
gerenciador possa criar a topologia da rede. Como apresentado na
\cref{sub:proposta-descoberta-configuracao-dispositivos}, os dispositivos são codificados seguindo
um padrão de nomes, os quais são separados pelo caractere "/". A convenção utilizada facilita a
definição de dispositivos que fazem parte de dispositivos com nível hierárquico maior. Também é
possível a definição de um nome para a representação de um grupo de dispositivos ou área onde os
dispositivos estão localizados.

A configuração da base de dados é fundamental para o correto funcionamento do sistema proposto. A
base pode ser configurada através da tela principal do Gerenciador de Dispositivos, sendo que, após
este procedimento, todos os dispositivos serão informados de como o acesso a base deve ser
realizado. O gerenciador também a utiliza para manter uma referência dos dispositivos encontrados na
rede. Caso um dos dispositivos encontrados não apresentar registro na base de dados, o gerenciador
deve criá-lo. Além do dispositivo, também são criados os registros de subdispositivos ou sensores.


%%%%
\subsubsection{Configuração dos dispositivos}

A configuração dos dispositivos é realizada pelo operador do sistema também através do Gerenciador
de Dispositivos. A configuração inclui o mapeamento dos sensores, para armazenar os dados de
interesse, a definição de comportamentos, que influenciam no modo de operação do equipamento, e o
envio de dados de treinamento. Na \cref{fig:device-config-dialog} é apresentado o diálogo de
configuração de um dispositivo. O diálogo é acessível através da tela principal do Gerenciador de
Dispositivos. Para ativá-lo, é necessário que o operador do sistema selecione um dos dispositivos da
lista antes do processo de configuração.

\includefiguretmp
  %{images/device-config-dialog}
  {Diálogo de configuração do dispositivo.}
  {fig:device-config-dialog}

No diálogo de configuração, além de outras informações pertinentes ao dispositivo selecionado, estão
a lista de sensores que o equipamento possui. As informações de um determinado sensor são exibidas
no momento em que o usuário o seleciona na lista de sensores. Dentre as informações, estão o
fabricante, modelo e versão do sensor, além de uma descrição e a unidade de medida utilizada na
aquisição dos dados. A interface também disponibiliza algumas configurações para o sensor
selecionado, como a taxa de amostragem e o número de dados que necessitam ser amostrados para envio
à base de dados. Estas configurações estão disponíveis em um outro diálogo, acessível através do
botão de configuração dos sensores.

A configuração dos comportamentos do equipamento, conforme descrito na
\cref{sub:proposta-gerenciador-dispositivos}, também é feita no Gerenciador de Dispositivos. Os
diferentes comportamentos definidos para um equipamento possibilitam que ele opere de diferentes
maneiras dependendo do nível de degradação. Os comportamentos configurados estão acessíveis através
de uma lista no diálogo de configuração do dispositivo (\cref{fig:device-config-dialog}). Ao
selecionar um dos elementos da lista, as informações adicionais são preenchidas. O diálogo permite
ainda o gerenciamento dos comportamentos, como o envio ou exclusão de um elemento existente. O envio
é feito através de um botão, que, quando acionado, abre um novo diálogo para a criação de um
comportamento. O operador deverá definir um nome para o novo comportamento e anexar um arquivo
contendo o código executável no formato \gls{JAR}. Dessa forma, o comportamento é enviado para o
dispositivo e poderá ser executado em alguma situação, dependendo do tipo de degradação observado.

Os dados de treinamento são enviados pelo mesmo dialogo de configuração. O diálogo apresenta um
sumário com informações pertinentes sobre o número de dados de treinamento disponíveis para o
dispositivo e possibilita que o usuário envie outros novos. Ao pressionar o botão de envio, um novo
diálogo é aberto, onde o operador poderá escolher para qual sensor os dados serão mapeados e qual o
tipo dos dados, conforme apresentado na \cref{fig:device-config-training-dialog}. O tipo de dados
pode ser normal ou falha, os quais representam, respectivamente, uma situação em que o equipamento
está se comportando conforme o esperado e quando existe algum comportamento indesejado. Também é
necessária a inclusão de informação sobre a origem dos dados, no campo denominado de descrição. O
diálogo aceita arquivos de dados no formato \gls{CSV}.

\includefiguretmp
  %{images/device-config-training-dialog}
  {Diálogo de envio de dados de treinamento.}
  {fig:device-config-training-dialog}


%%%%
\subsubsection{Definindo grupos de dispositivos}

A configuração ou envio de dados para um conjunto de dispositivos de mesma natureza é possível
através da criação de grupos de dispositivos. Dessa forma, dispositivos podem vir a ser agrupados
tendo em comum o modelo ou fabricante, por exemplo. A configuração dos grupos é acessível através de
um botão presente na tela principal do Gerenciador de Dispositivos, ilustrado na
\cref{fig:device-manager-main-screen}. Ao ser pressionado, um novo diálogo é aberto. Este diálogo é
apresentado na \cref{fig:device-config-groups-dialog}. Nele estão presentes a lista de grupos
criados pelo operador do sistema e as possibilidades de gerenciamento (criar, editar e/ou excluir).
Ao selecionar um grupo existente, na lista à esquerda da figura em questão, os dispositivos que o
compõe são apresentados. Pode-se adicionar ou remover dispositivos através da funcionalidade de
edição de grupos.

\includefiguretmp
  %{images/device-config-groups-dialog}
  {Diálogo de definição e configuração de grupos de dispositivos.}
  {fig:device-config-groups-dialog}

Um grupo de dispositivos é configurado através de um novo diálogo, disponível pelo pressionamento do
botão de configuração de dispositivos, o qual é apresentado na
\cref{fig:device-config-group-dialog}. O diálogo apresenta a lista de dispositivos e seus sensores
em árvore, possibilitando que o operador visualize estas informações de forma facilitada. Também é
possível a definição de novos comportamentos para o grupo de dispositivos e o envio de dados de
treinamento. O envio de novos comportamentos utiliza o mesmo mecanismo apresentado anteriormente,
quando abordado do diálogo de configuração de um dispositivo, exceto que, no caso de grupos, um
mesmo comportamento é enviado para um conjunto de dispositivos.

\includefiguretmp
  %{images/device-config-group-dialog}
  {Diálogo de configuração de um grupo de dispositivos.}
  {fig:device-config-group-dialog}

O envio de dados de treinamento também segue o mesmo padrão apresentado anteriormente na descrição
da configuração de somente um dispositivo. O caso particular se encontra no que diz respeito a
escolha do sensor que usufruirá dos novos dados. O Gerenciador de Dispositivos agrupa
automaticamente os sensores de mesmo nome, dessa forma, o envio de dados de treinamento é feito com
base no nome do sensor. Ao acessar o diálogo de envio de dados de comportamento, os sensores de
todos os dispositivos do grupo estarão disponíveis. Será ignorado o sensor escolhido para envio dos
dados que não estiver presente em nenhum dos dispositivos.


%%%%
\subsubsection{Obtenção de relatórios de saúde}

Os relatórios de saúde podem ser obtidos pelo operador do sistema através do diálogo apresentado na
\cref{fig:device-health-report-dialog}, acessível na tela inicial do Gerenciador de Dispositivos. O
diálogo permite definir o intervalo para geração do relatório, o qual contém os valores de confiança
obtidos para cada lote de dados. Os dados são obtidos diretamente da base de dados e o relatório é
fornecido ao operador do sistema em arquivo texto no formato \gls{CSV}.

\includefiguretmp
  %{images/device-health-report-dialog}
  {Diálogo para obtenção de relatórios de saúde.}
  {fig:device-health-report-dialog}


%%%
\subsection{Implementação do Gerenciador de Análises}

Como apresentado na \cref{sub:proposta-gerenciador-analises}, o Gerenciador de Análises é um dos
componentes de software presentes na arquitetura proposta que auxilia no processo de criação de
planos de análise para os diversos dispositivos presentes no sistema. Através dele, o operador do
sistema pode gerenciar o processo de análise de um equipamento, definindo as ferramentas utilizadas
para manipulação dos dados, bem como a origem dos dados utilizados. Além disso, é possível que o
equipamento assuma diferentes comportamentos baseado no resultado das análises realizadas. A tela
principal do Gerenciador de Análises é apresentada na \cref{fig:analysis-manager-main-screen}.

\includefiguretmp
  %{images/analysis-manager-main-screen}
  {Tela principal do Gerenciador de Análises.}
  {fig:analysis-manager-main-screen}

As análises cadastradas anteriormente são exibidas em uma lista na tela principal do Gerenciador de
Análises. Quando da seleção de um dos elementos da lista, informações adicionais, como nome,
descrição e número de dispositivos que utilizam a mesma análise, são fornecidas. Também são
fornecidas informações sobre a periodicidade de execução da análise, a data da última execução, a
data agendada para a próxima execução e se a análise está ativa ou não. As análises e respectivas
informações são obtidas diretamente da base de dados.


%%%%
\subsubsection{Definição de um plano de análise}

O operador do sistema poderá definir novas análises ou gerenciar as existentes, conforme
\cref{fig:analysis-manager-main-screen}. Na definição de uma nova análise, o diálogo, ilustrado na
\cref{fig:analysis-create-dialog}, é aberto. Através do diálogo, o operador deve definir um nome e
descrição para o plano de análise, bem como quais dispositivos e sensores farão parte dela. Também
deve definir as ferramentas computacionais utilizadas e, se pertinente, configurar quais
comportamentos os dispositivos devem assumir frente a diferentes situações de degradação. A
periodicidade da execução da análise também pode ser configurada.

\includefiguretmp
  %{images/analysis-create-dialog}
  {Diálogo de criação de uma nova análise.}
  {fig:analysis-create-dialog}

A configuração de dispositivos e sensores é realizada em um novo diálogo. O diálogo permite ao
operador configurar quais dispositivos serão analisados. Conjuntamente, ao selecionar os
dispositivos, também é possível definir quais sensores serão utilizados na análise de degradação. O
princípio de escolha dos dispositivos é herdado do Gerenciador de Dispositivos, exceto por não ser
possível a criação de novos grupos de dispositivos através do Gerenciador de Análises. O operador
pode optar pela escolha de somente um dispositivo, bem como de um grupo de dispositivos. No entanto,
o grupo de dispositivos deve ter sido criado previamente através do Gerenciador de Dispositivos.

As ferramentas de análise dizem respeito aos algoritmos utilizados para analisar os dados obtidos. O
processo de escolha é apresentado no diálogo da \cref{fig:analysis-tools-dialog}. Dois conjuntos de
ferramentas estão disponíveis na implementação do Gerenciador de Análises: processamento de sinais e
extração de características e avaliação do desempenho do sistema. Transformada rápida de Fourier,
análise tempo-frequência e energias da transformada wavelet packet fazem parte do primeiro conjunto.
Por conseguinte, regressão logística e reconhecimento estatístico de padrões fazem parte do segundo
conjunto de ferramentas.

Se implementado, as ferramentas permitem a configuração dos algoritmos. Após a seleção de uma delas,
o operador poderá optar por configurá-la. Para tanto, estão disponíveis botões na interface com o
usuário. Ao clicá-los, novos diálogos são abertos com as configurações específicas da ferramenta
selecionada.

\includefiguretmp
  %{images/analysis-tools-dialog}
  {Diálogo para escolha das ferramentas utilizadas na análise.}
  {fig:analysis-tools-dialog}

O operador poderá também definir os comportamentos utilizados como resposta aos níveis de degradação
obtidos. O diálogo da \cref{fig:analysis-behavior-dialog} ilustra alguns comportamentos e as
informações pertinentes àquele que foi selecionado. No procedimento para seleção dos comportamentos
utilizados pelo dispositivo, o operador deverá indicar, através das caixas de seleção, quais
comportamentos quer utilizar e definir as faixas de variação do valor de confiança onde o algoritmo
estará ativo. Como o valor de confiança pode assumir valores no intervalo ${[0; 1]}$, são
disponibilizados dois campos para a seleção do valor mínimo e máximo empregado na definição do
comportamento, os quais podem assumir valores dentro desse mesmo intervalo.

\includefiguretmp
  %{images/analysis-behavior-dialog}
  {Diálogo para a configuração dos comportamentos dos dispositivos.}
  {fig:analysis-behavior-dialog}

A configuração da periodicidade em que a tarefa será executada também pode ser acessada pelo diálogo
da \cref{fig:analysis-create-dialog}. Um novo diálogo é aberto, onde o operador poderá definir as
propriedades de repetição da análise. É possível definir um intervalo fechado, onde, após um
determinado período, as análises são interrompidas, ou a execução permanente, onde as análises
sempre serão executadas. A periodicidade da execução, ou seja, o intervalo entre uma execução e
outra, é definido pelo número de amostras coletadas dos sensores. Como apresentado no Gerenciador de
Dispositivos, o operador deve definir o tamanho de um lote de dados que será enviado à base de
dados. O dispositivo armazena estes dados e, quando o lote estiver completo, o envia a base de
dados. A base de dados é monitorada para a execução de uma análise sobre o novo lote de dados, a fim
de gerar o novo valor de confiança.


%%%
\subsection{Implementação do Analisador de Dispositivos}

O Analisador de Dispositivos é a entidade responsável por executar os planos de análise definidos
pelo operador do sistema no Gerenciador de Análises. Conforme definido na
\cref{sub:proposta-analisador-dispositivos}, o Analisador de Dispositivos concentra as ferramentas
necessárias para análise dos dados e obtenção dos níveis de degradação dos equipamentos. A tela
inicial do analisador é apresentada na \cref{fig:analyzer-main-screen}.

\includefiguretmp
  %{images/analyzer-main-screen}
  {Tela principal do Analisador de Dispositivos.}
  {fig:analyzer-main-screen}

O analisador mantém uma fila das análises ativas agendadas pelo operador do sistema. O Analisador de
Dispositivos implementa uma tarefa periódica que verifica a base de dados em busca de novos lotes de
dados relativos às análises ativas. Quando novos lotes são encontrados, os respectivos planos de
análise são postos em uma fila e executados sequencialmente. Os resultados gerados são armazenados
na base de dados.

A tela principal do Analisador de Dispositivos (\cref{fig:analyzer-main-screen}) apresenta algumas
informações úteis para o operador do sistema. São listados os planos atualmente enfileirados e
estatísticas sobre as execuções, como o número de planos agendados e quantas análises estão
pendentes para execução. Informações adicionais sobre um determinado plano podem ser obtidas pelo
operador através da seleção do elemento na lista da tela principal.


%%%%
\subsubsection{Ferramentas para análise dos dados e interoperabilidade entre diferentes
    fornecedores}

As ferramentas de análise dos dados disponíveis no analisador são construídas em forma de interface
para os algoritmos presentes no software Watchdog Agent, versão 3.3\todo{referencia}. Os algoritmos
são distribuídos na forma de \textit{scripts} integrados em um software que executa em ambiente
MATLAB. A solução empregada no desenvolvimento do analisador foi a de isolar os \textit{scripts} que
implementam as ferramentas desejadas e executá-los individualmente através do novo software. Dessa
forma, não foi necessária a reescrita dos algoritmos utilizando a linguagem Java.

A execução dos \textit{scripts} MATLAB foi possível utilizando a biblioteca
\emph{matlabcontrol}~\cite{matlabcontrol2013homepage}, versão 4.1.0. A biblioteca possibilita uma
ponte para troca de dados entre um programa escrito em Java e o MATLAB. O mecanismo empregado na
solução encontrada é ilustrado na \cref{fig:analyzer-matlab-interface}. Após a coleta dos dados,
obtidos da base de dados, para a execução de uma nova análise, o Analisador de Dispositivos
seleciona a ferramenta que deve ser empregada, a qual foi definida no plano de análise. Como a
ferramenta é um \textit{script} MATLAB, o analisador carrega os dados e executa o algoritmo
utilizando a biblioteca
\emph{matlabcontrol}. Do mesmo modo, ao final da execução do algoritmo, o analisador utiliza
novamente a biblioteca para recuperar os resultados gerados. O processo é repetido se o plano
estipular que os resultados anteriores necessitam de uma nova análise. Tanto os resultados
intermediários como o resultado final são armazenados novamente na base de dados.

\includefiguretmp
  %{images/analyzer-matlab-interface}
  {Diagrama de interoperabilidade entre o Analisador de Dispositivos e os \textit{scripts} do
      Watchdog Agent.}
  {fig:analyzer-matlab-interface}

A forma com que o Analisador de Dispositivos foi construído permite a utilização de ferramentas de
análise de outros fornecedores. Como exemplo, existe uma versão do Watchdog Agent que executa em
conjunto com o software LabVIEW. Devido a uma camada de abstração empregada na separação da
definição das ferramentas e na sua implementação, é possível a troca ou mesmo a execução simultânea
de ferramentas disponibilizadas por diferentes fornecedores, desde que uma nova interface seja
provida. A \cref{fig:class-matlab-interface} ilustra o diagrama de classes utilizado na definição da
interface de abstração entre as ferramentas\todo{explicar figura}.

\includefiguretmp
  %{images/class-matlab-interface}
  {Diagrama de classes para abstração das ferramentas de análise.}
  {fig:class-matlab-interface}


%%%%
\subsubsection{Alteração do comportamento do dispositivo}

Se definido no plano de análise, o dispositivo pode alterar o modo de operação dependendo do nível
de degradação obtido com a análise. Como visto anteriormente, o operador do sistema pode definir
comportamentos para o dispositivo, a fim de alterar o modo de operação em diferentes situações de
degradação. Os comportamentos são definidos através do Gerenciador de Dispositivos e ficam
disponíveis no dispositivo configurado. Após a análise de um plano, o nível de degradação expresso
pelo valor de confiança do equipamento é obtido e comparado com as faixas definidas para cada
comportamento. Se o valor calculado estiver em alguma das faixas cobertas por um dos comportamentos,
o Analisador de Dispositivos automaticamente informa para o dispositivo qual comportamento deve
utilizar.


%%%
\subsection{Implementação da camada de acesso a base de dados e definição da estrutura interna}

\todo[inline]{Apresentar a estrutura básica da base de dados.}
\todo[inline]{Camada de acesso a base de dados baseada em serviços.}
\todo[inline]{Grupos de dispositivos na base de dados.}
\todo[inline]{Dispositivos e subdispositivos (campo parent).}

Como apresentado na \cref{sub:proposta-base-dados}, é necessária a implementação de uma entidade de
software para abstrair o acesso a base de dados através de serviços. O software implementa, para
cada componente que necessite de persistência de dados, serviços com operações pertinentes. Por
exemplo, a busca ou manipulação de dispositivos pode ser feita com o uso de serviços especializados
que manipulam a tabela onde estão armazenadas as informações destes elementos. No entanto, a fim de
evitar a redefinição de todas as operações para acesso às diferentes tabelas, o acesso também pode
ser feito pelos métodos tradicionais.

Neste trabalho foi utilizada uma base de dados SQLite~\cite{sqlite2013homepage}. A base de dados
SQLite é baseada em arquivos, não necessitando de servidor dedicado, tampouco de um cliente
especializado. O conteúdo da base de dados é armazenado em um único arquivo, facilitando a
manipulação e distribuição. A escolha por esta base partiu do fato da simplicidade na sua utilização
e manutenção, visto que implementa todas as funcionalidades necessárias ao projeto.

Na definição do esquema das tabelas da base de dados, as as informações sobre dispositivos e dados
para análise foram separados de modo a facilitar a utilização pelas entidades da arquitetura
proposta. O esquema da base se dados, com algumas das entidades principais, é apresentado na
\cref{fig:database-schema}.

\includefiguretmp
  %{images/database-schema}
  {Esquema simplificado utilizado na base de dados.}
  {fig:database-schema}


%%%
\subsection{Implementação dos dispositivos}

Os dispositivos implementados neste trabalho se referem a atuadores elétricos. Os atuadores,
operando em conjunto com válvulas, como comentado na \cref{sec:estudo-caso}, são do modelo CSR6,
fabricados pela empresa Coester. Os equipamentos foram instrumentados e feitas simulações de
degradação em diferentes situações. Os dados da simulação das condições foram coletados, sendo
possível utilizá-los para simular o comportamento do equipamento sem a necessidade de montagem de
novos experimentos físicos em bancada de testes. No caso do embarque físico de um dispositivo no
atuador, a técnica utilizada apresenta vantagem, pois o software do dispositivo necessita de
modificações somente em pontos específicos, como a forma de obtenção dos dados dos sensores.

Através dos dados dos experimentos armazenados, fez-se necessário o desenvolvimento de uma entidade
de software para simular o comportamento do conjunto atuador elétrico e válvula. A entidade é
encarregada de obter os dados prévios e gerá-los nas mesmas condições em que foram obtidos. O
software tem por objetivo simular um dispositivo \gls{SOA} embarcado no equipamento, mas com dados
reais, facilitando o teste e validação do sistema. A \cref{fig:device-main-screen} apresenta a tela
principal da entidade para simulação dos dispositivos. A tela ilustra algumas informações sobre o
dispositivo, como nome, modelo, fabricante e número serial. Estas informações dizem respeito às
descritas na \cref{tab:device-metadata} e podem ser configuradas pelo operador do sistema. Outras
informações gerais, como qual experimento está sendo simulado e as ferramentas utilizadas para a
avaliação dos dados, também são apresentadas.

\includefiguretmp
  %{images/device-main-screen}
  {Tela principal do software para simulação dos dispositivos.}
  {fig:device-main-screen}

Para cada uma das situações apresentadas na \cref{tab:device-metadata}, estão disponíveis \num{50}
curvas. Desse total, metade são referentes ao fechamento e a outra parte ao processo de abertura da
válvula. O dispositivo simula o processo de abertura da válvula indefinidamente, gerando os dados
coletados nos experimentos de forma sequencial. Ao término da geração do conjunto de dados, o
dispositivo gera os mesmos dados inciais continuamente.


%%%%
\subsubsection{Definição das situações de simulação}

As diferentes situações possíveis de simulação estão definidas como apresentado na
\cref{sub:estudo-caso-aquisicao-dados}. De acordo com a descrição do estudo de caso, são possíveis
seis experimentos, sendo um deles referente a operação em condições ditas normais, sem interferência
no funcionamento, e outras cinco onde o equipamento é submetido a alterações de comportamento. Ao
selecionar um dos experimentos, são exibidas informações adicionais, como nome, descrição e as
condições a que o equipamento foi submetido. Dentre as condições, estão o acionamento do freio e o
uso de engrenagens defeituosas.


%%%%
\subsubsection{Geração dos dados de simulação}

Em cada um dos casos, os dados dos sensores simulados são gerados de acordo com o que foi definido
no Gerenciador de Dispositivos. O operador pode definir o número de dados que serão armazenados
antes do envio para a base de dados configurada. Neste caso, evitando o acesso contínuo ao serviço
de envio de dados da base.

O lote de dados é definido pela quantidade de amostras. Após a aquisição do número de amostras
estipulado pelo operador do sistema, o software de simulação agrupa-os em um arquivo \gls{CSV}. Um
arquivo específico para cada sensor é criado. No arquivo, estão discriminados, em colunas, a data e
hora de aquisição de cada uma das amostras e o valor amostrado.


%%%%
\subsubsection{Obtenção do valor de confiança}

Como forma de verificação do funcionamento do equipamento simulado, o software disponibiliza
informações sobre os valores de confiança obtidos. É possível a visualização do valor de confiança
atual, bem como do valor máximo e mínimo obtidos durante o experimento. A data e hora na qual os
valores foram gerados também é informada.

O valor de confiança é obtido pela entidade diretamente da base de dados. Dessa forma, fica claro
que é possível o próprio equipamento tomar decisões a partir da condição calculada. Em contraste com
os comportamentos definidos pelo operador do sistema no Gerenciador de Dispositivos, o equipamento
pode apresentar algoritmos padronizados para assumir diferentes comportamentos em situações
específicas de degradação.


%%
\section{Resultados obtidos com a arquitetura proposta}

Os resultados obtidos com este trabalho são analisados de forma subjetiva e têm por base as questões
levantadas na \cref{sec:experimentos-definicao}. De uma forma geral são analisados quesitos como a
interoperabilidade das entidades propostas e a comparação do sistema implementado com métodos
tradicionais usualmente utilizados. A interoperabilidade entre as ferramentas é analisada, onde
testes são realizados para verificar o correto funcionamento das entidades projetadas. Por fim, a
proposta é comparada com os métodos tradicionais de análise de degradação, além da verificação das
dificuldades e vantagens encontradas na utilização do sistema para monitoramento de um conjunto de
equipamentos.


%%%
\subsection{Verificação da interoperabilidade entre as entidades}

Em um primeiro momento, todas as entidades foram executadas para verificar a interoperabilidade
entre elas conforme sugerido na \cref{sub:experimentos-interoperabilidade}. O Gerenciador de
Dispositivos foi iniciado e colocado em modo de monitoramento. Colocado em mondo de monitoração da
rede, o gerenciador verifica quando novos dispositivos são iniciados, possibilitando obter maiores
informações sobre cada um.

Após o Gerenciador de Dispositivos, um dispositivo foi iniciado. O dispositivo representa um atuador
elétrico. Verificou-se que é possível a alteração das informações do dispositivo, tanto pelo
gerenciador, como também pela interface gráfica da entidade de simulação. Este fator aponta o
sucesso encontrado na comunicação entre as entidades.

Com o dispositivo configurado e iniciado, criou-se um plano de análise utilizando o Gerenciador de
Análises. A análise foi criada obedecendo as condições apresentadas na
\cref{sub:estudo-caso-analise-degradacao}. Dessa forma, o módulo de processamento de sinais e
extração de características foi configurado para utilizar a transformada de wavelet packet, enquanto
que, no módulo de avaliação de desempenho do sistema, utilizou-se a ferramenta de regressão
logística.

Também foi iniciado o Analisador de Dispositivos. Após a sua inicialização, o plano criado
anteriormente é apresentado para execução na lista de planos ativos e agendados. O analisador
apresenta as informações detalhadas corretamente, o que novamente comprovou que as entidades se
comunicam de forma coerente. Por conseguinte, o analisador foi iniciado, entrando no modo de espera
pelos dados do dispositivo.

Com as ferramentas configuradas, foi iniciada a simulação do dispositivo. Utilizando a tela gráfica
criada para interface do simulador do atuador elétrico, testou-se as seis situação de teste,
conforme apontadas na \cref{tab:data-description}. Como os algoritmos para análise de degradação não
foram treinados, o dispositivo não atualizou as informações sobre o valor de confiança calculado em
nenhuma das situações testadas.

Para que o dispositivo possa ser analisado corretamente, os algoritmos necessitam passar por um
treinamento. Este fato é apontado na \cref{sub:estudo-caso-analise-degradacao}. Para o treinamento,
foram utilizados alguns dos dados obtidos com os testes nas situações de funcionamento normal e
falha. Os dados para treinamento correspondem ao movimento de abertura da válvula. Do total dos dois
conjuntos de curvas de abertura, representados por comportamentos normal e em falha, utilizou-se
\num{12} curvas em falha e \num{12} em funcionamento normal para treinar os algoritmos. Como falha,
foram utilizadas as curvas com o pior nível de degradação, obtidas na situação envolvendo uma
engrenagem quebrada (\cref{tab:data-description}). Dessa forma, \num{24} curvas foram utilizadas
para o treinamento dos algoritmos.

As curvas de treinamento foram anexadas ao dispositivo através da interface disponibilizada pelo
Gerenciador de Dispositivos. Conforme apresentado anteriormente na
\cref{fig:device-config-training-dialog}, os dados de treinamento podem ser enviados através de uma
interface de configuração, sendo possível selecionar para qual sensor os dados serão mapeados e qual
o tipo (normal ou falha). Dessa forma, os arquivos de cada uma das curva descritas anteriormente
foram enviados e configurados para o respectivo sensor. No experimento, como descito na
\cref{sub:estudo-caso-analise-degradacao}, utilizou-se o acelerômetro acoplado ao eixo do motor, o
qual registra as variações de vibração.


%%%
\subsection{Comparação da proposta com métodos tradicionais}

A proposta de comparação com os métodos tradicionais é apresentada e ilustrada na
\cref{sub:experimentos-metodos-tradicionais}. São levantadas hipóteses sobre a facilidade de
configuração da análise de dispositivos utilizando a arquitetura proposta em comparação com os
métodos usuais. Por métodos usuais, entende-se aqueles que se utilizam de um software específico, o
qual concentra as ferramentas de análise de dados e é manipulado por operador especializado. Um
exemplo deste tipo de software é o Watchdog Agent.

Como anteriormente apresentado nas seções que mostram a implementação do sistema, foram definidos
softwares especializados para as tarefas identificadas no processo de obtenção dos níveis de
degradação utilizando um sistema de manutenção inteligente. Cada um dos softwares é empregado um uma
tarefa distinta na aplicação. Por exemplo, o Analisador de Dispositivos é a entidade que concentra
as ferramentas para análise de dados e obtém os níveis de degradação do equipamento analisado. Uma
vez que o plano de análise foi criado, o analisador o executa automaticamente conforme planejado.
Neste ponto existe a vantagem, em relação ao método tradicional, de que não é mais necessária a
obtenção dos dados por parte do operador, tampouco a configuração das ferramentas de análise.

Quanto a aquisição dos dados, o sistema proposto necessita de implementações específicas que não
estão disponíveis em todos os equipamentos. No método tradicional, os dados são obtidos diretamente
por sensores posicionados estrategicamente pelo fabricante do equipamento ou através de
instrumentação posterior por equipe especializada. Com os sensores interligados em placas de
aquisição, os dados são amostrados diretamente, sem a necessidade de uma camada de abstração que, no
caso da arquitetura proposta, é representada por serviços. Do ponto de vista da aquisição dos dados,
os serviços \gls{DPWS} representam um grande consumo de recursos em um dispositivo. Porém, com a
evolução da tecnologia e diminuição dos custos na construção de dispositivos embarcados, o uso de
padrões \gls{SOA} mostra-se viável. Se os dispositivos utilizarem-se desta tecnologia para o
fornecimento dos dados, a integração com o sistema de análise será facilitada, o que representa
novamente ponto positivo na utilização da arquitetura proposta.


%%%
\subsection{Verificação da escalabilidade da proposta}

Este experimento se refere à \cref{sub:experimentos-escalabilidade} e presume que, nos métodos
tradicionais, o aumento de dispositivos em um processo de análise de dados fica prejudicado. Também
é dada como hipótese o fato da utilização de um mesmo processo de análise para diferentes
equipamentos. Utilizando-se os métodos convencionais, a configuração de diversos equipamento pode
representar uma dificuldade.

O Gerenciador de Análises permite a criação de planos que são aplicados para mais de um dispositivo.
De certo modo, a análise de mais de um dispositivo pelo método tradicional não representa grande
empecilho. O fator principal é a reconfiguração das ferramentas e injeção de dados diferentes de
forma manual a cada nova análise, o que demanda, principalmente, tempo. Mesmo com equipamentos de
mesma natureza, a cada nova análise o operador precisa definir manualmente o processo que será
executado. Na maioria das vezes, somente é possível a execução dos testes de forma \textit{offline}.
A vantagem na utilização da arquitetura proposta reside no fato do processo automatizado criado. Com
o Analisador de Dispositivos executando automaticamente os planos, fica evidente a facilidade da
análise dos dados. Em adendo, um mesmo plano pode ser empregado para um conjunto de dispositivos, o
que, de certo modo, soluciona o problema de reconfiguração das ferramentas e injeção de dados de
teste para cada nova análise.
