\chapter{Conceituação teórica}

\section{Sistemas de manutenção inteligente}

Manutenção, no âmbito geral, consiste em uma série de medidas de prevenção, correção e predição de
falhas~\cite{lee2006intelligent}. Durante o uso, equipamentos ou máquinas tendem a deteriorar e
alterar o seu padrão de funcionamento devido a diversos fatores, como, por exemplo, desgaste,
rachaduras, corrosão e sujeira. Nestas condições, a restauração do sistema é de suma importância,
visto que, com o passar do tempo, podem apresentar defeitos e levar à falhas e indisponibilidades.
De acordo com~\cite{marcal2005detectando}, manutenção pode ser definida como todas as atividades
técnicas e organizacionais que garantam a operação das máquinas e equipamentos dentro da
confiabilidade esperada.

Tradicionalmente, são encontrados na literatura três tipos de estratégias de manutenção: corretiva,
preventiva e preditiva~\cite{goncalves2011desenvolvimento}. A manutenção corretiva visa
reestabelecer os sistemas danificados; a preventiva tem por objetivo manter os sistemas em
funcionamento, realizando pequenas correções; e a manutenção preditiva tem por base o monitoramento
do estado do sistema, detectando falhas insipientes e fornecendo subsídios para o planejamento de
ações de prevenção ou correção. Em termos gerais, a manutenção corretiva é aplicada somente quando
há falha e o sistema necessita de reposição de peças ou componentes para continuar operando
corretamente, enquanto que a manutenção preventiva visa o agendamento programado de intervenções no
sistema, afim de manter o funcionamento pelo maior tempo possível. Por outro lado, a manutenção
preditiva tem como foco o monitoramento do sistema continuamente e, desta forma, a intervenção é
feita somente quando necessário.

Nas três estratégias, pode-se citar vantagens e desvantagens. Na manutenção corretiva, a principal
vantagem está na dispensabilidade de realização de acompanhamentos ou inspeções no sistema. Isso
evita a geração de custos na alocação de pessoas ou equipamentos para desempenharem tarefas de
verificação do sistema, além da parada da linha de produção em intervalos agendados. Por outro lado,
a parada inesperada da linha de produção para uma manutenção emergencial pode gerar transtornos e
custos não programados. A manutenção preventiva visa sanar os problemas de paradas inesperadas,
utilizando-se de um modelo de agendamento das inspeções. O que para muitas situações é considerado
suficiente, se não for bem planejado, pode acarretar em custos excessivos devido às paradas
programadas e alocação de equipes de manutenção. Nesta situação, a estratégia de manutenção
preditiva busca o meio termo, utilizando-se da predição do estado do sistema, como análise de
tendências ou avaliações probabilísticas do estado de degradação dos equipamentos, para os
agendamentos de novas intervenções.

Mesmo com a programação das intervenções, as máquinas podem falhar de modo repentino, pondo em risco
os equipamentos e pessoas envolvidas com o processo produtivo~\cite{goncalves2011desenvolvimento}. A
falha no intervalo entre intervenções não é possível de prever através dos métodos clássicos de
manutenção. Logo, nos últimos anos, o que tem se visto é a substituição da estratégia de manutenção
preventiva por um novo paradigma: a manutenção proativa~\cite{lee2009informatics}. Esta nova
estratégia visa não somente a predição do estado do sistema, mas também o diagnóstico das falhas e,
em casos onde é aplicado, a intervenção de forma automática. Por intervenção, entende-se que o
padrão de funcionamento dos equipamentos monitorados pode ser alterado, visando minimizar os
possíveis agravantes até a realização da manutenção.

Neste cenário, emergem os sistemas de manutenção inteligente. Também conhecidos como sistemas de
manutenção baseados no conhecimento, visam capturar o conhecimento de um determinado sistema sob a
forma de regras e utilizá-las para construir um novo sistema baseado nestas regras. O novo sistema
é, então, utilizado para realização de um correto diagnóstico ou tomada de ação no caso da
ocorrência de algum defeito. Como exemplo, em~\cite{shikari2004automation} o padrão de vibração de
uma máquina de indução, de um atuador e de uma prensa são analisados e, realizado o diagnóstico
automático através de um sistema de manutenção inteligente, é determinado o motivo da falha, podendo
ser os rolamentos ou desalinhamentos.

\includefigure
    {images/maintenance-strategies}
    {Classificação das estratégias de manutenção.}
    {fig:maintenance-strategies}
\todo{Referenciar figura no texto.}

Com o intuito de auxiliar na migração do paradigma de conserto após falha para o paradigma de
predição e prevenção, foi criado, nos Estados Unidos, um centro de parceria entre universidades e
empresas, denominado \gls{IMSCenter}. Dentre as empresas integrantes da parceria \gls{IMSCenter},
pode-se citar, por exemplo, Boeing, Siemens, AMD, Toyota e Caterpillar. Entre as universidades,
fazem parte do consórcio a de Cincinnatti, Missouri-Rolla e Michigan.


%%%%%%%%
%\section{A ferramenta Watchdog Agent}

Um dos objetivos da parceria \gls{IMSCenter} foi o desenvolvimento de uma metodologia para abordagem
dos problemas de manutenção utilizando predição e prevenção. Para tanto, foi desenvolvido um
conjunto de ferramentas de análise denominado Watchdog Agent. Em termos gerais, o Watchdog Agent é
uma ferramenta de análise de desempenho. Aplicado a determinado equipamento, visa analisar sinais de
diversas partes da máquina, a fim de obter um índice de desempenho.

A extração das informações contidas nos sinais analisados são extraídas através das ferramentas
implementadas no Watchdog Agent. Primeiramente os dados dos sensores do equipamento são adquiridos.
Em um segundo momento, os dados são classificados com o auxílio de algoritmos. Com os dados
classificados, é possível determinar o índice de desempenho para a situação analisada. Estas etapas
são ilustradas na \cref{fig:data-processing-plot}.

\includefigure
    {images/data-processing-plot}
    {Processamento das informações utilizando a estratégia proposta pelo IMS~Center.}
    {fig:data-processing-plot}

A medida que o equipamento degrada, o índice de desempenho é alterado em comparação com o mesmo
indicador obtido com o equipamento em funcionamento normal. Um indicador normalmente utilizado para
identificação do estado de um equipamento é o valor de confiança. Este indicador é definido como uma
grandeza que varia no intervalo~${[0,1]}$. Valores próximos a~\num{1} representam funcionamento
normal do sistema, enquanto que valores próximos a~\num{0} equivalem a um funcionamento em falha. A
\cref{fig:confidence-value-concept} ilustra o conceito de valor de confiança. As duas curvas da
esquerda apresentam o comportamento normal e o comportamento recente de um determinado equipamento.
Ao cruzar as duas informações, é possível obter o valor de confiança, exemplificado no gráfico da
direita. À medida que o valor de confiança decai, a probabilidade de redução do desempenho do
sistema aumenta~\cite{djurdjanovic2003watchdog}.

\includefigure
    {images/confidence-value-concept}
    {Representação do conceito de valor de confiança.}
    {fig:confidence-value-concept}

%\missingfigure{Estrutura do Watchdog Agent.}

%\todo[inline]{Uma das vantagens no uso de técnicas proativas...}

Em comparação com estratégias de manutenção preventiva, um ponto importante a ser citado é aumento
da vida útil de peças de equipamentos~\cite{lazzaretti2012avaliacao}. Peças que poderiam ser
descartadas em função de uma intervenção preventiva.


%%%%%%%%
\section{Modelo OSA-CBM}

Como proposta de padronização de uma arquitetura aberta para troca de informações em um sistema
baseado em condição, surge o modelo \gls{OSACBM}~\cite{thurston2001open}. A arquitetura \gls{OSACBM}
visa facilitar a integração e interoperabilidade entre componentes e equipamentos de diferentes
fabricantes. Definida em sete camadas, possibilita a abstração de várias partes envolvidas em um
sistema de manutenção inteligente. A \cref{fig:osa-cbm-model} apresenta uma visão geral das camadas
do modelo juntamente com as suas interações. As camadas são numeradas de 1~(aquisição de dados) a
7~(apresentação).

%\todo[color=yellow, inline]{Outro todo.}

\includefigure
    {images/osa-cbm-model}
    {Modelo OSA-CBM.}
    {fig:osa-cbm-model}

A definição das funcionalidades de cada camada é apresentada por~\cite{thurston2001open}. Na camada
de aquisição de dados, as grandezas físicas são convertidas para sinais elétricos e digitalizadas. O
módulo consiste, normalmente, de um elemento sensor e um elemento de aquisição de dados. Além da
conversão física, a camada também pode armazenar os dados coletados em um banco de dados. A primeira
etapa de cálculos sob os dados obtidos é feita na camada de manipulação dos dados. Através do uso de
ferramentas de processamento de sinais, os dados adquiridos na camada anterior são manipulados,
podendo gerar resultados no domínios tempo, frequência ou tempo-frequência. Eventualmente, os
resultados das operações também podem ser armazenados em um banco de dados. O módulo de detecção do
estado do sistema analisa continuamente os indicadores de cada sistema, subsistema ou componente. De
posse dos dados processados pelas camadas anteriores, ao calcular os indicadores de estado, o módulo
de detecção pode gerar alarmes respeitando condições previamente estabelecidas. Novamente os dados
obtidos podem ser armazenados para uso posterior. Na camada de avaliação da saúde do sistema, o
resultado dos indicadores, obtidos no módulo de detecção do estado do sistema, são inseridos no
contexto das operações. A saúde do sistema monitorado é avaliada pelo uso dos indicadores atuais e
passados. Dessa forma, também é possível armazenar os resultados formando um histórico do
equipamento monitorado. Na camada de prognóstico, a saúde futura do sistema é estimada. Através de
um modelo estimado do sistema e dos dados obtidos nas camadas anteriores, o tempo de vida útil ou a
probabilidade de falha em um horizonte de predição são estimados. Como nas camadas anteriores, os
resultados podem ser armazenados em um banco de dados. O módulo de tomada de decisão utiliza os
dados obtidos na camada de prognóstico, além de outras informações, para sugerir ações recomendadas
de acordo com as implicações das decisões. São integrados, juntamente com os dados da camada de
prognóstico, informações de restrições externas, requisitos de funcionalidades do equipamento ou
sistema, condições financeiras, entre outros. No nível mais alto do modelo, está a camada de
aplicação. Definida como a interface homem-máquina do sistema, visa a apresentação dos dados obtidos
no processamento das informações. Nesta camada também podem ser utilizadas técnicas de realidade
aumentada~\cite{espindola2010visualizacao}.


%%%%%%%%
\section{Algoritmos de processamento da ferramenta Watchdog Agent}

Como mencionado anteriormente, o Watchdog Agent é um conjunto de algoritmos para processamento de sinais e extração de características.

\subsection{Energias da transformada Wavelet Packet}


