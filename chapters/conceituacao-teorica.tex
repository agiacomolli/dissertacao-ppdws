\chapter{Conceituação teórica}

Este capítulo apresenta os conceitos básicos utilizados ao longo do desenvolvimento do trabalho.
Basicamente são discutidas as duas áreas principais que envolvem o estudo proposto: sistemas de
manutenção inteligente e arquiteturas orientadas a serviços. Em se tratando de manutenção
inteligente, são discutidos os tipos de manutenção empregados e as novas tendências, que envolvem a
utilização de técnicas de predição e monitoramento contínuo do sistema. Também é apresentado o
software Watchdog Agent e as ferramentas utilizadas para manipulação dos dados.

Na conceituação teórica de arquiteturas orientadas a serviços, são explanadas as principais
características deste tipo de abordagem. Além disso, são apontados os elementos principais que fazem
parte do sistema e implementações disponíveis. Uma abordagem mais detalhada é ilustrada para a
utilização de \gls{DPWS}, que é a tecnologia utilizada neste trabalho para a construção da aplicação
\gls{SOA}.


%%
\section{Sistemas de manutenção inteligente}
\label{sec:manutencao-inteligente}

Manutenção, no âmbito geral, consiste em uma série de medidas de prevenção, correção e predição de
falhas~\cite{lee2006intelligent}. Durante o uso, equipamentos ou máquinas tendem a deteriorar e
alterar o seu padrão de funcionamento devido a diversos fatores, como, por exemplo, desgaste,
rachaduras, corrosão e sujeira. Nestas condições, a restauração do sistema é de suma importância,
visto que, com o passar do tempo, podem apresentar defeitos e levar à falhas e indisponibilidades.
De acordo com~\cite{marcal2005detectando}, manutenção pode ser definida como todas as atividades
técnicas e organizacionais que garantam a operação das máquinas e equipamentos dentro da
confiabilidade esperada.

Tradicionalmente, são encontrados na literatura três\todo{verificar} tipos de estratégias de
manutenção: corretiva, preventiva e preditiva~\cite{goncalves2011desenvolvimento}. A manutenção
corretiva visa reestabelecer os sistemas danificados; a preventiva tem por objetivo manter os
sistemas em funcionamento, realizando pequenas correções; e a manutenção preditiva tem por base o
monitoramento do estado do sistema, detectando falhas insipientes e fornecendo subsídios para o
planejamento de ações de prevenção ou correção. Em termos gerais, a manutenção corretiva é aplicada
somente quando há falha e o sistema necessita de reposição de peças ou componentes para continuar
operando corretamente, enquanto que a manutenção preventiva visa o agendamento programado de
intervenções no sistema, afim de manter o funcionamento pelo maior tempo possível. Por outro lado, a
manutenção preditiva tem como foco o monitoramento do sistema continuamente e, desta forma, a
intervenção é feita somente quando necessário.

Nas três estratégias, pode-se citar vantagens e desvantagens. Na manutenção corretiva, a principal
vantagem está na dispensabilidade de realização de acompanhamentos ou inspeções no sistema. Isso
evita a geração de custos na alocação de pessoas ou equipamentos para desempenharem tarefas de
verificação do sistema, além da parada da linha de produção em intervalos agendados. Por outro lado,
a parada inesperada da linha de produção para uma manutenção emergencial pode gerar transtornos e
custos não programados. A manutenção preventiva visa sanar os problemas de paradas inesperadas,
utilizando-se de um modelo de agendamento das inspeções. O que para muitas situações é considerado
suficiente, se não for bem planejado, pode acarretar em custos excessivos devido às paradas
programadas e alocação de equipes de manutenção. Nesta situação, a estratégia de manutenção
preditiva busca o meio termo, utilizando-se da predição do estado do sistema, como análise de
tendências ou avaliações probabilísticas do estado de degradação dos equipamentos, para os
agendamentos de novas intervenções.

Mesmo com a programação das intervenções, as máquinas podem falhar de modo repentino, pondo em risco
os equipamentos e pessoas envolvidas com o processo produtivo~\cite{goncalves2011desenvolvimento}. A
falha no intervalo entre intervenções não é possível de prever através dos métodos clássicos de
manutenção. Logo, nos últimos anos, o que tem se visto é a substituição da estratégia de manutenção
preventiva por um novo paradigma: a manutenção proativa~\cite{lee2009informatics}. Esta nova
estratégia visa não somente a predição do estado do sistema, mas também o diagnóstico das falhas e,
em casos onde é aplicado, a intervenção de forma automática. Por intervenção, entende-se que o
padrão de funcionamento dos equipamentos monitorados pode ser alterado, visando minimizar os
possíveis agravantes até a realização da manutenção.

Neste cenário, emergem os sistemas de manutenção inteligente. Também conhecidos como sistemas de
manutenção baseados no conhecimento, visam capturar o conhecimento de um determinado sistema sob a
forma de regras e utilizá-las para construir um novo sistema baseado nestas regras. O novo sistema
é, então, utilizado para realização de um correto diagnóstico ou tomada de ação no caso da
ocorrência de algum defeito. Como exemplo, em~\cite{shikari2004automation} o padrão de vibração de
uma máquina de indução, de um atuador e de uma prensa são analisados e, realizado o diagnóstico
automático através de um sistema de manutenção inteligente, é determinado o motivo da falha, podendo
ser os rolamentos ou desalinhamentos.

\includefigure
    {images/maintenance-strategies}
    {Classificação das estratégias de manutenção.}
    {fig:maintenance-strategies}
\todo{Referenciar figura no texto.}

Com o intuito de auxiliar na migração do paradigma de conserto após falha para o paradigma de
predição e prevenção, foi criado, nos Estados Unidos, um centro de parceria entre universidades e
empresas, denominado \gls{IMSCenter}. Dentre as empresas integrantes da parceria \gls{IMSCenter},
pode-se citar, por exemplo, Boeing, Siemens, AMD, Toyota e Caterpillar. Entre as universidades,
fazem parte do consórcio a de Cincinnatti, Missouri-Rolla e Michigan.


%%%
\subsection{A ferramenta Watchdog Agent}

Um dos objetivos da parceria \gls{IMSCenter} foi o desenvolvimento de uma metodologia para abordagem
dos problemas de manutenção utilizando predição e prevenção. Para tanto, foi desenvolvido um
conjunto de ferramentas de análise denominado Watchdog Agent. Em termos gerais, o Watchdog Agent é
uma ferramenta de análise de desempenho. Aplicado a determinado equipamento, visa analisar sinais de
diversas partes da máquina, a fim de obter um índice de desempenho.

A extração das informações contidas nos sinais analisados são extraídas através das ferramentas
implementadas no Watchdog Agent. Primeiramente os dados dos sensores do equipamento são adquiridos.
Em um segundo momento, os dados são classificados com o auxílio de algoritmos. Com os dados
classificados, é possível determinar o índice de desempenho para a situação analisada. Estas etapas
são ilustradas na \cref{fig:data-processing-plot}.

\includefigure
    {images/data-processing-plot}
    {Processamento das informações utilizando a estratégia proposta pelo IMS~Center.}
    {fig:data-processing-plot}

A medida que o equipamento degrada, o índice de desempenho é alterado em comparação com o mesmo
indicador obtido com o equipamento em funcionamento normal. Um indicador normalmente utilizado para
identificação do estado de um equipamento é o valor de confiança. Este indicador é definido como uma
grandeza que varia no intervalo~${[0; 1]}$. Valores próximos a~\num{1} representam funcionamento
normal do sistema, enquanto que valores próximos a~\num{0} equivalem a um funcionamento em falha. A
\cref{fig:confidence-value-concept} ilustra o conceito de valor de confiança. As duas curvas da
esquerda apresentam o comportamento normal e o comportamento recente de um determinado equipamento.
Ao cruzar as duas informações, é possível obter o valor de confiança, exemplificado no gráfico da
direita. À medida que o valor de confiança decai, a probabilidade de redução do desempenho do
sistema aumenta~\cite{djurdjanovic2003watchdog}.

\includefigure
    {images/confidence-value-concept}
    {Representação do conceito de valor de confiança.}
    {fig:confidence-value-concept}

%\missingfigure{Estrutura do Watchdog Agent.}

%\todo[inline]{Uma das vantagens no uso de técnicas proativas...}

Em comparação com estratégias de manutenção preventiva, um ponto importante a ser citado é aumento
da vida útil de peças de equipamentos~\cite{lazzaretti2012avaliacao}. Peças que poderiam ser
descartadas em função de uma intervenção preventiva\todo{terminar}.


%%%
\subsection{Modelo OSA-CBM}

Como proposta de padronização de uma arquitetura aberta para troca de informações em um sistema
baseado em condição, surge o modelo \gls{OSACBM}~\cite{thurston2001open}. A arquitetura \gls{OSACBM}
visa facilitar a integração e interoperabilidade entre componentes e equipamentos de diferentes
fabricantes. Definida em sete camadas, possibilita a abstração de várias partes envolvidas em um
sistema de manutenção inteligente. A \cref{fig:osa-cbm-model} apresenta uma visão geral das camadas
do modelo juntamente com as suas interações. As camadas são numeradas de 1~(aquisição de dados) a
7~(apresentação).

%\todo[color=yellow, inline]{Outro todo.}

\includefigure
    {images/osa-cbm-model}
    {Modelo OSA-CBM.}
    {fig:osa-cbm-model}

A definição das funcionalidades de cada camada é apresentada por~\cite{thurston2001open}. Na camada
de aquisição de dados, as grandezas físicas são convertidas para sinais elétricos e digitalizadas. O
módulo consiste, normalmente, de um elemento sensor e um elemento de aquisição de dados. Além da
conversão física, a camada também pode armazenar os dados coletados em um banco de dados. A primeira
etapa de cálculos sob os dados obtidos é feita na camada de manipulação dos dados. Através do uso de
ferramentas de processamento de sinais, os dados adquiridos na camada anterior são manipulados,
podendo gerar resultados no domínios tempo, frequência ou tempo-frequência. Eventualmente, os
resultados das operações também podem ser armazenados em um banco de dados. O módulo de detecção do
estado do sistema analisa continuamente os indicadores de cada sistema, subsistema ou componente. De
posse dos dados processados pelas camadas anteriores, ao calcular os indicadores de estado, o módulo
de detecção pode gerar alarmes respeitando condições previamente estabelecidas. Novamente os dados
obtidos podem ser armazenados para uso posterior. Na camada de avaliação da saúde do sistema, o
resultado dos indicadores, obtidos no módulo de detecção do estado do sistema, são inseridos no
contexto das operações. A saúde do sistema monitorado é avaliada pelo uso dos indicadores atuais e
passados. Dessa forma, também é possível armazenar os resultados formando um histórico do
equipamento monitorado. Na camada de prognóstico, a saúde futura do sistema é estimada. Através de
um modelo estimado do sistema e dos dados obtidos nas camadas anteriores, o tempo de vida útil ou a
probabilidade de falha em um horizonte de predição são estimados. Como nas camadas anteriores, os
resultados podem ser armazenados em um banco de dados. O módulo de tomada de decisão utiliza os
dados obtidos na camada de prognóstico, além de outras informações, para sugerir ações recomendadas
de acordo com as implicações das decisões. São integrados, juntamente com os dados da camada de
prognóstico, informações de restrições externas, requisitos de funcionalidades do equipamento ou
sistema, condições financeiras, entre outros. No nível mais alto do modelo, está a camada de
aplicação. Definida como a interface homem-máquina do sistema, visa a apresentação dos dados obtidos
no processamento das informações. Nesta camada também podem ser utilizadas técnicas de realidade
aumentada~\cite{espindola2011realidade}.


%%%
\subsection{Algoritmos de processamento da ferramenta Watchdog Agent}

Como mencionado anteriormente, o Watchdog Agent é um conjunto de algoritmos para processamento de
sinais e extração de características\todo{Terminar}.


%%%%
\subsubsection{Energias da transformada Wavelet Packet}

Uma das forma de se analisar sinais não estacionários no domínio tempo-frequência é utilizando a
transformada wavelet~\cite{antonini1992image}. Seu uso é indicado para sinais que apresentam
descontinuidades, tendências entre outros. É empregado nas mais diversas aplicações, desde a remoção
de ruídos em sinais ou imagens até a compressão de imagens médicas com pouca perda de qualidade.

Wavelets são formas de onda oscilantes com duração limitada e valor médio zero. São empregadas na
forma de wavelets mãe, definidas por \gls{wavelet-mother}. A função wavelet mãe pode ser dilatada ou
comprimida através de um parâmetro \gls{wavelet-scale-param} e transladada através de um
parâmetro \gls{wavelet-translation-param}. A mudança de escala e translação são apresentadas na
\cref{eq:wavelet-translation-compression}.

\begin{equation}
  \psi_{\alpha, \tau}(t) = \frac{1}{\sqrt{\alpha}} \psi \left ( \frac{t - \tau}{\alpha} \right )
  \label{eq:wavelet-translation-compression}
\end{equation}

A \cref{eq:wavelet-definition-continuous} apresenta a definição da transformada wavelet contínua
\gls{wavelet-transform-continuous} de um sinal contínuo \gls{signal-time-continuous}.

\begin{equation}
  \mathcal{W} \left \{ x, \psi \right \} =
      \frac{1}{\sqrt{\alpha}} \left \{ x(t), \psi_{\alpha, \tau}(t) \right \} =
      \frac{1}{\sqrt{\alpha}} \int_{-\infty}^{\infty} x(t) \cdot
        \psi^{*} \left ( \frac{t - \tau}{\alpha} \right ) \textrm{d}t
  \label{eq:wavelet-definition-continuous}
\end{equation}

A largura da wavelet é influenciada pelo fator de escala \gls{wavelet-scale-param}, o que também
contribui para a alteração da resolução empregada na análise. Quanto menor o valor de
\gls{wavelet-scale-param}, maior será a resolução empregada na detecção de eventos de alta
frequência. No caso contrário, quanto maior for o valor de \gls{wavelet-scale-param}, maior será a
dilatação empregada na wavelet mãe, o que é conveniente para a identificação de padrões de baixa
frequência. A \cref{fig:wavelet-time-frequency-representation} ilustra a mudança de escala para
análise do sinal em multiresolução.

\includefiguretmp
  %{images/wavelet-time-frequency-representation}
  {Representação da resolução tempo frequência da transformada wavelet.}
  {fig:wavelet-time-frequency-representation}

Como alternativa para a utilização da transformada para sinais discretos, é possível a utilização da
transformada wavelet discreta. Como a transformada wavelet contínua requer um esforço computacional
considerado exagerado para calcular os coeficientes de todas as possíveis escalas da transformada,
gerando informações redundantes, é possível a utilização de parâmetros de escalonamento e translação
discretos~\cite{mallat1989theory}. A transformada wavelet discreta \gls{wavelet-transform-discrete}
de um sinal contínuo \gls{signal-time-continuous} é apresentada na
\cref{eq:wavelet-definition-discrete}.

\begin{equation}
  \mathcal{V} \left \{ x, \psi \right \} =
      \left \{ x(t), \psi_{\kappa \beta} \right \} =
      \int_{-\infty}^{\infty} x(t) \cdot
        \psi_{\kappa \beta}(t) \textrm{d}t
  \label{eq:wavelet-definition-discrete}
\end{equation}

A função \gls{wavelet-mother}{\ensuremath{_{\kappa \beta}}} é a wavelet mãe criada a partir de
parâmetros de escala e translação discretos. A \cref{eq:wavelet-translation-compression-discrete}
apresenta a obtenção da função \gls{wavelet-mother}{\ensuremath{_{\kappa \beta}}}, onde \gls
{wavelet-scale-param} é a variação da escala e \gls{wavelet-translation-param} indica a translação;
\gls{dilatation-param} e \gls{scale-param} são constantes discretas que indicam, respectivamente, a
variação da escala e dilatação.

\begin{equation}
  \psi_{\kappa \beta}(t) =
  \frac{1}{\sqrt{\alpha^{\kappa}}} \psi \left (
    \frac{t - \beta \alpha^{\kappa} \tau}{\alpha^{\kappa}} \right )
  \label{eq:wavelet-translation-compression-discrete}
\end{equation}

Em termos gerais, a transformada wavelet é realizada através de um processo de filtragem de vários
níveis, onde cada nível apresenta um filtro em quadratura. O sinal decomposto em cada um dos níveis
apresenta duas informações, definidas como \emph{detalhe} (alta frequência) e \emph{aproximação}
(baixa frequência). A decomposição em vários níveis origina uma árvore, denominada de árvore de
decomposição wavelet packet~\cite{mallat1989theory}. Este processo é ilustrado na
\cref{fig:wavelet-decompose-representation}.

\includefiguretmp
  %{images/wavelet-decompose-representation}
  {Representação de dois níveis da árvore de decomposição wavelet packet.}
  {fig:wavelet-decompose-representation}

Como resultado do processo da árvore de decomposição, é gerado um vetor de elementos obtido através
do cálculo da energia dos coeficientes no nível mais baixo da decomposição da transformada. Este
vetor de características é denominado de energias da transformada wavelet
packet~\cite{ims2007documentation}.


%%%%
\subsubsection{Regressão logística}

A regressão logística é um método de classificação que permite uma classificação binária ou
dicotômica de um conjunto de dados~\cite{hosmer2013applied}. O método integra a categoria de modelos
chamados de Modelos Generalizados Lineares, e, portanto, como resultado da análise, é gerada uma
resposta de dois estados, que podem ser traduzidos, por exemplo, para sucesso ou falha ou
comportamento normal ou degradado.

O processo empregado pelo método da regressão logística é definido como a tentativa de ajustar o
espaço de \gls{dimension-index} dimensões da entrada para um espaço de saída de apenas uma dimensão.
A variável de saída ou resposta é representada por \gls{logistic-regression-output}, sendo que
\gls{logistic-regression-output}{\ensuremath{~= 1}} quando o conjunto de entrada possui a
característica de interesse e \gls{logistic-regression-output}{\ensuremath{~= 0}} quando não
possui~\cite{hosmer2013applied}. A \cref{fig:logistic-regression-representation} apresenta uma
curva típica da saída de um modelo de regressão logística.

\includefiguretmp
  %{images/logistic-regression-representation}
  {Representação de um modelo de regressão logística.}
  {fig:logistic-regression-representation}

A representação matemática do modelo de regressão logística é apresentada na
\cref{eq:logistic-regression-model}~\cite{ims2007documentation}. No modelo,
\gls{logistic-regression-input} é o vetor de entrada de \gls{dimension-index} dimensões e
\gls{logistic-regression-output} é a saída binária.

\begin{equation}
  p(r) =
  P(y = 1 | r) =
  \frac{1}{1 + e^{- \left (
    \alpha + \beta_{1} r_{1} + \beta_{2} r_{2} + \cdots + \beta_{k} r_{k} \right )}} =
  \frac{e^{\alpha + \beta_{1} r_{1} + \beta_{2} r_{2} + \cdots + \beta_{k} r_{k}}}
    {1 + e^{\alpha + \beta_{1} r_{1} + \beta_{2} r_{2} + \cdots + \beta_{k} r_{k}}}
  \label{eq:logistic-regression-model}
\end{equation}

O modelo apresentado na \cref{eq:logistic-regression-model} também pode ser representado em termos
das probabilidade de evento e de não evento. Neste caso, são definidos como $p(r)$ e $1 - p(r)$. A
\cref{eq:logistic-regression-model-linear} apresenta a nova função, onde o termo contendo o
logaritmo natural é conhecido como função \emph{logit}, cujo propósito é tornar a função linear.

\begin{equation}
  g(r) =
  \ln \left ( \frac{p(r)}{1 - p(r)} \right ) =
  \alpha + \beta_{1} r_{1} + \beta_{2} r_{2} + \cdots + \beta_{k} + r_{k}
  \label{eq:logistic-regression-model-linear}
\end{equation}

A partir da \cref{eq:logistic-regression-model} o valor de confiança é obtido. Para um comportamento
normal, o valor de confiança assume valores próximos a \num{1}, enquanto que para comportamento
ditos de falha, o valor de confiança fica concentrado próximo a \num{0}.


%%
\section{Arquiteturas orientadas a serviços}
\label{sec:arquiteturas-orientadas-servicos}

As técnicas para desenvolvimento de aplicações \gls{SOA} representam uma mudança de paradigma na
engenharia de software, onde os componentes são definidos como serviços~\cite{ramollari2007survey}.
Originalmente desenvolvido e utilizado para integração de sistemas no meio gerencial e corporativo,
logo teve aceitação entre diversos segmentos, como plataformas de negócio, telecomunicações,
transportes e na automação industrial.

O termo \gls{SOA} ainda possui uma definição concisa e única, diferindo conforme os conhecimentos
técnicos e a bagagem acumulada durante o desenvolvimento de diferentes aplicações por parte dos
autores~\cite{candido2013soa}. Ainda segundo \cite{candido2013soa}, a definição que mais se encaixa
no contexto de um trabalho que envolve integração em meio industrial é a
de~\cite{jammes2005service}, o qual expressa que "\gls{SOA} é um conjunto de princípios ou doutrinas
para a construção de sistemas interoperáveis e também autônomos". Esta percepção também descreve o
contexto deste trabalho, no qual os componentes envolvidos podem ser considerados peças
independentes do sistema, no entanto podem vir a representar um conjunto interoperável de entidades,
compartilhando recursos entre si.

Em termos gerais, a característica principal de uma arquitetura \gls{SOA} é a criação e
disponibilidade de serviços, que, quando agrupados, constituem um sistema
funcional~\cite{josuttis2009soa}. O termo serviço se refere a uma funcionalidade ou lógica que é
encapsulada e oferecida ao sistema através de uma interface. Dessa forma, outros serviços, entidades
ou programas podem obter o modo de acesso à esta funcionalidade e empregá-la na resolução de
determinada tarefa.


%%%
\subsection{Componentes de uma arquitetura orientada a serviços}

Em se tratando do contexto da aplicação, os serviços oferecidos, para que possam ser utilizados,
precisam ser encontrados ou expostos~\cite{papazoglou2007service}. Mesmo que, segundo a definição
adotada, serviços possam ser utilizados independentemente, a abordagem de utilizar um conjunto de
serviços trabalhando de forma cooperativa na resolução de um problema, é muito mais interessante.
Portanto, uma aplicação \gls{SOA} deve prover meios para que os serviços possam ser comunicar e
trocar informações via mensagens padronizadas. Dessa forma, é necessário que existam alguns
conceitos a serem cumpridos por todos os componentes de uma arquitetura
\gls{SOA}~\cite{erl2005service}:

\begin{itemize}
  \item \emph{Acoplamento mínimo}: serviços devem minimizar a dependência, armazenando somente as
  informações de outros serviços.

  \item \emph{Contrato de serviço}: devem utilizar um padrão de comunicação comum previamente
  definido e baseado em padrões abertos.

  \item \emph{Autonomia}: possibilidade de controle total da lógica que o serviço encapsula.

  \item \emph{Abstração}: possibilidade de esconder a lógica e os recursos utilizados pelo serviço
  do resto da aplicação.

  \item \emph{Reusabilidade}: utilização da mesma funcionalidade por diferentes partes do sistema ou
  em aplicações futuras, sem a necessidade de uma nova implementação.

  \item \emph{Composição}: organização de serviços para a construção de tarefas mais complexas podem
  ser feitas através da composição de serviços mais simples.

  \item \emph{Sem dependência de estado}: os serviços não devem reter nenhuma informação específica
  sobre as atividade executadas.

  \item \emph{Possibilidade de descoberta}: os serviços devem possibilitar a sua descoberta pelos
  mecanismos de busca.
\end{itemize}

As aplicações \gls{SOA} normalmente são construídas baseadas nos princípios de serviços
web~\cite{josuttis2009soa}. Não necessariamente as aplicações \gls{SOA} necessitam ser baseadas em
serviços web, porém, esta tecnologia começou a ser largamente utilizada devido, entre outros
fatores, a adoção de uma padronização. Por parte dos desenvolvedores, o encapsulamento das
funcionalidades que um serviço pode oferecer foi facilitado após a definição de interfaces e
protocolos de comunicação padrão. Dessa forma, é possível aos clientes acessar os serviços de forma
transparente, sem conhecimento prévio de detalhes de implementação.

Seguindo as convenções adotadas para serviços web, um serviço é uma entidade de software
identificada por uma \gls{URI}~\cite{bell2008service}. A \gls{URI} define o endereço do serviço na
aplicação, devendo ser única. O identificador permite a discriminação entre grupos de entidades,
através da utilização de um separador. Esta técnica é largamente utilizada em sistemas que usufruem
de identificadores baseados em \gls{URI}.

A troca de informações entre os serviços normalmente ocorre utilizando o protocolo \gls{SOAP}, onde
as mensagens são codificadas no formato \gls{XML}~\cite{josuttis2009soa}. O protocolo \gls{SOAP}
provê uma infraestrutura básica para a troca de mensagens entre serviços web. É definido por um
envelope, um conjunto de regras que definem os tipos de dados suportados e um meio de representar os
procedimentos ou funcionalidades disponíveis para execução. Por ser baseado em \gls{XML}, o
protocolo pode ser utilizado sobre diferentes protocolos de transporte, como, por exemplo, o
\gls{HTTP}.

A descrição das interfaces de cada serviço é normalmente definida por \gls{WSDL}. Documentos
\gls{WSDL} também são baseados em \gls{XML} e contém toda a informação necessária para a utilização
do serviço em questão. No documento, são especificadas todas as operações que o serviço possibilita,
bem como os tipos de dados suportados. Também é possível estender o documento definindo novos tipos
de dados para troca de mensagens.

Os serviços disponíveis na aplicação e a descrição de suas funcionalidades estão centralizadas no
\gls{UDDI}. O padrão \gls{UDDI} define o protocolo para os serviços de diretório, ou intermediadores
de serviço, onde são armazenadas todas as informações de cada um dos serviços da aplicação. Esta
entidade é utilizada para informar aos clientes quais serviços estão disponíveis, possibilitando
meios de descobri-los e obter seus metadados. A interoperabilidade entre os componentes de uma
aplicação \gls{SOA} é ilustrada na \cref{fig:soa-elements}.

\includefigure
  {images/soa-elements}
  {Interoperabilidade entre os elementos de uma aplicação SOA.}
  {fig:soa-elements}


%%%
\subsection{Device Profile for Web Services}

Especificações para web services normalmente são grafadas com o prefixo
"{WS-}"~\cite{candido2013soa}. É comum encontrar na literatura o termo "{WS-*}", o qual se refere ao
agrupamento de diferentes especificações para web services. Dentre as especificações, o \gls{DPWS}
define um conjunto mínimo de implementações que permitem a troca de mensagens, descoberta,
descrição, geração de eventos e autenticação para a utilização de web services em clientes com
recursos computacionais limitados. O \gls{DPWS} permite a integração destes clientes com outros, com
recursos mais flexíveis.

O \gls{DPWS} implementa um conjunto restrito dos padrões {WS-*}. A \cref{fig:dpws-stack} ilustra o
diagrama contendo a pilha de protocolos suportados pelo \gls{DPWS}. Dentre os padrões, é possível
destacar o WS-Adressing, utilizado para transferência de mensagens, WS-Security\todo{alterar a
figura}, para suprir as necessidades de autenticação nos web services, WS-Discovery, que possibilita
a descoberta de serviços em uma rede local, e WS-Eventing, que permite a utilização de eventos.

\includefigure
  {images/dpws-stack}
  {Pilha de protocolos suportados pelo DPWS.}
  {fig:dpws-stack}
