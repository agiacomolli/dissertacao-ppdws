\chapter{Análise do estado da arte}

\todo[inline]{Introdução do capítulo.}


%%
\section{Sistemas de manutenção inteligente}

\todo[inline]{Introdução da seção.}

O trabalho de~\cite{lee2006intelligent} introduz os conceitos de \textit{e-maintenance}, emergentes
à época, e os elementos críticos que o compõem. Considerando que o mercado global está mais
competitivo a cada ano, os autores afirmam que o uso de técnicas que minimizem os custos com a
produção é de fundamental importância. Entre os custos citados, estão em destaque os que dizem
respeito à falhas e quebras não programadas de maquinário. Também é sugerida a mudança de paradigma,
da tradicional \emph{falha e conserto} para \emph{predição e manutenção}. Para contornar estes
problemas, o artigo sugere o uso de técnicas computacionais, que, juntamente com o advento da
Internet, possibilitam o monitoramento da condição dos equipamento, ao invés de somente detectar os
equipamento em falha.

As ferramentas utilizadas para o monitoramento da condição dos equipamentos também são discutidas no
artigo de~\cite{lee2006intelligent}. Ferramentas de avaliação da condição e predição de falhas são
analisadas, a fim de se obter um monitoramento contínuo do equipamento. Além disso, são apresentados
os últimos avanços em relação às ferramentas utilizadas para este tipo de cenário e estudos de caso
para validação das técnicas empregadas.

Dando continuidade aos trabalhos na área de manutenção, o artigo de~\cite{muller2008concept} aponta
que a importância da manutenção em sistemas, a fim de aumentar a disponibilidade e segurança, bem
como a qualidade dos produtos, está em vias de crescimento. Como o artigo anterior, o trabalho
também ilustra os conceitos de \textit{e-maintenance}, provendo uma visão geral sobre os estudos e
desafios presentes neste campo de estudo. São apresentados os diferentes sistemas de manutenção
disponíveis, além de uma comparação entre as diferentes características de cada um. Como pontos
importantes, os autores veem grande potencial na adoção de novas tecnologias que auxiliem nas
tarefas de manutenção, os quais são definidos em termos de \emph{dispositivos inteligentes}. Além
disso, defendem a criação de padrões internacionais para integração entre os sistemas discutidos. E,
como contribuição maior, está a proposta da organização das ações no setor, em prol do avanço dos
estudos e definição de novos conceitos, apoiando o surgimento de uma nova disciplina.

\iffalse
Em~\cite{iung2009conceptual}, um \textit{framework} conceitual para integração entre as diferentes
tecnologias de \textit{e-maintenance} é proposto.
\fi

No trabalho de~\cite{goncalves2009design}, a detecção de falhas em um atuador eletromecânico é feita
através de um sistema composto por um microprocessador Microblaze sintetizado em um \gls{FPGA}. No
\gls{FPGA}, os sinais são processados e mapas auto-organizáveis são utilizados para detecção,
classificação e predição de falhas. O treinamento dos algoritmos é realizado em um computador
pessoal, enquanto que o monitoramento dos sinais é feito no \gls{FPGA}. O trabalho é estendido
em~\cite{goncalves2011desenvolvimento} com a implementação de um filtro adaptativo no \gls{FPGA}. A
expansão é em função da comparação das técnicas utilizadas para embarque na plataforma proposta,
onde são determinados a eficiência da identificação das falhas, a área de ocupação do \gls{FPGA} e o
tempo de execução dos algoritmos.

No trabalho de \cite{hua2013mobile}, é proposta uma arquitetura de manutenção inteligente baseada em
agentes móveis para monitoramento das condições dos equipamentos através de um sistema que imita o
sistema imunológico humano. Esta técnica é conhecida como sistemas imunológicos artificiais e foi
escolhida pelo fato de que muitos processos industriais, como mineração, óleo e gás, estão
geograficamente localizados em áreas remotas e de difícil acesso. Baseado no fato de que alocar
equipes de profissionais para inspeção destes processos demanda tempo e custo elevados, os autores
propõem uma arquitetura descentralizada para automatizar o processo de monitoramento das condições
dos equipamentos. O resultado dos experimentos visam determinar o desempenho do sistema proposto em
termos da acurácia na detecção de falhas e alocação de banda de comunicação. Os autores avaliam que
o sistema proposto não provê somente uma ferramenta viável para detecção de falhas, mas também
flexível e confiável com redução do uso da rede e dos recursos computacionais.


%%
\section{Arquiteturas orientadas a serviços}

\todo[inline]{Introdução da seção.}

Os desafios desta nova abordagem visam prover meios de interconexão entre dispositivos e aplicação.
Neste contexto, vários projetos foram iniciados. O projeto
\glsdisp{SIRENA}{SIRENA~(\glsdesc{SIRENA})}, foi a primeira iniciativa a aplicar os conceitos de
orientação a serviços em dispositivo físicos de chão de fábrica, com foco na implementação
distribuída de arquiteturas \gls{SOA} em ambientes heterogêneos~\cite{jammes2005service}. O projeto
investigou os diferentes domínios de aplicação, como automotivo, residencial e telefônico. A partir
deste projeto, diversos outros utilizaram os conceitos \gls{SOA} para integração entre equipamentos
em infraestruturas orientadas a serviços~\cite{zeeb2007service}.

Um dispositivo que possui recursos para utilização em ambientes \gls{SOA} apresenta ganhos em
interoperabilidade, autonomia além de se tornar uma importante peça na construção da aplicação em
que for baseado~\cite{candido2010industrial}. Dessa forma, a tecnologia empregada não só provê um
padrão de comunicação entre os componentes do sistema, mas também adiciona a capacidade de novas
aplicações serem construídas rapidamente mantendo a agilidade em relação às modificações impostas
por sistemas que apresentam muitas alterações durante o ciclo de vida.

O trabalho de~\cite{ribeiro2008generic} mostra o sucesso encontrado em pesquisas anteriores em
aplicações que trocam informações na forma de serviços, escondendo a complexidade dos componentes e
provendo interfaces limpas de comunicação. No entanto, os autores ilustram cenários onde são
inseridas desvantagens com o uso de \gls{SOA} em sistemas altamente configuráveis, onde a
reconfiguração dos dispositivos deve ser feita periodicamente. De acordo com o artigo, a
periodicidade da reconfiguração dos dispositivos acontece tipicamente em ambientes heterogêneos, o
que acarreta em perda de desempenho do sistema e alto tráfego nas redes de comunicação. Como solução
para este problema, é apresentado um modelo de comunicação genérica que minimiza os problemas
encontrados nestes sistemas.

Também é visto em~\cite{nagorny2013engineering} que o paradigma \gls{SOA} é uma alternativa
promissora na implementação e controle de ambientes ditos sistemas de sistemas. Sistemas de sistemas
são coleções de tarefas ou sistemas dedicados que unem seus recursos ou aptidões para desempenhar
determinada tarefa a fim de possibilitar a criação de um novo sistema que ofereça mais recursos e
possibilidades de desempenhar outras tarefas mais complexas. Como mostra o trabalho, utilizando uma
arquitetura orientada a serviços, é possível a integração de sistemas diferentes do ponto de vista
estrutural e dinâmico, como, por exemplo, sistema de aquecimento urbano e distribuição de energia.
Novamente, o uso e composição dos serviços existentes é feito de forma facilitada, o que contribui
para o sucesso da aplicação.


%%
\section{Uso de arquiteturas orientadas a serviços em sistemas de manutenção inteligente}

Nos últimos anos, a adoção de sistemas baseados em serviços na indústria está em constante
crescimento.

\todo[inline]{Introdução da seção.}

O trabalho de \cite{zhao2010soabased} apresenta o desenvolvimento de uma arquitetura baseada em
\gls{SOA} para monitoramento remoto de condição e diagnóstico de falhas em equipamentos. A
arquitetura proposta é construída sobre o conceito de Web Services, clientes inteligentes e
\gls{XML}. O artigo apresenta uma comparação entre programação orientada a objetos, baseada em
componentes e \gls{SOA}, como diferentes técnicas empregadas para construção de aplicações
distribuídas. Os autores confirmam o sucesso e vantagens no uso de \gls{SOA}, verificando a
possibilidade de integração de diferentes recursos de software, aumentando a disponibilidade e
escalabilidade do sistema e diminuindo o tempo de desenvolvimento.

\cite{arnaiz2010information}

\cite{lee2009informatics}
