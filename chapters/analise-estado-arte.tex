\chapter{Análise do estado da arte}

Os trabalhos relacionados com o tema desta dissertação dizem respeito às áreas de sistemas de
manutenção inteligente e arquiteturas orientadas a serviços. Neste capítulo são analisados os
trabalhos relevantes para as duas áreas distintas, evidenciando os pontos favoráveis de cada um. Por
fim, são apresentados trabalhos convergentes, que relacionam as duas áreas envolvidas. Os trabalhos
mostram que é possível a integração entre \gls{IMS} e \gls{SOA}.


%%
\section{Sistemas de manutenção inteligente}
\label{sec:arte-manutencao-inteligente}

As ações corretivas empregadas em manutenção sempre levam a parada da linha de produção, ocasionando
perdas~\cite{carvajal2011sobre}. Estas perdas podem ser programadas ou não programadas. Em ambos os
casos, existem meios de diminuir os impactos causados por estas intervenções. Ao longo dos anos,
vários trabalhos comprovam o sucesso de técnicas de manutenção inteligente aplicadas em diversos
cenários~\cite{muller2008concept}. Entre eles, o ponto em comum está no fato do crescimento da
importância do uso destas técnicas, tendo em vista a garantia da disponibilidade e segurança do
sistema mantendo a qualidade da linha de produção.

O trabalho de~\cite{lee2006intelligent} introduz os conceitos de \textit{e-maintenance}, emergentes
à época, e os elementos críticos que o compõem. Considerando que o mercado global está mais
competitivo a cada ano, os autores afirmam que o uso de técnicas que minimizem os custos com a
produção é de fundamental importância. Entre os custos citados, estão em destaque os que dizem
respeito à falhas e quebras não programadas de maquinário. Também é sugerida a mudança de paradigma,
da tradicional \emph{falha e conserto} para \emph{predição e manutenção}. Para contornar estes
problemas, o artigo sugere o uso de técnicas computacionais, que, juntamente com o advento da
Internet, possibilitam o monitoramento da condição dos equipamento, ao invés de somente detectar os
equipamento em falha.

As ferramentas utilizadas para o monitoramento da condição dos equipamentos também são discutidas no
artigo de~\cite{lee2006intelligent}. Ferramentas de avaliação da condição e predição de falhas são
analisadas, a fim de se obter um monitoramento contínuo do equipamento. Além disso, são apresentados
os últimos avanços em relação às ferramentas utilizadas para este tipo de cenário e estudos de caso
para validação das técnicas empregadas.

Dando continuidade aos trabalhos na área de manutenção, o artigo de~\cite{muller2008concept} aponta
que a importância da manutenção em sistemas, a fim de aumentar a disponibilidade e segurança, bem
como a qualidade dos produtos, está em vias de crescimento. Como o artigo anterior, o trabalho
também ilustra os conceitos de \textit{e-maintenance}, provendo uma visão geral sobre os estudos e
desafios presentes neste campo de estudo. São apresentados os diferentes sistemas de manutenção
disponíveis, além de uma comparação entre as diferentes características de cada um. Como pontos
importantes, os autores veem grande potencial na adoção de novas tecnologias que auxiliem nas
tarefas de manutenção, os quais são definidos em termos de \emph{dispositivos inteligentes}. Além
disso, defendem a criação de padrões internacionais para integração entre os sistemas discutidos. E,
como contribuição maior, está a proposta da organização das ações no setor, em prol do avanço dos
estudos e definição de novos conceitos, apoiando o surgimento de uma nova disciplina.

No trabalho de~\cite{goncalves2009design}, a detecção de falhas em um atuador eletromecânico é feita
através de um sistema composto por um microprocessador Microblaze sintetizado em um \gls{FPGA}. No
\gls{FPGA}, os sinais são processados e mapas auto-organizáveis são utilizados para detecção,
classificação e predição de falhas. O treinamento dos algoritmos é realizado em um computador
pessoal, enquanto que o monitoramento dos sinais é feito no \gls{FPGA}. O trabalho é estendido
em~\cite{goncalves2011desenvolvimento} com a implementação de um filtro adaptativo no \gls{FPGA}. A
expansão é em função da comparação das técnicas utilizadas para embarque na plataforma proposta,
onde são determinados a eficiência da identificação das falhas, a área de ocupação do \gls{FPGA} e o
tempo de execução dos algoritmos.

O artigo de \cite{hu2012prognostic} utiliza métodos baseados em redes imunológicas artificiais para
predizer a falha e o estado de saúde do sistema de propulsão de um navio. Considerando que, para o
correto funcionamento do navio por grandes períodos de tempo, cada componente precisa estar
condicionado ao correto funcionamento, os autores propõem um sistema de monitoramento para as
diferentes partes do sistema de propulsão. Conforme o artigo, em navios, para aumentar a
confiabilidade do sistema, são empregados muitos equipamentos redundantes, aumentando a complexidade
e dificultando a identificação de uma possível falha. Nessas circunstâncias, uma falha pode
acarretar em catástrofe. Com a proposta dos autores, o sistema de prognóstico ajuda a identificação
das possíveis falhas em meio a complexidade imposta pelo sistema.

A proposta uma arquitetura de manutenção inteligente baseada em agentes móveis para monitoramento
das condições dos equipamentos através de um sistema que imita o sistema imunológico humano também é
apresentada em \cite{hua2013mobile}. A técnica de sistemas imunológicos artificiais foi escolhida
pelo fato de que muitos processos industriais, como mineração, óleo e gás, estão geograficamente
localizados em áreas remotas e de difícil acesso. Baseado no fato de que alocar equipes de
profissionais para inspeção destes processos demanda tempo e custo elevados, os autores propõem uma
arquitetura descentralizada para automatizar o processo de monitoramento das condições dos
equipamentos. O resultado dos experimentos visam determinar o desempenho do sistema proposto em
termos da acurácia na detecção de falhas e alocação de banda de comunicação. Os autores avaliam que
o sistema proposto não provê somente uma ferramenta viável para detecção de falhas, mas também
flexível e confiável com redução do uso da rede e dos recursos computacionais.


%%
\section{Arquiteturas orientadas a serviços}

\Gls{SOA} é uma técnica emergente, padronizada, que possibilita acoplamento mínimo entre componentes
e comunicação distribuída independente de protocolo~\cite{papazoglou2007service}. O crescente uso
desta tecnologia em sistema de informação, nos últimos anos foi expandido para o segmento
industrial. A facilidade de interconexão e os altos níveis de abstração entre dispositivos fazem com
que a tecnologia \gls{SOA} possa ser empregada também em ambientes industriais com certa
facilidade~\cite{moritz2008web}.

Segundo~\cite{cannata2010dynamic}, a fábrica do futuro será conduzida pelo alto uso de arquiteturas
orientadas a serviços. O gerenciamento dos processos de negócio serão fortemente baseados em
informações provenientes do chão de fábrica, tornando as aplicações mais complexas e custosas.
Destas necessidades, surgem os conceitos de colaboração entre os diversos segmentos e componentes da
aplicação.

Os desafios desta nova abordagem visam prover meios de interconexão entre dispositivos e aplicação.
Neste contexto, vários projetos foram iniciados. O projeto
\glsdisp{SIRENA}{SIRENA~(\glsdesc{SIRENA})} foi a primeira iniciativa a aplicar os conceitos de
orientação a serviços em dispositivo físicos de chão de fábrica, com foco na implementação
distribuída de arquiteturas \gls{SOA} em ambientes heterogêneos~\cite{jammes2005service}. O projeto
investigou os diferentes domínios de aplicação, como automotivo, residencial e telefônico. A partir
deste projeto, diversos outros utilizaram os conceitos \gls{SOA} para integração entre equipamentos
em infraestruturas orientadas a serviços~\cite{zeeb2007service}.

Um dispositivo que possui recursos para utilização em ambientes \gls{SOA} apresenta ganhos em
interoperabilidade, autonomia além de se tornar uma importante peça na construção da aplicação em
que for baseado~\cite{candido2010industrial}. Dessa forma, a tecnologia empregada não só fornece um
padrão de comunicação entre os componentes do sistema, mas também adiciona a capacidade de novas
aplicações serem construídas rapidamente, mantendo a agilidade em relação às modificações impostas
por sistemas que apresentam muitas alterações durante o ciclo de vida.

No trabalho de~\cite{pathak2007service-oriented}, é apresentada uma arquitetura baseada em serviços
para o gerenciamento dos equipamentos de um sistema de transmissão de energia elétrica. A
arquitetura possibilita o sensoriamento, integração de informações, avaliação dos riscos e tomada de
decisão relativos à operação de um sistema elétrico de alta tensão. Segundo os autores, o sistema
proposto integra a aquisição de dados em tempo real, modelagem e funcionalidades de previsão.
Juntas, são utilizadas para determinar as políticas de operação, agendar manutenções e garantir o
correto funcionamento dos equipamentos que compõem o sistema de distribuição de energia.

O trabalho de~\cite{ribeiro2008generic} mostra o sucesso encontrado em pesquisas anteriores em
aplicações que trocam informações na forma de serviços, escondendo a complexidade dos componentes e
provendo interfaces limpas de comunicação. No entanto, os autores ilustram cenários onde são
inseridas desvantagens com o uso de \gls{SOA} em sistemas altamente configuráveis. Estes sistemas
apresentam a característica de reconfiguração dos dispositivos periodicamente. De acordo com o
artigo, a periodicidade da reconfiguração dos dispositivos acontece tipicamente em ambientes
heterogêneos, o que acarreta em perda de desempenho do sistema e ocasiona alto tráfego nas redes de
comunicação. Como solução para este problema, é apresentado um modelo de comunicação genérica que
minimiza os problemas encontrados nestes sistemas.

O artigo de~\cite{candido2010soa} aborda a interoperabilidade entre as especificações \gls{OPCUA} e
\gls{DPWS} do padrão \gls{SOA}. Os autores demonstram que, sozinha, nenhuma das duas tecnologias
consegue atender a todos os requisitos impostos a nível de dispositivos em aplicações orientadas a
serviços, mas, se combinadas, apresentam benefícios para o domínio em que são utilizadas. Uma
solução abrangendo as duas tecnologias é proposta e comprovada a convergência das duas tecnologias.

Também é visto em~\cite{nagorny2013engineering} que o paradigma \gls{SOA} é uma alternativa
promissora na implementação e controle de ambientes ditos sistemas de sistemas. Sistemas de sistemas
são coleções de tarefas ou sistemas dedicados que unem seus recursos ou aptidões para desempenhar
determinada tarefa a fim de possibilitar a criação de um novo sistema que ofereça mais recursos e
possibilidades de desempenhar outras tarefas mais complexas. Como mostra o trabalho, utilizando uma
arquitetura orientada a serviços, é possível a integração de sistemas diferentes do ponto de vista
estrutural e dinâmico, como, por exemplo, sistema de aquecimento urbano e distribuição de energia.
Novamente, o uso e composição dos serviços existentes é feito de forma facilitada, o que contribui
para o sucesso da aplicação.


%%
\section{Uso de arquiteturas orientadas a serviços em conjunto com sistemas de manutenção
inteligente}

Os trabalhos anteriores apresentam soluções empregadas nas duas áreas distintas. Como apresentado,
nos últimos anos, a adoção de sistemas baseados em serviços na indústria está em constante
crescimento. Ainda que as pesquisa nestas duas áreas esteja evoluindo, as iniciativas para o uso de
\gls{SOA} na integração de sistemas de manutenção inteligente ainda são escassas.

O trabalho de \cite{zhao2010soabased} apresenta o desenvolvimento de uma arquitetura baseada em
\gls{SOA} para monitoramento remoto de condição e diagnóstico de falhas em equipamentos. A
arquitetura proposta é construída sobre o conceito de Web Services, clientes inteligentes e
\gls{XML}. O artigo apresenta uma comparação entre programação orientada a objetos, baseada em
componentes e \gls{SOA}, como diferentes técnicas empregadas para construção de aplicações
distribuídas. Os autores confirmam o sucesso e vantagens no uso de \gls{SOA}, verificando a
possibilidade de integração de diferentes recursos de software, aumentando a disponibilidade e
escalabilidade do sistema e diminuindo o tempo de desenvolvimento.

No artigo de~\cite{cannata2010dynamic}, são apresentados os aspectos envolvidos em relação aos
conceitos de \textit{e-maintenance} em vistas a cumprir com a construção de dispositivos baseados em
\gls{SOA}. Os autores investigam os benefícios na adoção de arquiteturas orientadas a serviços para
a integração de camadas de negócio com sistemas de manutenção inteligente dispostos no chão de
fábrica. Através do uso destas tecnologias, é notável a facilidade e benefícios encontrados na
integração de sistemas, sugerindo o uso de \gls{SOA} em uma nova geração de sistemas industriais. Em
relação a \textit{e-maintenance}, o monitoramento das condições dos equipamentos de forma facilitada
faz com que o gerenciamento de planos de manutenção possam ser executados de maneira ágil e
eficiente, evitando paradas desnecessárias na linha de produção.
