\documentclass[oneside,diss]{deletex}

\usepackage[latin1]{inputenc}
\usepackage{parskip}
\usepackage{float}
\usepackage{outlines}
\usepackage{listings}
\usepackage{url}
\usepackage{amsmath}
\usepackage{booktabs}
\usepackage{siunitx}
\usepackage{mathptmx}
\usepackage[acronym,nonumberlist,sort=def]{glossaries}
\usepackage{setspace}
\usepackage{caption}
\usepackage[portuguese,shadow,textsize=footnotesize,backgroundcolor=yellow,bordercolor=black!50,
    linecolor=red,disable]{todonotes}
\usepackage{booktabs}
\usepackage{colortbl}
\usepackage{chngcntr}
\usepackage{blindtext}
\usepackage{hyperref}
\usepackage{cleveref}
%\usepackage[]{appendix}

\setlength{\parindent}{0.7cm}
\setlength{\parskip}{1.5pt plus0mm minus0mm}

\newcommand{\crefpairconjunction}{ e }
\newcommand{\crefmiddleconjunction}{, }
\newcommand{\creflastconjunction}{ e }
\newcommand{\crefrangeconjunction}{ a }

\crefname{figure}{figura}{figuras}
\Crefname{figure}{Figura}{Figuras}
\crefname{table}{tabela}{tabelas}
\Crefname{table}{Tabela}{Tabelas}
\crefname{equation}{equação}{equações}
\Crefname{equation}{Equação}{Equações}
\crefname{listing}{listagem}{listagens}
\Crefname{listing}{Listagem}{Listagens}
\crefname{chapter}{capítulo}{capítulos}
\Crefname{chapter}{Capítulo}{Capítulos}
\crefname{section}{seção}{seções}
\Crefname{section}{Seção}{Seções}
\crefname{subsection}{subseção}{subseções}
\Crefname{subsection}{Subseção}{Subseções}

% TODO: Conferir os dados do PDF.
\newcommand{\disstitle}{Proposta de arquitetura orientada a serviços para um sistema de manutenção
    inteligente}
\newcommand{\dissauthor}{Anderson Antônio Giacomolli}
\newcommand{\disssubject}{Proposta e desenvolvimento de uma arquitetura orientada a serviços para
    integração de equipamentos e sistemas de manutenção inteligente}
\newcommand{\disskeywords}{Sistema de Manutenção Inteligente, Arquitetura Orientada a Serviços,
    Integração de Sistemas, Automação Industrial}

%%
% Configuração do pacote para criação de links no PDF.
\hypersetup{%
  pdftitle={\disstitle},
  pdfauthor={\dissauthor},
  pdfproducer={\dissauthor},
  pdfsubject={\disssubject},
  pdfkeywords={\disskeywords},
  bookmarks=true,
  pdfmenubar=true,
  hidelinks=true
}

%%
% Configuração do pacote para representação numérica.
\sisetup{%
  detect-all,
  output-decimal-marker={,},
  inter-unit-product=\ensuremath{{}}
}

%%
% Configuração do pacote para inclusão das listagens.
% Configura o título da lista de listagens.
\renewcommand{\lstlistlistingname}{Listagens}

% Configura a legenda das listagens.
\renewcommand{\lstlistingname}{Listagem}

\makeatletter
\def\l@lstlisting#1#2{\@dottedtocline{1}{0em}{1em}{Listagem #1}{#2}}
\makeatother

% Define um estilo para códigos Java.
\lstdefinestyle{javacodestyle}{
  language=java,
  basicstyle=\ttfamily\small,
  keywordstyle=\color{black}\ttfamily\bfseries,
  backgroundcolor=\color{gray!10},
  captionpos=t,
  %float=!h,
  numberbychapter=false,
  showstringspaces=false,
  %columns=fullflexible
}

% Comando para inserir um bloco de codigo Java com legenda.
\newcommand{\includejavacode}[3]{%
  \lstinputlisting[%
      style=javacodestyle,%
      caption={#2},%
      label={#3}]{%
    #1
  }
}

\newglossary{symbols}{sym}{sbl}{List of symbols}

%%
% Definição do padrão para a lista de abreviaturas.
\newglossarystyle{ppgee}{%
  \setlength{\glsdescwidth}{0.8\linewidth}%
  \renewcommand*{\arraystretch}{1.5}%
  \renewenvironment{theglossary}%
    {\tablehead{}\tabletail{}%
      \begin{supertabular}{lp{\glsdescwidth}}}%
    {\end{supertabular}}%
  \renewcommand*{\glsnamefont}[1]{{\mdseries ##1}}%
  \renewcommand*{\glsgroupskip}{}%
  \renewcommand*{\glossaryheader}{}%
  \renewcommand*{\glspostdescription}{}%
  \renewcommand*{\glsgroupheading}[1]{}%
  \renewcommand*{\glossaryentryfield}[5]{%
    \glsentryitem{##1}\glstarget{##1}{##2} & ##3\glspostdescription\space ##5\\}%
  \renewcommand*{\glossarysubentryfield}[6]{%
    &
    \glssubentryitem{##2}%
    \glstarget{##2}{\strut}##4\glspostdescription\space ##6\\}%
  %\renewcommand*{\glsgroupskip}{ & \\}%
}

%%
% Comando para inserção de uma imagem.
\newcommand{\includefigure}[3]{%
  \begin{figure}[ht!]
    \centering
    \includegraphics{#1}
    \caption{#2}
    \label{#3}
  \end{figure}
}

%%
% Comando para inserção de uma imagem dos softwares.
\newcommand{\includefiguresoft}[3]{%
  \begin{figure}[ht!]
    \centering
    \includegraphics[scale=0.5]{#1}
    \caption{#2}
    \label{#3}
  \end{figure}
}

%%
% Comando para inserção de uma imagem temporária.
\newcommand{\includefiguretmp}[2]{%
  \begin{figure}[h]
    \centering
    \rule{12cm}{6cm}
    \caption{#1}
    \label{#2}
  \end{figure}
}

%%
% Comando para inserção de uma tabela.
\newcommand{\includetable}[3]{%
  \begin{table}[ht!]
    \centering
    \caption{#2}
    \input{#1}
    \label{#3}
  \end{table}
}

%%
% Título e autor do documento.
\title{\disstitle}
\author{Giacomolli}{Anderson Antônio}

%%
% Orientador.
\advisor[Prof.~Dr.]{Pereira}{Carlos Eduardo}
\advisorinfo{UFRGS}{Doutor pela Sttutgart University -- Sttutgart, Alemanha}
\advisorwidth{0.56\textwidth}

%%
% Banca examinadora
\examiner[Prof.~Dr.]{Antônio Barata de Oliveira}{José}
\examinerinfo{UNINOVA}{Doutor pela Universidade Nova de Lisboa -- Lisboa, Portugal}

\examiner[Prof.~Dr.]{Ventura Bayan Henriques}{Renato}
\examinerinfo{UFRGS}{Doutor pela Universidade Federal de Minas Gerais -- Belo Horizonte, Brasil}

\examiner[Prof.~Dr.]{João Brusamarello}{Valner}
\examinerinfo{UFRGS}{Doutor pela Universidade Federal de Santa Catarina -- Florianópolis, Brasil}

%%
% Data da defesa.
\date{março}{2014}

%%
% Área de concentração.
\topic{\ca}

%%
% Palavras-chave.
\keyword{Sistema de Manutenção Inteligente}
\keyword{Arquitetura Orientada a Serviços}
\keyword{Integração de Sistemas}
\keyword{Automação Industrial}

%%
% Cria os glossários.
\makeglossaries

%%
% Inclui o arquivo com a lista de abreviaturas.
\newacronym{ARMA}
  {ARMA}
  {Auto-Regressive Moving-Average}

\newacronym{CSV}
  {CSV}
  {Comma-Separated Values}

\newacronym{CV}
  {CV}
  {Confidence Value}

\newacronym{CWT}
  {CWT}
  {Continuous Wavelet Transform}

\newacronym{DPWS}
  {DPWS}
  {Devices Profile for Web-Services}

\newacronym{DWT}
  {DWT}
  {Discrete Wavelet Transform}

\newacronym{FPGA}
  {FPGA}
  {Field-Programmable Gate Array}

\newacronym{GSM}
  {GSM}
  {Global System for Mobile communication}

\newacronym{HAVi}
  {HAVi}
  {Home Audio/Video Interoperability}

\newacronym{HTTP}
  {HTTP}
  {Hypertext Transfer Protocol}

\newacronym{IEEE}
  {IEEE}
  {Institute of Electrical and Electronics Engineers}

\newacronym{IHM}
  {IHM}
  {Interface Homem-Máquina}

\newacronym{IMS}
  {IMS}
  {Intelligent Maintenance System}

\newacronym{IMSCenter}
  {IMS~Center}
  {Intelligent Maintenance System Center}

\newacronym{IP}
  {IP}
  {Internet Protocol}

\newacronym{IRI}
  {IRI}
  {Internationalized Resource Identifier}

\newacronym{JAR}
  {JAR}
  {Java Archive}

\newacronym{JINI}
  {JINI}
  {Java Intelligent Network Infrastructure}

\newacronym{JMEDS}
  {JMEDS}
  {Java Multi Edition DPWS Stack}

\newacronym{LAN}
  {LAN}
  {Local Area Network}

\newacronym{MTOM}
  {MTOM}
  {SOAP Message Transmission Optimization Mechanism}

\newacronym{OASIS}
  {OASIS}
  {Organization for the Advancement of Structured Information Standards}

\newacronym{OPCUA}
  {OPC~UA}
  {OPC Unified Architecture}

\newacronym{OSACBM}
  {OSA-CBM}
  {Open Systems Architecture for Condition-Based Maintenance}

\newacronym{OSGi}
  {OSGi}
  {Open Service Gateway initiative}

\newacronym{QoS}
  {QoS}
  {Quality of Service}

\newacronym{RL}
  {RL}
  {Regressão Logística}

\newacronym{SIRENA}
  {SIRENA}
  {Service Infrastructure for Real Time Embedded Networked Applications}

\newacronym{SOA}
  {SOA}
  {Service-Oriented Architecture}

\newacronym{SOAP}
  {SOAP}
  {Simple Object Access Protocol}

\newacronym{TCP}
  {TCP}
  {Transmission Control Protocol}

\newacronym{TDMA}
  {TDMA}
  {Time Division Multiple Access}

\newacronym{UDDI}
  {UDDI}
  {Universal Description Discovery Integration}

\newacronym{UDP}
  {UDP}
  {User Datagram Protocol}

\newacronym{UML}
  {UML}
  {Unified Modeling Language}


\newacronym{UPnP}
  {UPnP}
  {Universal Plug and Play}

\newacronym{URI}
  {URI}
  {Uniform Resource Identifier}

\newacronym{URL}
  {URL}
  {Uniform Resource Locator}

\newacronym{WAN}
  {WAN}
  {Wide Area Network}

\newacronym{WPAN}
  {WPAN}
  {Wireless Personal Area Network}

\newacronym{WPE}
  {WPE}
  {Wavelet Packet Energies}

\newacronym{WSA}
  {WSA}
  {Web Service Architecture}

\newacronym{WSD}
  {WSD}
  {Web Service Description}

\newacronym{WSDL}
  {WSDL}
  {Web Service Description Language}

\newacronym{W3C}
  {W3C}
  {World Wide Web Consortium}

\newacronym{XML}
  {XML}
  {Extensible Markup Language}

\newglossaryentry{time-instant}{%
  type = {symbols},
  name = {\ensuremath{t}},
  description = {Instante de tempo}
}

\newglossaryentry{sample-index-discrete}{%
  type = {symbols},
  name = {\ensuremath{n}},
  description = {Índice de amostragem discreto}
}

\newglossaryentry{dilatation-param}{%
  type = {symbols},
  name = {\ensuremath{\beta}},
  description = {Variação da dilatação}
}

\newglossaryentry{scale-param}{%
  type = {symbols},
  name = {\ensuremath{\kappa}},
  description = {Variação da escala}
}

\newglossaryentry{dimension-index}{%
  type = {symbols},
  name = {\ensuremath{k}},
  description = {Dimensão do espaço}
}

\newglossaryentry{logistic-regression-input}{%
  type = {symbols},
  name = {\ensuremath{r}},
  description = {Vetor de entrada do modelo de regressão logística}
}

\newglossaryentry{logistic-regression-output}{%
  type = {symbols},
  name = {\ensuremath{y}},
  description = {Saída do modelo de regressão logística}
}

\newglossaryentry{wavelet-mother}{%
  type = {symbols},
  name = {\ensuremath{\psi}},
  description = {Função wavelet mãe}
}

\newglossaryentry{wavelet-transform-continuous}{%
  type = {symbols},
  name = {\ensuremath{\mathcal{W}}},
  description = {Tranformada wavelet contínua}
}

\newglossaryentry{wavelet-transform-discrete}{%
  type = {symbols},
  name = {\ensuremath{\mathcal{V}}},
  description = {Tranformada wavelet discreta}
}

\newglossaryentry{signal-time-continuous}{%
  type = {symbols},
  name = {\ensuremath{x(t)}},
  description = {Sinal contínuo no domínio tempo}
}

\newglossaryentry{signal-time-discrete}{%
  type = {symbols},
  name = {\ensuremath{x[n]}},
  description = {Sinal discreto no domínio tempo}
}

\newglossaryentry{wavelet-scale-param}{%
  type = symbols,
  name = {\ensuremath{\alpha}},
  description = {Parâmetro de dilatação},
}

\newglossaryentry{wavelet-translation-param}{%
  type = symbols,
  name = {\ensuremath{\tau}},
  description = {Parâmetro de translação}
}

\newglossaryentry{funcao-valor-confianca}{%
  type=symbols,
  name={\ensuremath{{P(y = 1 | x)}}},
  %name=pi,
  %symbol={\ensuremath{\Omega}},
  description=Função probabilidade para o valor de confiança
}

\newglossaryentry{sinal-continuo-frequencia}{%
  type=symbols,
  name={\ensuremath{X(\omega)}},
  %name=pi,
  %symbol={\ensuremath{\Omega}},
  description=Sinal contínuo no domínio frequência
}

\newglossaryentry{sinal-discreto-frequencia}{%
  type=symbols,
  name={\ensuremath{X[n]}},
  %name=pi,
  %symbol={\ensuremath{\Omega}},
  description=Sinal discreto no domínio frequência
}


%%%%%%%%%%%%%%%%%%%%%%%%%%%%%%%%%%%%%%%%%%%%%%%%%%%%%%%%%%%%%%%%%%%%%%%%%%%%%%%%
%%%%%%%%%%%%%%%%%%%%%%%%%%%%%%%%%%%%%%%%%%%%%%%%%%%%%%%%%%%%%%%%%%%%%%%%%%%%%%%%
\begin{document}

\counterwithout{lstlisting}{chapter}

% O comando \maketile gera a capa, a folha de rosto e a folha de aprovacao
% (se for o caso)
% às vezes é necessário redefinir algum comando logo antes de produzir
% a Capa, folha de rosto e folha de aprovacao:
% \renewcommand{\coordname}{Coordenadora do Curso}
\maketitle

%%%%%%%%%%%%%%%%%%%%%%%%%%%%%%%%%%%%%%%%%%%%%%%%%%%%%%%%%%%%%%%%%%%%%%%%%%%%%%%%
%%%%%%%%%%%%%%%%%%%%%%%%%%%%%%%%%%%%%%%%%%%%%%%%%%%%%%%%%%%%%%%%%%%%%%%%%%%%%%%%
%\chapter*{Dedicatória}


%%%%%%%%%%%%%%%%%%%%%%%%%%%%%%%%%%%%%%%%%%%%%%%%%%%%%%%%%%%%%%%%%%%%%%%%%%%%%%%%
%%%%%%%%%%%%%%%%%%%%%%%%%%%%%%%%%%%%%%%%%%%%%%%%%%%%%%%%%%%%%%%%%%%%%%%%%%%%%%%%
%\chapter*{Agradecimentos}


% Resumo no idioma do documento.
%%%%%%%%%%%%%%%%%%%%%%%%%%%%%%%%%%%%%%%%%%%%%%%%%%%%%%%%%%%%%%%%%%%%%%%%%%%%%%%%
%%%%%%%%%%%%%%%%%%%%%%%%%%%%%%%%%%%%%%%%%%%%%%%%%%%%%%%%%%%%%%%%%%%%%%%%%%%%%%%%
\begin{abstract}
  No âmbito industrial, o custo empregado na manutenção de equipamentos ainda representa uma grande
parcela dos investimentos. Dessa forma, o desenvolvimento de técnicas de manutenção, e o seu correto
planejamento, cada vez mais estão assumindo papeis de grande importância nesse setor, visto que
impactam diretamente no fator econômico das empresas. Neste sentido, o trabalho em questão apresenta
a proposta de uma arquitetura orientada a serviços para um sistema de manutenção inteligente, a fim
de possibilitar a integração de forma facilitada entre equipamentos e as ferramentas de análise de
degradação. A arquitetura é composta por diferentes entidades, cada uma responsável por determinada
tarefa para o funcionamento do sistema. No trabalho, as entidades são implementadas e experimentos
são realizados a fim de validar a solução proposta.

\end{abstract}

% Abstract em inglês.
\begin{englishabstract}{Intelligent Maintenance Systems, Service-Oriented Architectures, Systems
    Integration, Industrial Automation}
  In the industrial field, the costs associated with equipment's maintenace still represents a large
portion of the resources available to a company. Therefore, new researches in maintenance systems,
and the correct task planning, are growing, since they impact directly on the economic side of the
companies. Thus, current work presents an service-oriented architecture for maintenance systems
integration. The proposed architecture intends to facilitate the integration of equipments and
degradation analysis tools. The architecture is comprised of several entities, where each one is
responsible for executing a given task in order to keep the system running properly. In this work,
all the entities are implemented and experiments are performed in order to validate the proposed
system.

\end{englishabstract}

% Lista de ilustrações.
\listoffigures

% Lista de tabelas.
\listoftables

% Lista de códigos.
%\lstlistoflistings

% Lista de abreviaturas e siglas.
%\begin{listofabbrv}{OSA-CBM}
%	\item[ABNT] Associação Brasileira de Normas Técnicas
%	\item[GCAR] Grupo de Controle, Automação e Robótica
%	\item[PPGEE] Programa de Pós-Graduação em Engenharia Elétrica
%	\item[OSA-CBM] Programa de Pós-Graduação em Engenharia Elétrica
%\end{listofabbrv}

% Lista de abreviaturas e siglas.
\setglossarysection{chapter}
\printglossary[style=ppgee,type=\acronymtype,title=Lista de abreviaturas]

% lista de símbolos é opcional
%\begin{listofsymbols}{$\alpha\beta\pi\omega$}
%       \item[$\sum$] Somatório
%       \item[$\alpha\beta\pi\omega$] Fator de inconstância do resultado
%\end{listofsymbols}

% Lista de símbolos.
%\setglossarysection{chapter}
\printglossary[style=ppgee,type=symbols,title=Lista de símbolos]
%\printglossaries

% Sumário.
\tableofcontents

%%%%%%%%%%%%%%%%%%%%%%%%%%%%%%%%%%%%%%%%%%%%%%%%%%%%%%%%%%%%%%%%%%%%%%%%%%%%%%%%
%%%%%%%%%%%%%%%%%%%%%%%%%%%%%%%%%%%%%%%%%%%%%%%%%%%%%%%%%%%%%%%%%%%%%%%%%%%%%%%%
\begin{onehalfspace}
  \chapter{Introdução}

\cite{lee2006intelligent}

\Blindtext[5][4]

  \chapter{Conceituação teórica}

\todo[inline]{Introdução do capítulo.}


%%
\section{Sistemas de manutenção inteligente}

Manutenção, no âmbito geral, consiste em uma série de medidas de prevenção, correção e predição de
falhas~\cite{lee2006intelligent}. Durante o uso, equipamentos ou máquinas tendem a deteriorar e
alterar o seu padrão de funcionamento devido a diversos fatores, como, por exemplo, desgaste,
rachaduras, corrosão e sujeira. Nestas condições, a restauração do sistema é de suma importância,
visto que, com o passar do tempo, podem apresentar defeitos e levar à falhas e indisponibilidades.
De acordo com~\cite{marcal2005detectando}, manutenção pode ser definida como todas as atividades
técnicas e organizacionais que garantam a operação das máquinas e equipamentos dentro da
confiabilidade esperada.

Tradicionalmente, são encontrados na literatura três\todo{verificar} tipos de estratégias de
manutenção: corretiva, preventiva e preditiva~\cite{goncalves2011desenvolvimento}. A manutenção
corretiva visa reestabelecer os sistemas danificados; a preventiva tem por objetivo manter os
sistemas em funcionamento, realizando pequenas correções; e a manutenção preditiva tem por base o
monitoramento do estado do sistema, detectando falhas insipientes e fornecendo subsídios para o
planejamento de ações de prevenção ou correção. Em termos gerais, a manutenção corretiva é aplicada
somente quando há falha e o sistema necessita de reposição de peças ou componentes para continuar
operando corretamente, enquanto que a manutenção preventiva visa o agendamento programado de
intervenções no sistema, afim de manter o funcionamento pelo maior tempo possível. Por outro lado, a
manutenção preditiva tem como foco o monitoramento do sistema continuamente e, desta forma, a
intervenção é feita somente quando necessário.

Nas três estratégias, pode-se citar vantagens e desvantagens. Na manutenção corretiva, a principal
vantagem está na dispensabilidade de realização de acompanhamentos ou inspeções no sistema. Isso
evita a geração de custos na alocação de pessoas ou equipamentos para desempenharem tarefas de
verificação do sistema, além da parada da linha de produção em intervalos agendados. Por outro lado,
a parada inesperada da linha de produção para uma manutenção emergencial pode gerar transtornos e
custos não programados. A manutenção preventiva visa sanar os problemas de paradas inesperadas,
utilizando-se de um modelo de agendamento das inspeções. O que para muitas situações é considerado
suficiente, se não for bem planejado, pode acarretar em custos excessivos devido às paradas
programadas e alocação de equipes de manutenção. Nesta situação, a estratégia de manutenção
preditiva busca o meio termo, utilizando-se da predição do estado do sistema, como análise de
tendências ou avaliações probabilísticas do estado de degradação dos equipamentos, para os
agendamentos de novas intervenções.

Mesmo com a programação das intervenções, as máquinas podem falhar de modo repentino, pondo em risco
os equipamentos e pessoas envolvidas com o processo produtivo~\cite{goncalves2011desenvolvimento}. A
falha no intervalo entre intervenções não é possível de prever através dos métodos clássicos de
manutenção. Logo, nos últimos anos, o que tem se visto é a substituição da estratégia de manutenção
preventiva por um novo paradigma: a manutenção proativa~\cite{lee2009informatics}. Esta nova
estratégia visa não somente a predição do estado do sistema, mas também o diagnóstico das falhas e,
em casos onde é aplicado, a intervenção de forma automática. Por intervenção, entende-se que o
padrão de funcionamento dos equipamentos monitorados pode ser alterado, visando minimizar os
possíveis agravantes até a realização da manutenção.

Neste cenário, emergem os sistemas de manutenção inteligente. Também conhecidos como sistemas de
manutenção baseados no conhecimento, visam capturar o conhecimento de um determinado sistema sob a
forma de regras e utilizá-las para construir um novo sistema baseado nestas regras. O novo sistema
é, então, utilizado para realização de um correto diagnóstico ou tomada de ação no caso da
ocorrência de algum defeito. Como exemplo, em~\cite{shikari2004automation} o padrão de vibração de
uma máquina de indução, de um atuador e de uma prensa são analisados e, realizado o diagnóstico
automático através de um sistema de manutenção inteligente, é determinado o motivo da falha, podendo
ser os rolamentos ou desalinhamentos.

\includefigure
    {images/maintenance-strategies}
    {Classificação das estratégias de manutenção.}
    {fig:maintenance-strategies}
\todo{Referenciar figura no texto.}

Com o intuito de auxiliar na migração do paradigma de conserto após falha para o paradigma de
predição e prevenção, foi criado, nos Estados Unidos, um centro de parceria entre universidades e
empresas, denominado \gls{IMSCenter}. Dentre as empresas integrantes da parceria \gls{IMSCenter},
pode-se citar, por exemplo, Boeing, Siemens, AMD, Toyota e Caterpillar. Entre as universidades,
fazem parte do consórcio a de Cincinnatti, Missouri-Rolla e Michigan.


%%%
\subsection{A ferramenta Watchdog Agent}

Um dos objetivos da parceria \gls{IMSCenter} foi o desenvolvimento de uma metodologia para abordagem
dos problemas de manutenção utilizando predição e prevenção. Para tanto, foi desenvolvido um
conjunto de ferramentas de análise denominado Watchdog Agent. Em termos gerais, o Watchdog Agent é
uma ferramenta de análise de desempenho. Aplicado a determinado equipamento, visa analisar sinais de
diversas partes da máquina, a fim de obter um índice de desempenho.

A extração das informações contidas nos sinais analisados são extraídas através das ferramentas
implementadas no Watchdog Agent. Primeiramente os dados dos sensores do equipamento são adquiridos.
Em um segundo momento, os dados são classificados com o auxílio de algoritmos. Com os dados
classificados, é possível determinar o índice de desempenho para a situação analisada. Estas etapas
são ilustradas na \cref{fig:data-processing-plot}.

\includefigure
    {images/data-processing-plot}
    {Processamento das informações utilizando a estratégia proposta pelo IMS~Center.}
    {fig:data-processing-plot}

A medida que o equipamento degrada, o índice de desempenho é alterado em comparação com o mesmo
indicador obtido com o equipamento em funcionamento normal. Um indicador normalmente utilizado para
identificação do estado de um equipamento é o valor de confiança. Este indicador é definido como uma
grandeza que varia no intervalo~${[0; 1]}$. Valores próximos a~\num{1} representam funcionamento
normal do sistema, enquanto que valores próximos a~\num{0} equivalem a um funcionamento em falha. A
\cref{fig:confidence-value-concept} ilustra o conceito de valor de confiança. As duas curvas da
esquerda apresentam o comportamento normal e o comportamento recente de um determinado equipamento.
Ao cruzar as duas informações, é possível obter o valor de confiança, exemplificado no gráfico da
direita. À medida que o valor de confiança decai, a probabilidade de redução do desempenho do
sistema aumenta~\cite{djurdjanovic2003watchdog}.

\includefigure
    {images/confidence-value-concept}
    {Representação do conceito de valor de confiança.}
    {fig:confidence-value-concept}

%\missingfigure{Estrutura do Watchdog Agent.}

%\todo[inline]{Uma das vantagens no uso de técnicas proativas...}

Em comparação com estratégias de manutenção preventiva, um ponto importante a ser citado é aumento
da vida útil de peças de equipamentos~\cite{lazzaretti2012avaliacao}. Peças que poderiam ser
descartadas em função de uma intervenção preventiva\todo{terminar}.


%%%
\subsection{Modelo OSA-CBM}

Como proposta de padronização de uma arquitetura aberta para troca de informações em um sistema
baseado em condição, surge o modelo \gls{OSACBM}~\cite{thurston2001open}. A arquitetura \gls{OSACBM}
visa facilitar a integração e interoperabilidade entre componentes e equipamentos de diferentes
fabricantes. Definida em sete camadas, possibilita a abstração de várias partes envolvidas em um
sistema de manutenção inteligente. A \cref{fig:osa-cbm-model} apresenta uma visão geral das camadas
do modelo juntamente com as suas interações. As camadas são numeradas de 1~(aquisição de dados) a
7~(apresentação).

%\todo[color=yellow, inline]{Outro todo.}

\includefigure
    {images/osa-cbm-model}
    {Modelo OSA-CBM.}
    {fig:osa-cbm-model}

A definição das funcionalidades de cada camada é apresentada por~\cite{thurston2001open}. Na camada
de aquisição de dados, as grandezas físicas são convertidas para sinais elétricos e digitalizadas. O
módulo consiste, normalmente, de um elemento sensor e um elemento de aquisição de dados. Além da
conversão física, a camada também pode armazenar os dados coletados em um banco de dados. A primeira
etapa de cálculos sob os dados obtidos é feita na camada de manipulação dos dados. Através do uso de
ferramentas de processamento de sinais, os dados adquiridos na camada anterior são manipulados,
podendo gerar resultados no domínios tempo, frequência ou tempo-frequência. Eventualmente, os
resultados das operações também podem ser armazenados em um banco de dados. O módulo de detecção do
estado do sistema analisa continuamente os indicadores de cada sistema, subsistema ou componente. De
posse dos dados processados pelas camadas anteriores, ao calcular os indicadores de estado, o módulo
de detecção pode gerar alarmes respeitando condições previamente estabelecidas. Novamente os dados
obtidos podem ser armazenados para uso posterior. Na camada de avaliação da saúde do sistema, o
resultado dos indicadores, obtidos no módulo de detecção do estado do sistema, são inseridos no
contexto das operações. A saúde do sistema monitorado é avaliada pelo uso dos indicadores atuais e
passados. Dessa forma, também é possível armazenar os resultados formando um histórico do
equipamento monitorado. Na camada de prognóstico, a saúde futura do sistema é estimada. Através de
um modelo estimado do sistema e dos dados obtidos nas camadas anteriores, o tempo de vida útil ou a
probabilidade de falha em um horizonte de predição são estimados. Como nas camadas anteriores, os
resultados podem ser armazenados em um banco de dados. O módulo de tomada de decisão utiliza os
dados obtidos na camada de prognóstico, além de outras informações, para sugerir ações recomendadas
de acordo com as implicações das decisões. São integrados, juntamente com os dados da camada de
prognóstico, informações de restrições externas, requisitos de funcionalidades do equipamento ou
sistema, condições financeiras, entre outros. No nível mais alto do modelo, está a camada de
aplicação. Definida como a interface homem-máquina do sistema, visa a apresentação dos dados obtidos
no processamento das informações. Nesta camada também podem ser utilizadas técnicas de realidade
aumentada~\cite{espindola2011realidade}.


%%%
\subsection{Algoritmos de processamento da ferramenta Watchdog Agent}

Como mencionado anteriormente, o Watchdog Agent é um conjunto de algoritmos para processamento de
sinais e extração de características\todo{Terminar}.


%%%%
\subsubsection{Energias da transformada Wavelet Packet}

Uma das forma de se analisar sinais não estacionários no domínio tempo-frequência é utilizando a
transformada wavelet~\cite{antonini1992image}. Seu uso é indicado para sinais que apresentam
descontinuidades, tendências entre outros. É empregado nas mais diversas aplicações, desde a remoção
de ruídos em sinais ou imagens até a compressão de imagens médicas com pouca perda de qualidade.

Wavelets são formas de onda oscilantes com duração limitada e valor médio zero. São empregadas na
forma de wavelets mãe, definidas por \gls{wavelet-mother}. A função wavelet mãe pode ser dilatada ou
comprimida através de um parâmetro \gls{wavelet-scale-param} e transladada através de um
parâmetro \gls{wavelet-translation-param}. A mudança de escala e translação são apresentadas na
\cref{eq:wavelet-translation-compression}.

\begin{equation}
  \psi_{\alpha, \tau}(t) = \frac{1}{\sqrt{\alpha}} \psi \left ( \frac{t - \tau}{\alpha} \right )
  \label{eq:wavelet-translation-compression}
\end{equation}

A \cref{eq:wavelet-definition-continuous} apresenta a definição da transformada wavelet contínua
\gls{wavelet-transform-continuous} de um sinal contínuo \gls{signal-time-continuous}.

\begin{equation}
  \mathcal{W} \left \{ x, \psi \right \} =
      \frac{1}{\sqrt{\alpha}} \left \{ x(t), \psi_{\alpha, \tau}(t) \right \} =
      \frac{1}{\sqrt{\alpha}} \int_{-\infty}^{\infty} x(t) \cdot
        \psi^{*} \left ( \frac{t - \tau}{\alpha} \right ) \textrm{d}t
  \label{eq:wavelet-definition-continuous}
\end{equation}

A largura da wavelet é influenciada pelo fator de escala \gls{wavelet-scale-param}, o que também
contribui para a alteração da resolução empregada na análise. Quanto menor o valor de
\gls{wavelet-scale-param}, maior será a resolução empregada na detecção de eventos de alta
frequência. No caso contrário, quanto maior for o valor de \gls{wavelet-scale-param}, maior será a
dilatação empregada na wavelet mãe, o que é conveniente para a identificação de padrões de baixa
frequência. A \cref{fig:wavelet-time-frequency-representation} ilustra a mudança de escala para
análise do sinal em multiresolução.

\includefiguretmp
  %{images/wavelet-time-frequency-representation}
  {Representação da resolução tempo frequência da transformada wavelet.}
  {fig:wavelet-time-frequency-representation}

Como alternativa para a utilização da transformada para sinais discretos, é possível a utilização da
transformada wavelet discreta. Como a transformada wavelet contínua requer um esforço computacional
considerado exagerado para calcular os coeficientes de todas as possíveis escalas da transformada,
gerando informações redundantes, é possível a utilização de parâmetros de escalonamento e translação
discretos~\cite{mallat1989theory}. A transformada wavelet discreta \gls{wavelet-transform-discrete}
de um sinal contínuo \gls{signal-time-continuous} é apresentada na
\cref{eq:wavelet-definition-discrete}.

\begin{equation}
  \mathcal{V} \left \{ x, \psi \right \} =
      \left \{ x(t), \psi_{\kappa \beta} \right \} =
      \int_{-\infty}^{\infty} x(t) \cdot
        \psi_{\kappa \beta}(t) \textrm{d}t
  \label{eq:wavelet-definition-discrete}
\end{equation}

A função \gls{wavelet-mother}{\ensuremath{_{\kappa \beta}}} é a wavelet mãe criada a partir de
parâmetros de escala e translação discretos. A \cref{eq:wavelet-translation-compression-discrete}
apresenta a obtenção da função \gls{wavelet-mother}{\ensuremath{_{\kappa \beta}}}, onde \gls
{wavelet-scale-param} é a variação da escala e \gls{wavelet-translation-param} indica a translação;
\gls{dilatation-param} e \gls{scale-param} são constantes discretas que indicam, respectivamente, a
variação da escala e dilatação.

\begin{equation}
  \psi_{\kappa \beta}(t) =
  \frac{1}{\sqrt{\alpha^{\kappa}}} \psi \left (
    \frac{t - \beta \alpha^{\kappa} \tau}{\alpha^{\kappa}} \right )
  \label{eq:wavelet-translation-compression-discrete}
\end{equation}

Em termos gerais, a transformada wavelet é realizada através de um processo de filtragem de vários
níveis, onde cada nível apresenta um filtro em quadratura. O sinal decomposto em cada um dos níveis
apresenta duas informações, definidas como \emph{detalhe} (alta frequência) e \emph{aproximação}
(baixa frequência). A decomposição em vários níveis origina uma árvore, denominada de árvore de
decomposição wavelet packet~\cite{mallat1989theory}. Este processo é ilustrado na
\cref{fig:wavelet-decompose-representation}.

\includefiguretmp
  %{images/wavelet-decompose-representation}
  {Representação de dois níveis da árvore de decomposição wavelet packet.}
  {fig:wavelet-decompose-representation}

Como resultado do processo da árvore de decomposição, é gerado um vetor de elementos obtido através
do cálculo da energia dos coeficientes no nível mais baixo da decomposição da transformada. Este
vetor de características é denominado de energias da transformada wavelet
packet~\cite{ims2007documentation}.


%%%%
\subsubsection{Regressão logística}

A regressão logística é um método de classificação que permite uma classificação binária ou
dicotômica de um conjunto de dados~\cite{hosmer2013applied}. O método integra a categoria de modelos
chamados de Modelos Generalizados Lineares, e, portanto, como resultado da análise, é gerada uma
resposta de dois estados, que podem ser traduzidos, por exemplo, para sucesso ou falha ou
comportamento normal ou degradado.

O processo empregado pelo método da regressão logística é definido como a tentativa de ajustar o
espaço de \gls{dimension-index} dimensões da entrada para um espaço de saída de apenas uma dimensão.
A variável de saída ou resposta é representada por \gls{logistic-regression-output}, sendo que
\gls{logistic-regression-output}{\ensuremath{~= 1}} quando o conjunto de entrada possui a
característica de interesse e \gls{logistic-regression-output}{\ensuremath{~= 0}} quando não
possui~\cite{hosmer2013applied}. A \cref{fig:logistic-regression-representation} apresenta uma
curva típica da saída de um modelo de regressão logística.

\includefiguretmp
  %{images/logistic-regression-representation}
  {Representação de um modelo de regressão logística.}
  {fig:logistic-regression-representation}

A representação matemática do modelo de regressão logística é apresentada na
\cref{eq:logistic-regression-model}~\cite{ims2007documentation}. No modelo,
\gls{logistic-regression-input} é o vetor de entrada de \gls{dimension-index} dimensões e
\gls{logistic-regression-output} é a saída binária.

\begin{equation}
  p(r) =
  P(y = 1 | r) =
  \frac{1}{1 + e^{- \left (
    \alpha + \beta_{1} r_{1} + \beta_{2} r_{2} + \cdots + \beta_{k} r_{k} \right )}} =
  \frac{e^{\alpha + \beta_{1} r_{1} + \beta_{2} r_{2} + \cdots + \beta_{k} r_{k}}}
    {1 + e^{\alpha + \beta_{1} r_{1} + \beta_{2} r_{2} + \cdots + \beta_{k} r_{k}}}
  \label{eq:logistic-regression-model}
\end{equation}

O modelo apresentado na \cref{eq:logistic-regression-model} também pode ser representado em termos
das probabilidade de evento e de não evento. Neste caso, são definidos como $p(r)$ e $1 - p(r)$. A
\cref{eq:logistic-regression-model-linear} apresenta a nova função, onde o termo contendo o
logaritmo natural é conhecido como função \emph{logit}, cujo propósito é tornar a função linear.

\begin{equation}
  g(r) =
  \ln \left ( \frac{p(r)}{1 - p(r)} \right ) =
  \alpha + \beta_{1} r_{1} + \beta_{2} r_{2} + \cdots + \beta_{k} + r_{k}
  \label{eq:logistic-regression-model-linear}
\end{equation}

A partir da \cref{eq:logistic-regression-model} o valor de confiança é obtido. Para um comportamento
normal, o valor de confiança assume valores próximos a \num{1}, enquanto que para comportamento
ditos de falha, o valor de confiança fica concentrado próximo a \num{0}.


%%
\section{Arquiteturas orientadas a serviços}

As técnicas para desenvolvimento de aplicações \gls{SOA} representam uma mudança de paradigma na
engenharia de software, onde os componentes são definidos como serviços~\cite{ramollari2007survey}.
Originalmente desenvolvido e utilizado para integração de sistemas no meio gerencial e corporativo,
logo teve aceitação entre diversos segmentos, como plataformas de negócio, telecomunicações,
transportes e na automação industrial.

O termo \gls{SOA} ainda possui uma definição concisa e única, diferindo conforme os conhecimentos
técnicos e a bagagem acumulada durante o desenvolvimento de diferentes aplicações por parte dos
autores~\cite{candido2013soa}. Ainda segundo \cite{candido2013soa}, a definição que mais se encaixa
no contexto de um trabalho que envolve integração em meio industrial é a
de~\cite{jammes2005service}, o qual expressa que "\gls{SOA} é um conjunto de princípios ou doutrinas
para a construção de sistemas interoperáveis e também autônomos". Esta percepção também descreve o
contexto deste trabalho, no qual os componentes envolvidos podem ser considerados peças
independentes do sistema, no entanto podem vir a representar um conjunto interoperável de entidades,
compartilhando recursos entre si.

Em termos gerais, a característica principal de uma arquitetura \gls{SOA} é a criação e
disponibilidade de serviços, que, quando agrupados, constituem um sistema
funcional~\cite{josuttis2009soa}. O termo serviço se refere a uma funcionalidade ou lógica que é
encapsulada e oferecida ao sistema através de uma interface. Dessa forma, outros serviços, entidades
ou programas podem obter o modo de acesso à esta funcionalidade e empregá-la na resolução de
determinada tarefa.


%%%
\subsection{Componentes de uma arquitetura orientada a serviços}

Em se tratando do contexto da aplicação, os serviços oferecidos, para que possam ser utilizados,
precisam ser encontrados ou expostos~\cite{papazoglou2007service}. Mesmo que, segundo a definição
adotada, serviços possam ser utilizados independentemente, a abordagem de utilizar um conjunto de
serviços trabalhando de forma cooperativa na resolução de um problema, é muito mais interessante.
Portanto, uma aplicação \gls{SOA} deve prover meios para que os serviços possam ser comunicar e
trocar informações via mensagens padronizadas. Dessa forma, é necessário que existam alguns
conceitos a serem cumpridos por todos os componentes de uma arquitetura
\gls{SOA}~\cite{erl2005service}:

\begin{itemize}
  \item \emph{Acoplamento mínimo}: serviços devem minimizar a dependência, armazenando somente as
  informações de outros serviços.

  \item \emph{Contrato de serviço}: devem utilizar um padrão de comunicação comum previamente
  definido e baseado em padrões abertos.

  \item \emph{Autonomia}: possibilidade de controle total da lógica que o serviço encapsula.

  \item \emph{Abstração}: possibilidade de esconder a lógica e os recursos utilizados pelo serviço
  do resto da aplicação.

  \item \emph{Reusabilidade}: utilização da mesma funcionalidade por diferentes partes do sistema ou
  em aplicações futuras, sem a necessidade de uma nova implementação.

  \item \emph{Composição}: organização de serviços para a construção de tarefas mais complexas podem
  ser feitas através da composição de serviços mais simples.

  \item \emph{Sem dependência de estado}: os serviços não devem reter nenhuma informação específica
  sobre as atividade executadas.

  \item \emph{Possibilidade de descoberta}: os serviços devem possibilitar a sua descoberta pelos
  mecanismos de busca.
\end{itemize}

As aplicações \gls{SOA} normalmente são construídas baseadas nos princípios de serviços
web~\cite{josuttis2009soa}. Não necessariamente as aplicações \gls{SOA} necessitam ser baseadas em
serviços web, porém, esta tecnologia começou a ser largamente utilizada devido, entre outros
fatores, a adoção de uma padronização. Por parte dos desenvolvedores, o encapsulamento das
funcionalidades que um serviço pode oferecer foi facilitado após a definição de interfaces e
protocolos de comunicação padrão. Dessa forma, é possível aos clientes acessar os serviços de forma
transparente, sem conhecimento prévio de detalhes de implementação.

Seguindo as convenções adotadas para serviços web, um serviço é uma entidade de software
identificada por uma \gls{URI}~\cite{bell2008service}. A \gls{URI} define o endereço do serviço na
aplicação, devendo ser única. O identificador permite a discriminação entre grupos de entidades,
através da utilização de um separador. Esta técnica é largamente utilizada em sistemas que usufruem
de identificadores baseados em \gls{URI}.

A troca de informações entre os serviços normalmente ocorre utilizando o protocolo \gls{SOAP}, onde
as mensagens são codificadas no formato \gls{XML}~\cite{josuttis2009soa}. O protocolo \gls{SOAP}
provê uma infraestrutura básica para a troca de mensagens entre serviços web. É definido por um
envelope, um conjunto de regras que definem os tipos de dados suportados e um meio de representar os
procedimentos ou funcionalidades disponíveis para execução. Por ser baseado em \gls{XML}, o
protocolo pode ser utilizado sobre diferentes protocolos de transporte, como, por exemplo, o
\gls{HTTP}.

A descrição das interfaces de cada serviço é normalmente definida por \gls{WSDL}. Documentos
\gls{WSDL} também são baseados em \gls{XML} e contém toda a informação necessária para a utilização
do serviço em questão. No documento, são especificadas todas as operações que o serviço possibilita,
bem como os tipos de dados suportados. Também é possível estender o documento definindo novos tipos
de dados para troca de mensagens.

Os serviços disponíveis na aplicação e a descrição de suas funcionalidades estão centralizadas no
\gls{UDDI}. O padrão \gls{UDDI} define o protocolo para os serviços de diretório, ou intermediadores
de serviço, onde são armazenadas todas as informações de cada um dos serviços da aplicação. Esta
entidade é utilizada para informar aos clientes quais serviços estão disponíveis, possibilitando
meios de descobri-los e obter seus metadados. A interoperabilidade entre os componentes de uma
aplicação \gls{SOA} é ilustrada na \cref{fig:soa-elements}.

\includefigure
  {images/soa-elements}
  {Interoperabilidade entre os elementos de uma aplicação SOA.}
  {fig:soa-elements}


%%%
\subsection{Device Profile for Web Services}

Especificações para web services normalmente são grafadas com o prefixo
"{WS-}"~\cite{candido2013soa}. É comum encontrar na literatura o termo "{WS-*}", o qual se refere ao
agrupamento de diferentes especificações para web services. Dentre as especificações, o \gls{DPWS}
define um conjunto mínimo de implementações que permitem a troca de mensagens, descoberta,
descrição, geração de eventos e autenticação para a utilização de web services em clientes com
recursos computacionais limitados. O \gls{DPWS} permite a integração destes clientes com outros, com
recursos mais flexíveis.

O \gls{DPWS} implementa um conjunto restrito dos padrões {WS-*}. A \cref{fig:dpws-stack} ilustra o
diagrama contendo a pilha de protocolos suportados pelo \gls{DPWS}. Dentre os padrões, é possível
destacar o WS-Adressing, utilizado para transferência de mensagens, WS-Security\todo{alterar a
figura}, para suprir as necessidades de autenticação nos web services, WS-Discovery, que possibilita
a descoberta de serviços em uma rede local, e WS-Eventing, que permite a utilização de eventos.

\includefigure
  {images/dpws-stack}
  {Pilha de protocolos suportados pelo DPWS.}
  {fig:dpws-stack}

  \chapter{Análise do estado da arte}

  %\chapter{Arquiteturas orientadas a serviços}

  \chapter{Arquitetura proposta}
\label{cha:arquitetura-proposta}

Conforme visto no capítulo anterior, é possível a integração de um sistema de manutenção inteligente
utilizando uma arquitetura orientada a serviços. Dessa forma, este capítulo apresenta a proposta de
uma arquitetura para um sistema de manutenção inteligente. São apresentadas as entidades principais
e a interligação entre elas através da utilização do padrão \gls{SOA}. As entidades que compõem a
arquitetura são descritas e inseridas no contexto de uma aplicação através de casos de uso. Através
dos casos de uso, são abordadas diferentes situações onde as entidades da arquitetura se encaixam
para resolver algum dos problemas.

Juntamente com a arquitetura proposta, é apresentada a programação dos experimentos e o estudo de
caso. O experimentos visam validar o sistema proposto, bem como todas as entidades aqui descritas.
Para validação, o estudo de caso, envolvendo um conjunto atuador elétrico e válvula, é proposto e
hipóteses sobre são levantadas.


%%
\section{Arquitetura orientada a serviços proposta}
\label{sec:arquitetura-proposta}

A arquitetura proposta neste trabalho tem por objetivo a integração de sistemas de manutenção
inteligente, dos diversos equipamentos que precisam ser monitorados, além de outras entidades que
auxiliam no funcionamento do sistema. A troca de informações entre todos os elementos que compõem a
arquitetura é abstraída na forma de serviços, o que facilita a integração, inserção e remoção de
novas entidades no sistema. Dessa forma, a especificação da arquitetura é definida utilizando os
padrões \gls{SOA}. Uma visão geral das entidades que fazem parte da arquitetura proposta é
apresentada na \cref{fig:soa-proposed-architecture}. Na figura, nota-se que cada entidade tem uma
função específica no sistema e o acesso às suas funcionalidades é realizado através de serviços. A
seguir, para melhor entendimento, todos os elementos serão detalhados.

\includefigure
  {images/soa-proposed-architecture}
  {Arquitetura orientada a serviços proposta para integração de um sistema de manutenção
      inteligente.}
  {fig:soa-proposed-architecture}


%%%
\subsection{Serviço}

No contexto da arquitetura proposta neste trabalho, serviço é um componente de software que
encapsula uma funcionalidade acessível através dos padrões definidos pela tecnologia \gls{SOA}.
Todos os componentes da arquitetura expõem as suas funcionalidades na forma de serviços,
possibilitando que a interação entre eles seja feita de forma transparente. Como parte do padrão
\gls{SOA}, serviços podem ser descobertos e utilizados por clientes que desejam executar uma
determinada tarefa. Neste contexto, também é possível a definição de serviços mais especializados
com base em outros serviços, praticando a técnica da composição de serviços. Estas características
se tornam inerentes à proposta, devido a utilização do padrão \gls{SOA}.


%%%
\subsection{Dispositivo}

Um dispositivo é um componente de software utilizado para encapsular um elemento físico da aplicação
proposta. Por ser executado em um dispositivo físico, o componente é denominado dispositivo lógico e
disponibiliza serviços para acesso à funcionalidades previamente definidas. As funcionalidades podem
ser relativas ao dispositivo físico em que o componente está executando ou outras que auxiliam em
alguma tarefa específica não relacionada diretamente com o hardware hospedeiro.

Do ponto de vista da aplicação, os dispositivos são entidades que hospedam serviços. Dentre os
serviços hospedados, alguns estão presentes em todos os dispositivos da arquitetura, servindo de
base para a comunicação entre todos os elementos desta classe. Dessa forma, conhecendo os serviços
básicos, uma interface mínima para troca de informações entre as entidades do sistema é definida,
facilitando a inserção de novos dispositivos.

Além dos serviços base, outros podem ser executados no dispositivo. A arquitetura permite o envio de
novos serviços para os dispositivos do sistema. O dispositivo recebe o novo serviço, que possui as
mesmas funcionalidades de um serviço padrão, e o carrega para ser executado normalmente.


%%%
%\subsection{Aplicação orientada a serviços}
%
%A aplicação orientada a serviços nada mais é do que o resultado da utilização dos diversos serviços
%propostos na arquitetura. Através da composição ou utilização direta dos serviços de uma forma
%coordenada, a aplicação é construída. De um modo minimalista, a aplicação pode ser vista como a
%interação progressiva de funcionalidades simples provenientes de serviços básicos, resultando em um
%componente complexo de software. Visto que o produto final da composição dos elementos do sistema
%não é definido \textit{a priori}, o resultado da aplicação pode ser qualquer construção ou
%interação entre os componentes.


%%%
\subsection{Gerenciador de Dispositivos}
\label{sub:proposta-gerenciador-dispositivos}

Como o nome sugere, o Gerenciador de Dispositivos é um componente de software utilizado para
gerenciar os dispositivos da arquitetura proposta. A configuração ou obtenção da lista dos
dispositivos na aplicação orientada a serviços são exemplos de funcionalidades do gerenciador. Como
os dispositivos são entidades que mapeiam funcionalidades dos dispositivos físicos encontrados em um
sistema de manutenção inteligente, o gerenciador pode, por exemplo, obter a lista de sensores do
equipamento e configurar alguns parâmetros, como a taxa de atualização dos dados dos sensores.
Também é possível definir comportamentos adicionais para o caso do equipamento estar operando em
diferentes níveis de degradação. Os diferentes comportamentos podem ser utilizados quando há
necessidade de operar o equipamento em condições onde manter os níveis de degradação estáveis é mais
importante do que a operação a pleno. Portanto, pode-se manter o equipamento funcionando até que uma
manutenção \textit{in loco} possa ser realizada.

Grupos de dispositivos podem ser criados e gerenciados. O Gerenciador de Dispositivos permite a
criação de grupos, a fim de facilitar a alteração simultânea de vários dispositivos. Supondo que a
aplicação é inerentemente escalável, o número de dispositivos tende a aumentar, o que dificulta o
gerenciamento ou configuração de dispositivos similares. Considerando que as configurações aplicadas
a uma mesma classe de dispositivos é muito parecida, o gerenciador permite que um grupo receba os
mesmos parâmetros, automatizando o processo de customização dos dispositivos.

Outra funcionalidade que o gerenciador provê é a obtenção da topologia dos componentes do sistema.
Em uma arquitetura \gls{SOA} todos os serviços estão localizados no mesmo nível hierárquico. Essa
característica pode ser considerada positiva do ponto de vista de integração e reuso de serviços. Em
contrapartida, ao definir níveis hierárquicos entre componentes de um mesmo dispositivo físico, a
complexidade para o estabelecimento de uma hierarquia lógica para representação das diferentes
partes desse dispositivo aumenta. Contudo, ao aplicar um identificador único para cada dispositivo,
ou parte de dispositivo, do sistema, torna-se possível definir e estabelecer uma relação
hierárquica. O identificador pode ser uma \gls{URI}, por exemplo, onde cada parte do endereço se
refere a um dos níveis hierárquicos.

O Gerenciador de Dispositivos também é utilizado na resolução de problemas encontrados durante a
execução dos componentes do sistema. A entidade se enquadra na categoria de \gls{IHM},
possibilitando que a rede seja mapeada em busca de dispositivos ou serviços de forma interativa.
Além disso, é possível obter o estado dos dispositivo, visualizar dados ou testar os serviços
encontrados.


%%%
\subsection{Gerenciador de Análises}
\label{sub:proposta-gerenciador-analises}

O Gerenciador de Análises é utilizado para definir o plano que será aplicado a determinado
equipamento, ou grupo de equipamentos, tendo em vista obter os níveis de degradação. No contexto
deste trabalho, um plano de análise é uma estrutura que armazena as informações necessárias para que
uma análise possa ser executada corretamente. Cada plano armazena as ferramentas que serão
utilizadas para análise de dados, bem como a ordem em que serão executadas. O gerenciador permite o
agendamento dos planos e a execução periódica em intervalos de tempo. Os planos também podem ser
gerenciados, possibilitando alteração e/ou exclusão.

No plano são definidos os comportamentos a serem adotados com base nos valores de degradação do
equipamento. O gerenciador obtém os comportamentos previamente definidos para o equipamento a ser
analisado, os quais podem ser mapeados pelo usuário para diferentes níveis de degradação. Dessa
forma, o equipamento pode ser adaptado a diferentes condições, por exemplo, evitando o aumento da
degradação até que uma manutenção possa ser realizada.

No geral, informações como nome, descrição e dispositivos afetados fazem parte do plano de análise.
Estas informações permitem ao operador do sistema identificar os planos criados e onde estão
empregados. Como o plano armazena os dispositivos que serão analisados, também é possível
identificar a origem dos dados utilizados. Dessa forma, pode-se mapear os sensores dos equipamentos
como fonte de dados para alimentar as ferramentas de análise. Portanto, a origem dos dados
utilizados também está contida no plano. Outra informação que faz parte desta estrutura é referente
às ferramentas utilizadas na análise. Além delas, o plano armazena a ordem em que serão utilizadas,
a qual é definida pelo operador do sistema.


O Gerenciador de Análises também coordena as análises realizadas em um grupo de equipamentos
similares. Em alguns casos, o processo de obtenção dos valores de confiança de um conjunto de
equipamentos pode ser simplificado pelo fato de que, se são similares, o nível de degradação de um
equipamento pode ser aproximado para o nível de degradação do grupo. Dessa maneira, é possível
elaborar um plano de análise que seja aplicável a vários equipamentos.


%%%
\subsection{Analisador de Dispositivos}
\label{sub:proposta-analisador-dispositivos}

O Analisador de Dispositivos concentra todas as ferramentas necessárias para a realização da análise
dos dados de determinado dispositivo. A entidade opera com base nos planos criados e agendados pelo
operador do sistema no Gerenciador de Análises. De posse do plano, o analisador executa as análises
utilizando os algoritmos necessários para obtenção do resultado esperado. A execução de um mesmo
plano é feita de forma automática.

O analisador permite verificar quais planos estão sendo executados ou na fila para execução. Sempre
que um novo plano é gerado, o analisador o coloca na fila de execução, a fim de que seja finalizado
o mais breve possível. A entidade permite o monitoramento destes planos pelo usuário, o qual pode
verificar se a execução está de acordo com o esperado.


%%%
\subsection{Base de dados}
\label{sub:proposta-base-dados}

A entidade Base de Dados é utilizada como uma interface para um banco de dados. O software encapsula
funcionalidades para  armazenar todos os dados relativos à aplicação. Desde os dados obtidos nas
análises de degradação dos equipamentos como também os planos de análise agendados pelo usuário. A
base centraliza as informações, mas não necessariamente faz com que a aplicação seja centralizada.
Nada impede a utilização de diferentes bases de dados e o compartilhamento de informações entre as
entidade do sistema. Por exemplo, o armazenamento dos dados de treinamento de um equipamento,
necessário para obtenção do nível de degradação, pode ser feito em diferentes bases de dados. O
Analisador de Dispositivos pode combinar estas informações, a fim de melhorar o resultado esperado.

A Base de Dados é defina como um componente \gls{SOA}, portanto o acesso aos dados é feito através
de serviços. A inclusão, deleção e modificação de dados são realizadas por serviços especializados.
Um dispositivo remoto pode acessar a base de dados a fim de obter o histórico dos últimos valores de
confiança calculados ou valores de testes intermediários para determinado equipamento, utilizando-os
para um novo tipo de análise.


%%%
\subsection{Repositório de serviços}

O repositório de serviços é uma entidade que armazena os dispositivos lógicos e seus serviços, os
quais podem ser obtidos dinamicamente e implantados em dispositivos físicos. Esta entidade pode
estar localizada na rede local ou remota, com acesso disponível para várias aplicações. Dessa forma,
uma nova aplicação, isolada de outra já existente, pode ser construída com o reúso de componentes
obtidos de um repositório de serviços compartilhado pelas duas. Esta abordagem também permite o
compartilhamento de uma base de dados de componentes lógicos entre aplicações.


%%
\section{Casos de uso para a arquitetura proposta}

Os elementos que compõem a arquitetura proposta encaixam-se em alguns casos de uso no contexto de um
sistema de manutenção inteligente. Os casos de uso propostos a seguir exemplificam a interação do
operador do sistema na busca ou configuração dos dispositivos e na criação dos planos de análise
para cada dispositivo. Para a maioria dos casos de uso, o operador do sistema é considerado como
ator. Além disso a ferramenta de análise de dados é utilizada para analisar os dados obtidos dos
equipamentos monitorados. Visto que as análises executam autonomamente durante o processo de
amostragem dos equipamentos, uma tarefa periódica é vista como ator do caso de uso.


%%%
\subsection{Descoberta e configuração de dispositivos}
\label{sub:proposta-descoberta-configuracao-dispositivos}

A descoberta dos dispositivos e serviços na rede pode ser feita a qualquer momento utilizando o
Gerenciador de Dispositivos. Cada um dos dispositivos encontrados é identificado e colocado em uma
lista de dispositivos ativos. Também são identificados os serviços hospedados em cada um dos
dispositivos. Dessa forma, é possível determinar quais dispositivos estão atualmente disponíveis na
rede e quais possuem serviços que poderão ser utilizados para suprir e executar determinada
funcionalidade do sistema.

A geração da hierarquia de recursos encontrados na rede também é possível. Ao aplicar um
identificador único para cada recurso, a reconstrução dos componentes do sistema pode ser obtida
agrupando-os em classes ou categorias. O identificador é utilizado para auxiliar a definir uma
topologia dos subdispositivos encontrados em um dispositivo lógico, facilitando a visualização e
configuração dos componentes.

O diagrama \gls{UML} de casos de uso para descoberta e configuração de dispositivos é apresentado na
\cref{fig:uml-discovery-setup-devices}. São ilustradas as interações que o operador do sistema pode
realizar nos dispositivos utilizando o Gerenciador de Dispositivos. Em vias gerais, o gerenciador
será utilizado pelo operador do sistema para descobrir os dispositivos da rede, obter o estado ou
realizar modificações nestes elementos.

\includefigure
  {images/uml-discovery-setup-devices}
  {Diagrama de casos de uso para descoberta e configuração de dispositivos.}
  {fig:uml-discovery-setup-devices}

Ao utilizar a descoberta de dispositivos, o gerenciador analisa todos dispositivos da aplicação
orientada a serviços. Os que forem encontrados são retornados, permitindo ao operador do sistema
verificar se a inclusão de um novo dispositivo foi concluída com êxito. Além disso, o gerenciador
permite obter o estado de cada elemento encontrado, o qual está relacionado ao funcionamento do
dispositivo, se está executando corretamente ou se está pronto para operação. Outras informações
também são obtidas utilizando os metadados. Esta funcionalidade lista dados como modelo, versão ou
fabricante do dispositivo.

Na configuração, o operador do sistema também deve informar os serviços de monitoramento dos
sensores do equipamento para enviar os dados à base de dados. Com as informações de acesso à base de
dados, o dispositivo pode enviar automaticamente os dados adquiridos dos sensores. Para evitar o
excesso de acessos à base, um montante de dados é adquirido e armazenado localmente pelo
dispositivo. Quando o montante estiver completo, é enviado diretamente para a base de dados. É
possibilitado ao operador do sistema configurar o número de amostras que o dispositivo irá adquirir
para compor o conjunto de dados. A \cref{fig:uml-configure-device-acquisition} apresenta o diagrama
\gls{UML} de casos de uso que ilustra a configuração do dispositivo para envio dos dados para a Base
de Dados.

\includefigure
  {images/uml-configure-device-acquisition}
  {Diagrama de casos de uso para a configuração do envio dos dados do dispositivo.}
  {fig:uml-configure-device-acquisition}

Outra funcionalidade que o Gerenciador de Dispositivos disponibiliza é a obtenção da topologia da
distribuição dos dispositivos na rede. Dispositivos e serviços no padrão \gls{SOA} sempre estão
localizados no mesmo nível hierárquico, o que dificulta o estabelecimento de uma hierarquia entre os
componentes. No entanto, pode-se definir uma hierarquia utilizando para isso o identificador de cada
dispositivo. O identificador -- uma \gls{URI}, por exemplo -- é único para cada dispositivo da
aplicação, possibilitando que, na inclusão de subdispositivos, o identificador seja estendido para
englobá-los. Além disso, pode-se criar classes e subclasses de dispositivos, organizando
funcionalidades comuns para determinado dispositivo. A visualização das duas topologias possíveis é
apresentada na \cref{fig:device-topology-visualization}. Em~(a), os dispositivos estão distribuídos
como são encontrados pelo Gerenciador de Dispositivos. Esta visão pode ser considerada a visão do
ponto de vista da rede. Em~(b), o gerenciador reconstrói a topologia baseado nos identificadores
únicos de cada dispositivo. Esta visão pode ser considerada a visão lógica do sistema. Nota-se que
os dispositivos, que na primeira topologia estão no mesmo nível hierárquico, agora são encontrados
como subdispositivos de outros dispositivos.

\includefigure
  {images/device-topology-visualization}
  {Possíveis visualizações da topologia dos dispositivos: (a)~visão da rede; (b)~visão lógica.}
  {fig:device-topology-visualization}

Tomando de exemplo a \cref{fig:device-topology-visualization}, os identificadores de cada
dispositivo poderiam ser construídos com um dispositivo pai englobando outros dois subdispositivos.
Tanto o dispositivo pai como os subdispositivos estão definidos dentro de escopos ou áreas e essas
informações estão incluídas no identificador. Por exemplo, os dispositivos sensores~(\emph{Sensor
freio}, \emph{Sensor engrenagem}, \emph{Sensor motor}) logicamente fazem parte do dispositivo
\emph{Atuador elétrico}. Dessa forma, é possível a construção de um identificador para cada um
deles, como segue:

\begin{outline}
  \1 Atuador elétrico:
    \2[] {\ttfamily\small http://system.com/equipment/ElectricActuator}

  \1 Sensor freio:
    \2[] {\ttfamily\small http://system.com/equipment/ElectricActuator/BrakeSensor}

  \1 Sensor engrenagem:
    \2[] {\ttfamily\small http://system.com/equipment/ElectricActuator/GearSensor}

  \1 Sensor motor:
    \2[] {\ttfamily\small http://system.com/equipment/ElectricActuator/MotorSensor}
\end{outline}

Com o exemplo, fica claro como os componentes são distribuídos hierarquicamente no sistema. Através
do endereço lógico, os sensores são agrupados em um nível hierárquico abaixo do dispositivo
\emph{Atuador elétrico}. Ao incluir mais dispositivos neste nível, o padrão de organização do
identificador deve seguir o mesmo formato. Também é possível que dispositivos abaixo do nível
hierárquico dos sensores sejam adicionados. Dessa forma, ao final do endereço do sensor deve-se
adicionar um separador e o novo identificador.

Outra forma de organizar hierarquicamente os dispositivos é através de grupos. Ainda referente ao
exemplo do atuador elétrico, caso fosse composto por sensores e atuadores, novos separadores de
grupos, definidos como \emph{sensors} e \emph{actuators}, poderiam ser adicionados ao endereço das
entidades. Dessa forma, os dispositivos distribuídos logicamente no nível hierárquico mais baixo
estariam organizados da seguinte maneira:

\begin{outline}
  \1 Sensores do atuador elétrico:
    \2[] {\ttfamily\small http://system.com/equipment/ElectricActuator/sensors}

  \1 Atuadores do atuador elétrico:
    \2[] {\ttfamily\small http://system.com/equipment/ElectricActuator/actuators}
\end{outline}

% TODO: Talvez não seja necessária a inclusão de outra figura.
%\includefiguretmp
%  {Identificação dos dispositivos na topologia hierárquica.}
%  {fig:device-topology-identification}

Quanto à configuração, cada dispositivo pode armazenar diferentes comportamentos, os quais dizem
respeito ao modo de operação que o equipamento vai assumir perante determinada situação. Os
comportamentos são definidos pelo operador do sistema. São diferentes tarefas que podem ser
executadas pelo equipamento em situações específicas. No caso de verificação de degradação excessiva
de um equipamento, é possível alterar o comportamento de funcionamento para outro que priorize a
manutenção do estado de saúde atual, evitando o aumento da degradação. Dessa forma, o equipamento
pode operar por um período maior de tempo até que uma manutenção possa ser realizada. Contudo,
medidas que alteram o comportamento podem levar à perda de desempenho nas tarefas executadas.
Portanto, o operador do sistema deve ponderar a escolha e utilização adequada de cada um, visando
manter a qualidade do nível de operação desejado.

%A \cref{fig:device-send-behavior} ilustra o envio de novos comportamentos pelo operador do sistema
%para um dispositivo. O dispositivo mantém uma lista de comportamentos e seleciona qual o mais
%adequado para a situação corrente. A situação é definida pelo nível de saúde obtido através das
%análises de dados e representa a degradação de todo ou parte do equipamento.
%
%\includefiguretmp
%  {Envio de comportamentos para o dispositivo.}
%  {fig:device-send-behavior}


%%%
\subsection{Envio de dados de treinamento}

Para obter o índice de degradação dos equipamento, as ferramentas de análise necessitam comparar os
dados obtidos durante o processo com dados de treinamento. Dentre os dados de treinamento, emergem
duas categorias: normal e falha. Os dados de funcionamento normal são obtidos quando o equipamento
está funcionando normalmente, enquanto que os dados de falha são obtidos quando o equipamento está
funcionando com algum tipo de degradação. A aquisição de ambos os tipos de dados são feitas em lugar
apropriado, tendo certeza das características de cada sinal.

O envio de dados de treinamento é feito através do Gerenciador de Dispositivos. O gerenciador
permite que o operador do sistema envie os dois tipo de dados, mapeando-os para um dispositivo ou
para um grupo de dispositivos. A \cref{fig:uml-send-training-data} apresenta os casos de uso para o
envio de dados de treinamento a um dispositivo.

\includefigure
  {images/uml-send-training-data}
  {Diagrama de casos de uso para envio de dados de treinamento para um dispositivo.}
  {fig:uml-send-training-data}


%%%
\subsection{Gerenciamento de análises}

O gerenciamento de análises é feito pelo operador do sistema utilizando a entidade Gerenciador de
Análises. As análises são definidas em planos, os quais apresentam informações sobre o dispositivo
analisado, quais dados serão utilizados e as ferramentas empregadas na manipulação dos dados. É
permitido o gerenciamento dos planos, tendo em vista a edição ou remoção, por parte do operador do
sistema. Como os planos são armazenadas em uma base de dados, facilmente podem ser compartilhados
com outras instâncias do Gerenciador de Análises e utilizados por outros operadores do sistema.

As informações contidas em um plano de análise ditam como determinado equipamento será analisado. Em
um primeiro momento, o operador do sistema faz a busca pelos dispositivos utilizando o Gerenciador
de Dispositivos. Com a lista de dispositivos ativos, o Gerenciador de Análises é utilizado para
definir as operações a serem realizadas em um equipamento específico. Preenchendo o plano de
análise, o operador seleciona quais dados do equipamento serão utilizados e quais as ferramentas
utilizadas para analisá-los. O plano também permite definir o intervalo de execução de cada análise
e por quanto tempo o equipamento será analisado. O diagrama \gls{UML} de casos de uso para algumas
funcionalidades do Gerenciador de Análises é apresentado na \cref{fig:uml-analysis-management}. A
estrutura do plano é armazenada na entidade Base de Dados, sendo acessível para as outras entidades
que utilizam as informações do plano.

\includefigure
  {images/uml-analysis-management}
  {Diagrama de casos de uso para o gerenciamento de análises pelo operador do sistema.}
  {fig:uml-analysis-management}

Na construção do plano de análise, os comportamentos de cada dispositivo são obtidos e incluídos na
execução. Os diferentes comportamentos podem ser mapeados para intervalos de resultados esperados
das análises. Como descrito pelo diagrama de casos de uso da \cref{fig:uml-analysis-management}, a
\cref{fig:device-create-plan} ilustra a criação de um plano de análise de dispositivo.
Primeiramente, o operador do sistema deve utilizar o Gerenciador de Análises~(1). Após a execução do
gerenciador, é possível realizar a busca pelos dispositivos na rede~(2). Os dispositivos encontrados
são listados para o operador, bem como outros subdispositivos que o compõe. No exemplo, fazem parte
da rede os dispositivos \emph{Atuador elétrico}, \emph{Sensor freio}, \emph{Sensor engrenagem} e
\emph{Sensor motor}. Contudo, os sensores, que na topologia de rede estão no mesmo nível
hierárquico, fazem parte do atuador.

Ao obter os dispositivos da rede, o operador do sistema pode criar um plano de análise~(3). O plano
é criado também utilizando o Gerenciador de Análises, como ilustrado ainda na
\cref{fig:device-create-plan}. No plano, é definido qual dispositivo será analisado, quais sensores
serão monitorados e as ferramentas utilizadas na análise dos dados. Após a criação do plano, o
operador tem a possibilidade de armazená-lo~(4). Sendo assim, o plano é enviado para a base de
dados, onde estará visível para utilização pelo Analisador de Dispositivos ou mesmo para
visualização ou modificação posterior por parte do operador.

\includefigure
  {images/device-create-plan}
  {Busca de dispositivo e criação de um plano de análise.}
  {fig:device-create-plan}


%%%
\subsection{Análise dos dados do dispositivo}

As análises dos equipamentos do sistema, anteriormente agendadas na ferramentas de Gerenciamento de
Análises, são executadas pela entidade Analisador de Dispositivos. O analisador obtém os planos e os
executa conforme programado pelo operador do sistema. Conhecendo a frequência em que cada plano deve
ser executado, uma tarefa periódica é definida. Dessa forma, chegada a hora da reexecução de um
plano, a tarefa se comunica com o Gerenciador de Análises e o dispara. Os possíveis casos de uso
para a tarefa periódica são apresentados no diagrama \gls{UML} da \cref{fig:uml-device-analysis}. A
tarefa tem a possibilidade de obter os planos de análise ativos ou selecionar diretamente um plano
conhecido. Após a carga, a tarefa pode iniciar a execução do plano. Na execução, o Analisador de
Dispositivos seleciona as ferramentas de análise, busca os dados do equipamento e, ao final,
armazena os resultados na Base de Dados.

\includefigure
  {images/uml-device-analysis}
  {Diagrama de casos de uso para a obtenção e execução das análises.}
  {fig:uml-device-analysis}

A \cref{fig:analysis-plan-execution} apresenta o procedimento utilizado quando da execução de um
plano de análise pelo Analisador de Dispositivos. O analisador verifica a base de dados em busca dos
planos ativos~(1). Ao encontrá-los, o analisador os analisa verificando para qual dispositivo o
plano deve ser empregado~(2). Com o plano, o Analisador de Dispositivos busca novamente os dados dos
sensores do dispositivo, indicados no plano e que estão armazenados na base de dados~(3). Ao término
da busca, os dados são analisados e o resultado da análise é novamente enviado para a base de
dados~(4). Se necessário, novas análises são realizadas sobre os dados que já foram manipulados. Com
o índice de degradação, que é o resultado final da análise, o Analisador de Dispositivos seleciona
um dos comportamentos definidos no plano. O equipamento é então posto para executar a tarefa
monitorada de acordo com o novo comportamento~(5).

\includefigure
  {images/analysis-plan-execution}
  {Diagrama de execução de uma análise de equipamento.}
  {fig:analysis-plan-execution}


%%%
%\subsection{Modos de operação do dispositivo}
%
%Os diferentes modos de operação dos dispositivos podem ser selecionados pelo operador do sistema e
%anexados ao plano de análise. O operador define, baseado nos níveis de degradação que podem ser
%obtidos, qual o melhor comportamento que o equipamento deve assumir. Com o plano, o Analisador de
%Dispositivos determina o índice de degradação do equipamento, verificando se está nos níveis
%desejados para determinado funcionamento. Dependendo do valor obtido na análise, o analisador pode
%alterar o funcionamento do equipamento, assumindo um dos comportamentos definidos anteriormente
%pelo operador do sistema. A \cref{fig:uml-device-operation-modes} apresenta o diagrama \gls{UML} de
%casos de uso para a seleção automática dos modos de operação do dispositivo baseado no plano de
%análise.
%
%\includefiguretmp
%  %{images/uml-device-operation-modes}
%  {Diagrama de casos de uso para seleção dos modos de operação suportados pelo dispositivo.}
%  {fig:uml-device-operation-modes}


%%%
\subsection{Relatórios de saúde}

A arquitetura proposta permite a obtenção dos relatórios de saúde dos dispositivos executando sobre
equipamentos monitorados pelo sistema de manutenção inteligente. Como todos os dados das análises
são armazenados na entidade Base de Dados, é possível, a qualquer momento, resgatá-los. A obtenção
dos relatórios dos dispositivos é possível com o uso do Gerenciador de Dispositivos. Os casos de uso
para obtenção dos relatórios de um dispositivo ou de um grupo de dispositivos são apresentados no
diagrama \gls{UML} da \cref{fig:uml-health-device-report}.

\includefigure
  {images/uml-health-device-report}
  {Diagrama de casos de uso para obtenção dos relatórios de saúde do dispositivo.}
  {fig:uml-health-device-report}


%%
\section{Estudo de caso}
\label{sec:estudo-caso}

O estudo de caso utilizado neste trabalho consiste de um conjunto atuador elétrico e válvula. O
atuador é utilizado para controle de fluxo em redes de distribuição de petróleo. Os atuadores
permitem a abertura e fechamento dos dutos através do movimento da haste pelo acionamento de um
motor. Devido a natureza do equipamento, é inerente a predisposição à degradação.

Como parte do trabalho, o atuador é definido como um dispositivo da arquitetura proposta. São
mapeadas as informações pertinentes e traduzidas para o projeto de um cliente. O cliente será um
software para simulação de dados, não sendo embarcado no equipamento físico.

Para o projeto do software de simulação do atuador, foram utilizados dados de operação do
equipamento real. Os dados foram obtidos através de instrumentação feita nos trabalhos anteriores
de~\cite{boesch2011deteccao} e \cite{faccin2011manutencao}. Com os dados reais, o software deve
utilizá-los para simular a aquisição de dados do equipamento em funcionamento.


%%%
\subsection{Visão geral do objeto de estudo de caso}

O objeto de estudo de caso deste trabalho é um conjunto atuador elétrico e válvula. Um atuador
elétrico permite a abertura e fechamento motorizado de válvulas através da movimentação de uma
haste. Dessa forma, válvulas são encontradas em diversas aplicações onde seja necessário o controle
de fluxo de fluidos, como, por exemplo, água, esgoto ou petróleo. Para cada aplicação, é indicado um
tipo diferente de válvula. Os tipos mais comuns são esfera, gaveta e
globo~\cite{campos2006controles}.

O conjunto atuador elétrico e válvula, utilizado no trabalho em questão, é utilizado para controle
de fluxo em redes de distribuição de petróleo, modelo CSR6, fabricado pela empresa Coester Automação
Ltda. A \cref{fig:csr6-device} ilustra o atuador utilizado. É possível a obtenção do torque mecânico
exercido pelo atuador sobre as engrenagens de comando da haste bem como a posição do obturador. O
acionamento do conjunto pode ser feito localmente, através de uma interface de programação local,
sendo que todos os elementos de controle estão incorporados no próprio equipamento. O atuador
utilizado é dito de comando inteligente, pois, além de estar instrumentado, possui a capacidade de
detecção de alguns problemas, como sobreaquecimento, torque excessivo ou falta de fase.

\includefigure
  {images/csr6-device}
  {Atuador elétrico modelo CSR6.}
  {fig:csr6-device}

Durante o funcionamento, os valores de torque e posição da haste são armazenados em uma memória
interna do atuador, sendo possível registrar até \num{500} operações de fechamento ou abertura. No
entanto, as medidas obtidas pelo atuador não são suficientes para utilização em um sistema de
detecção de falhas. Uma das causas é a baixa resolução com que os dados são adquiridos, sendo de
\SI{1}{\newton\meter} e a cada 5\% de abertura para um fundo de escala de \SI{60}{\newton\meter}.
Além disso, como forma de detectar facilmente o início e o fim do movimento do atuador, o torque é
truncado para \SI{10}{\newton\meter} quando os valores forem menores que \SI{10}{\newton\meter} e,
quando não há movimento, assume-se o valor de \SI{0}{\newton\meter}~\cite{lazzaretti2012avaliacao}.

O atuador foi montado em uma bancada de testes onde é possível a simulação das condições de desgaste
do conjunto e a instrumentação de partes que não são verificadas por padrão. Dessa forma, na
bancada, o atuador está instrumentado com mais três sensores, visando a obtenção dos níveis de
vibração da estrutura. Os sensores são acelerômetros, posicionados no eixo do motor do atuador, na
ponta do sem-fim e na pinça do freio. A \cref{fig:sensors-localization} ilustra o posicionamento dos
sensores instalados e a estrutura interna do atuador elétrico.

%\includefiguretmp
  %{images/study-case}
  %{Bancada de testes utilizada para obtenção dos dados do estudo de caso.}
  %{fig:study-case}

\includefigure
  {images/sensors-localization}
  {Diagrama da localização dos sensores instalados no atuador elétrico.}
  {fig:sensors-localization}

A bancada também conta com um sistema de freio a disco instalado no eixo do atuador, o que
possibilita a simulação dos esforços mecânicos exercidos pelo fluxo de fluido durante a abertura e
fechamento da válvula. A simulação do desgaste é feita pela troca de engrenagens com diferentes
níveis de degradação. No total são três engrenagens, uma em estado normal e outras duas apresentando
falha. Das engrenagens com falha, uma está desgastada e a outra possui uma parte quebrada. As três
engrenagens utilizadas no estudo de caso são ilustradas na \cref{fig:test-gears}.

\includefigure
  {images/test-gears}
  {Engrenagens utilizadas no estudo de caso -- adaptado de \cite{lazzaretti2012avaliacao}.}
  {fig:test-gears}

As engrenagens fazem parte do conjunto de engrenagens satélite do atuador. Este conjunto é
responsável pela transmissão do movimento do motor para a haste da válvula. A posição do conjunto de
engrenagens satélite é apresentada na \cref{fig:sattelite-gears-location}. O sensor de torque, que
faz parte do conjunto sem necessidade de modificação na estrutura, está acoplado ao eixo de
acionamento do torque e também está identificado na mesma figura.

\includefigure
  {images/sattelite-gears-location}
  {Localização das engrenagens satélite no atuador elétrico.}
  {fig:sattelite-gears-location}


%%%
\subsection{Definição dos dispositivos}

No contexto deste trabalho, um dispositivo é a entidade lógica principal que abstrai um elemento da
aplicação. Pode representar uma entidade física, como um sensor ou atuador, ou lógica, como uma
máquina, composta por diversas entidades de hardware mapeadas como entidades lógicas. Cada um dos
dispositivo hospeda serviços, representando tarefas ou funcionalidades específicas possíveis de
serem executadas. Tanto os dispositivos, bem como os serviços por eles hospedados, podem ser
descobertos e identificados na rede. Os dispositivos podem descobrir outros dispositivos e utilizar
os serviços do segundo, a fim de criar um serviço composto mais complexo para execução de
determinada tarefa. A \cref{fig:device-services-overview} ilustra a topologia para os dispositivos
utilizada neste trabalho. Os clientes utilizam os serviços hospedados pelos dispositivos
diretamente. Um serviço também pode ser considerado um cliente caso utilize de uma funcionalidade
remota para prover a sua funcionalidade ou tarefa. Além disso, um dispositivo também pode ser
considerado um cliente, o que flexibiliza a integração entre os componentes do sistema.

\includefigure
    {images/device-services-overview}
    {Visão geral dos clientes, dispositivos e serviços hospedados.}
    {fig:device-services-overview}

O conjunto utilizado como estudo de caso foi mapeado para a representação de dispositivos da
arquitetura proposta. Foram verificadas algumas propriedades passíveis de serem incluídas em todos
os dispositivos que integram o sistema. Todas as informações mapeadas são acessíveis em forma de
serviço. Dessa forma, permitindo a integração do dispositivo com as diversas entidades de software
projetadas. A \cref{tab:device-metadata} apresenta as informações relativas aos dispositivos da
arquitetura proposta. Têm-se informações sobre o equipamento, como nome, modelo e número de série, e
também sobre o seu fabricante.

\includetable
  {tables/device-metadata}
  {Metadados disponíveis para configuração em cada dispositivo.}
  {tab:device-metadata}

Da mesma forma que para os dispositivos, os sensores que compõem o equipamento também possuem
informações que devem estar disponíveis para a aplicação. Os sensores representam componentes
importantes no sistema proposto, pois é onde os dados utilizados nas análises são gerados. Como
exemplo, são utilizados pelo Gerenciador de Análises para definir um plano de análise. As
informações dos sensores, que estão presentes em todos os elementos deste tipo na arquitetura
proposta, são apresentadas na \cref{tab:sensor-info}. Dentre as informações, estão o modelo, versão
e descrição, além do fabricante e a unidade de medida disponibilizada para os dados adquiridos.

\includetable
  {tables/sensor-info}
  {Parâmetros configuráveis dos sensores.}
  {tab:sensor-info}


%%%
\subsection{Aquisição dos dados}
\label{sub:estudo-caso-aquisicao-dados}

Os dados foram obtidos de acordo com os trabalhos de~\cite{boesch2011deteccao} e
\cite{faccin2011manutencao}. Segundo os autores, a aquisição dos dados foi feita através de um
sistema desenvolvido em LabVIEW utilizando o chassis {cRIO-9104} e o módulo para aquisição dos dados
de vibração {NI-9233}. Os dados foram coletados durante seis situações distintas, onde, para cada
situação, foram obtidos \num{25} conjuntos de dados de abertura e \num{25} de fechamento da válvula,
o que representa um total de \num{300} curvas de experimentos. As situações foram: utilização de
engrenagens em perfeito estado e sem acionamento do freio mecânico; engrenagens em perfeito estado e
acionamento do freio mecânico com pressão de \SI{1}{\bar}; engrenagens em perfeito estado e
acionamento do freio mecânico com pressão de \SI{3}{\bar}; utilização de uma engrenagem desgastada
sem acionamento do freio; três engrenagens desgastadas sem acionamento do freio; uma das engrenagens
com dentes quebrados sem acionamento do freio mecânico. A \cref{tab:data-description} ilustra as
seis situações de coleta dos dados.

\includetable
  {tables/data-description}
  {Situações definidas para coleta dos dados.}
  {tab:data-description}

Os dados de cada experimento foram amostrados a uma taxa de \SI{2048}{S\per\second}. Dessa forma,
cada conjunto de abertura ou fechamento da válvula tem duração aproximada de \SI{45}{\second}. Para
tornar o processamento dos dados menos custoso, uma subamostragem para a taxa de
\SI{1024}{S\per\second} foi realizada. Segundo~\cite{faccin2011manutencao}, a subamostragem dos
dados não prejudica o resultado da análise, porém torna o processamento menos custoso.


%%%
\subsection{Análise de degradação}
\label{sub:estudo-caso-analise-degradacao}

A análise de degradação do atuador é feita utilizando somente um dos três sensores citados
anteriormente. De acordo com os trabalhos de~\cite{boesch2011deteccao}
e~\cite{faccin2011manutencao}, o acelerômetro acoplado ao eixo do motor é o que apresenta melhores
resultados para a detecção na degradação das engrenagens do atuador. Como apresentado
em~\cite{lazzaretti2012avaliacao}, os dados que apresentam melhores resultados para a obtenção dos
níveis de degradação são as curvas de abertura da válvula, devido à maior força que é exercida na
sede da válvula.

Os sinais de vibração não são estacionários. Para analisá-los, obtendo as características variantes
no tempo, técnicas de processamento de tempo e frequência são as mais indicadas. Portanto, a
extração das características dos sinais pode ser feita utilizando a transformada de wavelet packet.
A eficácia do uso deste método para obtenção das características de sinais variantes no tempo é
descrita no trabalho de~\cite{qiu2006wavelet}.

Considerando a forma como os dados foram obtidos, pode-se supor que o atuador opera em três
situações distintas: em falha, em funcionamento normal e em teste. O funcionamento em falha se
refere à operação utilizando uma das engrenagens com dentes quebrados sem acionamento do freio
mecânico. O modo normal é definido como a operação quando é sabido que nenhuma das engrenagens
apresenta estado de degradação e o freio mecânico não é utilizado. Por fim, o modo de teste é aquele
onde o equipamento está sendo analisado, a fim de determinar o nível de degradação.

Os três tipos de situações definidas pela extração dos dados podem ser utilizadas para treinamento
do equipamento. Dessa forma, como apontado em~\cite{lazzaretti2012avaliacao}, uma das forma de se
fazer o treinamento do equipamento é utilizando o método da Regressão Logística. Para utilização, o
método necessita tanto de dados de funcionamento em falha, como de funcionamento normal. Dessa
forma, a Regressão Logística pode ser utilizada.


%%
\section{Definição dos experimentos}
\label{sec:experimentos-definicao}

Os experimentos a serem realizados tem por base validar a proposta apresentada neste trabalho. Uma
das formas de validação é a verificação do correto funcionamento de todas as entidades após a
implementação. Esta validação engloba o teste de comunicação entre as entidades e o correto
funcionamento dos serviços implementados.

Outro ponto importante aplicado à validação da arquitetura proposta é a comparação com os métodos
tradicionais empregados em trabalhos sobre manutenção inteligente. Conforme apresentado na
\cref{sec:arte-manutencao-inteligente}, os trabalhos relacionados com manutenção inteligente visam a
resolução de um problema específico. Dessa forma, a utilização das ferramentas de análise de dados
são desenvolvidas ou implementadas especialmente para o caso de uso proposto. Contudo, é possível
avaliar o impacto da utilização do sistema proposto para resolução deste tipo de problema.

Além da comparação com os métodos tradicionais, é possível a verificação do comportamento do sistema
proposto com o aumento do número de dispositivos monitorados por parte do operador do sistema. Por
premissa, pelo sistema apresentar entidades que possibilitam a configuração e análise automática dos
dados dos dispositivos, tem-se que o aumento de estações monitoradas não dificulta o processo de
análise. Porém, o aumento excessivo pode ocasionar outros problemas, como indisponibilidade da rede
ou das entidades de configuração e análise de dados.


%%%
\subsection{Verificação da interoperabilidade entre as entidades}
\label{sub:experimentos-interoperabilidade}

Como forma de validar o correto funcionamento dos componentes da arquitetura proposta, propõe-se um
teste de interoperabilidade. O teste visa verificar se todas as entidades estão operando
corretamente. Também permite identificar a troca de mensagens entre as entidades e o correto
funcionamento dos serviços implementados.

Para validação, o teste deve iniciar e executar uma aplicação mínima. Além da configuração das
entidades básicas, como o Gerenciador de Dispositivos e o Gerenciador de Análises, o teste proposto
visa validar a análise de um dispositivo. Dessa forma, o dispositivo deve ser corretamente
configurado através do Gerenciador de Dispositivos e analisado através do Analisador de
Dispositivos. Para a análise, um plano deve ser criado e dados de treinamento enviados.

Com o dispositivo configurado, deve-se verificar os resultados do nível de degradação. O valor de
confiança deve ser calculado através da análise dos dados simulados. Os dados são processados pelas
ferramentas de análise juntamente com os dados de treinamento.


%%%
\subsection{Comparação da proposta com métodos tradicionais}
\label{sub:experimentos-metodos-tradicionais}

O experimento visa comparar o método tradicional empregado na análise de degradação de um
equipamento com a arquitetura proposta. Nos métodos tradicionais, um software contendo as
ferramentas necessárias para a aplicação é utilizado para análise dos dados. Neste caso, é
necessário um operador especializado para realizar coleta dos dados, a configuração das ferramentas
de análise e a interpretação dos resultados obtidos. Normalmente, em cada nova verificação do
coeficiente de degradação, o processo deve ser repetido de forma manual ou semi-automática. Um
exemplo deste tipo de software é o Watchdog Agent.

Com a arquitetura proposta e a utilização dos componentes de software apresentados anteriormente, o
método de análise dos dados é possível de ser melhorado em comparação com os métodos tradicionais. O
experimento visa verificar qual a influência na configuração de uma análise de dados de um
equipamento e a forma como os resultados são obtidos.


%%%
\subsection{Verificação da escalabilidade da proposta}
\label{sub:experimentos-escalabilidade}

Com o aumento do número de dispositivos monitorados, utilizando os métodos tradicionais, é possível
que exista aumento da dificuldade de monitoração de todos. Dessa forma, o experimento visa
determinar qual a influência no aumento do número de equipamentos que necessitam de monitoramento no
método tradicional e com a utilização da arquitetura proposta.

Também é objetivo do teste verificar as formas de configuração e definição dos planos de análise
para um conjunto de equipamentos. Em métodos tradicionais, para cada novo equipamento que necessita
ser testado, os dados precisam ser analisados individualmente pelo operador do sistema. Como forma
de melhorar esta condição, o sistema proposto pode apresentar um ganho na configuração e
monitoramento de mais de um equipamento de forma simultânea.

  \chapter{Implementação e resultados}

\Blindtext[5][10]

  \chapter{Conclusão}
\label{cha:conclusao}

Este trabalho apresentou a proposta de uma arquitetura orientada a serviços para um sistema de
manutenção inteligente. Na dissertação, foi apresentado o projeto das diversas entidades que fazem
parte da arquitetura e propostos casos de uso onde o sistema se encaixa. Através dos casos de uso,
foi possível determinar as funcionalidades que cada entidade de software deveria apresentar. Por
fim, o sistema foi implementado e verificou-se que a solução proposta é viável, visto que integra um
ambiente que auxilia na determinação dos coeficientes de degradação de equipamentos de forma
facilitada.

Com as entidades propostas, mais especificamente o Gerenciador de Dispositivos, foi apresentado que
é possível gerenciar e configurar os dispositivos. É ofertado ao operador do sistema um software
dedicado para a configuração dos dispositivos remotamente. O gerenciador possibilita, entre outras
funcionalidades, a capacidade de obtenção de todos os dispositivos presentes na rede e a
apresentação de informações detalhadas sobre cada um deles ao usuário. Além disso, é possível a
obtenção da topologia encontrada na rede. O gerenciador organiza os dispositivos apresentando-os de
forma hierárquica, facilitando a verificação da estrutura dos equipamentos. Também no Gerenciador de
Dispositivos, é disponibilizada a funcionalidade de obtenção de relatórios de análise dos
equipamentos. Dessa forma, o operador pode verificar o histórico de degradação dos equipamentos e,
futuramente, utilizar os dados para tarefas de prognóstico.

Como forma de gerenciar as análises empregadas nos dispositivos, foi proposto o Gerenciador de
Análises. O gerenciador possibilita a criação de planos de análise para dispositivos individuais ou
em grupo. Esta entidade integra todo o processo de gerenciamento das análises a que os equipamentos
estão submetidos. É disponibilizado ao operador do sistema a definição das ferramentas que serão
utilizadas na análise dos dados, bem como a configuração de comportamentos em função do nível de
degradação obtido com a análise. Novamente, como no caso anterior, o Gerenciador de Análises é uma
ferramenta de configuração remota, permitindo, assim, o gerenciamento dos planos de análise de forma
facilitada.

As análises agendadas pelo operador no Gerenciador de Análises são executadas pela entidade
Analisador de Dispositivos. A arquitetura proposta dispões desta entidade dedicada para a execução
dos planos de análise dos equipamentos. O analisador verifica os dados dos dispositivos e executa as
análises com base no plano definido pelo operador do sistema. Como apresentado, o analisador utiliza
os algoritmos de análise presentes na ferramenta Watchdog Agent. Porém, levando em conta a forma com
o que software foi projetado, a extensão do módulo de análise para outras implementações comerciais
ou específicas é possível de forma facilitada.

Por fim, coma definição do estudo de caso e implementação dos dispositivos, foi possível testar a
interoperabilidade da solução proposta. O estudo de caso apresentado foi implementado na forma de
dispositivos \gls{DPWS}. A fim de facilitar os testes da arquitetura, os dados foram obtidos
anteriormente em situações distintas. Com isso, o dispositivo se comportou como um simulador,
gerando os dados dos sensores na mesma forma como foram obtidos.

Em relação aos experimentos propostos, verificou-se a vantagem no uso da arquitetura proposta em
relação a utilização dos métodos tradicionais para verificação da degradação em equipamentos
utilizando as técnicas de manutenção inteligente. Ficou claro que, nos métodos tradicionais,
utilizando somente o software Watchdog Agent, por exemplo, é necessário o emprego de um operador
qualificado para a configuração das ferramentas e obtenção dos resultados. Isso se deve ao fato de
que, para cada nova análise, o software precisa ser reconfigurado. Com as hipóteses levantadas, foi
possível concluir que a utilização de planos de análise, empregados nesta proposta, facilitam o
processo de análise dos dados, visto que, após a criação de um plano, o usuário não necessita
modificá-lo nem reconfigurar o software de análise. Também foi constatada a facilidade de obtenção
dos níveis de degradação. Ao passo que, nos métodos tradicionais o operador necessita coordenar a
análise, com a entidade proposta essas são feitas de maneira automática. O operador do sistema
necessita somente obter os relatórios com os valores calculados durante o período estipulado.

As vantagens na utilização da proposta são mais evidentes quando comparadas à configuração e análise
de múltiplos equipamentos. As dificuldades encontradas na utilização do software convencional para
obtenção dos níveis de degradação de um equipamento são expandidas quando se faz necessária a mesma
avaliação para um conjunto de dispositivos. Dessa forma, o operador necessita, a cada nova análise,
iniciar todo o processo de configuração das ferramentas e obtenção dos dados. Com a definição dos
planos de análise no Gerenciador de Análises, é possível a utilização de um mesmo plano para um
grupo de dispositivos. Os dispositivo que fazem parte do grupo de análise serão monitorados de forma
igual pelo Analisador de Dispositivos. Dessa forma, o analisador se encarrega de obter os níveis de
degradação de forma automática.

Visando a continuidade do desenvolvimento da proposta, pode-se citar alguns pontos verificados como
passíveis de melhoramento. Um deles diz respeito aos modos de aquisição de dados de treinamento. No
estado atual, o operador do sistema necessita enviar os dados de treinamento para os dispositivos de
forma manual. A cada novo treinamento que deseja-se incluir na base de dados, é necessária a
utilização das funcionalidades empregadas no Gerenciador de Dispositivos para envio dos dados.
Portanto, com o aumento no número de treinamentos, aumenta também a dificuldade de envio dos dados,
por representar tarefas repetitivas. Uma melhoria constatada está na utilização da entidade que
representa o dispositivo configurada em um modo específico de aquisição de dados de treinamento. Ao
contrário do que é utilizado na implementação atual, o dispositivo poderia ser configurado para
enviar os dados das aquisições diretamente para a base de dados como dados de treinamento.

O aumento no número de dispositivos também é considerado um problema na visão do Analisador de
Dispositivos. Se muitos dispositivos estão em processo de análise, o analisador manterá uma fila
demasiadamente grande. O processamento das análises enfileiradas pode representar atraso
considerável na obtenção dos valores de degradação. Considerando isso, a proposta de inclusão de
mais entidades para análise de dispositivos pode solucionar o problema. Uma possibilidade é a
comunicação ente os processos e verificação da disponibilidade. Dessa forma, evitando que a
sobrecarga dos planos de análise agendados recaiam sobre somente um ponto de processamento de dados.

Por último, como forma de melhorias, a reimplementação das ferramentas de análise se faz necessário.
Neste estudo, foram utilizadas as ferramentas que integram o software Watchdog Agent. Essas são
disponibilizadas em forma de arquivos de \textit{script} para execução no software Matlab. Pelo fato
de que arquivos deste tipo são interpretados, verificou-se perda considerável no desempenho em
função do aumento do lote de dados analisado. Dessa forma, propõe-se a reimplementação dos
algoritmos utilizando linguagem compiladas, como C ou C++, e, em pontos onde for possível,
processamento paralelo, como CUDA ou OpenCL~\cite{kirk2012programming}.

\end{onehalfspace}

\bibliographystyle{abnt}
\bibliography{bibliography/biblio}

\begin{onehalfspace}
  \appendix
  \chapter*{Apêndice A: Códigos utilizados na proposta}
\label{app:codigo-entidades}

Este apêndice apresenta alguns dos códigos utilizados no desenvolvimento da proposta. Algumas
informações foram omitidas a fim de destacar somente os pontos importantes da implementação das
classes.

\includejavacode
  {listings/waveletpackageenergies.java}
  {Implementação do serviço para cálculo das energias da transformada wavelet.}
  {lst:wavelet-package-energies}

\includejavacode
  {listings/datamanipulationdevice.java}
  {Implementação do dispositivo SOA contendo os serviços para manipulação de dados.}
  {lst:data-manipulation-device}

\includejavacode
  {listings/datamanipulationserviceprovider.java}
  {Implementação do provedor de serviços para a manipulação de dados.}
  {lst:data-manipulation-service-provider}

\includejavacode
  {listings/behaviorplugin.java}
  {Interface para a construção de novos comportamentos.}
  {lst:behavior-plugin-interface}

\includejavacode
  {listings/normalbehavior.java}
  {Implementação de um comportamento definido como normal seguindo a interface para construção de
      comportamentos.}
  {lst:normal-behavior-implementation}

\end{onehalfspace}

\end{document}
