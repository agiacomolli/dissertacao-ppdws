\documentclass[oneside,diss]{deletex}

\usepackage[latin1]{inputenc}
\usepackage{parskip}
\usepackage{float}
\usepackage{url}
\usepackage{amsmath}
\usepackage{booktabs}
\usepackage{siunitx}
\usepackage{mathptmx}
\usepackage{cleveref}
\usepackage[acronym,nonumberlist]{glossaries}
\usepackage[portuguese,shadow,textsize=footnotesize,backgroundcolor=yellow,bordercolor=black!50,
    linecolor=red]{todonotes}
\usepackage{setspace}

\setlength{\parindent}{0.7cm}
\setlength{\parskip}{1.5pt plus0mm minus0mm}

\newcommand{\crefpairconjunction}{ e }
\newcommand{\crefmiddleconjunction}{, }
\newcommand{\creflastconjunction}{ e }
\newcommand{\crefrangeconjunction}{ a }

\crefname{figure}{figura}{figuras}
\Crefname{figure}{Figura}{Figuras}
\crefname{table}{tabela}{tabelas}
\Crefname{table}{Tabela}{Tabelas}
\crefname{equation}{equação}{equações}
\Crefname{equation}{Equação}{Equações}

% TODO: Conferir os dados do PDF.
\newcommand{\disstitle}{Proposta de arquitetura orientada a serviços para um sistema de manutenção
    inteligente}
\newcommand{\dissauthor}{Anderson Antônio Giacomolli}
\newcommand{\disssubject}{Resumo}
\newcommand{\disskeywords}{Palavra-chave 1, Palavra-chave 2...}

%%
% Configuração do pacote para criação de links no PDF.
\hypersetup{%
  pdftitle={\disstitle},
  pdfauthor={\dissauthor},
  pdfproducer={\dissauthor},
  pdfsubject={\disssubject},
  pdfkeywords={\disskeywords},
  bookmarks=true,
  pdfmenubar=true,
  pdfborder={0,0,0}
}

%%
% Configuração do pacote para representação numérica.
\sisetup{%
  detect-all,
  output-decimal-marker={,}
}

\newglossary{symbols}{sym}{sbl}{List of symbols}

%%
% Definição do padrão para a lista de abreviaturas.
\newglossarystyle{ppgee}{%
  \setlength{\glsdescwidth}{0.8\linewidth}%
  \renewcommand*{\arraystretch}{1.5}%
  \renewenvironment{theglossary}%
    {\tablehead{}\tabletail{}%
      \begin{supertabular}{lp{\glsdescwidth}}}%
    {\end{supertabular}}%
  \renewcommand*{\glsnamefont}[1]{{\mdseries ##1}}%
  \renewcommand*{\glsgroupskip}{}%
  \renewcommand*{\glossaryheader}{}%
  \renewcommand*{\glspostdescription}{}%
  \renewcommand*{\glsgroupheading}[1]{}%
  \renewcommand*{\glossaryentryfield}[5]{%
    \glsentryitem{##1}\glstarget{##1}{##2} & ##3\glspostdescription\space ##5\\}%
  \renewcommand*{\glossarysubentryfield}[6]{%
    &
    \glssubentryitem{##2}%
    \glstarget{##2}{\strut}##4\glspostdescription\space ##6\\}%
  %\renewcommand*{\glsgroupskip}{ & \\}%
}

%%
% Comando para inserção de uma imagem.
\newcommand{\includefigure}[3]{%
  \begin{figure}[H]
    \centering
    \includegraphics{#1}
    \caption{#2}
    \label{#3}
  \end{figure}
}

%%
% Comando para inserção de uma tabela.
% TODO

%%
% Título e autor do documento.
\title{\disstitle}
\author{Giacomolli}{Anderson Antônio}

%%
% Orientador.
\advisor[Prof.~Dr.]{Pereira}{Carlos Eduardo}
\advisorinfo{UFRGS}{Doutor pela Sttutgart University -- Sttutgart, Alemanha}
\advisorwidth{0.56\textwidth}

%%
% Banca examinadora
\examiner[Prof.~Dr.]{do professor)}{(nome}
\examinerinfo{sigla da Instituição onde atual}{Doutor pela (Instituição onde obteve o título --
  Cidade, País)}
\examiner[Prof.~Dr.]{do professor)}{(nome}
\examinerinfo{sigla da Instituição onde atual}{Doutor pela (Instituição onde obteve o título --
  Cidade, País)}
\examiner[Prof.~Dr.]{do professor)}{(nome}
\examinerinfo{sigla da Instituição onde atual}{Doutor pela (Instituição onde obteve o título --
  Cidade, País)}

%%
% Data da defesa.
\date{março}{2014}

%%
% Área de concentração.
\topic{\ca}

%%
% Palavras-chave.
\keyword{Engenharia Elétrica}
\keyword{Processamento de Sinais}
\keyword{Automação e Controle}
\keyword{Eletrônica e Instrumentação}

%%
% Cria os glossários.
\makeglossaries

%%
% Inclui o arquivo com a lista de abreviaturas.
\newacronym{ARMA}
  {ARMA}
  {Auto-Regressive Moving-Average}

\newacronym{CSV}
  {CSV}
  {Comma-Separated Values}

\newacronym{CV}
  {CV}
  {Confidence Value}

\newacronym{CWT}
  {CWT}
  {Continuous Wavelet Transform}

\newacronym{DPWS}
  {DPWS}
  {Devices Profile for Web-Services}

\newacronym{DWT}
  {DWT}
  {Discrete Wavelet Transform}

\newacronym{FPGA}
  {FPGA}
  {Field-Programmable Gate Array}

\newacronym{GSM}
  {GSM}
  {Global System for Mobile communication}

\newacronym{HAVi}
  {HAVi}
  {Home Audio/Video Interoperability}

\newacronym{HTTP}
  {HTTP}
  {Hypertext Transfer Protocol}

\newacronym{IEEE}
  {IEEE}
  {Institute of Electrical and Electronics Engineers}

\newacronym{IHM}
  {IHM}
  {Interface Homem-Máquina}

\newacronym{IMS}
  {IMS}
  {Intelligent Maintenance System}

\newacronym{IMSCenter}
  {IMS~Center}
  {Intelligent Maintenance System Center}

\newacronym{IP}
  {IP}
  {Internet Protocol}

\newacronym{IRI}
  {IRI}
  {Internationalized Resource Identifier}

\newacronym{JAR}
  {JAR}
  {Java Archive}

\newacronym{JINI}
  {JINI}
  {Java Intelligent Network Infrastructure}

\newacronym{JMEDS}
  {JMEDS}
  {Java Multi Edition DPWS Stack}

\newacronym{LAN}
  {LAN}
  {Local Area Network}

\newacronym{MTOM}
  {MTOM}
  {SOAP Message Transmission Optimization Mechanism}

\newacronym{OASIS}
  {OASIS}
  {Organization for the Advancement of Structured Information Standards}

\newacronym{OPCUA}
  {OPC~UA}
  {OPC Unified Architecture}

\newacronym{OSACBM}
  {OSA-CBM}
  {Open Systems Architecture for Condition-Based Maintenance}

\newacronym{OSGi}
  {OSGi}
  {Open Service Gateway initiative}

\newacronym{QoS}
  {QoS}
  {Quality of Service}

\newacronym{RL}
  {RL}
  {Regressão Logística}

\newacronym{SIRENA}
  {SIRENA}
  {Service Infrastructure for Real Time Embedded Networked Applications}

\newacronym{SOA}
  {SOA}
  {Service-Oriented Architecture}

\newacronym{SOAP}
  {SOAP}
  {Simple Object Access Protocol}

\newacronym{TCP}
  {TCP}
  {Transmission Control Protocol}

\newacronym{TDMA}
  {TDMA}
  {Time Division Multiple Access}

\newacronym{UDDI}
  {UDDI}
  {Universal Description Discovery Integration}

\newacronym{UDP}
  {UDP}
  {User Datagram Protocol}

\newacronym{UML}
  {UML}
  {Unified Modeling Language}


\newacronym{UPnP}
  {UPnP}
  {Universal Plug and Play}

\newacronym{URI}
  {URI}
  {Uniform Resource Identifier}

\newacronym{URL}
  {URL}
  {Uniform Resource Locator}

\newacronym{WAN}
  {WAN}
  {Wide Area Network}

\newacronym{WPAN}
  {WPAN}
  {Wireless Personal Area Network}

\newacronym{WPE}
  {WPE}
  {Wavelet Packet Energies}

\newacronym{WSA}
  {WSA}
  {Web Service Architecture}

\newacronym{WSD}
  {WSD}
  {Web Service Description}

\newacronym{WSDL}
  {WSDL}
  {Web Service Description Language}

\newacronym{W3C}
  {W3C}
  {World Wide Web Consortium}

\newacronym{XML}
  {XML}
  {Extensible Markup Language}

\newglossaryentry{time-instant}{%
  type = {symbols},
  name = {\ensuremath{t}},
  description = {Instante de tempo}
}

\newglossaryentry{sample-index-discrete}{%
  type = {symbols},
  name = {\ensuremath{n}},
  description = {Índice de amostragem discreto}
}

\newglossaryentry{dilatation-param}{%
  type = {symbols},
  name = {\ensuremath{\beta}},
  description = {Variação da dilatação}
}

\newglossaryentry{scale-param}{%
  type = {symbols},
  name = {\ensuremath{\kappa}},
  description = {Variação da escala}
}

\newglossaryentry{dimension-index}{%
  type = {symbols},
  name = {\ensuremath{k}},
  description = {Dimensão do espaço}
}

\newglossaryentry{logistic-regression-input}{%
  type = {symbols},
  name = {\ensuremath{r}},
  description = {Vetor de entrada do modelo de regressão logística}
}

\newglossaryentry{logistic-regression-output}{%
  type = {symbols},
  name = {\ensuremath{y}},
  description = {Saída do modelo de regressão logística}
}

\newglossaryentry{wavelet-mother}{%
  type = {symbols},
  name = {\ensuremath{\psi}},
  description = {Função wavelet mãe}
}

\newglossaryentry{wavelet-transform-continuous}{%
  type = {symbols},
  name = {\ensuremath{\mathcal{W}}},
  description = {Tranformada wavelet contínua}
}

\newglossaryentry{wavelet-transform-discrete}{%
  type = {symbols},
  name = {\ensuremath{\mathcal{V}}},
  description = {Tranformada wavelet discreta}
}

\newglossaryentry{signal-time-continuous}{%
  type = {symbols},
  name = {\ensuremath{x(t)}},
  description = {Sinal contínuo no domínio tempo}
}

\newglossaryentry{signal-time-discrete}{%
  type = {symbols},
  name = {\ensuremath{x[n]}},
  description = {Sinal discreto no domínio tempo}
}

\newglossaryentry{wavelet-scale-param}{%
  type = symbols,
  name = {\ensuremath{\alpha}},
  description = {Parâmetro de dilatação},
}

\newglossaryentry{wavelet-translation-param}{%
  type = symbols,
  name = {\ensuremath{\tau}},
  description = {Parâmetro de translação}
}

\newglossaryentry{funcao-valor-confianca}{%
  type=symbols,
  name={\ensuremath{{P(y = 1 | x)}}},
  %name=pi,
  %symbol={\ensuremath{\Omega}},
  description=Função probabilidade para o valor de confiança
}

\newglossaryentry{sinal-continuo-frequencia}{%
  type=symbols,
  name={\ensuremath{X(\omega)}},
  %name=pi,
  %symbol={\ensuremath{\Omega}},
  description=Sinal contínuo no domínio frequência
}

\newglossaryentry{sinal-discreto-frequencia}{%
  type=symbols,
  name={\ensuremath{X[n]}},
  %name=pi,
  %symbol={\ensuremath{\Omega}},
  description=Sinal discreto no domínio frequência
}


%%%%%%%%%%%%%%%%%%%%%%%%%%%%%%%%%%%%%%%%%%%%%%%%%%%%%%%%%%%%%%%%%%%%%%%%%%%%%%%%
%%%%%%%%%%%%%%%%%%%%%%%%%%%%%%%%%%%%%%%%%%%%%%%%%%%%%%%%%%%%%%%%%%%%%%%%%%%%%%%%
\begin{document}

% O comando \maketile gera a capa, a folha de rosto e a folha de aprovacao
% (se for o caso)
% às vezes é necessário redefinir algum comando logo antes de produzir
% a Capa, folha de rosto e folha de aprovacao:
% \renewcommand{\coordname}{Coordenadora do Curso}
\maketitle

%%%%%%%%%%%%%%%%%%%%%%%%%%%%%%%%%%%%%%%%%%%%%%%%%%%%%%%%%%%%%%%%%%%%%%%%%%%%%%%%
%%%%%%%%%%%%%%%%%%%%%%%%%%%%%%%%%%%%%%%%%%%%%%%%%%%%%%%%%%%%%%%%%%%%%%%%%%%%%%%%
%\chapter*{Dedicatória}


%%%%%%%%%%%%%%%%%%%%%%%%%%%%%%%%%%%%%%%%%%%%%%%%%%%%%%%%%%%%%%%%%%%%%%%%%%%%%%%%
%%%%%%%%%%%%%%%%%%%%%%%%%%%%%%%%%%%%%%%%%%%%%%%%%%%%%%%%%%%%%%%%%%%%%%%%%%%%%%%%
%\chapter*{Agradecimentos}


% Resumo no idioma do documento.
%%%%%%%%%%%%%%%%%%%%%%%%%%%%%%%%%%%%%%%%%%%%%%%%%%%%%%%%%%%%%%%%%%%%%%%%%%%%%%%%
%%%%%%%%%%%%%%%%%%%%%%%%%%%%%%%%%%%%%%%%%%%%%%%%%%%%%%%%%%%%%%%%%%%%%%%%%%%%%%%%
\begin{abstract}
  No âmbito industrial, o custo empregado na manutenção de equipamentos ainda representa uma grande
parcela dos investimentos. Dessa forma, o desenvolvimento de técnicas de manutenção, e o seu correto
planejamento, cada vez mais estão assumindo papeis de grande importância nesse setor, visto que
impactam diretamente no fator econômico das empresas. Neste sentido, o trabalho em questão apresenta
a proposta de uma arquitetura orientada a serviços para um sistema de manutenção inteligente, a fim
de possibilitar a integração de forma facilitada entre equipamentos e as ferramentas de análise de
degradação. A arquitetura é composta por diferentes entidades, cada uma responsável por determinada
tarefa para o funcionamento do sistema. No trabalho, as entidades são implementadas e experimentos
são realizados a fim de validar a solução proposta.

\end{abstract}

% Abstract em inglês.
\begin{englishabstract}{Electrical Engineering, Signal Processing, Automation and Control,
    Electronic and Instrumentation}
  In the industrial field, the costs associated with equipment's maintenace still represents a large
portion of the resources available to a company. Therefore, new researches in maintenance systems,
and the correct task planning, are growing, since they impact directly on the economic side of the
companies. Thus, current work presents an service-oriented architecture for maintenance systems
integration. The proposed architecture intends to facilitate the integration of equipments and
degradation analysis tools. The architecture is comprised of several entities, where each one is
responsible for executing a given task in order to keep the system running properly. In this work,
all the entities are implemented and experiments are performed in order to validate the proposed
system.

\end{englishabstract}

% Lista de ilustrações.
\listoffigures

% Lista de tabelas.
\listoftables

% Lista de abreviaturas e siglas.
%\begin{listofabbrv}{OSA-CBM}
%	\item[ABNT] Associação Brasileira de Normas Técnicas
%	\item[GCAR] Grupo de Controle, Automação e Robótica
%	\item[PPGEE] Programa de Pós-Graduação em Engenharia Elétrica
%	\item[OSA-CBM] Programa de Pós-Graduação em Engenharia Elétrica
%\end{listofabbrv}

% Lista de abreviaturas e siglas.
\setglossarysection{chapter}
\printglossary[style=ppgee,type=\acronymtype,title=Lista de abreviaturas]

% lista de símbolos é opcional
%\begin{listofsymbols}{$\alpha\beta\pi\omega$}
%       \item[$\sum$] Somatório
%       \item[$\alpha\beta\pi\omega$] Fator de inconstância do resultado
%\end{listofsymbols}

% Lista de símbolos.
%\setglossarysection{chapter}
\printglossary[style=ppgee,type=symbols,title=Lista de símbolos]
%\printglossaries

% Sumário.
\tableofcontents

%%%%%%%%%%%%%%%%%%%%%%%%%%%%%%%%%%%%%%%%%%%%%%%%%%%%%%%%%%%%%%%%%%%%%%%%%%%%%%%%
%%%%%%%%%%%%%%%%%%%%%%%%%%%%%%%%%%%%%%%%%%%%%%%%%%%%%%%%%%%%%%%%%%%%%%%%%%%%%%%%
\begin{onehalfspace}
  \chapter{Introdução}

\cite{lee2006intelligent}

\Blindtext[5][4]

  \chapter{Sistemas de manutenção inteligente}

  \input{chapters/arquiteturas-orientadas-a-servicos}
  %\chapter{Arquitetura proposta para um sistema de manutenção inteligente}
\chapter{Arquitetura proposta}

\todo[inline]{Introdução do capítulo. Revisar o capítulo para se adequar às novas alterações.}


%%
\section{Arquitetura orientada a serviços proposta}

A arquitetura proposta neste trabalho tem por objetivo a integração de sistemas de manutenção
inteligente, dos diversos equipamentos que precisam ser monitorados, além de outras entidades que
auxiliam no funcionamento do sistema. A troca de informações entre todos os elementos que compõem a
arquitetura é abstraída na forma de serviços, o que facilita a integração, inserção e remoção de
novas entidades no sistema. Dessa forma, a especificação da arquitetura é definida utilizando os
padrões \gls{SOA}. Uma visão geral das entidades que fazem parte da arquitetura proposta é
apresentada na \cref{fig:soa-proposed-architecture}\todo{Alterar figura. A aplicação SOA deve
englobar mais entidades}. Na figura, nota-se que cada entidade tem uma função específica no sistema
e o acesso às suas funcionalidades é realizado através de serviços. A seguir, para melhor
entendimento, todos os elementos serão detalhados.

\includefigure
  {images/soa-proposed-architecture}
  {Arquitetura orientada a serviços proposta para integração de um sistema de manutenção
      inteligente.}
  {fig:soa-proposed-architecture}


%%%
\subsection{Serviço}

No contexto da arquitetura proposta neste trabalho, serviço é um componente de software que
encapsula uma funcionalidade acessível através dos padrões definidos pela tecnologia \gls{SOA}.
Todos os componentes da arquitetura expõem as suas funcionalidades na forma de serviços,
possibilitando que a interação entre eles seja feita de forma transparente. Como parte do padrão
\gls{SOA}, serviços podem ser descobertos e utilizados por clientes que desejam executar uma
determinada tarefa. Neste contexto, também é possível a definição de serviços mais especializados
com base em outros serviços, praticando a técnica da composição de serviços. Estas características
se tornam inerentes à proposta, devido a utilização do padrão \gls{SOA}.


%%%
\subsection{Dispositivo}

Um dispositivo é um componente de software utilizado para encapsular um elemento físico da aplicação
proposta. Por ser executado em um dispositivo físico, o componente é denominado dispositivo lógico e
disponibiliza serviços para acesso à funcionalidades previamente definidas. As funcionalidades podem
ser relativas ao dispositivo físico em que o componente está executando ou outras que auxiliam em
alguma tarefa específica não relacionada diretamente com o hardware hospedeiro.

Do ponto de vista da aplicação, os dispositivos são entidades que hospedam serviços. Dentre os
serviços hospedados, alguns estão presentes em todos os dispositivos da arquitetura, servindo de
base para a comunicação entre todos os elementos desta classe. Dessa forma, conhecendo os serviços
básicos, uma interface mínima para troca de informações entre as entidades do sistema é definida,
facilitando a inserção de novos dispositivos.

Além dos serviços base, outros podem ser executados no dispositivo. A arquitetura permite o envio de
novos serviços para os dispositivos do sistema. O dispositivo recebe o novo serviço, que possui as
mesmas funcionalidades de um serviço padrão, e o carrega para ser executado normalmente.


%%%
\subsection{Aplicação orientada a serviços}

A \todo{Alterar texto para englobar mais entidades da arquitetura}aplicação orientada a serviços
nada mais é do que o resultado da utilização dos diversos serviços propostos na arquitetura. Através
da composição ou utilização direta dos serviços de uma forma coordenada, a aplicação é construída.
De um modo minimalista, a aplicação pode ser vista como a interação progressiva de funcionalidades
simples provenientes de serviços básicos, resultando em um componente complexo de software. Visto
que o produto final da composição dos elementos do sistema não é definido \textit{a priori}, o
resultado da aplicação pode ser qualquer construção ou interação entre os componentes.


%%%
\subsection{Gerenciador de dispositivos}

Como o nome sugere, o Gerenciador de Dispositivos é um componente de software utilizado para
gerenciar os dispositivos da arquitetura proposta. A configuração ou obtenção da lista dos
dispositivos na aplicação orientada a serviços são exemplos de funcionalidades do gerenciador. Como
os dispositivos são entidades que mapeiam funcionalidades dos dispositivos físicos encontrados em um
sistema de manutenção inteligente, o gerenciador pode, por exemplo, obter a lista de sensores do
equipamento e configurar alguns parâmetros, como a taxa de atualização dos dados dos sensores.
Também é possível definir comportamentos adicionais para o caso do equipamento estar operando em
diferentes níveis de degradação. Os diferentes comportamentos podem ser utilizados quando há
necessidade de operar o equipamento em condições onde manter os níveis de degradação estáveis é mais
importante do que a operação a pleno. Portanto, pode-se manter o equipamento funcionando até que uma
manutenção \textit{in loco} possa ser realizada.

Grupos de dispositivos podem ser criados e gerenciados. O Gerenciador de Dispositivos permite a
criação de grupos, a fim de facilitar a alteração simultânea de vários dispositivos. Supondo que a
aplicação é inerentemente escalável, o número de dispositivos tente a aumentar, o que dificulta o
gerenciamento ou configuração de dispositivos similares. Considerando que as configurações aplicadas
a uma mesma classe de dispositivos é muito parecida, o gerenciador permite que um grupo receba os
mesmos parâmetros, automatizando o processo de customização dos dispositivos.

Outra funcionalidade que o gerenciador provê é a obtenção da topologia dos componentes do sistema.
Em uma arquitetura \gls{SOA} todos os serviços estão localizados no mesmo nível hierárquico. Essa
característica pode ser considerada positiva do ponto de vista de integração e reuso de serviços. Em
contrapartida, ao definir níveis hierárquicos entre componentes de um mesmo dispositivo físico, a
complexidade para o estabelecimento de uma hierarquia lógica para representação das diferentes
partes desse dispositivo aumenta. Contudo, ao aplicar um identificador único para cada dispositivo,
ou parte de dispositivo, do sistema, torna-se possível definir e estabelecer uma relação
hierárquica. O identificador pode ser uma \gls{URI}, por exemplo, onde cada parte do endereço se
refere a um dos níveis hierárquicos.

O Gerenciador de Dispositivos também é utilizado na resolução de problemas encontrados durante a
execução dos componentes do sistema. A entidade se enquadra na categoria de \gls{IHM},
possibilitando que a rede seja mapeada em busca de dispositivos ou serviços de forma interativa.
Além disso, é possível obter o estado dos dispositivo, visualizar dados ou testar os serviços
encontrados.

\todo[inline]{Falar sobre os valores de confiança. É interessante que o gerenciador de dispositivos
forneça os históricos da saúde do equipamento.}


%%%
\subsection{Gerenciador de análises}

O Gerenciador de Análises é utilizado para definir o plano que será aplicado a determinado
equipamento, ou grupo de equipamentos, \todo{Em ou com?}com vista a obter os níveis de degradação.
Nesta entidade, é possível selecionar e definir a ordem em que serão executadas as ferramentas de
análise de dados através de um plano. O gerenciador permite o agendamento dos planos e a execução
periódica em intervalos de tempo. Os planos também podem ser gerenciados, possibilitando alteração
e/ou exclusão.

No plano são definidos os comportamentos a serem adotados com base nos valores de degradação do
equipamento. O gerenciador obtém os comportamentos previamente definidos para o equipamento a ser
analisado, os quais podem ser mapeados pelo usuário para diferentes níveis de degradação. Dessa
forma, o equipamento pode ser adaptado a diferentes condições, por exemplo, evitando o aumento da
degradação até que uma manutenção possa ser realizada.

Esta entidade também coordena as análises realizadas em um grupo de equipamentos similares. O
processo de obtenção dos valores de confiança de um conjunto de equipamentos pode ser simplificado
pelo fato de que, se são similares, o nível de degradação de um equipamento pode ser aproximado para
o nível de degradação do grupo. Dessa maneira, é possível elaborar um plano de análise que seja
aplicável a vários equipamentos.

\todo[inline]{Utilização da base de dados para armazenamento dos planos.}


%%%
\subsection{Analisador de dispositivos}

O Analisador de Dispositivos concentra todas as ferramentas necessárias para a realização da análise
dos dados de determinado dispositivo. A entidade opera com base nos planos criados e agendados pelo
\todo{Operador do sistema?}usuário no Gerenciador de Análises. De posse do plano, o analisador
executa as análises utilizando os algoritmos necessários para obtenção do resultado esperado. A
execução de um mesmo plano é feita de forma automática.

O analisador permite verificar quais planos estão sendo executados ou na fila para execução. Sempre
que um novo plano é encontrado, o analisador o coloca na fila de execução, a fim de que seja
finalizado o mais breve possível. A entidade permite o monitoramento destes planos pelo usuário, o
qual pode verificar se a execução está de acordo com o esperado.


%%%
\subsection{Base de dados}

A entidade Base de Dados é utilizada como uma interface para um banco de dados. O software encapsula
funcionalidades para  armazenar todos os dados relativos à aplicação. Desde os dados obtidos nas
análises de degradação dos equipamentos como também os planos de análise agendados pelo usuário. A
base centraliza as informações, mas não necessariamente faz com que a aplicação seja centralizada.
Nada impede a utilização de diferentes \todo{Remotas e locais? Alterar figura}bases de dados e o
compartilhamento de informações entre as entidade do sistema. Por exemplo, o armazenamento dos dados
de treinamento de um equipamento, necessário para obtenção do nível de degradação, pode ser feito em
diferentes bases de dados. O Analisador de Dispositivos pode combinar estas informações, a fim de
melhorar o resultado esperado.

Por ser defina como um componente \gls{SOA}, o acesso aos dados é feito através de serviços. A
inclusão, deleção e modificação de dados são realizadas por serviços especializados. Um dispositivo
remoto pode acessar a base de dados a fim de obter o histórico dos últimos valores de confiança
calculados ou valores de testes intermediários para determinado equipamento, utilizando-os para um
novo tipo de análise.


%%%
\subsection{Repositório de serviços}

O repositório de serviços é uma entidade que armazena os dispositivos lógicos e seus serviços, os
quais podem ser obtidos dinamicamente e implantados em dispositivos físicos. Esta entidade pode
estar localizada na rede local ou remota, com acesso disponível para várias aplicações. Dessa forma,
uma nova aplicação, isolada de outra já existente, pode ser construída com o reúso de componentes
obtidos de um repositório de serviços compartilhado pelas duas. Esta abordagem também permite o
compartilhamento de uma base de dados de componentes lógicos entre aplicações.


%%
\section{Casos de uso para a arquitetura proposta}

Os elementos que compõem a arquitetura proposta encaixam-se em alguns casos de uso no contexto de um
sistema de manutenção inteligente. Os casos de uso propostos a seguir exemplificam a interação do
operador do sistema, quando há a necessidade de busca ou configuração dos dispositivos, além da
ferramenta de análise de dados, que executa autonomamente sobre os dados obtidos durante o processo
de amostragem dos equipamentos.\todo{Introdução da seção com definição dos atores da aplicação.}


%%%
\subsection{Descoberta e configuração de dispositivos}

A descoberta dos dispositivos e serviços na rede pode ser feita a qualquer momento utilizando o
Gerenciador de Dispositivos. Cada um dos dispositivos encontrados é identificado e colocado em uma
lista de dispositivos ativos. Também são identificados os serviços hospedados em cada um dos
dispositivos. Dessa forma, é possível determinar quais dispositivos estão atualmente disponíveis na
rede e quais possuem serviços que poderão ser utilizados para suprir e executar determinada
funcionalidade do sistema.

A geração da hierarquia de recursos encontrados na rede também é possível. Ao aplicar um
identificador único para cada recurso, a reconstrução dos componentes do sistema pode ser obtida
agrupando-os em classes ou categorias. O identificador é utilizado para auxiliar a definir uma
topologia dos subdispositivos encontrados em um dispositivo lógico, facilitando a visualização e
configuração dos componentes.

O diagrama \gls{UML} de casos de uso para descoberta e configuração de dispositivos é apresentado na
\cref{fig:uml-discovery-setup-devices}. São ilustradas as interações que o operador do sistema pode
realizar nos dispositivos utilizando o Gerenciador de Dispositivos. Em vias gerais, o gerenciador
será utilizado pelo operador do sistema para descobrir os dispositivos da rede, obter o estado ou
realizar modificações nestes elementos.

\includefigure
  {images/uml-discovery-setup-devices}
  {Diagrama de casos de uso para descoberta e configuração de dispositivos.}
  {fig:uml-discovery-setup-devices}

Ao utilizar a descoberta de dispositivos, o gerenciador analisa todos dispositivos da aplicação
orientada a serviços. Os que forem encontrados são retornados, permitindo ao operador do sistema
verificar se a inclusão de um novo dispositivo foi concluída com êxito. Além disso, o gerenciador
permite obter o estado de cada elemento encontrado, o qual está relacionado ao funcionamento do
dispositivo, se está executando corretamente ou se está pronto para operação. Outras informações
também são obtidas utilizando os metadados. Esta funcionalidade lista dados como modelo, versão ou
fabricante do dispositivo.

Outra funcionalidade que o Gerenciador de Dispositivos disponibiliza é a obtenção da topologia da
distribuição dos dispositivos na rede. Dispositivos e serviços no padrão \gls{SOA} sempre estão
localizados no mesmo nível hierárquico, o que dificulta o estabelecimento de uma hierarquia entre os
componentes. No entanto, pode-se definir uma hierarquia utilizando para isso o identificador de cada
dispositivo. O identificador -- uma \gls{URI}, por exemplo -- é único para cada dispositivo da
aplicação, possibilitando que, na inclusão de subdispositivos, o identificador seja estendido para
englobá-los. Além disso, pode-se criar classes e subclasses de dispositivos, organizando
funcionalidades comuns para determinado dispositivo. A visualização das duas topologias possíveis é
apresentada na \cref{fig:device-topology-visualization}. À esquerda, os dispositivos estão
distribuídos como são encontrados pelo Gerenciador de Dispositivos. Na direita, o gerenciador
reconstrói a topologia baseado nos identificadores únicos de cada dispositivo. Nota-se que os
dispositivos, que na primeira topologia estão no mesmo nível hierárquico, agora são encontrados como
subdispositivos de outros dispositivos.

\includefiguretmp
  {Possíveis visualizações da topologia dos dispositivos.}
  {fig:device-topology-visualization}

Tomando de exemplo a \cref{fig:device-topology-visualization}, os identificadores de cada
dispositivo poderiam ser construídos com um dispositivo pai englobando outros dois subdispositivos.
Tanto o dispositivo pai como os subdispositivos estão definidos dentro de escopos ou áreas e essas
informações estão incluídas no identificador. A \cref{fig:device-topology-identification} demonstra
os identificadores utilizados para cada dispositivo, obedecendo os níveis hierárquicos\todo{Terminar
de falar sobre a figura, mencionando os componentes utilizados no exemplo (atuador elétrico)}.

\includefiguretmp
  {Identificação dos dispositivos na topologia hierárquica.}
  {fig:device-topology-identification}

Cada dispositivo pode armazenar diferentes comportamentos. Os comportamentos dizem respeito ao modo
de operação que o equipamento vai assumir perante determinada situação, sendo definidos pelo
operador do sistema. São diferentes tarefas que podem ser executadas pelo equipamento em situações
específicas. No caso de verificação de degradação excessiva de um equipamento, é possível alterar o
comportamento de funcionamento para outro que priorize a manutenção do estado de saúde atual,
evitando o aumento da degradação. Dessa forma, o equipamento pode operar por um período maior de
tempo até que uma manutenção possa ser realizada. Contudo, medidas que alteram o comportamento podem
levar a perda de desempenho nas tarefas executadas. A \cref{fig:device-send-behavior} ilustra o
envio de novos comportamentos pelo operador do sistema para um dispositivo. O dispositivo mantém uma
lista de comportamentos e seleciona qual o mais adequado para a situação corrente. A situação é
definida pelo nível de saúde obtido através das análises de dados e representa a degradação de todo
ou parte do equipamento.

\includefiguretmp
  {Envio de comportamentos para o dispositivo.}
  {fig:device-send-behavior}


%%%
\subsection{Envio de dados de treinamento}

Para obter o índice de degradação dos equipamento, as ferramentas de análise necessitam comparar os
dados obtidos durante o processo com dados de treinamento. Dentre os dados de treinamento, emergem
duas categorias: normal e falha. Os dados de funcionamento normal são obtidos quando o equipamento
está funcionando normalmente, enquanto que os dados de falha são obtidos quando o equipamento está
funcionando com algum tipo de degradação. A aquisição de ambos os tipos de dados são feitas em lugar
apropriado, tendo certeza das características de cada sinal.

O envio de dados de treinamento é feito através do Gerenciador de Dispositivos. O gerenciador
permite que o operador do sistema envie os dois tipo de dados, mapeando-os para um dispositivo ou
para um grupo de dispositivos. A \cref{fig:uml-training-data-send} apresenta os casos de uso para o
envio de dados de treinamento a um dispositivo.

\includefiguretmp
  {Diagrama de casos de uso para envio de dados de treinamento para um dispositivo.}
  {fig:uml-training-data-send}


%%%
\subsection{Gerenciamento de análises}

O gerenciamento de análises é feito pelo operador do sistema utilizando a entidade Gerenciador de
Análises. As análises são definidas em planos, os quais apresentam informações sobre o dispositivo
analisado, quais dados serão utilizados e as ferramentas empregadas na manipulação dos dados. É
permitido o gerenciamento dos planos, como a criação, edição ou remoção, pelo operador do sistema.
Como os planos são armazenadas em uma base de dados, facilmente podem ser compartilhados com outras
instâncias do Gerenciador de Análises e utilizados por outros operadores do sistema.

As informações contidas em um plano de análise ditam como determinado equipamento será analisado. Em
um primeiro momento, o operador do sistema faz a busca pelos dispositivos utilizando o Gerenciador
de Dispositivos. Com a lista de dispositivos ativos, o Gerenciador de Análises é utilizado para
definir as operações a serem realizadas em um equipamento específico. Preenchendo o plano de
análise, o operador seleciona quais dados do equipamento serão utilizados e quais as ferramentas
utilizadas para analisá-los. O plano também permite definir o intervalo de execução de cada análise
e por quanto tempo o equipamento será analisado. O diagrama \gls{UML} de casos de uso para algumas
funcionalidades do Gerenciador de Análises é apresentado na \cref{fig:uml-analysis-management}. A
estrutura do plano é armazenada na entidade Base de Dados, sendo acessível para as outras entidades
que utilizam as informações do plano.

\todo[inline]{Incluir o analisador de dispositivos na figura. Neles estão armazenadas as ferramentas
que podem ser utilizadas.}

\includefigure
  {images/uml-analysis-management}
  {Diagrama de casos de uso para o gerenciamento de análises pelo operador do sistema.}
  {fig:uml-analysis-management}

Na construção do plano de análise, os comportamentos de cada dispositivo são obtidos e incluídos na
execução. Os diferentes comportamentos podem ser mapeados para intervalos de resultados esperados
das análises. Juntamente com a \cref{fig:uml-analysis-management}, a \cref{fig:device-create-plan}
ilustra a criação de um plano de análise de dispositivo. Após obter todas as informações do
dispositivo a ser analisado, o plano é criado pelo operador do sistema. Na criação, são definidos os
intervalos de reexecução do plano, bem como a ordem das ferramentas utilizadas na manipulação dos
dados.

\includefiguretmp
  {Busca de dispositivo e criação de um plano de análise.}
  {fig:device-create-plan}

\todo[inline]{Falar da análise de um grupo de dispositivos.}

%%%
\subsection{Análise dos dados do dispositivo}

As análises dos equipamentos do sistema, anteriormente agendadas na ferramentas de Gerenciamento de
Análises, são executadas pela entidade Analisador de Dispositivos. O analisador obtém os planos e os
executa conforme programado pelo operador do sistema. Conhecendo a frequência em que cada plano deve
ser executado, uma tarefa periódica é definida. Dessa forma, chegada a hora da reexecução de um
plano, a tarefa se comunica com o Gerenciador de Análises e o dispara. Os possíveis casos de uso
para a tarefa periódica são apresentados no diagrama \gls{UML} da \cref{fig:uml-device-analysis}. A
tarefa tem a possibilidade de obter os planos de análise ativos ou selecionar diretamente um plano
conhecido. Após a carga, a tarefa pode iniciar a execução do plano. Na execução, o Analisador de
Dispositivos seleciona as ferramentas de análise, busca os dados do equipamento e, ao final,
armazena os resultados na Base de Dados.

\includefigure
  {images/uml-device-analysis}
  {Diagrama de casos de uso para a obtenção e execução das análises.}
  {fig:uml-device-analysis}

A \cref{fig:analysis-plan-execution} apresenta o fluxo de dados quando da execução de um plano de
análise pelo Analisador de Dispositivos.

\includefiguretmp
  {Diagrama de execução de uma análise de equipamento.}
  {fig:analysis-plan-execution}

O operador do sistema também pode interagir com o Analisador de Dispositivos. O diagrama de casos de
uso da \cref{fig:uml-analysis-monitoring} ilustra essa interação.

\includefiguretmp
  {Diagrama de casos de uso para o monitoramento da execução das análises.}
  {fig:uml-analysis-monitoring}


%%%
\subsection{Modos de operação do dispositivo}

Os diferentes modos de operação dos dispositivos podem ser selecionados pelo operador do sistema e
anexados ao plano de análise. O operador define, baseado nos níveis de degradação que podem ser
obtidos, qual o melhor comportamento que o equipamento deve assumir. Com o plano, o Analisador de
Dispositivos determina o índice de degradação do equipamento, verificando se está nos níveis
desejados para determinado funcionamento. Dependendo do valor obtido na análise, o analisador pode
alterar o funcionamento do equipamento, assumindo um dos comportamentos definidos anteriormente pelo
operador do sistema. A \cref{fig:uml-device-operation-modes} apresenta o diagrama \gls{UML} de casos
de uso para a seleção automática dos modos de operação do dispositivo baseado no plano de análise.

\includefiguretmp
  {Diagrama de casos de uso para seleção dos modos de operação suportados pelo dispositivo.}
  {fig:uml-device-operation-modes}


%%%
\subsection{Relatórios de saúde}

A arquitetura proposta permite a obtenção dos relatórios de saúde dos dispositivos executando sobre
equipamentos monitorados pelo sistema de manutenção inteligente. Como todos os dados das análises
são armazenados na entidade Base de Dados, é possível, a qualquer momento, resgatá-los.

\includefiguretmp
  {Diagrama de casos de uso para obtenção dos relatórios de saúde do dispositivo.}
  {fig:uml-health-device-report}

  \chapter{Modelagem das entidades e estudo de caso}

Este capítulo apresenta a definição das entidades presentes na arquitetura proposta com base no que foi abordado no capítulo anterior\todo{Terminar a introdução do capítulo}.



  \input{chapters/implementacao-e-resultados}
  \chapter{Conclusão}
\label{cha:conclusao}

Este trabalho apresentou a proposta de uma arquitetura orientada a serviços para um sistema de
manutenção inteligente. Na dissertação, foi apresentado o projeto das diversas entidades que fazem
parte da arquitetura e propostos casos de uso onde o sistema se encaixa. Através dos casos de uso,
foi possível determinar as funcionalidades que cada entidade de software deveria apresentar. Por
fim, o sistema foi implementado e verificou-se que a solução proposta é viável, visto que integra um
ambiente que auxilia na determinação dos coeficientes de degradação de equipamentos de forma
facilitada.

Com as entidades propostas, mais especificamente o Gerenciador de Dispositivos, foi apresentado que
é possível gerenciar e configurar os dispositivos. É ofertado ao operador do sistema um software
dedicado para a configuração dos dispositivos remotamente. O gerenciador possibilita, entre outras
funcionalidades, a capacidade de obtenção de todos os dispositivos presentes na rede e a
apresentação de informações detalhadas sobre cada um deles ao usuário. Além disso, é possível a
obtenção da topologia encontrada na rede. O gerenciador organiza os dispositivos apresentando-os de
forma hierárquica, facilitando a verificação da estrutura dos equipamentos. Também no Gerenciador de
Dispositivos, é disponibilizada a funcionalidade de obtenção de relatórios de análise dos
equipamentos. Dessa forma, o operador pode verificar o histórico de degradação dos equipamentos e,
futuramente, utilizar os dados para tarefas de prognóstico.

Como forma de gerenciar as análises empregadas nos dispositivos, foi proposto o Gerenciador de
Análises. O gerenciador possibilita a criação de planos de análise para dispositivos individuais ou
em grupo. Esta entidade integra todo o processo de gerenciamento das análises a que os equipamentos
estão submetidos. É disponibilizado ao operador do sistema a definição das ferramentas que serão
utilizadas na análise dos dados, bem como a configuração de comportamentos em função do nível de
degradação obtido com a análise. Novamente, como no caso anterior, o Gerenciador de Análises é uma
ferramenta de configuração remota, permitindo, assim, o gerenciamento dos planos de análise de forma
facilitada.

As análises agendadas pelo operador no Gerenciador de Análises são executadas pela entidade
Analisador de Dispositivos. A arquitetura proposta dispões desta entidade dedicada para a execução
dos planos de análise dos equipamentos. O analisador verifica os dados dos dispositivos e executa as
análises com base no plano definido pelo operador do sistema. Como apresentado, o analisador utiliza
os algoritmos de análise presentes na ferramenta Watchdog Agent. Porém, levando em conta a forma com
o que software foi projetado, a extensão do módulo de análise para outras implementações comerciais
ou específicas é possível de forma facilitada.

Por fim, coma definição do estudo de caso e implementação dos dispositivos, foi possível testar a
interoperabilidade da solução proposta. O estudo de caso apresentado foi implementado na forma de
dispositivos \gls{DPWS}. A fim de facilitar os testes da arquitetura, os dados foram obtidos
anteriormente em situações distintas. Com isso, o dispositivo se comportou como um simulador,
gerando os dados dos sensores na mesma forma como foram obtidos.

Em relação aos experimentos propostos, verificou-se a vantagem no uso da arquitetura proposta em
relação a utilização dos métodos tradicionais para verificação da degradação em equipamentos
utilizando as técnicas de manutenção inteligente. Ficou claro que, nos métodos tradicionais,
utilizando somente o software Watchdog Agent, por exemplo, é necessário o emprego de um operador
qualificado para a configuração das ferramentas e obtenção dos resultados. Isso se deve ao fato de
que, para cada nova análise, o software precisa ser reconfigurado. Com as hipóteses levantadas, foi
possível concluir que a utilização de planos de análise, empregados nesta proposta, facilitam o
processo de análise dos dados, visto que, após a criação de um plano, o usuário não necessita
modificá-lo nem reconfigurar o software de análise. Também foi constatada a facilidade de obtenção
dos níveis de degradação. Ao passo que, nos métodos tradicionais o operador necessita coordenar a
análise, com a entidade proposta essas são feitas de maneira automática. O operador do sistema
necessita somente obter os relatórios com os valores calculados durante o período estipulado.

As vantagens na utilização da proposta são mais evidentes quando comparadas à configuração e análise
de múltiplos equipamentos. As dificuldades encontradas na utilização do software convencional para
obtenção dos níveis de degradação de um equipamento são expandidas quando se faz necessária a mesma
avaliação para um conjunto de dispositivos. Dessa forma, o operador necessita, a cada nova análise,
iniciar todo o processo de configuração das ferramentas e obtenção dos dados. Com a definição dos
planos de análise no Gerenciador de Análises, é possível a utilização de um mesmo plano para um
grupo de dispositivos. Os dispositivo que fazem parte do grupo de análise serão monitorados de forma
igual pelo Analisador de Dispositivos. Dessa forma, o analisador se encarrega de obter os níveis de
degradação de forma automática.

Visando a continuidade do desenvolvimento da proposta, pode-se citar alguns pontos verificados como
passíveis de melhoramento. Um deles diz respeito aos modos de aquisição de dados de treinamento. No
estado atual, o operador do sistema necessita enviar os dados de treinamento para os dispositivos de
forma manual. A cada novo treinamento que deseja-se incluir na base de dados, é necessária a
utilização das funcionalidades empregadas no Gerenciador de Dispositivos para envio dos dados.
Portanto, com o aumento no número de treinamentos, aumenta também a dificuldade de envio dos dados,
por representar tarefas repetitivas. Uma melhoria constatada está na utilização da entidade que
representa o dispositivo configurada em um modo específico de aquisição de dados de treinamento. Ao
contrário do que é utilizado na implementação atual, o dispositivo poderia ser configurado para
enviar os dados das aquisições diretamente para a base de dados como dados de treinamento.

O aumento no número de dispositivos também é considerado um problema na visão do Analisador de
Dispositivos. Se muitos dispositivos estão em processo de análise, o analisador manterá uma fila
demasiadamente grande. O processamento das análises enfileiradas pode representar atraso
considerável na obtenção dos valores de degradação. Considerando isso, a proposta de inclusão de
mais entidades para análise de dispositivos pode solucionar o problema. Uma possibilidade é a
comunicação ente os processos e verificação da disponibilidade. Dessa forma, evitando que a
sobrecarga dos planos de análise agendados recaiam sobre somente um ponto de processamento de dados.

Por último, como forma de melhorias, a reimplementação das ferramentas de análise se faz necessário.
Neste estudo, foram utilizadas as ferramentas que integram o software Watchdog Agent. Essas são
disponibilizadas em forma de arquivos de \textit{script} para execução no software Matlab. Pelo fato
de que arquivos deste tipo são interpretados, verificou-se perda considerável no desempenho em
função do aumento do lote de dados analisado. Dessa forma, propõe-se a reimplementação dos
algoritmos utilizando linguagem compiladas, como C ou C++, e, em pontos onde for possível,
processamento paralelo, como CUDA ou OpenCL~\cite{kirk2012programming}.

\end{onehalfspace}

\bibliographystyle{abnt}
\bibliography{bibliography/biblio}

\end{document}
