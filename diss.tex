\documentclass[oneside,diss]{deletex}

\usepackage[latin1]{inputenc}
\usepackage{parskip}
\usepackage{float}
\usepackage{outlines}
\usepackage{listings}
\usepackage{url}
\usepackage{amsmath}
\usepackage{booktabs}
\usepackage{siunitx}
\usepackage{mathptmx}
\usepackage[acronym,nonumberlist,sort=def]{glossaries}
\usepackage{setspace}
\usepackage{caption}
\usepackage[portuguese,shadow,textsize=footnotesize,backgroundcolor=yellow,bordercolor=black!50,
    linecolor=red,disable]{todonotes}
\usepackage{booktabs}
\usepackage{colortbl}
\usepackage{chngcntr}
\usepackage{blindtext}
\usepackage{hyperref}
\usepackage{cleveref}
%\usepackage[]{appendix}

\setlength{\parindent}{0.7cm}
\setlength{\parskip}{1.5pt plus0mm minus0mm}

\newcommand{\crefpairconjunction}{ e }
\newcommand{\crefmiddleconjunction}{, }
\newcommand{\creflastconjunction}{ e }
\newcommand{\crefrangeconjunction}{ a }

\crefname{figure}{figura}{figuras}
\Crefname{figure}{Figura}{Figuras}
\crefname{table}{tabela}{tabelas}
\Crefname{table}{Tabela}{Tabelas}
\crefname{equation}{equação}{equações}
\Crefname{equation}{Equação}{Equações}
\crefname{listing}{listagem}{listagens}
\Crefname{listing}{Listagem}{Listagens}
\crefname{chapter}{capítulo}{capítulos}
\Crefname{chapter}{Capítulo}{Capítulos}
\crefname{section}{seção}{seções}
\Crefname{section}{Seção}{Seções}
\crefname{subsection}{subseção}{subseções}
\Crefname{subsection}{Subseção}{Subseções}

% TODO: Conferir os dados do PDF.
\newcommand{\disstitle}{Proposta de arquitetura orientada a serviços para um sistema de manutenção
    inteligente}
\newcommand{\dissauthor}{Anderson Antônio Giacomolli}
\newcommand{\disssubject}{Proposta e desenvolvimento de uma arquitetura orientada a serviços para
    integração de equipamentos e sistemas de manutenção inteligente}
\newcommand{\disskeywords}{Sistema de Manutenção Inteligente, Arquitetura Orientada a Serviços,
    Integração de Sistemas, Automação Industrial}

%%
% Configuração do pacote para criação de links no PDF.
\hypersetup{%
  pdftitle={\disstitle},
  pdfauthor={\dissauthor},
  pdfproducer={\dissauthor},
  pdfsubject={\disssubject},
  pdfkeywords={\disskeywords},
  bookmarks=true,
  pdfmenubar=true,
  hidelinks=true
}

%%
% Configuração do pacote para representação numérica.
\sisetup{%
  detect-all,
  output-decimal-marker={,},
  inter-unit-product=\ensuremath{{}}
}

%%
% Configuração do pacote para inclusão das listagens.
% Configura o título da lista de listagens.
\renewcommand{\lstlistlistingname}{Listagens}

% Configura a legenda das listagens.
\renewcommand{\lstlistingname}{Listagem}

\makeatletter
\def\l@lstlisting#1#2{\@dottedtocline{1}{0em}{1em}{Listagem #1}{#2}}
\makeatother

% Define um estilo para códigos Java.
\lstdefinestyle{javacodestyle}{
  language=java,
  basicstyle=\ttfamily\small,
  keywordstyle=\color{black}\ttfamily\bfseries,
  backgroundcolor=\color{gray!10},
  captionpos=t,
  %float=!h,
  numberbychapter=false,
  showstringspaces=false,
  %columns=fullflexible
}

% Comando para inserir um bloco de codigo Java com legenda.
\newcommand{\includejavacode}[3]{%
  \lstinputlisting[%
      style=javacodestyle,%
      caption={#2},%
      label={#3}]{%
    #1
  }
}

\newglossary{symbols}{sym}{sbl}{List of symbols}

%%
% Definição do padrão para a lista de abreviaturas.
\newglossarystyle{ppgee}{%
  \setlength{\glsdescwidth}{0.8\linewidth}%
  \renewcommand*{\arraystretch}{1.5}%
  \renewenvironment{theglossary}%
    {\tablehead{}\tabletail{}%
      \begin{supertabular}{lp{\glsdescwidth}}}%
    {\end{supertabular}}%
  \renewcommand*{\glsnamefont}[1]{{\mdseries ##1}}%
  \renewcommand*{\glsgroupskip}{}%
  \renewcommand*{\glossaryheader}{}%
  \renewcommand*{\glspostdescription}{}%
  \renewcommand*{\glsgroupheading}[1]{}%
  \renewcommand*{\glossaryentryfield}[5]{%
    \glsentryitem{##1}\glstarget{##1}{##2} & ##3\glspostdescription\space ##5\\}%
  \renewcommand*{\glossarysubentryfield}[6]{%
    &
    \glssubentryitem{##2}%
    \glstarget{##2}{\strut}##4\glspostdescription\space ##6\\}%
  %\renewcommand*{\glsgroupskip}{ & \\}%
}

%%
% Comando para inserção de uma imagem.
\newcommand{\includefigure}[3]{%
  \begin{figure}[ht!]
    \centering
    \includegraphics{#1}
    \caption{#2}
    \label{#3}
  \end{figure}
}

%%
% Comando para inserção de uma imagem dos softwares.
\newcommand{\includefiguresoft}[3]{%
  \begin{figure}[ht!]
    \centering
    \includegraphics[scale=0.5]{#1}
    \caption{#2}
    \label{#3}
  \end{figure}
}

%%
% Comando para inserção de uma imagem temporária.
\newcommand{\includefiguretmp}[2]{%
  \begin{figure}[h]
    \centering
    \rule{12cm}{6cm}
    \caption{#1}
    \label{#2}
  \end{figure}
}

%%
% Comando para inserção de uma tabela.
\newcommand{\includetable}[3]{%
  \begin{table}[ht!]
    \centering
    \caption{#2}
    \input{#1}
    \label{#3}
  \end{table}
}

%%
% Título e autor do documento.
\title{\disstitle}
\author{Giacomolli}{Anderson Antônio}

%%
% Orientador.
\advisor[Prof.~Dr.]{Pereira}{Carlos Eduardo}
\advisorinfo{UFRGS}{Doutor pela Sttutgart University -- Sttutgart, Alemanha}
\advisorwidth{0.56\textwidth}

%%
% Banca examinadora
\examiner[Prof.~Dr.]{Antônio Barata de Oliveira}{José}
\examinerinfo{UNINOVA}{Doutor pela Universidade Nova de Lisboa -- Lisboa, Portugal}

\examiner[Prof.~Dr.]{Ventura Bayan Henriques}{Renato}
\examinerinfo{UFRGS}{Doutor pela Universidade Federal de Minas Gerais -- Belo Horizonte, Brasil}

\examiner[Prof.~Dr.]{João Brusamarello}{Valner}
\examinerinfo{UFRGS}{Doutor pela Universidade Federal de Santa Catarina -- Florianópolis, Brasil}

%%
% Data da defesa.
\date{março}{2014}

%%
% Área de concentração.
\topic{\ca}

%%
% Palavras-chave.
\keyword{Sistema de Manutenção Inteligente}
\keyword{Arquitetura Orientada a Serviços}
\keyword{Integração de Sistemas}
\keyword{Automação Industrial}

%%
% Cria os glossários.
\makeglossaries

%%
% Inclui o arquivo com a lista de abreviaturas.
\newacronym{ARMA}
  {ARMA}
  {Auto-Regressive Moving-Average}

\newacronym{CSV}
  {CSV}
  {Comma-Separated Values}

\newacronym{CV}
  {CV}
  {Confidence Value}

\newacronym{CWT}
  {CWT}
  {Continuous Wavelet Transform}

\newacronym{DPWS}
  {DPWS}
  {Devices Profile for Web-Services}

\newacronym{DWT}
  {DWT}
  {Discrete Wavelet Transform}

\newacronym{FPGA}
  {FPGA}
  {Field-Programmable Gate Array}

\newacronym{GSM}
  {GSM}
  {Global System for Mobile communication}

\newacronym{HAVi}
  {HAVi}
  {Home Audio/Video Interoperability}

\newacronym{HTTP}
  {HTTP}
  {Hypertext Transfer Protocol}

\newacronym{IEEE}
  {IEEE}
  {Institute of Electrical and Electronics Engineers}

\newacronym{IHM}
  {IHM}
  {Interface Homem-Máquina}

\newacronym{IMS}
  {IMS}
  {Intelligent Maintenance System}

\newacronym{IMSCenter}
  {IMS~Center}
  {Intelligent Maintenance System Center}

\newacronym{IP}
  {IP}
  {Internet Protocol}

\newacronym{IRI}
  {IRI}
  {Internationalized Resource Identifier}

\newacronym{JAR}
  {JAR}
  {Java Archive}

\newacronym{JINI}
  {JINI}
  {Java Intelligent Network Infrastructure}

\newacronym{JMEDS}
  {JMEDS}
  {Java Multi Edition DPWS Stack}

\newacronym{LAN}
  {LAN}
  {Local Area Network}

\newacronym{MTOM}
  {MTOM}
  {SOAP Message Transmission Optimization Mechanism}

\newacronym{OASIS}
  {OASIS}
  {Organization for the Advancement of Structured Information Standards}

\newacronym{OPCUA}
  {OPC~UA}
  {OPC Unified Architecture}

\newacronym{OSACBM}
  {OSA-CBM}
  {Open Systems Architecture for Condition-Based Maintenance}

\newacronym{OSGi}
  {OSGi}
  {Open Service Gateway initiative}

\newacronym{QoS}
  {QoS}
  {Quality of Service}

\newacronym{RL}
  {RL}
  {Regressão Logística}

\newacronym{SIRENA}
  {SIRENA}
  {Service Infrastructure for Real Time Embedded Networked Applications}

\newacronym{SOA}
  {SOA}
  {Service-Oriented Architecture}

\newacronym{SOAP}
  {SOAP}
  {Simple Object Access Protocol}

\newacronym{TCP}
  {TCP}
  {Transmission Control Protocol}

\newacronym{TDMA}
  {TDMA}
  {Time Division Multiple Access}

\newacronym{UDDI}
  {UDDI}
  {Universal Description Discovery Integration}

\newacronym{UDP}
  {UDP}
  {User Datagram Protocol}

\newacronym{UML}
  {UML}
  {Unified Modeling Language}


\newacronym{UPnP}
  {UPnP}
  {Universal Plug and Play}

\newacronym{URI}
  {URI}
  {Uniform Resource Identifier}

\newacronym{URL}
  {URL}
  {Uniform Resource Locator}

\newacronym{WAN}
  {WAN}
  {Wide Area Network}

\newacronym{WPAN}
  {WPAN}
  {Wireless Personal Area Network}

\newacronym{WPE}
  {WPE}
  {Wavelet Packet Energies}

\newacronym{WSA}
  {WSA}
  {Web Service Architecture}

\newacronym{WSD}
  {WSD}
  {Web Service Description}

\newacronym{WSDL}
  {WSDL}
  {Web Service Description Language}

\newacronym{W3C}
  {W3C}
  {World Wide Web Consortium}

\newacronym{XML}
  {XML}
  {Extensible Markup Language}

\newglossaryentry{time-instant}{%
  type = {symbols},
  name = {\ensuremath{t}},
  description = {Instante de tempo}
}

\newglossaryentry{sample-index-discrete}{%
  type = {symbols},
  name = {\ensuremath{n}},
  description = {Índice de amostragem discreto}
}

\newglossaryentry{dilatation-param}{%
  type = {symbols},
  name = {\ensuremath{\beta}},
  description = {Variação da dilatação}
}

\newglossaryentry{scale-param}{%
  type = {symbols},
  name = {\ensuremath{\kappa}},
  description = {Variação da escala}
}

\newglossaryentry{dimension-index}{%
  type = {symbols},
  name = {\ensuremath{k}},
  description = {Dimensão do espaço}
}

\newglossaryentry{logistic-regression-input}{%
  type = {symbols},
  name = {\ensuremath{r}},
  description = {Vetor de entrada do modelo de regressão logística}
}

\newglossaryentry{logistic-regression-output}{%
  type = {symbols},
  name = {\ensuremath{y}},
  description = {Saída do modelo de regressão logística}
}

\newglossaryentry{wavelet-mother}{%
  type = {symbols},
  name = {\ensuremath{\psi}},
  description = {Função wavelet mãe}
}

\newglossaryentry{wavelet-transform-continuous}{%
  type = {symbols},
  name = {\ensuremath{\mathcal{W}}},
  description = {Tranformada wavelet contínua}
}

\newglossaryentry{wavelet-transform-discrete}{%
  type = {symbols},
  name = {\ensuremath{\mathcal{V}}},
  description = {Tranformada wavelet discreta}
}

\newglossaryentry{signal-time-continuous}{%
  type = {symbols},
  name = {\ensuremath{x(t)}},
  description = {Sinal contínuo no domínio tempo}
}

\newglossaryentry{signal-time-discrete}{%
  type = {symbols},
  name = {\ensuremath{x[n]}},
  description = {Sinal discreto no domínio tempo}
}

\newglossaryentry{wavelet-scale-param}{%
  type = symbols,
  name = {\ensuremath{\alpha}},
  description = {Parâmetro de dilatação},
}

\newglossaryentry{wavelet-translation-param}{%
  type = symbols,
  name = {\ensuremath{\tau}},
  description = {Parâmetro de translação}
}

\newglossaryentry{funcao-valor-confianca}{%
  type=symbols,
  name={\ensuremath{{P(y = 1 | x)}}},
  %name=pi,
  %symbol={\ensuremath{\Omega}},
  description=Função probabilidade para o valor de confiança
}

\newglossaryentry{sinal-continuo-frequencia}{%
  type=symbols,
  name={\ensuremath{X(\omega)}},
  %name=pi,
  %symbol={\ensuremath{\Omega}},
  description=Sinal contínuo no domínio frequência
}

\newglossaryentry{sinal-discreto-frequencia}{%
  type=symbols,
  name={\ensuremath{X[n]}},
  %name=pi,
  %symbol={\ensuremath{\Omega}},
  description=Sinal discreto no domínio frequência
}


%%%%%%%%%%%%%%%%%%%%%%%%%%%%%%%%%%%%%%%%%%%%%%%%%%%%%%%%%%%%%%%%%%%%%%%%%%%%%%%%
%%%%%%%%%%%%%%%%%%%%%%%%%%%%%%%%%%%%%%%%%%%%%%%%%%%%%%%%%%%%%%%%%%%%%%%%%%%%%%%%
\begin{document}

\counterwithout{lstlisting}{chapter}

% O comando \maketile gera a capa, a folha de rosto e a folha de aprovacao
% (se for o caso)
% às vezes é necessário redefinir algum comando logo antes de produzir
% a Capa, folha de rosto e folha de aprovacao:
% \renewcommand{\coordname}{Coordenadora do Curso}
\maketitle

%%%%%%%%%%%%%%%%%%%%%%%%%%%%%%%%%%%%%%%%%%%%%%%%%%%%%%%%%%%%%%%%%%%%%%%%%%%%%%%%
%%%%%%%%%%%%%%%%%%%%%%%%%%%%%%%%%%%%%%%%%%%%%%%%%%%%%%%%%%%%%%%%%%%%%%%%%%%%%%%%
%\chapter*{Dedicatória}


%%%%%%%%%%%%%%%%%%%%%%%%%%%%%%%%%%%%%%%%%%%%%%%%%%%%%%%%%%%%%%%%%%%%%%%%%%%%%%%%
%%%%%%%%%%%%%%%%%%%%%%%%%%%%%%%%%%%%%%%%%%%%%%%%%%%%%%%%%%%%%%%%%%%%%%%%%%%%%%%%
%\chapter*{Agradecimentos}


% Resumo no idioma do documento.
%%%%%%%%%%%%%%%%%%%%%%%%%%%%%%%%%%%%%%%%%%%%%%%%%%%%%%%%%%%%%%%%%%%%%%%%%%%%%%%%
%%%%%%%%%%%%%%%%%%%%%%%%%%%%%%%%%%%%%%%%%%%%%%%%%%%%%%%%%%%%%%%%%%%%%%%%%%%%%%%%
\begin{abstract}
  % TODO

\end{abstract}

% Abstract em inglês.
\begin{englishabstract}{Intelligent Maintenance Systems, Service-Oriented Architectures, Systems
    Integration, Industrial Automation}
  In the industrial field, the costs associated with equipment's maintenace still represents a large
portion of the resources available to a company. Therefore, new researches in maintenance systems,
and the correct task planning, are growing, since they impact directly on the economic side of the
companies. Thus, current work presents an service-oriented architecture for maintenance systems
integration. The proposed architecture intends to facilitate the integration of equipments and
degradation analysis tools. The architecture is comprised of several entities, where each one is
responsible for executing a given task in order to keep the system running properly. In this work,
all the entities are implemented and experiments are performed in order to validate the proposed
system.

\end{englishabstract}

% Lista de ilustrações.
\listoffigures

% Lista de tabelas.
\listoftables

% Lista de códigos.
%\lstlistoflistings

% Lista de abreviaturas e siglas.
%\begin{listofabbrv}{OSA-CBM}
%	\item[ABNT] Associação Brasileira de Normas Técnicas
%	\item[GCAR] Grupo de Controle, Automação e Robótica
%	\item[PPGEE] Programa de Pós-Graduação em Engenharia Elétrica
%	\item[OSA-CBM] Programa de Pós-Graduação em Engenharia Elétrica
%\end{listofabbrv}

% Lista de abreviaturas e siglas.
\setglossarysection{chapter}
\printglossary[style=ppgee,type=\acronymtype,title=Lista de abreviaturas]

% lista de símbolos é opcional
%\begin{listofsymbols}{$\alpha\beta\pi\omega$}
%       \item[$\sum$] Somatório
%       \item[$\alpha\beta\pi\omega$] Fator de inconstância do resultado
%\end{listofsymbols}

% Lista de símbolos.
%\setglossarysection{chapter}
\printglossary[style=ppgee,type=symbols,title=Lista de símbolos]
%\printglossaries

% Sumário.
\tableofcontents

%%%%%%%%%%%%%%%%%%%%%%%%%%%%%%%%%%%%%%%%%%%%%%%%%%%%%%%%%%%%%%%%%%%%%%%%%%%%%%%%
%%%%%%%%%%%%%%%%%%%%%%%%%%%%%%%%%%%%%%%%%%%%%%%%%%%%%%%%%%%%%%%%%%%%%%%%%%%%%%%%
\begin{onehalfspace}
  \chapter{Introdução}

Atualmente, a importância do emprego de técnicas de manutenção no âmbito industrial está em
constante ascensão devido à necessidade de aumentar a disponibilidade e segurança dos equipamentos,
bem como a qualidade do processo produtivo~\cite{muller2008formalisation}. O custo empregado
anualmente com processos de manutenção está na faixa de 15\% para a indústria de manufatura, entre
20\% a 30\% para a indústria química e na faixa de 40\% para a indústria do aço e siderúrgica, como
ilustrado nos trabalhos de~\cite{chu1998predictive} e \cite{nguyen2008new}. Dessa forma, o
desenvolvimento de novas técnicas de manutenção para uso nas mais diversas áreas e o correto
planejamento dos processos de manutenção estão cada vez mais importantes, uma vez que impactam
diretamente no fator econômico, alterando a disponibilidade do sistema e também a
segurança~\cite{zhao2010soabased}.

Nos últimos anos, tem-se observado um crescimento no uso de um novo paradigma de manutenção
denominado de manutenção inteligente~\cite{zhang2013performance}. Este novo paradigma visa
transformar a forma como as técnicas de manutenção são utilizadas. Diferentemente dos métodos
tradicionais, conhecidos por aplicar o conserto aos equipamento somente após a falha ou por manterem
processos de manutenção agendados baseado no histórico de falhas dos componentes, o paradigma de
manutenção inteligente visa predizer a condição do sistema e prevenir uma possível falha.
Segundo~\cite{bloch2012machinery}, 99\% das falhas em sistemas mecânicos podem ser observadas por
indicadores perceptíveis. Consequentemente é possível a utilização de técnicas de manutenção
inteligente empregadas no monitoramento contínuo da saúde do sistema, de forma a não interromper a
operação dos equipamentos.

As tecnologias empregadas na manutenção contínua do sistema, bem como diagnóstico de falhas, tiveram
grande desenvolvimento nas últimas décadas e visam predizer o estado do
sistema~\cite{heng2009rotating}. Assinaturas de sinais de vibração e de emissão acústica puderam ser
obtidas, processadas e analisadas através de sensores e softwares computacionais. Novas pesquisas
nestas áreas estão em constante evolução, utilizando técnicas modernas para análise e processamento
de sinais. Isso se torna possível com os avanços da eletrônica e computação, que cada vez mais
propiciam ferramentas e técnicas para a resolução de problemas~\cite{zhao2010predictive}. Como
exemplo, pode-se citar o uso de métodos baseados em redes neurais ou de mapas auto-organizáveis para
detecção de padrões em sinais~\cite{goncalves2011fault}.

Outra linha de pesquisa também está em constante crescimento: o uso de \gls{SOA} ou Arquiteturas
Orientadas a Serviços. O uso do padrão \gls{SOA} está evoluindo e está cada vez mais presente em
aplicações nos mais diversos segmentos, sejam eles a nível de dispositivos, na implementação de
camadas de negócios ou mesmo no setor industrial, como apresentado em~\cite{candido2010soa},
\cite{choi2010impact}, \cite{ragavan2012service} e \cite{papazoglou2007service}. É um conceito de
arquitetura que suporta acoplamento mínimo entre componentes, possibilitando ganhos em flexibilidade
e interoperabilidade. Por conseguinte, qualquer tipo de aplicação pode ser representada como um
conjunto complexo de serviços.

Com a utilização de \gls{SOA}, um recurso ou componente é identificado como um serviço. Cada
entidade apresenta comportamento bem definido e é composta por módulos autocontidos, os quais
permitem que um determinado serviço seja independente do estado ou contexto de outros
serviços~\cite{papazoglou2007service}. As funcionalidades agregadas a um serviço são publicadas e
disponibilizadas através de uma interface padrão, o que possibilita a troca de informações ou
requisição da execução de alguma tarefa entre os componentes~\cite{ragavan2012service}.

O uso de \gls{SOA} no âmbito industrial, como forma de integração dos sistemas, se mostra factível
pelos sucesso de vários projetos, dos quais pode-se citar~\cite{karnouskos2010towards},
\cite{bohn2006sirena}, \cite{de2006soda} e \cite{colombo2010factory}. Os projetos demonstram a
viabilidade na utilização de serviços em sistemas embarcados a fim de integrá-los com sistemas
\gls{MES} e \gls{ERP}, localizados nos níveis mais altos da corporação. A utilização de \gls{SOA} em
ambientes industriais possibilita o aumento da flexibilidade do sistema, resultado em rápida
adaptação para situações onde são impostas demandas do mercado~\cite{starke2013flexible}. A
configuração de equipamentos de forma flexível utilizando
\gls{SOA} possibilita aumento na agilidade como os processos desta natureza são executados.


%Motivação

Nesse contexto, um sistema de manutenção inteligente também pode se valer da utilização dos
conceitos empregados pelo padrão \gls{SOA}. Do ponto de vista da arquitetura \gls{SOA}, o sistema de
manutenção pode conter serviços para relatórios de saúde e falhas, informações sobre o prognóstico
do tempo de operação sem necessidade de manutenção, além de serviços de configuração de ferramentas
de diagnóstico ou dos modos de operação suportados pelo equipamento. O monitoramento remoto do
sistema também possibilita a integração das informações de saúde dos equipamentos em sistemas
\gls{MES} e \gls{ERP}, a fim de se obter o correto gerenciamento da cadeia de suprimentos de peças
de reposição~\cite{oldham2003delivering}.


%Proposta do trabalho

Mesmo que as pesquisas envolvendo as áreas de sistemas de manutenção e arquiteturas orientadas a
serviços estejam em constante evolução, as iniciativas para integração das duas tecnologias ainda
são escassas. Isso posto, o trabalho em questão apresenta a proposta de uma arquitetura orientada a
serviços para um sistema de manutenção inteligente. A proposta tem por objetivo possibilitar a
integração de equipamentos com um sistema de manutenção inteligente de forma facilitada e flexível.
Portanto, todas as entidades propostas foram construídas obedecendo os padrões definidos pelas
arquiteturas \gls{SOA}.

A arquitetura proposta engloba diversas entidades, cada uma destinada a um propósito específico. As
entidades foram construídas conforme a demanda encontrada para a integração entre os sistemas aqui
descritos. Dentre as principais, estão o Analisador de Dispositivos, que possibilita a análise
automáticas de equipamentos e o Gerenciador de Análises, o qual permite a criação de planos de
análise pelo operador do sistema, com a definição da utilização dos dados e ferramentas de análise.


%Objetivos

Como objetivos do trabalho, estão a definição da arquitetura proposta e implementação das entidades.
A validação do sistema é feita com a implementação de um dispositivo que representa um conjunto
atuador elétrico e válvula, utilizado para controle de fluxo em redes de distribuição de petróleo.
Com a definição do dispositivo que representa o atuador elétrico, o sistema é posto em
funcionamento, onde o operador do sistema tem acesso às configurações dos equipamentos e planos de
análise, podendo criar novos planos e verificar o resultado dos níveis de degradação obtidos com as
análises.

Por fim, são ilustrados experimentos para determinar a viabilidade da solução proposta. Os
experimentos tem por objetivo verificar a interoperabilidade entre as entidades propostas, bem como
determinar o correto funcionamento dos componentes implementados. São analisados pontos positivos em
relação à utilização do sistema proposto, tendo em vista a comparação com a utilização de métodos
tradicionais de obtenção dos níveis de degradação em equipamentos. Também são verificadas as
vantagens na utilização da arquitetura proposta em relação ao aumento do número de dispositivos
monitorados, levando em conta o gerenciamento de vários dispositivos similares para a obtenção dos
estados de saúde de todos.


% Estrutura

Dando continuidade ao capítulo introdutório, esta dissertação está estruturada da seguinte forma: no
\cref{cha:conceituacao-teorica} são apresentados os conteúdos teóricos utilizados ao longo do
trabalho, divididos nas duas áreas de estudo, sendo elas a de manutenção inteligente e também
arquiteturas orientadas a serviços; a análise do estado da arte é apresentada no
\cref{cha:estado-arte}, onde são descritos os estudos atuais envolvendo os dois assuntos
pertinentes a este trabalho; a proposta de uma arquitetura orientada a serviços faz parte do
\cref{cha:arquitetura-proposta}, onde é apresentado o projeto dos componentes utilizados no sistema
proposto, além da definição do estudo de caso e dos experimentos para validação; no
\cref{cha:implementacao-resultados} são apresentados os resultados obtidos com a implementação do
sistema proposto e o resultado dos experimentos definidos no capítulo anterior; por fim, o
\cref{cha:conclusao} apresenta as conclusões e pontos relevantes para a continuidade do trabalho.

  \chapter{Conceituação teórica}

\todo[inline]{Introdução do capítulo.}


%%
\section{Sistemas de manutenção inteligente}

Manutenção, no âmbito geral, consiste em uma série de medidas de prevenção, correção e predição de
falhas~\cite{lee2006intelligent}. Durante o uso, equipamentos ou máquinas tendem a deteriorar e
alterar o seu padrão de funcionamento devido a diversos fatores, como, por exemplo, desgaste,
rachaduras, corrosão e sujeira. Nestas condições, a restauração do sistema é de suma importância,
visto que, com o passar do tempo, podem apresentar defeitos e levar à falhas e indisponibilidades.
De acordo com~\cite{marcal2005detectando}, manutenção pode ser definida como todas as atividades
técnicas e organizacionais que garantam a operação das máquinas e equipamentos dentro da
confiabilidade esperada.

Tradicionalmente, são encontrados na literatura três\todo{verificar} tipos de estratégias de
manutenção: corretiva, preventiva e preditiva~\cite{goncalves2011desenvolvimento}. A manutenção
corretiva visa reestabelecer os sistemas danificados; a preventiva tem por objetivo manter os
sistemas em funcionamento, realizando pequenas correções; e a manutenção preditiva tem por base o
monitoramento do estado do sistema, detectando falhas insipientes e fornecendo subsídios para o
planejamento de ações de prevenção ou correção. Em termos gerais, a manutenção corretiva é aplicada
somente quando há falha e o sistema necessita de reposição de peças ou componentes para continuar
operando corretamente, enquanto que a manutenção preventiva visa o agendamento programado de
intervenções no sistema, afim de manter o funcionamento pelo maior tempo possível. Por outro lado, a
manutenção preditiva tem como foco o monitoramento do sistema continuamente e, desta forma, a
intervenção é feita somente quando necessário.

Nas três estratégias, pode-se citar vantagens e desvantagens. Na manutenção corretiva, a principal
vantagem está na dispensabilidade de realização de acompanhamentos ou inspeções no sistema. Isso
evita a geração de custos na alocação de pessoas ou equipamentos para desempenharem tarefas de
verificação do sistema, além da parada da linha de produção em intervalos agendados. Por outro lado,
a parada inesperada da linha de produção para uma manutenção emergencial pode gerar transtornos e
custos não programados. A manutenção preventiva visa sanar os problemas de paradas inesperadas,
utilizando-se de um modelo de agendamento das inspeções. O que para muitas situações é considerado
suficiente, se não for bem planejado, pode acarretar em custos excessivos devido às paradas
programadas e alocação de equipes de manutenção. Nesta situação, a estratégia de manutenção
preditiva busca o meio termo, utilizando-se da predição do estado do sistema, como análise de
tendências ou avaliações probabilísticas do estado de degradação dos equipamentos, para os
agendamentos de novas intervenções.

Mesmo com a programação das intervenções, as máquinas podem falhar de modo repentino, pondo em risco
os equipamentos e pessoas envolvidas com o processo produtivo~\cite{goncalves2011desenvolvimento}. A
falha no intervalo entre intervenções não é possível de prever através dos métodos clássicos de
manutenção. Logo, nos últimos anos, o que tem se visto é a substituição da estratégia de manutenção
preventiva por um novo paradigma: a manutenção proativa~\cite{lee2009informatics}. Esta nova
estratégia visa não somente a predição do estado do sistema, mas também o diagnóstico das falhas e,
em casos onde é aplicado, a intervenção de forma automática. Por intervenção, entende-se que o
padrão de funcionamento dos equipamentos monitorados pode ser alterado, visando minimizar os
possíveis agravantes até a realização da manutenção.

Neste cenário, emergem os sistemas de manutenção inteligente. Também conhecidos como sistemas de
manutenção baseados no conhecimento, visam capturar o conhecimento de um determinado sistema sob a
forma de regras e utilizá-las para construir um novo sistema baseado nestas regras. O novo sistema
é, então, utilizado para realização de um correto diagnóstico ou tomada de ação no caso da
ocorrência de algum defeito. Como exemplo, em~\cite{shikari2004automation} o padrão de vibração de
uma máquina de indução, de um atuador e de uma prensa são analisados e, realizado o diagnóstico
automático através de um sistema de manutenção inteligente, é determinado o motivo da falha, podendo
ser os rolamentos ou desalinhamentos.

\includefigure
    {images/maintenance-strategies}
    {Classificação das estratégias de manutenção.}
    {fig:maintenance-strategies}
\todo{Referenciar figura no texto.}

Com o intuito de auxiliar na migração do paradigma de conserto após falha para o paradigma de
predição e prevenção, foi criado, nos Estados Unidos, um centro de parceria entre universidades e
empresas, denominado \gls{IMSCenter}. Dentre as empresas integrantes da parceria \gls{IMSCenter},
pode-se citar, por exemplo, Boeing, Siemens, AMD, Toyota e Caterpillar. Entre as universidades,
fazem parte do consórcio a de Cincinnatti, Missouri-Rolla e Michigan.


%%%
\subsection{A ferramenta Watchdog Agent}

Um dos objetivos da parceria \gls{IMSCenter} foi o desenvolvimento de uma metodologia para abordagem
dos problemas de manutenção utilizando predição e prevenção. Para tanto, foi desenvolvido um
conjunto de ferramentas de análise denominado Watchdog Agent. Em termos gerais, o Watchdog Agent é
uma ferramenta de análise de desempenho. Aplicado a determinado equipamento, visa analisar sinais de
diversas partes da máquina, a fim de obter um índice de desempenho.

A extração das informações contidas nos sinais analisados são extraídas através das ferramentas
implementadas no Watchdog Agent. Primeiramente os dados dos sensores do equipamento são adquiridos.
Em um segundo momento, os dados são classificados com o auxílio de algoritmos. Com os dados
classificados, é possível determinar o índice de desempenho para a situação analisada. Estas etapas
são ilustradas na \cref{fig:data-processing-plot}.

\includefigure
    {images/data-processing-plot}
    {Processamento das informações utilizando a estratégia proposta pelo IMS~Center.}
    {fig:data-processing-plot}

A medida que o equipamento degrada, o índice de desempenho é alterado em comparação com o mesmo
indicador obtido com o equipamento em funcionamento normal. Um indicador normalmente utilizado para
identificação do estado de um equipamento é o valor de confiança. Este indicador é definido como uma
grandeza que varia no intervalo~${[0; 1]}$. Valores próximos a~\num{1} representam funcionamento
normal do sistema, enquanto que valores próximos a~\num{0} equivalem a um funcionamento em falha. A
\cref{fig:confidence-value-concept} ilustra o conceito de valor de confiança. As duas curvas da
esquerda apresentam o comportamento normal e o comportamento recente de um determinado equipamento.
Ao cruzar as duas informações, é possível obter o valor de confiança, exemplificado no gráfico da
direita. À medida que o valor de confiança decai, a probabilidade de redução do desempenho do
sistema aumenta~\cite{djurdjanovic2003watchdog}.

\includefigure
    {images/confidence-value-concept}
    {Representação do conceito de valor de confiança.}
    {fig:confidence-value-concept}

%\missingfigure{Estrutura do Watchdog Agent.}

%\todo[inline]{Uma das vantagens no uso de técnicas proativas...}

Em comparação com estratégias de manutenção preventiva, um ponto importante a ser citado é aumento
da vida útil de peças de equipamentos~\cite{lazzaretti2012avaliacao}. Peças que poderiam ser
descartadas em função de uma intervenção preventiva\todo{terminar}.


%%%
\subsection{Modelo OSA-CBM}

Como proposta de padronização de uma arquitetura aberta para troca de informações em um sistema
baseado em condição, surge o modelo \gls{OSACBM}~\cite{thurston2001open}. A arquitetura \gls{OSACBM}
visa facilitar a integração e interoperabilidade entre componentes e equipamentos de diferentes
fabricantes. Definida em sete camadas, possibilita a abstração de várias partes envolvidas em um
sistema de manutenção inteligente. A \cref{fig:osa-cbm-model} apresenta uma visão geral das camadas
do modelo juntamente com as suas interações. As camadas são numeradas de 1~(aquisição de dados) a
7~(apresentação).

%\todo[color=yellow, inline]{Outro todo.}

\includefigure
    {images/osa-cbm-model}
    {Modelo OSA-CBM.}
    {fig:osa-cbm-model}

A definição das funcionalidades de cada camada é apresentada por~\cite{thurston2001open}. Na camada
de aquisição de dados, as grandezas físicas são convertidas para sinais elétricos e digitalizadas. O
módulo consiste, normalmente, de um elemento sensor e um elemento de aquisição de dados. Além da
conversão física, a camada também pode armazenar os dados coletados em um banco de dados. A primeira
etapa de cálculos sob os dados obtidos é feita na camada de manipulação dos dados. Através do uso de
ferramentas de processamento de sinais, os dados adquiridos na camada anterior são manipulados,
podendo gerar resultados no domínios tempo, frequência ou tempo-frequência. Eventualmente, os
resultados das operações também podem ser armazenados em um banco de dados. O módulo de detecção do
estado do sistema analisa continuamente os indicadores de cada sistema, subsistema ou componente. De
posse dos dados processados pelas camadas anteriores, ao calcular os indicadores de estado, o módulo
de detecção pode gerar alarmes respeitando condições previamente estabelecidas. Novamente os dados
obtidos podem ser armazenados para uso posterior. Na camada de avaliação da saúde do sistema, o
resultado dos indicadores, obtidos no módulo de detecção do estado do sistema, são inseridos no
contexto das operações. A saúde do sistema monitorado é avaliada pelo uso dos indicadores atuais e
passados. Dessa forma, também é possível armazenar os resultados formando um histórico do
equipamento monitorado. Na camada de prognóstico, a saúde futura do sistema é estimada. Através de
um modelo estimado do sistema e dos dados obtidos nas camadas anteriores, o tempo de vida útil ou a
probabilidade de falha em um horizonte de predição são estimados. Como nas camadas anteriores, os
resultados podem ser armazenados em um banco de dados. O módulo de tomada de decisão utiliza os
dados obtidos na camada de prognóstico, além de outras informações, para sugerir ações recomendadas
de acordo com as implicações das decisões. São integrados, juntamente com os dados da camada de
prognóstico, informações de restrições externas, requisitos de funcionalidades do equipamento ou
sistema, condições financeiras, entre outros. No nível mais alto do modelo, está a camada de
aplicação. Definida como a interface homem-máquina do sistema, visa a apresentação dos dados obtidos
no processamento das informações. Nesta camada também podem ser utilizadas técnicas de realidade
aumentada~\cite{espindola2011realidade}.


%%%
\subsection{Algoritmos de processamento da ferramenta Watchdog Agent}

Como mencionado anteriormente, o Watchdog Agent é um conjunto de algoritmos para processamento de
sinais e extração de características\todo{Terminar}.


%%%%
\subsubsection{Energias da transformada Wavelet Packet}

Uma das forma de se analisar sinais não estacionários no domínio tempo-frequência é utilizando a
transformada wavelet~\cite{antonini1992image}. Seu uso é indicado para sinais que apresentam
descontinuidades, tendências entre outros. É empregado nas mais diversas aplicações, desde a remoção
de ruídos em sinais ou imagens até a compressão de imagens médicas com pouca perda de qualidade.

Wavelets são formas de onda oscilantes com duração limitada e valor médio zero. São empregadas na
forma de wavelets mãe, definidas por \gls{wavelet-mother}. A função wavelet mãe pode ser dilatada ou
comprimida através de um parâmetro \gls{wavelet-scale-param} e transladada através de um
parâmetro \gls{wavelet-translation-param}. A mudança de escala e translação são apresentadas na
\cref{eq:wavelet-translation-compression}.

\begin{equation}
  \psi_{\alpha, \tau}(t) = \frac{1}{\sqrt{\alpha}} \psi \left ( \frac{t - \tau}{\alpha} \right )
  \label{eq:wavelet-translation-compression}
\end{equation}

A \cref{eq:wavelet-definition-continuous} apresenta a definição da transformada wavelet contínua
\gls{wavelet-transform-continuous} de um sinal contínuo \gls{signal-time-continuous}.

\begin{equation}
  \mathcal{W} \left \{ x, \psi \right \} =
      \frac{1}{\sqrt{\alpha}} \left \{ x(t), \psi_{\alpha, \tau}(t) \right \} =
      \frac{1}{\sqrt{\alpha}} \int_{-\infty}^{\infty} x(t) \cdot
        \psi^{*} \left ( \frac{t - \tau}{\alpha} \right ) \textrm{d}t
  \label{eq:wavelet-definition-continuous}
\end{equation}

A largura da wavelet é influenciada pelo fator de escala \gls{wavelet-scale-param}, o que também
contribui para a alteração da resolução empregada na análise. Quanto menor o valor de
\gls{wavelet-scale-param}, maior será a resolução empregada na detecção de eventos de alta
frequência. No caso contrário, quanto maior for o valor de \gls{wavelet-scale-param}, maior será a
dilatação empregada na wavelet mãe, o que é conveniente para a identificação de padrões de baixa
frequência. A \cref{fig:wavelet-time-frequency-representation} ilustra a mudança de escala para
análise do sinal em multiresolução.

\includefiguretmp
  %{images/wavelet-time-frequency-representation}
  {Representação da resolução tempo frequência da transformada wavelet.}
  {fig:wavelet-time-frequency-representation}

Como alternativa para a utilização da transformada para sinais discretos, é possível a utilização da
transformada wavelet discreta. Como a transformada wavelet contínua requer um esforço computacional
considerado exagerado para calcular os coeficientes de todas as possíveis escalas da transformada,
gerando informações redundantes, é possível a utilização de parâmetros de escalonamento e translação
discretos~\cite{mallat1989theory}. A transformada wavelet discreta \gls{wavelet-transform-discrete}
de um sinal contínuo \gls{signal-time-continuous} é apresentada na
\cref{eq:wavelet-definition-discrete}.

\begin{equation}
  \mathcal{V} \left \{ x, \psi \right \} =
      \left \{ x(t), \psi_{\kappa \beta} \right \} =
      \int_{-\infty}^{\infty} x(t) \cdot
        \psi_{\kappa \beta}(t) \textrm{d}t
  \label{eq:wavelet-definition-discrete}
\end{equation}

A função \gls{wavelet-mother}{\ensuremath{_{\kappa \beta}}} é a wavelet mãe criada a partir de
parâmetros de escala e translação discretos. A \cref{eq:wavelet-translation-compression-discrete}
apresenta a obtenção da função \gls{wavelet-mother}{\ensuremath{_{\kappa \beta}}}, onde \gls
{wavelet-scale-param} é a variação da escala e \gls{wavelet-translation-param} indica a translação;
\gls{dilatation-param} e \gls{scale-param} são constantes discretas que indicam, respectivamente, a
variação da escala e dilatação.

\begin{equation}
  \psi_{\kappa \beta}(t) =
  \frac{1}{\sqrt{\alpha^{\kappa}}} \psi \left (
    \frac{t - \beta \alpha^{\kappa} \tau}{\alpha^{\kappa}} \right )
  \label{eq:wavelet-translation-compression-discrete}
\end{equation}

Em termos gerais, a transformada wavelet é realizada através de um processo de filtragem de vários
níveis, onde cada nível apresenta um filtro em quadratura. O sinal decomposto em cada um dos níveis
apresenta duas informações, definidas como \emph{detalhe} (alta frequência) e \emph{aproximação}
(baixa frequência). A decomposição em vários níveis origina uma árvore, denominada de árvore de
decomposição wavelet packet~\cite{mallat1989theory}. Este processo é ilustrado na
\cref{fig:wavelet-decompose-representation}.

\includefiguretmp
  %{images/wavelet-decompose-representation}
  {Representação de dois níveis da árvore de decomposição wavelet packet.}
  {fig:wavelet-decompose-representation}

Como resultado do processo da árvore de decomposição, é gerado um vetor de elementos obtido através
do cálculo da energia dos coeficientes no nível mais baixo da decomposição da transformada. Este
vetor de características é denominado de energias da transformada wavelet
packet~\cite{ims2007documentation}.


%%%%
\subsubsection{Regressão logística}

A regressão logística é um método de classificação que permite uma classificação binária ou
dicotômica de um conjunto de dados~\cite{hosmer2013applied}. O método integra a categoria de modelos
chamados de Modelos Generalizados Lineares, e, portanto, como resultado da análise, é gerada uma
resposta de dois estados, que podem ser traduzidos, por exemplo, para sucesso ou falha ou
comportamento normal ou degradado.

O processo empregado pelo método da regressão logística é definido como a tentativa de ajustar o
espaço de \gls{dimension-index} dimensões da entrada para um espaço de saída de apenas uma dimensão.
A variável de saída ou resposta é representada por \gls{logistic-regression-output}, sendo que
\gls{logistic-regression-output}{\ensuremath{~= 1}} quando o conjunto de entrada possui a
característica de interesse e \gls{logistic-regression-output}{\ensuremath{~= 0}} quando não
possui~\cite{hosmer2013applied}. A \cref{fig:logistic-regression-representation} apresenta uma
curva típica da saída de um modelo de regressão logística.

\includefiguretmp
  %{images/logistic-regression-representation}
  {Representação de um modelo de regressão logística.}
  {fig:logistic-regression-representation}

A representação matemática do modelo de regressão logística é apresentada na
\cref{eq:logistic-regression-model}~\cite{ims2007documentation}. No modelo,
\gls{logistic-regression-input} é o vetor de entrada de \gls{dimension-index} dimensões e
\gls{logistic-regression-output} é a saída binária.

\begin{equation}
  p(r) =
  P(y = 1 | r) =
  \frac{1}{1 + e^{- \left (
    \alpha + \beta_{1} r_{1} + \beta_{2} r_{2} + \cdots + \beta_{k} r_{k} \right )}} =
  \frac{e^{\alpha + \beta_{1} r_{1} + \beta_{2} r_{2} + \cdots + \beta_{k} r_{k}}}
    {1 + e^{\alpha + \beta_{1} r_{1} + \beta_{2} r_{2} + \cdots + \beta_{k} r_{k}}}
  \label{eq:logistic-regression-model}
\end{equation}

O modelo apresentado na \cref{eq:logistic-regression-model} também pode ser representado em termos
das probabilidade de evento e de não evento. Neste caso, são definidos como $p(r)$ e $1 - p(r)$. A
\cref{eq:logistic-regression-model-linear} apresenta a nova função, onde o termo contendo o
logaritmo natural é conhecido como função \emph{logit}, cujo propósito é tornar a função linear.

\begin{equation}
  g(r) =
  \ln \left ( \frac{p(r)}{1 - p(r)} \right ) =
  \alpha + \beta_{1} r_{1} + \beta_{2} r_{2} + \cdots + \beta_{k} + r_{k}
  \label{eq:logistic-regression-model-linear}
\end{equation}

A partir da \cref{eq:logistic-regression-model} o valor de confiança é obtido. Para um comportamento
normal, o valor de confiança assume valores próximos a \num{1}, enquanto que para comportamento
ditos de falha, o valor de confiança fica concentrado próximo a \num{0}.


%%
\section{Arquiteturas orientadas a serviços}

As técnicas para desenvolvimento de aplicações \gls{SOA} representam uma mudança de paradigma na
engenharia de software, onde os componentes são definidos como serviços~\cite{ramollari2007survey}.
Originalmente desenvolvido e utilizado para integração de sistemas no meio gerencial e corporativo,
logo teve aceitação entre diversos segmentos, como plataformas de negócio, telecomunicações,
transportes e na automação industrial.

O termo \gls{SOA} ainda possui uma definição concisa e única, diferindo conforme os conhecimentos
técnicos e a bagagem acumulada durante o desenvolvimento de diferentes aplicações por parte dos
autores~\cite{candido2013soa}. Ainda segundo \cite{candido2013soa}, a definição que mais se encaixa
no contexto de um trabalho que envolve integração em meio industrial é a
de~\cite{jammes2005service}, o qual expressa que "\gls{SOA} é um conjunto de princípios ou doutrinas
para a construção de sistemas interoperáveis e também autônomos". Esta percepção também descreve o
contexto deste trabalho, no qual os componentes envolvidos podem ser considerados peças
independentes do sistema, no entanto podem vir a representar um conjunto interoperável de entidades,
compartilhando recursos entre si.

Em termos gerais, a característica principal de uma arquitetura \gls{SOA} é a criação e
disponibilidade de serviços, que, quando agrupados, constituem um sistema
funcional~\cite{josuttis2009soa}. O termo serviço se refere a uma funcionalidade ou lógica que é
encapsulada e oferecida ao sistema através de uma interface. Dessa forma, outros serviços, entidades
ou programas podem obter o modo de acesso à esta funcionalidade e empregá-la na resolução de
determinada tarefa.


%%%
\subsection{Componentes de uma arquitetura orientada a serviços}

Em se tratando do contexto da aplicação, os serviços oferecidos, para que possam ser utilizados,
precisam ser encontrados ou expostos~\cite{papazoglou2007service}. Mesmo que, segundo a definição
adotada, serviços possam ser utilizados independentemente, a abordagem de utilizar um conjunto de
serviços trabalhando de forma cooperativa na resolução de um problema, é muito mais interessante.
Portanto, uma aplicação \gls{SOA} deve prover meios para que os serviços possam ser comunicar e
trocar informações via mensagens padronizadas. Dessa forma, é necessário que existam alguns
conceitos a serem cumpridos por todos os componentes de uma arquitetura
\gls{SOA}~\cite{erl2005service}:

\begin{itemize}
  \item \emph{Acoplamento mínimo}: serviços devem minimizar a dependência, armazenando somente as
  informações de outros serviços.

  \item \emph{Contrato de serviço}: devem utilizar um padrão de comunicação comum previamente
  definido e baseado em padrões abertos.

  \item \emph{Autonomia}: possibilidade de controle total da lógica que o serviço encapsula.

  \item \emph{Abstração}: possibilidade de esconder a lógica e os recursos utilizados pelo serviço
  do resto da aplicação.

  \item \emph{Reusabilidade}: utilização da mesma funcionalidade por diferentes partes do sistema ou
  em aplicações futuras, sem a necessidade de uma nova implementação.

  \item \emph{Composição}: organização de serviços para a construção de tarefas mais complexas podem
  ser feitas através da composição de serviços mais simples.

  \item \emph{Sem dependência de estado}: os serviços não devem reter nenhuma informação específica
  sobre as atividade executadas.

  \item \emph{Possibilidade de descoberta}: os serviços devem possibilitar a sua descoberta pelos
  mecanismos de busca.
\end{itemize}

As aplicações \gls{SOA} normalmente são construídas baseadas nos princípios de serviços
web~\cite{josuttis2009soa}. Não necessariamente as aplicações \gls{SOA} necessitam ser baseadas em
serviços web, porém, esta tecnologia começou a ser largamente utilizada devido, entre outros
fatores, a adoção de uma padronização. Por parte dos desenvolvedores, o encapsulamento das
funcionalidades que um serviço pode oferecer foi facilitado após a definição de interfaces e
protocolos de comunicação padrão. Dessa forma, é possível aos clientes acessar os serviços de forma
transparente, sem conhecimento prévio de detalhes de implementação.

Seguindo as convenções adotadas para serviços web, um serviço é uma entidade de software
identificada por uma \gls{URI}~\cite{bell2008service}. A \gls{URI} define o endereço do serviço na
aplicação, devendo ser única. O identificador permite a discriminação entre grupos de entidades,
através da utilização de um separador. Esta técnica é largamente utilizada em sistemas que usufruem
de identificadores baseados em \gls{URI}.

A troca de informações entre os serviços normalmente ocorre utilizando o protocolo \gls{SOAP}, onde
as mensagens são codificadas no formato \gls{XML}~\cite{josuttis2009soa}. O protocolo \gls{SOAP}
provê uma infraestrutura básica para a troca de mensagens entre serviços web. É definido por um
envelope, um conjunto de regras que definem os tipos de dados suportados e um meio de representar os
procedimentos ou funcionalidades disponíveis para execução. Por ser baseado em \gls{XML}, o
protocolo pode ser utilizado sobre diferentes protocolos de transporte, como, por exemplo, o
\gls{HTTP}.

A descrição das interfaces de cada serviço é normalmente definida por \gls{WSDL}. Documentos
\gls{WSDL} também são baseados em \gls{XML} e contém toda a informação necessária para a utilização
do serviço em questão. No documento, são especificadas todas as operações que o serviço possibilita,
bem como os tipos de dados suportados. Também é possível estender o documento definindo novos tipos
de dados para troca de mensagens.

Os serviços disponíveis na aplicação e a descrição de suas funcionalidades estão centralizadas no
\gls{UDDI}. O padrão \gls{UDDI} define o protocolo para os serviços de diretório, ou intermediadores
de serviço, onde são armazenadas todas as informações de cada um dos serviços da aplicação. Esta
entidade é utilizada para informar aos clientes quais serviços estão disponíveis, possibilitando
meios de descobri-los e obter seus metadados. A interoperabilidade entre os componentes de uma
aplicação \gls{SOA} é ilustrada na \cref{fig:soa-elements}.

\includefigure
  {images/soa-elements}
  {Interoperabilidade entre os elementos de uma aplicação SOA.}
  {fig:soa-elements}


%%%
\subsection{Device Profile for Web Services}

Especificações para web services normalmente são grafadas com o prefixo
"{WS-}"~\cite{candido2013soa}. É comum encontrar na literatura o termo "{WS-*}", o qual se refere ao
agrupamento de diferentes especificações para web services. Dentre as especificações, o \gls{DPWS}
define um conjunto mínimo de implementações que permitem a troca de mensagens, descoberta,
descrição, geração de eventos e autenticação para a utilização de web services em clientes com
recursos computacionais limitados. O \gls{DPWS} permite a integração destes clientes com outros, com
recursos mais flexíveis.

O \gls{DPWS} implementa um conjunto restrito dos padrões {WS-*}. A \cref{fig:dpws-stack} ilustra o
diagrama contendo a pilha de protocolos suportados pelo \gls{DPWS}. Dentre os padrões, é possível
destacar o WS-Adressing, utilizado para transferência de mensagens, WS-Security\todo{alterar a
figura}, para suprir as necessidades de autenticação nos web services, WS-Discovery, que possibilita
a descoberta de serviços em uma rede local, e WS-Eventing, que permite a utilização de eventos.

\includefigure
  {images/dpws-stack}
  {Pilha de protocolos suportados pelo DPWS.}
  {fig:dpws-stack}

  \chapter{Análise do estado da arte}

Os trabalhos relacionados com o tema desta dissertação dizem respeito às áreas de sistemas de
manutenção inteligente e arquiteturas orientadas a serviços. Neste capítulo são analisados os
trabalhos relevantes para as duas áreas distintas, evidenciando os pontos favoráveis de cada um. Por
fim, são apresentados trabalhos convergentes, que relacionam as duas áreas envolvidas. Os trabalhos
mostram que é possível a integração entre \gls{IMS} e \gls{SOA}.


%%
\section{Sistemas de manutenção inteligente}

As ações corretivas empregadas em manutenção sempre levam a parada da linha de produção, ocasionando
perdas~\cite{carvajal2011sobre}. Estas perdas podem ser programadas ou não programadas. Em ambos os
casos, existem meios de diminuir os impactos causados por estas intervenções. Ao longo dos anos,
vários trabalhos comprovam o sucesso de técnicas de manutenção inteligente aplicadas em diversos
cenários~\cite{muller2008concept}. Entre eles, o ponto em comum está no fato do crescimento da
importância do uso destas técnicas, tendo em vista a garantia da disponibilidade e segurança do
sistema mantendo a qualidade da linha de produção.

O trabalho de~\cite{lee2006intelligent} introduz os conceitos de \textit{e-maintenance}, emergentes
à época, e os elementos críticos que o compõem. Considerando que o mercado global está mais
competitivo a cada ano, os autores afirmam que o uso de técnicas que minimizem os custos com a
produção é de fundamental importância. Entre os custos citados, estão em destaque os que dizem
respeito à falhas e quebras não programadas de maquinário. Também é sugerida a mudança de paradigma,
da tradicional \emph{falha e conserto} para \emph{predição e manutenção}. Para contornar estes
problemas, o artigo sugere o uso de técnicas computacionais, que, juntamente com o advento da
Internet, possibilitam o monitoramento da condição dos equipamento, ao invés de somente detectar os
equipamento em falha.

As ferramentas utilizadas para o monitoramento da condição dos equipamentos também são discutidas no
artigo de~\cite{lee2006intelligent}. Ferramentas de avaliação da condição e predição de falhas são
analisadas, a fim de se obter um monitoramento contínuo do equipamento. Além disso, são apresentados
os últimos avanços em relação às ferramentas utilizadas para este tipo de cenário e estudos de caso
para validação das técnicas empregadas.

Dando continuidade aos trabalhos na área de manutenção, o artigo de~\cite{muller2008concept} aponta
que a importância da manutenção em sistemas, a fim de aumentar a disponibilidade e segurança, bem
como a qualidade dos produtos, está em vias de crescimento. Como o artigo anterior, o trabalho
também ilustra os conceitos de \textit{e-maintenance}, provendo uma visão geral sobre os estudos e
desafios presentes neste campo de estudo. São apresentados os diferentes sistemas de manutenção
disponíveis, além de uma comparação entre as diferentes características de cada um. Como pontos
importantes, os autores veem grande potencial na adoção de novas tecnologias que auxiliem nas
tarefas de manutenção, os quais são definidos em termos de \emph{dispositivos inteligentes}. Além
disso, defendem a criação de padrões internacionais para integração entre os sistemas discutidos. E,
como contribuição maior, está a proposta da organização das ações no setor, em prol do avanço dos
estudos e definição de novos conceitos, apoiando o surgimento de uma nova disciplina.

No trabalho de~\cite{goncalves2009design}, a detecção de falhas em um atuador eletromecânico é feita
através de um sistema composto por um microprocessador Microblaze sintetizado em um \gls{FPGA}. No
\gls{FPGA}, os sinais são processados e mapas auto-organizáveis são utilizados para detecção,
classificação e predição de falhas. O treinamento dos algoritmos é realizado em um computador
pessoal, enquanto que o monitoramento dos sinais é feito no \gls{FPGA}. O trabalho é estendido
em~\cite{goncalves2011desenvolvimento} com a implementação de um filtro adaptativo no \gls{FPGA}. A
expansão é em função da comparação das técnicas utilizadas para embarque na plataforma proposta,
onde são determinados a eficiência da identificação das falhas, a área de ocupação do \gls{FPGA} e o
tempo de execução dos algoritmos.

O artigo de \cite{hu2012prognostic} utiliza métodos baseados em redes imunológicas artificiais para
predizer a falha e o estado de saúde do sistema de propulsão de um navio. Considerando que, para o
correto funcionamento do navio por grandes períodos de tempo, cada componente precisa estar
condicionado ao correto funcionamento, os autores propõem um sistema de monitoramento para as
diferentes partes do sistema de propulsão. Conforme o artigo, em navios, para aumentar a
confiabilidade do sistema, são empregados muitos equipamentos redundantes, aumentando a complexidade
e dificultando a identificação de uma possível falha. Nessas circunstâncias, uma falha pode
acarretar em catástrofe. Com a proposta dos autores, o sistema de prognóstico ajuda a identificação
das possíveis falhas em meio a complexidade imposta pelo sistema.

A proposta uma arquitetura de manutenção inteligente baseada em agentes móveis para monitoramento
das condições dos equipamentos através de um sistema que imita o sistema imunológico humano também é
apresentada em \cite{hua2013mobile}. A técnica de sistemas imunológicos artificiais foi escolhida
pelo fato de que muitos processos industriais, como mineração, óleo e gás, estão geograficamente
localizados em áreas remotas e de difícil acesso. Baseado no fato de que alocar equipes de
profissionais para inspeção destes processos demanda tempo e custo elevados, os autores propõem uma
arquitetura descentralizada para automatizar o processo de monitoramento das condições dos
equipamentos. O resultado dos experimentos visam determinar o desempenho do sistema proposto em
termos da acurácia na detecção de falhas e alocação de banda de comunicação. Os autores avaliam que
o sistema proposto não provê somente uma ferramenta viável para detecção de falhas, mas também
flexível e confiável com redução do uso da rede e dos recursos computacionais.


%%
\section{Arquiteturas orientadas a serviços}

\Gls{SOA} é uma técnica emergente, padronizada, que possibilita acoplamento mínimo entre componentes
e comunicação distribuída independente de protocolo~\cite{papazoglou2007service}. O crescente uso
desta tecnologia em sistema de informação, nos últimos anos foi expandido para o segmento
industrial. A facilidade de interconexão e os altos níveis de abstração entre dispositivos fazem com
que a tecnologia \gls{SOA} possa ser empregada também em ambientes industriais com certa
facilidade~\cite{moritz2008web}.

Segundo~\cite{cannata2010dynamic}, a fábrica do futuro será conduzida pelo alto uso de arquiteturas
orientadas a serviços. O gerenciamento dos processos de negócio serão fortemente baseados em
informações provenientes do chão de fábrica, tornando as aplicações mais complexas e custosas.
Destas necessidades, surgem os conceitos de colaboração entre os diversos segmentos e componentes da
aplicação.

Os desafios desta nova abordagem visam prover meios de interconexão entre dispositivos e aplicação.
Neste contexto, vários projetos foram iniciados. O projeto
\glsdisp{SIRENA}{SIRENA~(\glsdesc{SIRENA})} foi a primeira iniciativa a aplicar os conceitos de
orientação a serviços em dispositivo físicos de chão de fábrica, com foco na implementação
distribuída de arquiteturas \gls{SOA} em ambientes heterogêneos~\cite{jammes2005service}. O projeto
investigou os diferentes domínios de aplicação, como automotivo, residencial e telefônico. A partir
deste projeto, diversos outros utilizaram os conceitos \gls{SOA} para integração entre equipamentos
em infraestruturas orientadas a serviços~\cite{zeeb2007service}.

Um dispositivo que possui recursos para utilização em ambientes \gls{SOA} apresenta ganhos em
interoperabilidade, autonomia além de se tornar uma importante peça na construção da aplicação em
que for baseado~\cite{candido2010industrial}. Dessa forma, a tecnologia empregada não só fornece um
padrão de comunicação entre os componentes do sistema, mas também adiciona a capacidade de novas
aplicações serem construídas rapidamente, mantendo a agilidade em relação às modificações impostas
por sistemas que apresentam muitas alterações durante o ciclo de vida.

No trabalho de~\cite{pathak2007service-oriented}, é apresentada uma arquitetura baseada em serviços
para o gerenciamento dos equipamentos de um sistema de transmissão de energia elétrica. A
arquitetura possibilita o sensoriamento, integração de informações, avaliação dos riscos e tomada de
decisão relativos à operação de um sistema elétrico de alta tensão. Segundo os autores, o sistema
proposto integra a aquisição de dados em tempo real, modelagem e funcionalidades de previsão.
Juntas, são utilizadas para determinar as políticas de operação, agendar manutenções e garantir o
correto funcionamento dos equipamentos que compõem o sistema de distribuição de energia.

O trabalho de~\cite{ribeiro2008generic} mostra o sucesso encontrado em pesquisas anteriores em
aplicações que trocam informações na forma de serviços, escondendo a complexidade dos componentes e
provendo interfaces limpas de comunicação. No entanto, os autores ilustram cenários onde são
inseridas desvantagens com o uso de \gls{SOA} em sistemas altamente configuráveis. Estes sistemas
apresentam a característica de reconfiguração dos dispositivos periodicamente. De acordo com o
artigo, a periodicidade da reconfiguração dos dispositivos acontece tipicamente em ambientes
heterogêneos, o que acarreta em perda de desempenho do sistema e ocasiona alto tráfego nas redes de
comunicação. Como solução para este problema, é apresentado um modelo de comunicação genérica que
minimiza os problemas encontrados nestes sistemas.

O artigo de~\cite{candido2010soa} aborda a interoperabilidade entre as especificações \gls{OPCUA} e
\gls{DPWS} do padrão \gls{SOA}. Os autores demonstram que, sozinha, nenhuma das duas tecnologias
consegue atender a todos os requisitos impostos a nível de dispositivos em aplicações orientadas a
serviços, mas, se combinadas, apresentam benefícios para o domínio em que são utilizadas. Uma
solução abrangendo as duas tecnologias é proposta e comprovada a convergência das duas tecnologias.

Também é visto em~\cite{nagorny2013engineering} que o paradigma \gls{SOA} é uma alternativa
promissora na implementação e controle de ambientes ditos sistemas de sistemas. Sistemas de sistemas
são coleções de tarefas ou sistemas dedicados que unem seus recursos ou aptidões para desempenhar
determinada tarefa a fim de possibilitar a criação de um novo sistema que ofereça mais recursos e
possibilidades de desempenhar outras tarefas mais complexas. Como mostra o trabalho, utilizando uma
arquitetura orientada a serviços, é possível a integração de sistemas diferentes do ponto de vista
estrutural e dinâmico, como, por exemplo, sistema de aquecimento urbano e distribuição de energia.
Novamente, o uso e composição dos serviços existentes é feito de forma facilitada, o que contribui
para o sucesso da aplicação.


%%
\section{Uso de arquiteturas orientadas a serviços em conjunto com sistemas de manutenção
inteligente}

Os trabalhos anteriores apresentam soluções empregadas nas duas áreas distintas. Como apresentado,
nos últimos anos, a adoção de sistemas baseados em serviços na indústria está em constante
crescimento. Ainda que as pesquisa nestas duas áreas esteja evoluindo, as iniciativas para o uso de
\gls{SOA} na integração de sistemas de manutenção inteligente ainda são escassas.

O trabalho de \cite{zhao2010soabased} apresenta o desenvolvimento de uma arquitetura baseada em
\gls{SOA} para monitoramento remoto de condição e diagnóstico de falhas em equipamentos. A
arquitetura proposta é construída sobre o conceito de Web Services, clientes inteligentes e
\gls{XML}. O artigo apresenta uma comparação entre programação orientada a objetos, baseada em
componentes e \gls{SOA}, como diferentes técnicas empregadas para construção de aplicações
distribuídas. Os autores confirmam o sucesso e vantagens no uso de \gls{SOA}, verificando a
possibilidade de integração de diferentes recursos de software, aumentando a disponibilidade e
escalabilidade do sistema e diminuindo o tempo de desenvolvimento.

No artigo de~\cite{cannata2010dynamic}, são apresentados os aspectos envolvidos em relação aos
conceitos de \textit{e-maintenance} em vistas a cumprir com a construção de dispositivos baseados em
\gls{SOA}. Os autores investigam os benefícios na adoção de arquiteturas orientadas a serviços para
a integração de camadas de negócio com sistemas de manutenção inteligente dispostos no chão de
fábrica. Através do uso destas tecnologias, é notável a facilidade e benefícios encontrados na
integração de sistemas, sugerindo o uso de \gls{SOA} em uma nova geração de sistemas industriais. Em
relação a \textit{e-maintenance}, o monitoramento das condições dos equipamentos de forma facilitada
faz com que o gerenciamento de planos de manutenção possam ser executados de maneira ágil e
eficiente, evitando paradas desnecessárias na linha de produção.

  %\chapter{Arquiteturas orientadas a serviços}

  \chapter{Arquitetura proposta}

Conforme visto no capítulo anterior, é possível a integração de um sistema de manutenção inteligente
utilizando uma arquitetura orientada a serviços. Dessa forma, este capítulo apresenta a proposta de
uma arquitetura para um sistema de manutenção inteligente. São apresentadas as entidades principais
e a interligação entre elas através da utilização do padrão \gls{SOA}. As entidades que compõem a
arquitetura são descritas e inseridas no contexto de uma aplicação através de casos de uso. Através
dos casos de uso, são abordadas diferentes situações onde as entidades da arquitetura se encaixam
para resolver algum dos problemas.

\blindtext

\todo[inline]{Falar sobre o estudo de caso e programação dos experimentos.}


%%
\section{Arquitetura orientada a serviços proposta}
\label{sec:arquitetura-proposta}

A arquitetura proposta neste trabalho tem por objetivo a integração de sistemas de manutenção
inteligente, dos diversos equipamentos que precisam ser monitorados, além de outras entidades que
auxiliam no funcionamento do sistema. A troca de informações entre todos os elementos que compõem a
arquitetura é abstraída na forma de serviços, o que facilita a integração, inserção e remoção de
novas entidades no sistema. Dessa forma, a especificação da arquitetura é definida utilizando os
padrões \gls{SOA}. Uma visão geral das entidades que fazem parte da arquitetura proposta é
apresentada na \cref{fig:soa-proposed-architecture}\todo{Alterar figura. A aplicação SOA deve
englobar mais entidades}. Na figura, nota-se que cada entidade tem uma função específica no sistema
e o acesso às suas funcionalidades é realizado através de serviços. A seguir, para melhor
entendimento, todos os elementos serão detalhados.

\includefigure
  {images/soa-proposed-architecture}
  {Arquitetura orientada a serviços proposta para integração de um sistema de manutenção
      inteligente.}
  {fig:soa-proposed-architecture}


%%%
\subsection{Serviço}

No contexto da arquitetura proposta neste trabalho, serviço é um componente de software que
encapsula uma funcionalidade acessível através dos padrões definidos pela tecnologia \gls{SOA}.
Todos os componentes da arquitetura expõem as suas funcionalidades na forma de serviços,
possibilitando que a interação entre eles seja feita de forma transparente. Como parte do padrão
\gls{SOA}, serviços podem ser descobertos e utilizados por clientes que desejam executar uma
determinada tarefa. Neste contexto, também é possível a definição de serviços mais especializados
com base em outros serviços, praticando a técnica da composição de serviços. Estas características
se tornam inerentes à proposta, devido a utilização do padrão \gls{SOA}.


%%%
\subsection{Dispositivo}

Um dispositivo é um componente de software utilizado para encapsular um elemento físico da aplicação
proposta. Por ser executado em um dispositivo físico, o componente é denominado dispositivo lógico e
disponibiliza serviços para acesso à funcionalidades previamente definidas. As funcionalidades podem
ser relativas ao dispositivo físico em que o componente está executando ou outras que auxiliam em
alguma tarefa específica não relacionada diretamente com o hardware hospedeiro.

Do ponto de vista da aplicação, os dispositivos são entidades que hospedam serviços. Dentre os
serviços hospedados, alguns estão presentes em todos os dispositivos da arquitetura, servindo de
base para a comunicação entre todos os elementos desta classe. Dessa forma, conhecendo os serviços
básicos, uma interface mínima para troca de informações entre as entidades do sistema é definida,
facilitando a inserção de novos dispositivos.

Além dos serviços base, outros podem ser executados no dispositivo. A arquitetura permite o envio de
novos serviços para os dispositivos do sistema. O dispositivo recebe o novo serviço, que possui as
mesmas funcionalidades de um serviço padrão, e o carrega para ser executado normalmente.


%%%
\subsection{Aplicação orientada a serviços}

A \todo{Alterar texto para englobar mais entidades da arquitetura}aplicação orientada a serviços
nada mais é do que o resultado da utilização dos diversos serviços propostos na arquitetura. Através
da composição ou utilização direta dos serviços de uma forma coordenada, a aplicação é construída.
De um modo minimalista, a aplicação pode ser vista como a interação progressiva de funcionalidades
simples provenientes de serviços básicos, resultando em um componente complexo de software. Visto
que o produto final da composição dos elementos do sistema não é definido \textit{a priori}, o
resultado da aplicação pode ser qualquer construção ou interação entre os componentes.


%%%
\subsection{Gerenciador de Dispositivos}
\label{sub:proposta-gerenciador-dispositivos}

Como o nome sugere, o Gerenciador de Dispositivos é um componente de software utilizado para
gerenciar os dispositivos da arquitetura proposta. A configuração ou obtenção da lista dos
dispositivos na aplicação orientada a serviços são exemplos de funcionalidades do gerenciador. Como
os dispositivos são entidades que mapeiam funcionalidades dos dispositivos físicos encontrados em um
sistema de manutenção inteligente, o gerenciador pode, por exemplo, obter a lista de sensores do
equipamento e configurar alguns parâmetros, como a taxa de atualização dos dados dos sensores.
Também é possível definir comportamentos adicionais para o caso do equipamento estar operando em
diferentes níveis de degradação. Os diferentes comportamentos podem ser utilizados quando há
necessidade de operar o equipamento em condições onde manter os níveis de degradação estáveis é mais
importante do que a operação a pleno. Portanto, pode-se manter o equipamento funcionando até que uma
manutenção \textit{in loco} possa ser realizada.

Grupos de dispositivos podem ser criados e gerenciados. O Gerenciador de Dispositivos permite a
criação de grupos, a fim de facilitar a alteração simultânea de vários dispositivos. Supondo que a
aplicação é inerentemente escalável, o número de dispositivos tente a aumentar, o que dificulta o
gerenciamento ou configuração de dispositivos similares. Considerando que as configurações aplicadas
a uma mesma classe de dispositivos é muito parecida, o gerenciador permite que um grupo receba os
mesmos parâmetros, automatizando o processo de customização dos dispositivos.

Outra funcionalidade que o gerenciador provê é a obtenção da topologia dos componentes do sistema.
Em uma arquitetura \gls{SOA} todos os serviços estão localizados no mesmo nível hierárquico. Essa
característica pode ser considerada positiva do ponto de vista de integração e reuso de serviços. Em
contrapartida, ao definir níveis hierárquicos entre componentes de um mesmo dispositivo físico, a
complexidade para o estabelecimento de uma hierarquia lógica para representação das diferentes
partes desse dispositivo aumenta. Contudo, ao aplicar um identificador único para cada dispositivo,
ou parte de dispositivo, do sistema, torna-se possível definir e estabelecer uma relação
hierárquica. O identificador pode ser uma \gls{URI}, por exemplo, onde cada parte do endereço se
refere a um dos níveis hierárquicos.

O Gerenciador de Dispositivos também é utilizado na resolução de problemas encontrados durante a
execução dos componentes do sistema. A entidade se enquadra na categoria de \gls{IHM},
possibilitando que a rede seja mapeada em busca de dispositivos ou serviços de forma interativa.
Além disso, é possível obter o estado dos dispositivo, visualizar dados ou testar os serviços
encontrados.


%%%
\subsection{Gerenciador de Análises}
\label{sub:proposta-gerenciador-analises}

O Gerenciador de Análises é utilizado para definir o plano que será aplicado a determinado
equipamento, ou grupo de equipamentos, \todo{Em ou com?}com vista a obter os níveis de degradação.
Nesta entidade, é possível selecionar e definir a ordem em que serão executadas as ferramentas de
análise de dados através de um plano. O gerenciador permite o agendamento dos planos e a execução
periódica em intervalos de tempo. Os planos também podem ser gerenciados, possibilitando alteração
e/ou exclusão.

No plano são definidos os comportamentos a serem adotados com base nos valores de degradação do
equipamento. O gerenciador obtém os comportamentos previamente definidos para o equipamento a ser
analisado, os quais podem ser mapeados pelo usuário para diferentes níveis de degradação. Dessa
forma, o equipamento pode ser adaptado a diferentes condições, por exemplo, evitando o aumento da
degradação até que uma manutenção possa ser realizada.

Esta entidade também coordena as análises realizadas em um grupo de equipamentos similares. O
processo de obtenção dos valores de confiança de um conjunto de equipamentos pode ser simplificado
pelo fato de que, se são similares, o nível de degradação de um equipamento pode ser aproximado para
o nível de degradação do grupo. Dessa maneira, é possível elaborar um plano de análise que seja
aplicável a vários equipamentos.


%%%
\subsection{Analisador de Dispositivos}
\label{sub:proposta-analisador-dispositivos}

O Analisador de Dispositivos concentra todas as ferramentas necessárias para a realização da análise
dos dados de determinado dispositivo. A entidade opera com base nos planos criados e agendados pelo
operador do sistema no Gerenciador de Análises. De posse do plano, o analisador executa as análises
utilizando os algoritmos necessários para obtenção do resultado esperado. A execução de um mesmo
plano é feita de forma automática.

O analisador permite verificar quais planos estão sendo executados ou na fila para execução. Sempre
que um novo plano é gerado, o analisador o coloca na fila de execução, a fim de que seja
finalizado o mais breve possível. A entidade permite o monitoramento destes planos pelo usuário, o
qual pode verificar se a execução está de acordo com o esperado.


%%%
\subsection{Base de dados}

A entidade Base de Dados é utilizada como uma interface para um banco de dados. O software encapsula
funcionalidades para  armazenar todos os dados relativos à aplicação. Desde os dados obtidos nas
análises de degradação dos equipamentos como também os planos de análise agendados pelo usuário. A
base centraliza as informações, mas não necessariamente faz com que a aplicação seja centralizada.
Nada impede a utilização de diferentes \todo{Remotas e locais? Alterar figura}bases de dados e o
compartilhamento de informações entre as entidade do sistema. Por exemplo, o armazenamento dos dados
de treinamento de um equipamento, necessário para obtenção do nível de degradação, pode ser feito em
diferentes bases de dados. O Analisador de Dispositivos pode combinar estas informações, a fim de
melhorar o resultado esperado.

Por ser defina como um componente \gls{SOA}, o acesso aos dados é feito através de serviços. A
inclusão, deleção e modificação de dados são realizadas por serviços especializados. Um dispositivo
remoto pode acessar a base de dados a fim de obter o histórico dos últimos valores de confiança
calculados ou valores de testes intermediários para determinado equipamento, utilizando-os para um
novo tipo de análise.


%%%
\subsection{Repositório de serviços}

O repositório de serviços é uma entidade que armazena os dispositivos lógicos e seus serviços, os
quais podem ser obtidos dinamicamente e implantados em dispositivos físicos. Esta entidade pode
estar localizada na rede local ou remota, com acesso disponível para várias aplicações. Dessa forma,
uma nova aplicação, isolada de outra já existente, pode ser construída com o reúso de componentes
obtidos de um repositório de serviços compartilhado pelas duas. Esta abordagem também permite o
compartilhamento de uma base de dados de componentes lógicos entre aplicações.


%%
\section{Casos de uso para a arquitetura proposta}

Os elementos que compõem a arquitetura proposta encaixam-se em alguns casos de uso no contexto de um
sistema de manutenção inteligente. Os casos de uso propostos a seguir exemplificam a interação do
operador do sistema na busca ou configuração dos dispositivos e na criação dos planos de análise
para cada dispositivo. Para a maioria dos casos de uso, o operador do sistema é considerado como
ator. Além disso a ferramenta de análise de dados é utilizada para analisar os dados obtidos dos
equipamentos monitorados. Visto que as análises executam autonomamente durante o processo de
amostragem dos equipamentos, uma tarefa periódica é vista como ator do caso de uso.


%%%
\subsection{Descoberta e configuração de dispositivos}
\label{sub:proposta-descoberta-configuracao-dispositivos}

A descoberta dos dispositivos e serviços na rede pode ser feita a qualquer momento utilizando o
Gerenciador de Dispositivos. Cada um dos dispositivos encontrados é identificado e colocado em uma
lista de dispositivos ativos. Também são identificados os serviços hospedados em cada um dos
dispositivos. Dessa forma, é possível determinar quais dispositivos estão atualmente disponíveis na
rede e quais possuem serviços que poderão ser utilizados para suprir e executar determinada
funcionalidade do sistema.

A geração da hierarquia de recursos encontrados na rede também é possível. Ao aplicar um
identificador único para cada recurso, a reconstrução dos componentes do sistema pode ser obtida
agrupando-os em classes ou categorias. O identificador é utilizado para auxiliar a definir uma
topologia dos subdispositivos encontrados em um dispositivo lógico, facilitando a visualização e
configuração dos componentes.

O diagrama \gls{UML} de casos de uso para descoberta e configuração de dispositivos é apresentado na
\cref{fig:uml-discovery-setup-devices}. São ilustradas as interações que o operador do sistema pode
realizar nos dispositivos utilizando o Gerenciador de Dispositivos. Em vias gerais, o gerenciador
será utilizado pelo operador do sistema para descobrir os dispositivos da rede, obter o estado ou
realizar modificações nestes elementos.

\includefigure
  {images/uml-discovery-setup-devices}
  {Diagrama de casos de uso para descoberta e configuração de dispositivos.}
  {fig:uml-discovery-setup-devices}

Ao utilizar a descoberta de dispositivos, o gerenciador analisa todos dispositivos da aplicação
orientada a serviços. Os que forem encontrados são retornados, permitindo ao operador do sistema
verificar se a inclusão de um novo dispositivo foi concluída com êxito. Além disso, o gerenciador
permite obter o estado de cada elemento encontrado, o qual está relacionado ao funcionamento do
dispositivo, se está executando corretamente ou se está pronto para operação. Outras informações
também são obtidas utilizando os metadados. Esta funcionalidade lista dados como modelo, versão ou
fabricante do dispositivo.

Na configuração, o operador do sistema também deve informar os serviços de monitoramento dos
sensores do equipamento para enviar os dados à base de dados. Com as informações de acesso à base de
dados, o dispositivo pode enviar automaticamente os dados adquiridos dos sensores. Para evitar o
excesso de acessos à base, um montante de dados é adquirido e armazenado localmente pelo
dispositivo. Quando o montante estiver completo, é enviado diretamente para a base de dados. É
possibilitado ao operador do sistema configurar o número de amostras que o dispositivo irá adquirir
para compor o conjunto de dados. A \cref{fig:uml-configure-device-acquisition} apresenta o diagrama
\gls{UML} de casos de uso que ilustra a configuração do dispositivo para envio dos dados para a Base
de Dados.

\includefigure
  {images/uml-configure-device-acquisition}
  {Diagrama de casos de uso para a configuração do envio dos dados do dispositivo.}
  {fig:uml-configure-device-acquisition}

Outra funcionalidade que o Gerenciador de Dispositivos disponibiliza é a obtenção da topologia da
distribuição dos dispositivos na rede. Dispositivos e serviços no padrão \gls{SOA} sempre estão
localizados no mesmo nível hierárquico, o que dificulta o estabelecimento de uma hierarquia entre os
componentes. No entanto, pode-se definir uma hierarquia utilizando para isso o identificador de cada
dispositivo. O identificador -- uma \gls{URI}, por exemplo -- é único para cada dispositivo da
aplicação, possibilitando que, na inclusão de subdispositivos, o identificador seja estendido para
englobá-los. Além disso, pode-se criar classes e subclasses de dispositivos, organizando
funcionalidades comuns para determinado dispositivo. A visualização das duas topologias possíveis é
apresentada na \cref{fig:device-topology-visualization}. À esquerda, os dispositivos estão
distribuídos como são encontrados pelo Gerenciador de Dispositivos. Na direita, o gerenciador
reconstrói a topologia baseado nos identificadores únicos de cada dispositivo. Nota-se que os
dispositivos, que na primeira topologia estão no mesmo nível hierárquico, agora são encontrados como
subdispositivos de outros dispositivos.

\includefiguretmp
  {Possíveis visualizações da topologia dos dispositivos.}
  {fig:device-topology-visualization}

Tomando de exemplo a \cref{fig:device-topology-visualization}, os identificadores de cada
dispositivo poderiam ser construídos com um dispositivo pai englobando outros dois subdispositivos.
Tanto o dispositivo pai como os subdispositivos estão definidos dentro de escopos ou áreas e essas
informações estão incluídas no identificador.

% TODO: Talvez não seja necessária a inclusão de outra figura.
%\includefiguretmp
%  {Identificação dos dispositivos na topologia hierárquica.}
%  {fig:device-topology-identification}

Cada dispositivo pode armazenar diferentes comportamentos. Os comportamentos dizem respeito ao modo
de operação que o equipamento vai assumir perante determinada situação, sendo definidos pelo
operador do sistema. São diferentes tarefas que podem ser executadas pelo equipamento em situações
específicas. No caso de verificação de degradação excessiva de um equipamento, é possível alterar o
comportamento de funcionamento para outro que priorize a manutenção do estado de saúde atual,
evitando o aumento da degradação. Dessa forma, o equipamento pode operar por um período maior de
tempo até que uma manutenção possa ser realizada. Contudo, medidas que alteram o comportamento podem
levar a perda de desempenho nas tarefas executadas.

\iffalse
A \cref{fig:device-send-behavior} ilustra o envio de novos comportamentos pelo operador do sistema
para um dispositivo. O dispositivo mantém uma lista de comportamentos e seleciona qual o mais
adequado para a situação corrente. A situação é definida pelo nível de saúde obtido através das
análises de dados e representa a degradação de todo ou parte do equipamento.

\includefiguretmp
  {Envio de comportamentos para o dispositivo.}
  {fig:device-send-behavior}
\fi

%%%
\subsection{Envio de dados de treinamento}

Para obter o índice de degradação dos equipamento, as ferramentas de análise necessitam comparar os
dados obtidos durante o processo com dados de treinamento. Dentre os dados de treinamento, emergem
duas categorias: normal e falha. Os dados de funcionamento normal são obtidos quando o equipamento
está funcionando normalmente, enquanto que os dados de falha são obtidos quando o equipamento está
funcionando com algum tipo de degradação. A aquisição de ambos os tipos de dados são feitas em lugar
apropriado, tendo certeza das características de cada sinal.

O envio de dados de treinamento é feito através do Gerenciador de Dispositivos. O gerenciador
permite que o operador do sistema envie os dois tipo de dados, mapeando-os para um dispositivo ou
para um grupo de dispositivos. A \cref{fig:uml-send-training-data} apresenta os casos de uso para o
envio de dados de treinamento a um dispositivo.

\includefigure
  {images/uml-send-training-data}
  {Diagrama de casos de uso para envio de dados de treinamento para um dispositivo.}
  {fig:uml-send-training-data}


%%%
\subsection{Gerenciamento de análises}

O gerenciamento de análises é feito pelo operador do sistema utilizando a entidade Gerenciador de
Análises. As análises são definidas em planos, os quais apresentam informações sobre o dispositivo
analisado, quais dados serão utilizados e as ferramentas empregadas na manipulação dos dados. É
permitido o gerenciamento dos planos, como a criação, edição ou remoção, pelo operador do sistema.
Como os planos são armazenadas em uma base de dados, facilmente podem ser compartilhados com outras
instâncias do Gerenciador de Análises e utilizados por outros operadores do sistema.

As informações contidas em um plano de análise ditam como determinado equipamento será analisado. Em
um primeiro momento, o operador do sistema faz a busca pelos dispositivos utilizando o Gerenciador
de Dispositivos. Com a lista de dispositivos ativos, o Gerenciador de Análises é utilizado para
definir as operações a serem realizadas em um equipamento específico. Preenchendo o plano de
análise, o operador seleciona quais dados do equipamento serão utilizados e quais as ferramentas
utilizadas para analisá-los. O plano também permite definir o intervalo de execução de cada análise
e por quanto tempo o equipamento será analisado. O diagrama \gls{UML} de casos de uso para algumas
funcionalidades do Gerenciador de Análises é apresentado na \cref{fig:uml-analysis-management}. A
estrutura do plano é armazenada na entidade Base de Dados, sendo acessível para as outras entidades
que utilizam as informações do plano.

\includefigure
  {images/uml-analysis-management}
  {Diagrama de casos de uso para o gerenciamento de análises pelo operador do sistema.}
  {fig:uml-analysis-management}

Na construção do plano de análise, os comportamentos de cada dispositivo são obtidos e incluídos na
execução. Os diferentes comportamentos podem ser mapeados para intervalos de resultados esperados
das análises. Como descrito pelo diagrama de casos de uso da \cref{fig:uml-analysis-management}, a
\cref{fig:device-create-plan} ilustra a criação de um plano de análise de dispositivo. Após obter
todas as informações do dispositivo a ser analisado, o plano é criado pelo operador do sistema. Na
criação, são definidos os intervalos de reexecução do plano, bem como a ordem das ferramentas
utilizadas na manipulação dos dados.

\includefiguretmp
  {Busca de dispositivo e criação de um plano de análise.}
  {fig:device-create-plan}


%%%
\subsection{Análise dos dados do dispositivo}

As análises dos equipamentos do sistema, anteriormente agendadas na ferramentas de Gerenciamento de
Análises, são executadas pela entidade Analisador de Dispositivos. O analisador obtém os planos e os
executa conforme programado pelo operador do sistema. Conhecendo a frequência em que cada plano deve
ser executado, uma tarefa periódica é definida. Dessa forma, chegada a hora da reexecução de um
plano, a tarefa se comunica com o Gerenciador de Análises e o dispara. Os possíveis casos de uso
para a tarefa periódica são apresentados no diagrama \gls{UML} da \cref{fig:uml-device-analysis}. A
tarefa tem a possibilidade de obter os planos de análise ativos ou selecionar diretamente um plano
conhecido. Após a carga, a tarefa pode iniciar a execução do plano. Na execução, o Analisador de
Dispositivos seleciona as ferramentas de análise, busca os dados do equipamento e, ao final,
armazena os resultados na Base de Dados.

\includefigure
  {images/uml-device-analysis}
  {Diagrama de casos de uso para a obtenção e execução das análises.}
  {fig:uml-device-analysis}

A \cref{fig:analysis-plan-execution} apresenta o fluxo de dados quando da execução de um plano de
análise pelo Analisador de Dispositivos. O dispositivo é identificado e os seus dados são obtidos da
base de dados. As ferramentas de análise são selecionadas de acordo com o plano e os dados brutos
são analisados. O resultado da análise é novamente enviado para a base de dados. Se necessário,
novas análises são realizadas sobre os dados que já foram manipulados. Com o índice de degradação,
que é o resultado final da análise, o Analisador de Dispositivos seleciona um dos comportamentos
definidos no plano. O equipamento é então posto para executar a tarefa monitorada de acordo com o
novo comportamento.

\includefiguretmp
  {Diagrama de execução de uma análise de equipamento.}
  {fig:analysis-plan-execution}


%%%
%\subsection{Modos de operação do dispositivo}
%
%Os diferentes modos de operação dos dispositivos podem ser selecionados pelo operador do sistema e
%anexados ao plano de análise. O operador define, baseado nos níveis de degradação que podem ser
%obtidos, qual o melhor comportamento que o equipamento deve assumir. Com o plano, o Analisador de
%Dispositivos determina o índice de degradação do equipamento, verificando se está nos níveis
%desejados para determinado funcionamento. Dependendo do valor obtido na análise, o analisador pode
%alterar o funcionamento do equipamento, assumindo um dos comportamentos definidos anteriormente pelo
%operador do sistema. A \cref{fig:uml-device-operation-modes} apresenta o diagrama \gls{UML} de casos
%de uso para a seleção automática dos modos de operação do dispositivo baseado no plano de análise.
%
%\includefiguretmp
%  {Diagrama de casos de uso para seleção dos modos de operação suportados pelo dispositivo.}
%  {fig:uml-device-operation-modes}


%%%
\subsection{Relatórios de saúde}

A arquitetura proposta permite a obtenção dos relatórios de saúde dos dispositivos executando sobre
equipamentos monitorados pelo sistema de manutenção inteligente. Como todos os dados das análises
são armazenados na entidade Base de Dados, é possível, a qualquer momento, resgatá-los. A obtenção
dos relatórios dos dispositivos é possível com o uso do Gerenciador de Dispositivos. Os casos de uso
para obtenção dos relatórios de um dispositivo ou de um grupo de dispositivos são apresentados no
diagrama \gls{UML} da \cref{fig:uml-health-device-report}.

\includefigure
  {images/uml-health-device-report}
  {Diagrama de casos de uso para obtenção dos relatórios de saúde do dispositivo.}
  {fig:uml-health-device-report}


%%
\section{Estudo de caso}

\todo[inline]{Introdução da seção mostrando que a sistema será simulado com os dados obtidos em
trabalhos anteriores.}


%%%
\subsection{Visão geral do objeto de estudo de caso}

O objeto de estudo de caso deste trabalho é um conjunto atuador elétrico e válvula. Um atuador
elétrico permite a abertura e fechamento motorizado de válvulas através da movimentação de uma
haste. Dessa forma, válvulas são encontradas em diversas aplicações onde seja necessário o controle
de fluxo de fluidos, como, por exemplo, água, esgoto ou petróleo. Para cada aplicação, é indicado um
tipo diferente de válvula. Os tipos mais comuns são esfera, gaveta e
globo~\cite{campos2006controles}.

O conjunto atuador elétrico e válvula, utilizado no trabalho em questão, é utilizado para controle
de fluxo em redes de distribuição de petróleo\todo{Modelo do atuador}. É possível a obtenção do
torque mecânico exercido pelo atuador sobre as engrenagens de comando da haste bem como a posição do
obturador. O acionamento do conjunto pode ser feito localmente, através de uma interface de
programação local, sendo que todos os elementos de controle estão incorporados no próprio
equipamento. O atuador utilizado é dito de comando inteligente, pois, além de estar instrumentado,
possui a capacidade de detecção de alguns problemas, como sobreaquecimento, torque excessivo ou
falta de fase.

Durante o funcionamento, os valores de torque e posição da haste são armazenados em uma memória
interna do atuador, sendo possível registrar até \num{500} operações de fechamento ou abertura. No
entanto, as medidas obtidas pelo atuador não são suficientes para utilização em um sistema de
detecção de falhas. Uma das causas é a baixa resolução com que os dados são adquiridos, sendo de
\SI{1}{\newton\meter} e a cada 5\% de abertura para um fundo de escala de \SI{60}{\newton\meter}.
Além disso, como forma de detectar facilmente o início e o fim do movimento do atuador, o torque é
truncado para \SI{10}{\newton\meter} quando os valores forem menores que \SI{10}{\newton\meter} e,
quando não há movimento, assume o valor de \SI{0}{\newton\meter}~\cite{lazzaretti2012avaliacao}.

\includefigure
  {images/csr6-device}
  {Atuador elétrico modelo CSR6.}
  {fig:csr6-device}

O atuador foi montado em uma bancada de testes onde é possível a simulação das condições de desgaste
do conjunto e a instrumentação de partes que não são verificadas por padrão. Dessa forma, na
bancada, o atuador está instrumentado com mais três sensores, visando a obtenção dos níveis de
vibração da estrutura. Os sensores são acelerômetros, posicionados no eixo do motor do atuador, na
ponta do sem-fim e na pinça do freio. A \cref{fig:study-case} apresenta a bancada de testes
utilizada para obtenção dos dados utilizados no estudo de caso enquanto que a
\cref{fig:sensors-localization} ilustra o posicionamento dos sensores instalados e a estrutura
interna do atuador elétrico.

\todo[inline]{Modelo do atuador elétrico no SolidWorks.}

\includefiguretmp
  %{images/study-case}
  {Bancada de testes utilizada para obtenção dos dados do estudo de caso.}
  {fig:study-case}

\includefiguretmp
  %{images/sensors-localization}
  {Diagrama da localização dos sensores instalados no atuador elétrico.}
  {fig:sensors-localization}

A bancada também conta com um sistema de freio a disco instalado no eixo do atuador, o que
possibilita a simulação dos esforços mecânicos exercidos pelo fluxo de fluido durante a abertura e
fechamento da válvula. A simulação do desgaste é feita pela troca de engrenagens com diferentes
níveis de degradação. No total são três engrenagens, uma em estado normal e outras duas apresentando
falha. Das engrenagens com falha, uma está desgastada e a outra possui uma parte quebrada. As três
engrenagens utilizadas no estudo de caso são ilustradas na \cref{fig:test-gears}.

\includefiguretmp
  %{images/test-gears}
  {Engrenagens utilizadas no estudo de caso.}
  {fig:test-gears}

As engrenagens fazem parte do conjunto de engrenagens satélite do atuador. Este conjunto é
responsável pela transmissão do movimento do motor para a haste da válvula. A posição do conjunto de
engrenagens satélite é apresentada na \cref{fig:sattelite-gears-location}. O sensor de torque, que
faz parte do conjunto sem necessidade de modificação na estrutura, está acoplado ao eixo de
acionamento do torque e também está identificado na mesma figura.

\includefigure
  {images/sattelite-gears-location}
  {Localização das engrenagens satélite no atuador elétrico.}
  {fig:sattelite-gears-location}


%%%
\subsection{Definição dos dispositivos}

O conjunto utilizado como estudo de caso foi mapeado para a representação de dispositivos da
arquitetura proposta.

No contexto deste trabalho, um dispositivo é a entidade lógica principal que abstrai um elemento da
aplicação. Pode representar uma entidade física, como um sensor ou atuador, ou lógica, como uma
máquina, composta por diversas entidades de hardware mapeadas como entidades lógicas. Cada um dos
dispositivo hospeda serviços, representando tarefas ou funcionalidades específicas possíveis de
serem executadas. Tanto os dispositivos, bem como os serviços por eles hospedados, podem ser
descobertos e identificados na rede. Os dispositivos podem descobrir outros dispositivos e utilizar
os serviços do segundo, a fim de criar um serviço composto mais complexo para execução de
determinada tarefa. A \cref{fig:device-services-overview} ilustra a topologia para os dispositivos
utilizada neste trabalho. Os clientes utilizam os serviços hospedados pelos dispositivos
diretamente. Um serviço também pode ser considerado um cliente caso utilize de uma funcionalidade
remota para prover a sua funcionalidade ou tarefa. Além disso, da mesma forma, um dispositivo também
pode ser considerado um cliente, o que flexibiliza a integração entre os componentes do sistema.

\includefigure
    {images/device-services-overview}
    {Visão geral dos clientes, dispositivos e serviços hospedados.}
    {fig:device-services-overview}

\iffalse
O modelo de dispositivos empregados neste estudo é apresentado na figura\todo{Incluir figura}. Nela,
o dispositivo físico é abstraído por um dispositivo lógico, identificado por\textit{XX*}. O
dispositivo lógico inclui alguns serviços padrão, referenciando funcionalidades que podem ser
encontradas em todos os dispositivos presentes na arquitetura proposta. Dentre os serviços padrão,
tem-se o serviço de implantação de novos serviços. Este serviço permite que outros serviços sejam
adicionados ao dispositivo físico. Além disso, é possível a implantação de novos dispositivos
lógicos, a fim de abstrair componentes da aplicação, juntamente com serviços do
usuário\todo{Melhorar o texto com base na imagem que será incluída.}.
\fi

\includetable
  {tables/device-metadata}
  {Metadados disponíveis para configuração em cada dispositivo.}
  {tab:device-metadata}


\includetable
  {tables/sensor-info}
  {Parâmetros configuráveis dos sensores.}
  {tab:sensor-info}



%%%
\subsection{Aquisição dos dados}

Os dados foram obtidos de acordo com os trabalhos de~\cite{boesch2011deteccao} e
\cite{faccin2011manutencao}. Segundo os autores, a aquisição dos dados foi feita através de um
sistema desenvolvido em LabVIEW utilizando o chassis {cRIO-9104} e o módulo para aquisição dos dados
de vibração {NI-9233}. Os dados foram coletados durante seis situações distintas, onde, para cada
situação, foram obtidos \num{25} conjuntos de dados de abertura e \num{25} de fechamento da válvula,
o que representa um total de \num{300} curvas de experimentos. As situações foram: utilização de
engrenagens em perfeito estado e sem acionamento do freio mecânico; engrenagens em perfeito estado e
acionamento do freio mecânico com pressão de \SI{1}{\bar}; engrenagens em perfeito estado e
acionamento do freio mecânico com pressão de \SI{3}{\bar}; utilização de uma engrenagem desgastada
sem acionamento do freio; três engrenagens desgastadas sem acionamento do freio; uma das engrenagens
com dentes quebrados sem acionamento do freio mecânico.

Os dados de cada experimento foram amostrados a uma taxa de \SI{2048}{S\per\second}. Dessa forma,
cada conjunto de abertura ou fechamento da válvula tem duração aproximada de \SI{45}{\second}. Para
tornar o processamento dos dados menos custoso, uma subamostragem para a taxa de
\SI{1024}{S\per\second} foi realizada. Segundo~\cite{faccin2011manutencao}, a subamostragem dos
dados não prejudica o resultado da análise, porém torna o processamento menos custoso.


%%%
\subsection{Análise de degradação}

A análise de degradação do atuador é feita utilizando somente um dos três sensores citados
anteriormente. De acordo com os trabalhos de~\cite{boesch2011deteccao} e
\cite{faccin2011manutencao}, o acelerômetro acoplado ao eixo do motor é o que apresenta melhores
resultados para a detecção na degradação das engrenagens do atuador. Como apresentado
em~\cite{lazzaretti2012avaliacao}, os dados que apresentam melhores resultados para a obtenção dos
níveis de degradação são as curvas de abertura da válvula, devido a maior força que é exercida na
sede da válvula.

Os sinais de vibração não são estacionários. Para analisá-los, obtendo as características variantes
no tempo, técnicas de processamento de tempo e frequência são as mais indicadas. Portanto, a
extração das características dos sinais pode ser feita utilizando \gls{WPE}. A eficácia do uso deste
método para obtenção das características de sinais variantes no tempo é descrita no trabalho
de~\cite{qiu2006wavelet}.

Considerando a forma como os dados foram obtidos, pode-se supor que o atuador opera em três
situações distintas: em falha, em funcionamento normal e em teste. O funcionamento em falha se
refere à operação utilizando uma das engrenagens com dentes quebrados sem acionamento do freio
mecânico. O modo normal é definido com a operação quando é sabido que nenhuma das engrenagens
apresenta estado de degradação e o freio mecânico não é utilizado. Por fim, o modo de teste é aquele
onde o equipamento está sendo analisado, a fim de determinar o nível de degradação.

Os três tipos de situações definidas pela extração dos dados podem ser utilizadas para treinamento
do equipamento. Dessa forma, como apontado em~\cite{lazzaretti2012avaliacao}, uma das forma de se
fazer o treinamento do equipamento é utilizando o método da Regressão Logística.

%%
\section{Definição dos experimentos}

Os experimentos a serem realizados tem por base validar a proposta apresentada neste trabalho.


%%%
\subsection{Comparação da proposta em relação aos métodos tradicionais}


%%%
\subsection{Comparação dos métodos em relação a escalabilidade}


  \chapter{Implementação e resultados}
\label{cha:implementacao-resultados}

Este capítulo descreve a implementação e resultados obtidos com o sistema proposto. A implementação
faz referências diretas sobre o que foi apresentado no capítulo anterior, onde a proposta geral do
trabalho foi definida. A forma como as entidades que fazem parte da arquitetura orientada a serviços
foram implementadas é descrita e são apresentados resultados.

Alguns experimentos visam comprovar as vantagens na utilização da proposta definida neste estudo.
São analisadas questões como a facilidade no gerenciamento das análises e equipamentos, bem como a
escalabilidade da solução. Também são testadas as entidades implementadas, verificando a
interoperabilidade entre os diferentes softwares desenvolvidos.


%%
\section{Implementação das entidades da arquitetura proposta}

Para a implementação de todas as entidades de software utilizou-se a linguagem de programação
Java~\cite{java2013homepage}. Aplicativos construídos utilizando esta linguagem executam em uma
máquina virtual. Dessa forma, é possível a execução em diferentes sistemas operacionais sem a
necessidade de reescrita ou reimplementação de partes do software. A única premissa é que exista uma
versão compatível da máquina virtual para o sistema operacional alvo.

Neste trabalho, a especificação \gls{SOA} escolhida foi a \gls{DPWS}. Conforme visto na
\cref{sec:arquiteturas-orientadas-servicos}, como \gls{SOA} é um padrão que não define a forma de
implementação, é possível encontrar diversas. Dentre as várias opções, algumas são voltadas para o
cenário corporativo, para a integração entre grandes empresas, e que estão focadas na manipulação e
gerência de um grande volume de dados e operações. Outras são empregadas em cenários mais simples do
ponto de vista de número de tarefas e conjunto diferenciado de serviços. Dentre as várias
especificações, o \gls{DPWS} foi o mais indicado para este trabalho devido às funcionalidades
disponibilizadas, as quais cumprem com os requisitos estipulados no projeto da arquitetura.

A especificação \gls{DPWS} possui algumas diferentes implementações em meio à diferentes projetos.
Dentre os projetos, pode-se citar WS4D, SOA4D e .Net Microframework~\cite{marcelo2013analise}. Os
dois primeiros contemplam implementações em linguagens Java, C e C++, o que reduz a dependência por
um sistema operacional específico. A terceira faz parte de um conjunto de ferramentas desenvolvido
pela Microsoft para sistemas embarcados, possuindo implementações em C\#. Neste trabalho optou-se
pela utilização da implementação \gls{JMEDS}, que faz parte do projeto
WS4D~\cite{jmeds2013homepage}. Esta escolha partiu do fato de que o projeto em questão conta com
grande atividade e contribuidores e, como as entidades de software são implementadas em Java, a
integração entre a biblioteca se dá de forma facilitada.


%%%
\subsection{Implementação do Gerenciador de Dispositivos}

Como visto na \cref{sub:proposta-gerenciador-dispositivos}, o Gerenciador de Dispositivos se trata
de um componente de software responsável pelo gerenciamento dos dispositivos na arquitetura
proposta. De forma geral, o gerenciador é utilizado pelo operador do sistema para obter os
dispositivos presentes na rede e configurá-los. A \cref{fig:device-manager-main-screen} ilustra a
tela principal do Gerenciador de Dispositivos.

\includefiguresoft
  {images/device-manager-main-screen}
  {Tela principal do Gerenciador de Dispositivos.}
  {fig:device-manager-main-screen}


%%%%
\subsubsection{Obtenção da lista de dispositivos e serviços}

Uma das funcionalidades do gerenciador é a obtenção da lista de dispositivos presentes na rede. O
conjunto de ferramentas \gls{JMEDS}~\cite{jmeds2013homepage} implementa a especificação
{WS-Discovery}, permitindo a descoberta de serviços através de um protocolo padronizado. Dessa forma,
toda a busca e troca de mensagens para a descoberta e identificação dos serviços é disponibilizada
de forma transparente para o desenvolvedor da aplicação.

Com a lista de dispositivos, é possível saber o estado de cada um. Ainda conforme a
\cref{fig:device-manager-main-screen}, é possível observar que o gerenciador apresenta uma área onde
são exibidas as informações pertinentes do dispositivo, como nome, modelo e fabricante do
dispositivo. Estas informações são apresentadas quando o operador do sistema seleciona um dos
elementos da lista de dispositivos encontrados e referem-se à \cref{tab:device-metadata}.

O operador tem a possibilidade de visualizar informações mais detalhadas sobre o dispositivo
selecionado na tela principal do gerenciador. Esta funcionalidade é disponibilizada através do botão
\emph{Mais informações}, localizado na área de informações do dispositivo. Ao ser pressionado, um
novo diálogo é aberto. Em complementação às informações do dispositivo, apresentadas na tela
principal do gerenciador, o novo diálogo exibe informações mais detalhadas, como uma \gls{URL} para
o número do modelo e outra para o endereço eletrônico do fabricante. A fim de atualizar estas
informações, o operador do sistema poderá editá-las utilizando outro botão. Ao ser pressionado, um
novo diálogo é aberto, possibilitando a inserção, edição ou atualização de todas as informações do
dispositivo selecionado. A \cref{fig:device-manager-sensors-dialog} ilustra a janela de diálogo
contendo as informações do dispositivo, a lista de sensores e os botões mencionados anteriormente.

\includefiguresoft
  {images/device-manager-sensors-dialog}
  {Diálogo com informações detalhadas sobre o dispositivo e lista de sensores disponíveis.}
  {fig:device-manager-sensors-dialog}

%Dentre os serviços listados no diálogo de informações do dispositivo, estão aqueles comuns a todos
%os dispositivos da arquitetura proposta juntamente com os específicos. Os serviços específicos são
%aqueles que diferem entre os dispositivos, como, por exemplo, o serviço para obtenção de dados de um
%determinado sensor. É possível listar somente os serviços específicos do dispositivo através de uma
%caixa de seleção. Dessa forma, lista presente no diálogo da
%\cref{fig:device-manager-services-dialog} exibirá somente os serviços não padronizados.

Informações detalhadas sobre cada um dos sensores presente no dispositivo também podem ser obtidas
através do Gerenciador de Dispositivos. No diálogo de informações do dispositivo, após a seleção de
determinado sensor, os campos à direita são preenchidos com informações adicionais sobre o
componente. Esta funcionalidade também é ilustrada na \cref{fig:device-manager-sensors-dialog}.


%%%%
\subsubsection{Topologia da rede e configuração da base de dados}

A topologia da rede é obtida pelo Gerenciador de Dispositivos no momento em que o operador do
sistema realiza uma nova busca de dispositivos. Como anteriormente apresentado na
\cref{fig:device-manager-main-screen}, a tela principal do gerenciador lista todos os dispositivos
encontrados, bem como os subdispositivos, localizados em níveis hierárquicos inferiores. Isso é
possível pelo fato de que cada dispositivo da arquitetura proposta deve apresentar um identificador
único. O identificador, no caso implementado, uma \gls{URL}, é utilizada para construir a hierarquia
dos dispositivos da rede.

Na \gls{URL} de identificação de cada dispositivo, estão codificadas informações para que o
gerenciador possa criar a topologia da rede. Como apresentado na
\cref{sub:proposta-descoberta-configuracao-dispositivos}, os dispositivos são codificados seguindo
um padrão de nomes, os quais são separados pelo caractere "/". A convenção utilizada facilita a
definição de dispositivos que fazem parte de dispositivos com nível hierárquico maior. Também é
possível a definição de um nome para a representação de um grupo de dispositivos ou área onde os
dispositivos estão localizados.

A configuração da base de dados é fundamental para o correto funcionamento do sistema proposto. A
base pode ser configurada através da tela principal do Gerenciador de Dispositivos, pelo principal
de configuração do gerenciador, sendo que, após este procedimento, todos os dispositivos serão
informados de como o acesso a base deve ser realizado. O gerenciador também a utiliza para manter
uma referência dos dispositivos encontrados na rede. Caso um dos dispositivos encontrados não
apresentar registro na base de dados, o gerenciador deve criá-lo. Além do dispositivo, também são
criados os registros de subdispositivos ou sensores. O diálogo de configuração do Gerenciador de
Dispositivos é apresentado na \cref{fig:device-manager-config-dialog}.

\includefiguresoft
  {images/device-manager-config-dialog}
  {Diálogo de configuração do Gerenciador de Dispositivos.}
  {fig:device-manager-config-dialog}


%%%%
\subsubsection{Configuração dos dispositivos}

A configuração dos dispositivos é realizada pelo operador do sistema também através do Gerenciador
de Dispositivos. A configuração inclui o mapeamento dos sensores, para armazenar os dados de
interesse, a definição de comportamentos, que influenciam no modo de operação do equipamento, e o
envio de dados de treinamento. As configurações do dispositivo são acessadas pelo diálogo
apresentado anteriormente na \cref{fig:device-manager-sensors-dialog}. O diálogo é acessível através
da tela principal do Gerenciador de Dispositivos. Para ativá-lo, é necessário que o operador do
sistema selecione um dos dispositivos da lista antes do processo de configuração.

%\includefiguretmp
%  %{images/device-config-dialog}
%  {Diálogo de configuração do dispositivo.}
%  {fig:device-config-dialog}

No diálogo de configuração, além de outras informações pertinentes ao dispositivo selecionado, estão
a lista de sensores que o equipamento possui. As informações de um determinado sensor são exibidas
no momento em que o usuário o seleciona na lista de sensores. Dentre as informações, estão o
fabricante, modelo e versão do sensor, além de uma descrição e a unidade de medida utilizada na
aquisição dos dados. A interface também disponibiliza algumas configurações para o sensor
selecionado, como a taxa de amostragem e o número de dados que necessitam ser amostrados para envio
à base de dados. Estas configurações estão disponíveis em um outro diálogo, acessível através do
botão identificado como \emph{Configurar aquisição}.

A configuração dos comportamentos do equipamento, conforme descrito na
\cref{sub:proposta-gerenciador-dispositivos}, também é feita no Gerenciador de Dispositivos. Os
diferentes comportamentos definidos para um equipamento possibilitam que ele opere de diferentes
maneiras dependendo do nível de degradação. Os comportamentos configurados estão acessíveis através
do diálogo de configuração do dispositivo (\cref{fig:device-manager-sensors-dialog}). O diálogo de
configuração dos comportamentos é apresentado na \cref{fig:device-config-behavior-dialog}. Ao
selecionar um dos elementos da lista, as informações adicionais são preenchidas. O diálogo permite
ainda o gerenciamento dos comportamentos, como o envio ou exclusão de um elemento existente. O envio
é feito através de um botão, que, quando acionado, abre um novo diálogo para a criação de um
comportamento. O operador deverá definir um nome para o novo comportamento e anexar um arquivo
contendo o código executável no formato \gls{JAR}. Dessa forma, o comportamento é enviado para o
dispositivo e poderá ser executado em alguma situação, dependendo do tipo de degradação observado.

\includefiguresoft
  {images/device-config-behavior-dialog}
  {Diálogo para configuração dos comportamentos do dispositivo.}
  {fig:device-config-behavior-dialog}

Os dados de treinamento são acessíveis pelo mesmo diálogo de configuração dos dispositivos, conforme
\cref{fig:device-manager-sensors-dialog}. No diálogo de dados de treinamento, o operador poderá
escolher para qual sensor os dados serão mapeados e qual o tipo dos dados, conforme apresentado na
\cref{fig:device-config-training-dialog}. O tipo de dados pode ser normal ou falha, os quais
representam, respectivamente, uma situação em que o equipamento está se comportando conforme o
esperado e quando existe algum comportamento indesejado. Também é necessária a inclusão de
informação sobre a origem dos dados, no campo denominado de descrição. O diálogo aceita arquivos de
dados no formato \gls{CSV}. A \cref{fig:device-config-training-dialog} ilustra a interface de envio
de dados de treinamento.

\includefiguresoft
  {images/device-config-training-dialog}
  {Diálogo de envio de dados de treinamento.}
  {fig:device-config-training-dialog}


%%%%
\subsubsection{Definindo grupos de dispositivos}

A configuração ou envio de dados para um conjunto de dispositivos de mesma natureza é possível
através da criação de grupos de dispositivos. Dessa forma, dispositivos podem vir a ser agrupados
tendo em comum o modelo ou fabricante, por exemplo. A configuração dos grupos é acessível através do
diálogo de configuração do Gerenciador de Dispositivos, ilustrado na
\cref{fig:device-manager-config-dialog}. A interface para configuração e gerenciamento dos grupos de
dispositivos é apresentada na \cref{fig:device-config-groups-dialog}. Nela estão presentes a lista de
grupos criados pelo operador do sistema e as possibilidades de gerenciamento (criar, editar e/ou
excluir). Ao selecionar um grupo existente, na lista à esquerda da figura em questão, os
dispositivos que o compõe são apresentados. Pode-se adicionar ou remover dispositivos através da
funcionalidade de edição de grupos.

\includefiguresoft
  {images/device-config-groups-dialog}
  {Diálogo de definição e configuração de grupos de dispositivos.}
  {fig:device-config-groups-dialog}

As mesmas operações permitidas para um dispositivo individual é possível para os grupos. Sendo
assim, é possível a definição de novos comportamentos para o grupo de dispositivos e o envio de
dados de treinamento. O envio de novos comportamentos utiliza o mesmo mecanismo apresentado
anteriormente, quando abordado do diálogo de configuração de um dispositivo, exceto que, no caso de
grupos, um mesmo comportamento é enviado para um conjunto de dispositivos.

%\includefiguretmp
%  %{images/device-config-group-dialog}
%  {Diálogo de configuração de um grupo de dispositivos.}
%  {fig:device-config-group-dialog}

O envio de dados de treinamento também segue o mesmo padrão apresentado anteriormente na descrição
da configuração de somente um dispositivo. O caso particular se encontra no que diz respeito a
escolha do sensor que usufruirá dos novos dados. O Gerenciador de Dispositivos agrupa
automaticamente os sensores de mesmo nome, dessa forma, o envio de dados de treinamento é feito com
base no nome do sensor. Ao acessar o diálogo de envio de dados de comportamento, os sensores de
todos os dispositivos do grupo estarão disponíveis. Será ignorado o sensor escolhido para envio dos
dados que não estiver presente em nenhum dos dispositivos.


%%%%
\subsubsection{Obtenção de relatórios de saúde}

Os relatórios de saúde podem ser obtidos pelo operador do sistema através do diálogo apresentado na
\cref{fig:device-health-report-dialog}, acessível pela interface de configuração de um dispositivo.
O diálogo permite definir o intervalo para geração do relatório, o qual contém os valores de
confiança obtidos para cada lote de dados. Os dados são obtidos diretamente da base de dados e o
relatório é fornecido ao operador do sistema em arquivo texto no formato \gls{CSV}.

\includefiguresoft
  {images/device-health-report-dialog}
  {Diálogo para obtenção de relatórios de saúde.}
  {fig:device-health-report-dialog}


%%%
\subsection{Implementação do Gerenciador de Análises}

Como apresentado na \cref{sub:proposta-gerenciador-analises}, o Gerenciador de Análises é um dos
componentes de software presentes na arquitetura proposta que auxilia no processo de criação de
planos de análise para os diversos dispositivos presentes no sistema. Através dele, o operador do
sistema pode gerenciar o processo de análise de um equipamento, definindo as ferramentas utilizadas
para manipulação dos dados, bem como a origem dos dados utilizados. Além disso, é possível que o
equipamento assuma diferentes comportamentos baseado no resultado das análises realizadas. A tela
principal do Gerenciador de Análises é apresentada na \cref{fig:analysis-manager-main-screen}.

\includefiguresoft
  {images/analysis-manager-main-screen}
  {Tela principal do Gerenciador de Análises.}
  {fig:analysis-manager-main-screen}

As análises cadastradas anteriormente são exibidas em uma lista na tela principal do Gerenciador de
Análises. Quando da seleção de um dos elementos da lista, informações adicionais, como nome,
descrição e número de dispositivos que utilizam a mesma análise, são fornecidas. Também são
fornecidas informações sobre a execução da análise, como a última vez em que o plano de análise foi
executado, e se a análise está ativa ou não. As análises e respectivas informações são obtidas
diretamente da base de dados.


%%%%
\subsubsection{Definição de um plano de análise}

O operador do sistema poderá definir novas análises ou gerenciar as existentes, conforme
\cref{fig:analysis-manager-main-screen}. Na definição de uma nova análise, o diálogo, ilustrado na
\cref{fig:analysis-create-dialog}, é aberto. Através do diálogo, o operador deve definir um nome e
descrição para o plano de análise, bem como quais dispositivos e sensores farão parte dela. Também
deve definir as ferramentas computacionais utilizadas e, se pertinente, configurar quais
comportamentos os dispositivos devem assumir frente a diferentes situações de degradação. A
periodicidade da execução da análise também pode ser configurada.

\includefiguresoft
  {images/analysis-create-dialog}
  {Diálogo de criação de uma nova análise.}
  {fig:analysis-create-dialog}

A configuração de dispositivos e sensores é realizada em um novo diálogo. O diálogo permite ao
operador configurar quais dispositivos serão analisados. Conjuntamente, ao selecionar os
dispositivos, também é possível definir quais sensores serão utilizados na análise de degradação. O
princípio de escolha dos dispositivos é herdado do Gerenciador de Dispositivos, exceto por não ser
possível a criação de novos grupos de dispositivos através do Gerenciador de Análises. O operador
pode optar pela escolha de somente um dispositivo, bem como de um grupo de dispositivos. No entanto,
o grupo de dispositivos deve ter sido criado previamente através do Gerenciador de Dispositivos.

As ferramentas de análise dizem respeito aos algoritmos utilizados para analisar os dados obtidos. O
processo de escolha é apresentado no diálogo da \cref{fig:analysis-tools-dialog}. Dois conjuntos de
ferramentas estão disponíveis na implementação do Gerenciador de Análises: processamento de sinais e
extração de características e avaliação do desempenho do sistema. Transformada rápida de Fourier,
análise tempo-frequência e energias da transformada wavelet packet fazem parte do primeiro conjunto.
Por conseguinte, regressão logística e reconhecimento estatístico de padrões fazem parte do segundo
conjunto de ferramentas.

Se implementado, as ferramentas permitem a configuração dos algoritmos. Após a seleção de uma delas,
o operador poderá optar por configurá-la. Para tanto, estão disponíveis botões na interface com o
usuário. Ao clicá-los, novos diálogos são abertos com as configurações específicas da ferramenta
selecionada.

\includefiguresoft
  {images/analysis-tools-dialog}
  {Diálogo para escolha das ferramentas utilizadas na análise.}
  {fig:analysis-tools-dialog}

O operador poderá também definir os comportamentos utilizados como resposta aos níveis de degradação
obtidos. O diálogo da \cref{fig:analysis-behavior-dialog} ilustra alguns comportamentos e as
informações pertinentes àquele que foi selecionado. No procedimento para seleção dos comportamentos
utilizados pelo dispositivo, o operador deverá indicar, através das caixas de seleção, quais
comportamentos quer utilizar e definir as faixas de variação do valor de confiança onde o algoritmo
estará ativo. Como o valor de confiança pode assumir valores no intervalo ${[0; 1]}$, são
disponibilizados dois campos para a seleção do valor mínimo e máximo empregado na definição do
comportamento, os quais podem assumir valores dentro desse mesmo intervalo.

\includefiguresoft
  {images/analysis-behavior-dialog}
  {Diálogo para a configuração dos comportamentos dos dispositivos.}
  {fig:analysis-behavior-dialog}

A configuração da periodicidade em que a tarefa será executada também pode ser acessada pelo diálogo
da \cref{fig:analysis-create-dialog}. Um novo diálogo é aberto, onde o operador poderá definir as
propriedades de repetição da análise. É possível definir um intervalo fechado, onde, após um
determinado período, as análises são interrompidas, ou a execução permanente, onde as análises
sempre serão executadas. A periodicidade da execução, ou seja, o intervalo entre uma execução e
outra, é definido pelo número de amostras coletadas dos sensores. Como apresentado no Gerenciador de
Dispositivos, o operador deve definir o tamanho de um lote de dados que será enviado à base de
dados. O dispositivo armazena estes dados e, quando o lote estiver completo, o envia a base de
dados. A base de dados é monitorada para a execução de uma análise sobre o novo lote de dados, a fim
de gerar o novo valor de confiança.


%%%
\subsection{Implementação do Analisador de Dispositivos}

O Analisador de Dispositivos é a entidade responsável por executar os planos de análise definidos
pelo operador do sistema no Gerenciador de Análises. Conforme definido na
\cref{sub:proposta-analisador-dispositivos}, o Analisador de Dispositivos concentra as ferramentas
necessárias para análise dos dados e obtenção dos níveis de degradação dos equipamentos. A tela
inicial do analisador é apresentada na \cref{fig:analyzer-main-screen}.

\includefiguresoft
  {images/analyzer-main-screen}
  {Tela principal do Analisador de Dispositivos.}
  {fig:analyzer-main-screen}

O analisador mantém uma fila das análises ativas agendadas pelo operador do sistema. O Analisador de
Dispositivos implementa uma tarefa periódica que verifica a base de dados em busca de novos lotes de
dados relativos às análises ativas. Quando novos lotes são encontrados, os respectivos planos de
análise são postos em uma fila e executados sequencialmente. Os resultados gerados são armazenados
na base de dados.

A tela principal do Analisador de Dispositivos (\cref{fig:analyzer-main-screen}) apresenta algumas
informações úteis para o operador do sistema. São listados os planos atualmente enfileirados e
estatísticas sobre as execuções, como o número de planos agendados e quantas análises estão
pendentes para execução. Informações adicionais sobre um determinado plano podem ser obtidas pelo
operador através da seleção do elemento na lista da tela principal.


%%%%
\subsubsection{Ferramentas para análise dos dados e interoperabilidade entre diferentes
    fornecedores}

As ferramentas de análise dos dados disponíveis no analisador são construídas em forma de interface
para os algoritmos presentes no software Watchdog Agent, versão 3.3\todo{referencia}. Os algoritmos
são distribuídos na forma de \textit{scripts} integrados em um software que executa em ambiente
MATLAB. A solução empregada no desenvolvimento do analisador foi a de isolar os \textit{scripts} que
implementam as ferramentas desejadas e executá-los individualmente através do novo software. Dessa
forma, não foi necessária a reescrita dos algoritmos utilizando a linguagem Java.

A execução dos \textit{scripts} MATLAB foi possível utilizando a biblioteca
\emph{matlabcontrol}~\cite{matlabcontrol2013homepage}, versão 4.1.0. A biblioteca possibilita uma
ponte para troca de dados entre um programa escrito em Java e o MATLAB. O mecanismo empregado na
solução encontrada é ilustrado na \cref{fig:analyzer-matlab-interface}. Após a coleta dos dados,
obtidos da base de dados, para a execução de uma nova análise, o Analisador de Dispositivos
seleciona a ferramenta que deve ser empregada, a qual foi definida no plano de análise~(1). Como a
ferramenta é um \textit{script} Matlab, o analisador carrega os dados e os envia para o Matlab~(2),
que, por sua vez, repassa-os para os algoritmos presentes no Watchdog Agent~(3). Os algoritmos são
disparados para execução através da utilização da biblioteca \emph{matlabcontrol}~(4). Do mesmo
modo, ao final da execução dos algoritmos, o analisador utiliza novamente a biblioteca
\emph{matlabcontrol} para recuperar os resultados gerados~(5), sendo disponibilizados novamente para
acesso no Matlab~(6) pelo Analisador de Dispositivos. O processo é repetido se o plano estipular que
os resultados anteriores necessitam de uma nova análise. Tanto os resultados intermediários como o
resultado final são armazenados novamente na base de dados.

\includefigure
  {images/analyzer-matlab-interface}
  {Diagrama de interoperabilidade entre o Analisador de Dispositivos e os \textit{scripts} do
      Watchdog Agent executando no software Matlab.}
  {fig:analyzer-matlab-interface}

A forma com que o Analisador de Dispositivos foi construído permite a utilização de ferramentas de
análise de outros fornecedores. Como exemplo, existe uma versão do Watchdog Agent que executa em
conjunto com o software LabVIEW. Devido a uma camada de abstração empregada na separação da
definição das ferramentas e na sua implementação, é possível a troca ou mesmo a execução simultânea
de ferramentas disponibilizadas por diferentes fornecedores, desde que uma nova interface seja
provida. A \cref{fig:class-matlab-interface} ilustra o diagrama de classes utilizado na definição da
interface de abstração entre as ferramentas. A interface \emph{IMSTools} é utilizada como base para
a construção das ferramentas de análise. Nela, estão definidas todas as interfaces para os
algoritmos necessários para o processamento de dados. Já a classe de interface \emph{WatchdogAgent}
estende a interface principal para prover uma abstração para a implementação das funcionalidades da
classe principal pelas ferramentas do Watchdog Agent. As outras duas subclasses, que utilizam a
interface anterior, implementam as as ferramentas para manipulação dos dados. Dessa forma, o exemplo
ilustra que é possível a extensão do sistema, simplesmente fornecendo uma interface para a
implementação das ferramentas de análise.

\includefigure
  {images/class-matlab-interface}
  {Diagrama de classes para abstração das ferramentas de análise.}
  {fig:class-matlab-interface}


%%%%
\subsubsection{Alteração do comportamento do dispositivo}

Se definido no plano de análise, o dispositivo pode alterar o modo de operação dependendo do nível
de degradação obtido com a análise. Como visto anteriormente, o operador do sistema pode definir
comportamentos para o dispositivo, a fim de alterar o modo de operação em diferentes situações de
degradação. Os comportamentos são definidos através do Gerenciador de Dispositivos e ficam
disponíveis no dispositivo configurado. Após a análise de um plano, o nível de degradação expresso
pelo valor de confiança do equipamento é obtido e comparado com as faixas definidas para cada
comportamento. Se o valor calculado estiver em alguma das faixas cobertas por um dos comportamentos,
o Analisador de Dispositivos automaticamente informa para o dispositivo qual comportamento deve
utilizar.


%%%
\subsection{Implementação da camada de acesso a base de dados e definição da estrutura interna}

Como apresentado na \cref{sub:proposta-base-dados}, é necessária a implementação de uma entidade de
software para abstrair o acesso a base de dados através de serviços. O software implementa, para
cada componente que necessite de persistência de dados, serviços com operações pertinentes. Por
exemplo, a busca ou manipulação de dispositivos pode ser feita com o uso de serviços especializados
que manipulam a tabela onde estão armazenadas as informações destes elementos. No entanto, a fim de
evitar a redefinição de todas as operações para acesso às diferentes tabelas, o acesso também pode
ser feito pelos métodos tradicionais.

Neste trabalho foi utilizada uma base de dados SQLite~\cite{sqlite2013homepage}. A base de dados
SQLite é baseada em arquivos, não necessitando de servidor dedicado, tampouco de um cliente
especializado. O conteúdo da base de dados é armazenado em um único arquivo, facilitando a
manipulação e distribuição. A escolha por esta base partiu do fato da simplicidade na sua utilização
e manutenção, visto que implementa todas as funcionalidades necessárias ao projeto.

Na definição do esquema das tabelas da base de dados, as as informações sobre dispositivos e dados
para análise foram separados de modo a facilitar a utilização pelas entidades da arquitetura
proposta.
%O esquema da base se dados, com algumas das entidades principais, é apresentado na
%\cref{fig:database-schema}.

%\includefiguretmp
%  %{images/database-schema}
%  {Esquema simplificado utilizado na base de dados.}
%  {fig:database-schema}


%%%
\subsection{Implementação dos dispositivos}

Os dispositivos implementados neste trabalho se referem a atuadores elétricos. Os atuadores,
operando em conjunto com válvulas, como comentado na \cref{sec:estudo-caso}, são do modelo CSR6,
fabricados pela empresa Coester. Os equipamentos foram instrumentados e feitas simulações de
degradação em diferentes situações. Os dados da simulação das condições foram coletados, sendo
possível utilizá-los para simular o comportamento do equipamento sem a necessidade de montagem de
novos experimentos físicos em bancada de testes. No caso do embarque físico de um dispositivo no
atuador, a técnica utilizada apresenta vantagem, pois o software do dispositivo necessita de
modificações somente em pontos específicos, como a forma de obtenção dos dados dos sensores.

Através dos dados dos experimentos armazenados, fez-se necessário o desenvolvimento de uma entidade
de software para simular o comportamento do conjunto atuador elétrico e válvula. A entidade é
encarregada de obter os dados prévios e gerá-los nas mesmas condições em que foram obtidos. O
software tem por objetivo simular um dispositivo \gls{SOA} embarcado no equipamento, mas com dados
reais, facilitando o teste e validação do sistema. A \cref{fig:device-main-screen} apresenta a tela
principal da entidade para simulação dos dispositivos. A tela ilustra algumas informações sobre o
dispositivo, como nome, modelo, fabricante e número serial. Estas informações dizem respeito às
descritas na \cref{tab:device-metadata} e podem ser configuradas pelo operador do sistema. Outras
informações gerais, como qual experimento está sendo simulado e as ferramentas utilizadas para a
avaliação dos dados, também são apresentadas.

\includefiguresoft
  {images/device-main-screen}
  {Tela principal do software para simulação dos dispositivos.}
  {fig:device-main-screen}

Para cada uma das situações apresentadas na \cref{tab:device-metadata}, estão disponíveis \num{50}
curvas. Desse total, metade são referentes ao fechamento e a outra parte ao processo de abertura da
válvula. O dispositivo simula o processo de abertura da válvula indefinidamente, gerando os dados
coletados nos experimentos de forma sequencial. Ao término da geração do conjunto de dados, o
dispositivo gera os mesmos dados inciais continuamente.


%%%%
\subsubsection{Definição das situações de simulação}

As diferentes situações possíveis de simulação estão definidas como apresentado na
\cref{sub:estudo-caso-aquisicao-dados}. De acordo com a descrição do estudo de caso, são possíveis
seis experimentos, sendo um deles referente a operação em condições ditas normais, sem interferência
no funcionamento, e outras cinco onde o equipamento é submetido a alterações de comportamento. Ao
selecionar um dos experimentos, são exibidas informações adicionais, como nome, descrição e as
condições a que o equipamento foi submetido. Dentre as condições, estão o acionamento do freio e o
uso de engrenagens defeituosas.


%%%%
\subsubsection{Geração dos dados de simulação}

Em cada um dos casos, os dados dos sensores simulados são gerados de acordo com o que foi definido
no Gerenciador de Dispositivos. O operador pode definir o número de dados que serão armazenados
antes do envio para a base de dados configurada. Neste caso, evitando o acesso contínuo ao serviço
de envio de dados da base.

O lote de dados é definido pela quantidade de amostras. Após a aquisição do número de amostras
estipulado pelo operador do sistema, o software de simulação agrupa-os em um arquivo \gls{CSV}. Um
arquivo específico para cada sensor é criado. No arquivo, estão discriminados, em colunas, a data e
hora de aquisição de cada uma das amostras e o valor amostrado.


%%%%
\subsubsection{Obtenção do valor de confiança}

Como forma de verificação do funcionamento do equipamento simulado, o software disponibiliza
informações sobre os valores de confiança obtidos. É possível a visualização do valor de confiança
atual, bem como do valor máximo e mínimo obtidos durante o experimento. A data e hora na qual os
valores foram gerados também é informada.

O valor de confiança é obtido pela entidade diretamente da base de dados. Dessa forma, fica claro
que é possível o próprio equipamento tomar decisões a partir da condição calculada. Em contraste com
os comportamentos definidos pelo operador do sistema no Gerenciador de Dispositivos, o equipamento
pode apresentar algoritmos padronizados para assumir diferentes comportamentos em situações
específicas de degradação.


%%
\section{Resultados obtidos com a arquitetura proposta}

Os resultados obtidos com este trabalho são analisados de forma subjetiva e têm por base as questões
levantadas na \cref{sec:experimentos-definicao}. De uma forma geral são analisados quesitos como a
interoperabilidade das entidades propostas e a comparação do sistema implementado com métodos
tradicionais usualmente utilizados. A interoperabilidade entre as ferramentas é analisada, onde
testes são realizados para verificar o correto funcionamento das entidades projetadas. Por fim, a
proposta é comparada com os métodos tradicionais de análise de degradação, além da verificação das
dificuldades e vantagens encontradas na utilização do sistema para monitoramento de um conjunto de
equipamentos.


%%%
\subsection{Verificação da interoperabilidade entre as entidades}

Em um primeiro momento, todas as entidades foram executadas para verificar a interoperabilidade
entre elas conforme sugerido na \cref{sub:experimentos-interoperabilidade}. O Gerenciador de
Dispositivos foi iniciado e colocado em modo de monitoramento. Colocado em mondo de monitoração da
rede, o gerenciador verifica quando novos dispositivos são iniciados, possibilitando obter maiores
informações sobre cada um.

Após o Gerenciador de Dispositivos, um dispositivo foi iniciado. O dispositivo representa um atuador
elétrico. Verificou-se que é possível a alteração das informações do dispositivo, tanto pelo
gerenciador, como também pela interface gráfica da entidade de simulação. Este fator aponta o
sucesso encontrado na comunicação entre as entidades.

Com o dispositivo configurado e iniciado, criou-se um plano de análise utilizando o Gerenciador de
Análises. A análise foi criada obedecendo as condições apresentadas na
\cref{sub:estudo-caso-analise-degradacao}. Dessa forma, o módulo de processamento de sinais e
extração de características foi configurado para utilizar a transformada de wavelet packet, enquanto
que, no módulo de avaliação de desempenho do sistema, utilizou-se a ferramenta de regressão
logística.

Também foi iniciado o Analisador de Dispositivos. Após a sua inicialização, o plano criado
anteriormente é apresentado para execução na lista de planos ativos e agendados. O analisador
apresenta as informações detalhadas corretamente, o que novamente comprovou que as entidades se
comunicam de forma coerente. Por conseguinte, o analisador foi iniciado, entrando no modo de espera
pelos dados do dispositivo.

Com as ferramentas configuradas, foi iniciada a simulação do dispositivo. Utilizando a tela gráfica
criada para interface do simulador do atuador elétrico, testou-se as seis situação de teste,
conforme apontadas na \cref{tab:data-description}. Como os algoritmos para análise de degradação não
foram treinados, o dispositivo não atualizou as informações sobre o valor de confiança calculado em
nenhuma das situações testadas.

Para que o dispositivo possa ser analisado corretamente, os algoritmos necessitam passar por um
treinamento. Este fato é apontado na \cref{sub:estudo-caso-analise-degradacao}. Para o treinamento,
foram utilizados alguns dos dados obtidos com os testes nas situações de funcionamento normal e
falha. Os dados para treinamento correspondem ao movimento de abertura da válvula. Do total dos dois
conjuntos de curvas de abertura, representados por comportamentos normal e em falha, utilizou-se
\num{12} curvas em falha e \num{12} em funcionamento normal para treinar os algoritmos. Como falha,
foram utilizadas as curvas com o pior nível de degradação, obtidas na situação envolvendo uma
engrenagem quebrada (\cref{tab:data-description}). Dessa forma, \num{24} curvas foram utilizadas
para o treinamento dos algoritmos.

As curvas de treinamento foram anexadas ao dispositivo através da interface disponibilizada pelo
Gerenciador de Dispositivos. Conforme apresentado anteriormente na
\cref{fig:device-config-training-dialog}, os dados de treinamento podem ser enviados através de uma
interface de configuração, sendo possível selecionar para qual sensor os dados serão mapeados e qual
o tipo (normal ou falha). Dessa forma, os arquivos de cada uma das curva descritas anteriormente
foram enviados e configurados para o respectivo sensor. No experimento, como descito na
\cref{sub:estudo-caso-analise-degradacao}, utilizou-se o acelerômetro acoplado ao eixo do motor, o
qual registra as variações de vibração.


%%%
\subsection{Comparação da proposta com métodos tradicionais}

A proposta de comparação com os métodos tradicionais é apresentada e ilustrada na
\cref{sub:experimentos-metodos-tradicionais}. São levantadas hipóteses sobre a facilidade de
configuração da análise de dispositivos utilizando a arquitetura proposta em comparação com os
métodos usuais. Por métodos usuais, entende-se aqueles que se utilizam de um software específico, o
qual concentra as ferramentas de análise de dados e é manipulado por operador especializado. Um
exemplo deste tipo de software é o Watchdog Agent.

Como anteriormente apresentado nas seções que mostram a implementação do sistema, foram definidos
softwares especializados para as tarefas identificadas no processo de obtenção dos níveis de
degradação utilizando um sistema de manutenção inteligente. Cada um dos softwares é empregado um uma
tarefa distinta na aplicação. Por exemplo, o Analisador de Dispositivos é a entidade que concentra
as ferramentas para análise de dados e obtém os níveis de degradação do equipamento analisado. Uma
vez que o plano de análise foi criado, o analisador o executa automaticamente conforme planejado.
Neste ponto existe a vantagem, em relação ao método tradicional, de que não é mais necessária a
obtenção dos dados por parte do operador, tampouco a configuração das ferramentas de análise.

Quanto a aquisição dos dados, o sistema proposto necessita de implementações específicas que não
estão disponíveis em todos os equipamentos. No método tradicional, os dados são obtidos diretamente
por sensores posicionados estrategicamente pelo fabricante do equipamento ou através de
instrumentação posterior por equipe especializada. Com os sensores interligados em placas de
aquisição, os dados são amostrados diretamente, sem a necessidade de uma camada de abstração que, no
caso da arquitetura proposta, é representada por serviços. Do ponto de vista da aquisição dos dados,
os serviços \gls{DPWS} representam um grande consumo de recursos em um dispositivo. Porém, com a
evolução da tecnologia e diminuição dos custos na construção de dispositivos embarcados, o uso de
padrões \gls{SOA} mostra-se viável. Se os dispositivos utilizarem-se desta tecnologia para o
fornecimento dos dados, a integração com o sistema de análise será facilitada, o que representa
novamente ponto positivo na utilização da arquitetura proposta.


%%%
\subsection{Verificação da escalabilidade da proposta}

Este experimento se refere à \cref{sub:experimentos-escalabilidade} e presume que, nos métodos
tradicionais, o aumento de dispositivos em um processo de análise de dados fica prejudicado. Também
é dada como hipótese o fato da utilização de um mesmo processo de análise para diferentes
equipamentos. Utilizando-se os métodos convencionais, a configuração de diversos equipamento pode
representar uma dificuldade.

O Gerenciador de Análises permite a criação de planos que são aplicados para mais de um dispositivo.
De certo modo, a análise de mais de um dispositivo pelo método tradicional não representa grande
empecilho. O fator principal é a reconfiguração das ferramentas e injeção de dados diferentes de
forma manual a cada nova análise, o que demanda, principalmente, tempo. Mesmo com equipamentos de
mesma natureza, a cada nova análise o operador precisa definir manualmente o processo que será
executado. Na maioria das vezes, somente é possível a execução dos testes de forma \textit{offline}.
A vantagem na utilização da arquitetura proposta reside no fato do processo automatizado criado. Com
o Analisador de Dispositivos executando automaticamente os planos, fica evidente a facilidade da
análise dos dados. Em adendo, um mesmo plano pode ser empregado para um conjunto de dispositivos, o
que, de certo modo, soluciona o problema de reconfiguração das ferramentas e injeção de dados de
teste para cada nova análise.

  \chapter{Conclusão}

Trabalhos futuros:

Modo de aquisição de dados de treinamento.

Com a arquitetura proposta, no caso de um aumento excessivo do número de dispositivos que necessitam
de análise, é possível a utilização de mais de uma entidade Analisador de Dispositivos.

\end{onehalfspace}

\bibliographystyle{abnt}
\bibliography{bibliography/biblio}

\begin{onehalfspace}
  \appendix
  \chapter*{Apêndice A: Códigos utilizados na proposta}
\label{app:codigo-entidades}

Este apêndice apresenta alguns dos códigos utilizados no desenvolvimento da proposta. Algumas
informações foram omitidas a fim de destacar somente os pontos importantes da implementação das
classes.

\includejavacode
  {listings/waveletpackageenergies.java}
  {Implementação do serviço para cálculo das energias da transformada wavelet.}
  {lst:wavelet-package-energies}

\includejavacode
  {listings/datamanipulationdevice.java}
  {Implementação do dispositivo SOA contendo os serviços para manipulação de dados.}
  {lst:data-manipulation-device}

\includejavacode
  {listings/datamanipulationserviceprovider.java}
  {Implementação do provedor de serviços para a manipulação de dados.}
  {lst:data-manipulation-service-provider}

\includejavacode
  {listings/behaviorplugin.java}
  {Interface para a construção de novos comportamentos.}
  {lst:behavior-plugin-interface}

\includejavacode
  {listings/normalbehavior.java}
  {Implementação de um comportamento definido como normal seguindo a interface para construção de
      comportamentos.}
  {lst:normal-behavior-implementation}

\end{onehalfspace}

\end{document}
